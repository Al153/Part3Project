\documentclass{report}

%% Don't import the header multiple times

\ifdefined\HEADERIMPORTED
\else
\newcommand\HEADERIMPORTED[0]{This file is HEADERIMPORTED}
\usepackage{amssymb}

\usepackage{amsmath}


% For typesetting tree rules
\usepackage{mathpartir}

% For colouring code
\usepackage{xcolor}


\usepackage{array}   % for \newcolumntype macro
\usepackage{tikz-cd}
\usepackage{tabstackengine}
\usepackage{breqn}
\usepackage{stmaryrd}

\usepackage{float} % extra options for figure placement

% For drawing boxed
\usepackage{framed}

% for code fragments + highlighting
\usepackage{listings}

% For roman numerals
\usepackage{enumitem}


\usepackage{amsthm}
%Theorems
\usepackage[utf8]{inputenc}
\usepackage[english]{babel}

\ifdefined\PRESENTATIONMODE
\else
\usepackage[a4paper,includeheadfoot,margin=2.54cm]{geometry}
\newtheorem{theorem}{Theorem}[section]
\newtheorem{corollary}{Corollary}[theorem]
\newtheorem{lemma}[theorem]{Lemma}
\newtheorem{definition}{Definition}[section]

\newtheorem{aside}{Aside}[section]
\newtheorem{property}[theorem]{Property}
\theoremstyle{definition}
\fi



\usepackage{tikz}

\definecolor{grey}{rgb}{0.75, 0.75, 0.75}
\definecolor{DarkGreen}{rgb}{0.1, 0.6, 0.1}

\usetikzlibrary{shapes.geometric,fit}
\usetikzlibrary{arrows,automata,positioning}
\usetikzlibrary{decorations.pathreplacing,calc}



\setstackEOL{\cr}
\setstackgap{L}{\normalbaselineskip}

\newcommand\todo[1]{\textbf{TODO: #1}}
\newcommand\needsRef[1]{\textbf{Reference Needed: (#1)}}
\newcommand\fixLayout[1]{\textbf{Fix Layout: #1}}


%% Rule Names
% Prefixes
\newcommand{\tprefix}[0]{T-}
\newcommand{\eprefix}[0]{E-}
\newcommand{\sprefix}[0]{S-}
\newcommand\equationalprefix[0]{Eq-}
\newcommand\envprefix[0]{Env-}
\newcommand\pprefix[0]{\eprefix\envprefix}

\newcommand\subprefix[0]{Sb-}
\newcommand\weakenprefix[0]{Wk-}

% Base  rule names
\newcommand\basenil[0]{Nil}
\newcommand\baseextend[0]{Extend}

\newcommand{\baseground}[0]{Ground}
\newcommand{\baseweaken}[0]{Weaken}
\newcommand{\basevar}[0]{Var}
\newcommand\basefn[0]{Fn}
\newcommand\baseeffect[0]{Effect}
\newcommand\basequant[0]{Quantification}


\newcommand\baseunit[0]{Unit}
\newcommand\basetrue[0]{True}
\newcommand\basefalse[0]{False}
\newcommand\baseconst[0]{Const}
\newcommand\basesubtype[0]{Subtype}
\newcommand\basegen[0]{Effect-Gen}
\newcommand\basespec[0]{Effect-Spec}
\newcommand\basereturn[0]{Return}
\newcommand\baseapply[0]{Apply}
\newcommand\baseif[0]{If}
\newcommand\basebind[0]{Bind}

\newcommand\basetransitive[0]{Transitive}
\newcommand\basereflexive[0]{Reflexive}

\newcommand{\baseid}[0]{Id}
\newcommand\baseproject[0]{Project}

% Effect Weakening Rule Names
\newcommand{\eid}[0]{\eprefix\baseid}
\newcommand{\eproject}[0]{\eprefix\baseproject}
\newcommand{\eextend}[0]{\eprefix\baseextend}

% Term Weakening Rule Names
\newcommand{\tid}[0]{\tprefix\baseid}
\newcommand{\tproject}[0]{\tprefix\baseproject}
\newcommand{\textend}[0]{\tprefix\baseextend}

% Effect Substitution Rule Names
\newcommand\esubnil[0]{\eprefix\basenil}
\newcommand\esubextend[0]{\eprefix\baseextend}

% Term Substitution Rule Names

\newcommand\tsubnil[0]{\tprefix\basenil}
\newcommand\tsubextend[0]{\tprefix\baseextend}

% Type environment Rule Names
\newcommand\envnil[0]{\envprefix\basenil}
\newcommand\envextend[0]{\envprefix\baseextend}
% Effect Environment rule names
\newcommand\pnil[0]{\pprefix\basenil}
\newcommand\pextend[0]{\pprefix\baseextend}
% Equational equality rule names
\newcommand{\eqbeta}[0]{\equationalprefix Lambda-Beta}
\newcommand{\eqeta}[0]{\equationalprefix Lambda-Eta}
\newcommand{\eqeffbeta}[0]{\equationalprefix Effect-Beta}
\newcommand{\eqeffeta}[0]{\equationalprefix Effect-Eta}
\newcommand\eqleftunit[0]{\equationalprefix Left-Unit}
\newcommand\eqrightunit[0]{\equationalprefix Right-Unit}
\newcommand\equnitequiv[0]{\equationalprefix Unit}
\newcommand\eqiftrue[0]{\equationalprefix If-True}
\newcommand\eqiffalse[0]{\equationalprefix If-False}
\newcommand\eqifeta[0]{\equationalprefix If-Eta}
\newcommand\eqassociativity[0]{\equationalprefix Associativity}

\newcommand{\eqreflexive}[0]{\equationalprefix\basereflexive}
\newcommand\eqtransitive[0]{\equationalprefix\basetransitive}
\newcommand\eqsymmetric[0]{\equationalprefix Symmetric}

\newcommand\equnit[0]{\equationalprefix\baseunit}
\newcommand\eqtrue[0]{\equationalprefix\basetrue}
\newcommand\eqfalse[0]{\equationalprefix\basefalse}
\newcommand\eqconst[0]{\equationalprefix\baseconst}
\newcommand{\eqvar}[0]{\equationalprefix\basevar}
\newcommand\eqweaken[0]{\equationalprefix\baseweaken}
\newcommand\eqfun[0]{\equationalprefix\basefn}
\newcommand\eqsubtype[0]{\equationalprefix\basesubtype}
\newcommand\eqgen[0]{\equationalprefix\basegen}
\newcommand\eqspec[0]{\equationalprefix\basespec}
\newcommand\eqreturn[0]{\equationalprefix\basereturn}
\newcommand\eqapply[0]{\equationalprefix\baseapply}
\newcommand\eqif[0]{\equationalprefix\baseif}
\newcommand\eqbind[0]{\equationalprefix\basebind}

% Term rule names
\newcommand\vunit[0]{\baseunit}
\newcommand\vtrue[0]{\basetrue}
\newcommand\vfalse[0]{\basefalse}
\newcommand\vconst[0]{\baseconst}
\newcommand{\vvar}[0]{\basevar}
\newcommand\vweaken[0]{\baseweaken}
\newcommand\vfun[0]{\basefn}
\newcommand\vsubtype[0]{\basesubtype}
\newcommand\vgen[0]{\basegen}
\newcommand\vspec[0]{\basespec}
\newcommand\vreturn[0]{\basereturn}
\newcommand\vapply[0]{\baseapply}
\newcommand\vif[0]{\baseif}
\newcommand\vbind[0]{\basebind}

%Effect rule names
\newcommand\eground[0]{\eprefix\baseground}
\newcommand\evar[0]{\eprefix\basevar}
\newcommand\eweaken[0]{\eprefix\baseweaken}
\newcommand\ecompose[0]{\eprefix Compose}

% Type rule names
\newcommand{\tground}[0]{\tprefix\baseground}
\newcommand{\tfun}[0]{\tprefix\basefn}
\newcommand{\teffect}[0]{\tprefix\baseeffect}
\newcommand{\tquant}[0]{\tprefix\basequant}

% Subtyping rule names
\newcommand{\stransitive}[0]{\sprefix\basetransitive}
\newcommand{\sreflexive}[0]{\sprefix\basereflexive}
\newcommand{\sground}[0]{\sprefix\baseground}
\newcommand{\sfun}[0]{\sprefix\basefn}
\newcommand{\seffect}[0]{\sprefix\baseeffect}
\newcommand{\squant}[0]{\sprefix\basequant}


\newcommand{\s}{\;}
\newcommand{\doin}[3]{\texttt{do}\s #1 \leftarrow #2 \s\texttt{in}\s #3\s}
\newcommand\apply[2]{#1\s#2}
\newcommand{\pifthenelse}[4]{\texttt{if}_{\textcolor{purple}{#1}}\s#2\s \texttt{then}\s #3 \s\texttt{else} \s#4\s}
\newcommand\ifthenelse[5]{\pifthenelse{#1, #2}{#3}{#4}{#5}}
\newcommand\const[1]{\texttt{k}^{\color{purple} #1}}
\newcommand\return[1]{\texttt{return} \s#1\s}


\newcommand\lam[3]{\lambda #1 \colon {\color{purple}#2}. #3\s}
\renewcommand\u[0]{\texttt{()}}
\newcommand{\U}[0]{\texttt{Unit}}
\renewcommand\t[0]{\texttt{true}}
\newcommand\f[0]{\texttt{false}}
\newcommand{\B}[0]{\texttt{Bool}}
\newcommand{\G}[0]{\Gamma}
\newcommand\D{\Delta}


% draw type relations
\newcommand{\typerelation}[3]{{\color{DarkGreen}#1} \vdash #2 \colon {\color{blue}#3}}
\newcommand\wellformed[2]{{\color{DarkGreen}#1}\vdash {\color{blue}#2}}
\newcommand\wellformedok[2]{\ok{{\color{DarkGreen}#1}\vdash {\color{blue} #2}}}

\newcommand{\wellformedtype}[2]{\typerelation{#1}{#2}{\type}}
\newcommand{\wellformedeffect}[2]{\typerelation{#1}{#2}{\effect}}
\newcommand{\wellformedF}[2]{\typerelation{#1}{#2}{F}}



\newcommand{\gtyperelation}[2]{\typerelation{\G}{#1}{#2}}
 

\newcommand\treerulez[1]{\inferrule{ }{#1}}
\newcommand\treeruleI[2]{\inferrule{#1}{#2}}
\newcommand\treeruleII[3]{\inferrule{#1 \\ #2}{#3}}
\newcommand\treeruleIII[4]{\inferrule{#1 \\ #2 \\ #3}{#4}}
\newcommand\treeruleIV[5]{\inferrule{#1 \\ #2 \\ #3 \\ #4}{#5}}
\newcommand\treeruleV[6]{\inferrule{#1 \\ #2 \\ #3 \\ #4 \\ #5}{#6}}

\newcommand\ntreerulez[2]{(\text{#1})\inferrule{ }{#2}}
\newcommand\ntreeruleI[3]{(\text{#1})\inferrule{#2}{#3}}
\newcommand\ntreeruleII[4]{(\text{#1})\inferrule{#2 \\ #3}{#4}}
\newcommand\ntreeruleIII[5]{(\text{#1})\inferrule{#2 \\ #3 \\ #4}{#5}}
\newcommand\ntreeruleIV[6]{(\text{#1})\inferrule{#2 \\ #3 \\ #4 \\ #5}{#6}}
\newcommand\ntreeruleV[7]{(\text{#1})\inferrule{#2 \\ #3 \\ #4 \\ #5 \\ #6}{#7}}

\newcommand\condtreerulez[3]{(\text{#1})\inferrule{ }{#2}(\text{if } #3)}
\newcommand\condtreeruleI[4]{(\text{#1})\inferrule{#2}{#3}(\text{if } #4)}
\newcommand\condtreeruleII[5]{(\text{#1})\inferrule{#2 \\ #3}{#4}(\text{if } #5)}
\newcommand\condtreeruleIII[6]{(\text{#1})\inferrule{#2 \\ #3 \\ #4}{#5}(\text{if } #6)}
\newcommand\condtreeruleIV[7]{(\text{#1})\inferrule{#2 \\ #3 \\ #4 \\ #5}{#6}(\text{if } #7)}
\newcommand\condtreeruleV[8]{(\text{#1})\inferrule{ #2 \\ #3 \\ #4 \\ #5 \\ #6 }{#7}(\text{if } #8)}



\newcommand{\subtype}[0]{\leq\colon}
\newcommand\subeffect[0]{\leq}

\newcommand{\M}[2]{\texttt{M}_{#1}{#2}}

\newcommand\lamtype[3]{#1 \rightarrow \M{#2}{#3}}
\newcommand{\1}[0]{\texttt{1}}

\newcommand\e[0]{\epsilon}

\newcommand{\db}[1]{{\bf [\![}#1{\bf ]\!]}}
\newcommand{\deno}[1]{\db{#1}}
\newcommand\after\circ
\newcommand\term[1]{\langle\rangle_{#1}}

\newcommand\bindmu[0]{\mu}
\newcommand\point[1]{\eta_{#1}}
\newcommand\bind[3]{\bindmu_{#1, #2, #3}}

\newcommand\T[2]{T_{#1}{#2}}

\newcommand\pr[2]{\langle#1, #2\rangle}
\newcommand\finpr[2]{\langle #1\rangle_{#2}}

\newcommand\strengtht[0]{\texttt{t}}
% tensor strength Nat-tran
\newcommand\tstrength[3]{\strengtht_{#1, #2, #3}}

% Id morphism
\newcommand\Id[1]{\texttt{Id}_{#1}}

\newcommand\idg[0]{\Id{\G}}
% beta-eta equivalence
\newcommand\beequiv[0]{\approx}
% Substitutions
\newcommand\si{\sigma}

\newcommand{\sub}[1]{[#1]}
\newcommand{\ssub}[2]{[#2 / #1]}
\newcommand{\ssi}[0]{\sub{\si}}

% beta-eta equivalence relation
\newcommand{\berelation}[4]{\typerelation{#1}{#2 \beequiv #3}{#4}}
\newcommand{\gberelation}[3]{\gtyperelation{#1 \beequiv #2}{#3}}


% Shortcuts for denotational equality
\newcommand{\denoequality}[4]{\deno{\typerelation{#1}{#2}{#4}} = \deno{\typerelation{#1}{#3}{#4}}}
\newcommand{\gdenoequality}[3]{\denoequality{\G}{#1}{#2}{#3}}

% Shorthand for monad types
\newcommand\mea[0]{\M{\e}{A}}
\newcommand\meb[0]{\M{\e}{B}}
\newcommand\mec[0]{\M{\e}{C}}

\newcommand\tea[0]{\T{\e}{A}}
\newcommand\teb[0]{\T{\e}{B}}
\newcommand\tec[0]{\T{\e}{C}}


\newcommand\moa[0]{\M{\1}{A}}
\newcommand\mob[0]{\M{\1}{B}}
\newcommand\moc[0]{\M{\1}{C}}

\newcommand\toa[0]{\T{\1}{A}}
\newcommand\tob[0]{\T{\1}{B}}
\newcommand\toc[0]{\T{\1}{C}}

\newcommand\aeb[0]{\lamtype{A}{\e}{B}}

% Shorthand for Gammas
\newcommand{\gax}[0]{\G, x\colon A}
\newcommand{\gby}[0]{\G, y\colon B}

% reduction function
\newcommand{\reduce}[0]{reduce}



% Combinators for building delta-based tree proof terms
\newcommand{\deltavrule}[4]{
    \ntreeruleII{\vsubtype}{\treeruleI{\D}{\typerelation{#1}{#2}{#3}}}{#3 \subtype #4}{\typerelation{#1}{#2}{#4}}}

\newcommand{\deltavruleprime}[4]{
    \ntreeruleII{\vsubtype}{\treeruleI{\D'}{\typerelation{#1}{#2}{#3}}}{#3 \subtype #4}{\typerelation{#1}{#2}{#4}}}

\newcommand{\deltavruleprimeprime}[4]{
        \ntreeruleII{\vsubtype}{\treeruleI{\D'}{\typerelation{#1}{#2}{#3}}}{#3 \subtype #4}{\typerelation{#1}{#2}{#4}}}
    
\newcommand{\deltacrule}[6]{
            \ntreeruleII{Subeffect}{\treeruleI{\D}{\typerelation{#1}{#2}{\M{#3}{#4}}}}{\subeffecttree{#3}{#4}{#5}{#6}}{\typerelation{#1}{#2}{\M{#5}{#6}}}}
\newcommand{\deltacruleprime}[6]{
    \ntreeruleII{Subeffect}{\treeruleI{\D'}{\typerelation{#1}{#2}{\M{#3}{#4}}}}{
    \subeffecttree{#3}{#4}{#5}{#6}}{\typerelation{#1}{#2}{\M{#5}{#6}}}}
\newcommand{\deltacruleprimeprime}[6]{
    \ntreeruleII{\vsubtype}{\treeruleI{\D''}{\typerelation{#1}{#2}{\M{#3}{#4}}}}{
        \subeffecttree{#3}{#4}{#5}{#6}}{\typerelation{#1}{#2}{\M{#5}{#6}}}}
                            

\newcommand{\p}[0]{\pi_1}
\newcommand{\pp}[0]{\pi_2}

% short-hands for weakening
\newcommand{\wrel}[3]{#1 \colon {\color{blue}#2} \triangleright {\color {blue} #3}}
\newcommand{\ok}[1]{{\color{blue} #1} \texttt{ Ok}}
\newcommand\okt[0]{\texttt{Ok}}
\renewcommand\i[0]{\iota}
\newcommand\w{\omega}
\newcommand\dom[1]{\texttt{dom}(#1)}
\newcommand\x{\times}


\newcommand\fev[1]{fev(#1)}
\newcommand\union[0]{\cup}


% Combinators to build tree proofs
\newcommand{\truleconst}[0]{\ntreeruleI{\vconst}{\ok{\G}}{\gtyperelation{\const{A}}{A}}}
\newcommand{\truleunit}[0]{\ntreeruleI{\vunit}{\ok{\G}}{\typerelation{\G}{\u}{\U}}}
\newcommand{\truletrue}[0]{\ntreeruleI{\vtrue}{\ok{\G}}{\typerelation{\G}{\t}{\B}}}
\newcommand{\trulefalse}[0]{\ntreeruleI{\vfalse}{\ok{\G}}{\typerelation{\G}{\f}{\B}}}


\newcommand{\E}[0]{\mathbb{E}}
\renewcommand{\dot}{\cdot}
\newcommand{\gens}[0]{\colon\colon=}
\newcommand{\nil}[0]{\diamond}
\newcommand{\ground}[0]{\gamma}

% Terminal object of C
\newcommand{\terminal}[0]{\texttt{\1}}

% The category C
\newcommand{\C}[0]{\mathbb{C}}
\newcommand{\Cz}[0]{\C_0}
\newcommand\DC[0]{\mathbb{D}}

% The category of locally-small categories
\newcommand{\Cat}[0]{\texttt{Cat}}
% Sub-effect Nat-trans
\newcommand{\dse}[2]{\db{#1 \subeffect #2}}

\newcommand\app[0]{\texttt{app}}
\newcommand\cur[1]{\texttt{cur}(#1)}
\newcommand{\ifnt}[1]{\texttt{If}_{#1}}


\newcommand{\setto}{\colon=}
\newcommand{\fv}[1]{\texttt{fv}(#1)}

% shorthand for inserting text to equations
\newcommand\qt[1]{\quad\text{#1}}

% Co-product short-hands
\newcommand\inr[0]{\texttt{inr}}
\newcommand\inl[0]{\texttt{inl}}
    
\newcommand\fld[2]{[#1,#2]}
\newcommand{\diag}[1]{\delta_{#1}}
\newcommand{\twist}[2]{\tau_{#1, #2}}

\newcommand\ifMorph[3]{\app\after((\fld{\cur{#2\after\pp}}{\cur{#3\after\pp}}\after #1)\times \idg)\after \diag{\G}}


% Polymorphic short-hands
\newcommand\elam[2]{\Lambda #1. #2}
\newcommand{\eapp}[2]{#1\s#2}
\renewcommand{\a}[0]{\alpha}
\newcommand{\all}[2]{\forall #1. #2}
\renewcommand{\P}[0]{\Phi}

\renewcommand{\b}[0]{\beta}
\newcommand{\g}[0]{\gamma}
\renewcommand\d[0]{\delta}
\newcommand\oke[2]{\wellformedok{#1}{#2}}
\newcommand\etyperelation[4]{\typerelation{#1\mid#2}{#3}{#4}}
\newcommand{\gpetyperelation}[2]{\etyperelation{\P}{\G}{#1}{#2}}
\newcommand{\gppetyperelation}[2]{\etyperelation{\P'}{\G}{#1}{#2}}


\newcommand{\eberelation}[5]{\berelation{#1\mid#2}{#3}{#4}{#5}}
\newcommand{\gpeberelation}[3]{\berelation{\P\mid\G}{#1}{#2}{#3}}
\newcommand{\gppeberelation}[3]{\berelation{\P'\mid\G}{#1}{#2}{#3}}

\newcommand{\dotp}[0]{\dot_\P}
\newcommand{\fntype}[2]{#1\rightarrow #2}
\newcommand{\ab}[0]{\fntype{A}{B}}

\newcommand\wrelw[2]{\wrel{\w}{#1}{#2}}
\renewcommand\proof[0]{\paragraph{Proof:}}
\newcommand{\case}[1]{\paragraph{Case #1:}}
\newcommand{\subcase}[1]{\subparagraph{Case: #1}}
\newcommand\bi[0]{By inversion}

%pre-filled effect-weakening relations
\newcommand\ewrel[4]{\wellformed{#1}{\color{black}\wrel{#2}{#3}{#4}}}
\newcommand\pewrel[3]{\ewrel{\P}{#1}{#2}#3}
\newcommand\ppewrel[3]{\ewrel{\P'}{#1}{#2}#3}

\newcommand\subtypep[0]{\subtype_\P}
\newcommand\subtypepp[0]{\subtype_{\P'}}
\newcommand\subeffectp[0]{\subeffect_{\P}}
\newcommand\subeffectpp[0]{\subeffect_{\P'}}
\newcommand\subeffectn[0]{\subeffect_{n}}
\newcommand\subeffectz[0]{\subeffect_{0}}

\newcommand{\allI}[0]{\forall_I}
\newcommand{\allII}[0]{\forall_{I'}}
\newcommand\allIU[0]{\forall_{I\times U}}
\newcommand\type[0]{\texttt{Type}}
\newcommand\effect[0]{\texttt{Effect}}
\newcommand\ciw[0]{\C(I, W)}
\newcommand\ciu[0]{\C(I, U)}
\newcommand\ciuw[0]{\C(I\times U, W)}
\newcommand\cipw[0]{\C(I', W)}
\newcommand\cipu[0]{\C(I', U)}
\newcommand\ciuu[0]{\C(I\times U, U)}
\newcommand\cii[0]{\C(I', I)}
\newcommand\Eff[0]{\texttt{Eff}}
\newcommand\Mul[0]{\texttt{Mul}}
\newcommand\singleton[0]{\ast}
\renewcommand\star[0]{^*}
\renewcommand\bar[1]{\overline{#1}}

\newcommand\subtypeg[0]{\subtype_\g}
\newcommand\subtypepa[0]{\subtype_{\P, \a}}
\newcommand\subtypeppa[0]{\subtype_{\P', \a}}

\newcommand\subtypez[0]{\subtype_{0}}
\newcommand\subtypen[0]{\subtype_{n}}

\usepackage{scalerel,stackengine}
\stackMath
\renewcommand\widehat[1]{%
\savestack{\tmpbox}{\stretchto{%
  \scaleto{%
    \scalerel*[\widthof{\ensuremath{#1}}]{\kern.1pt\mathchar"0362\kern.1pt}%
    {\rule{0ex}{\textheight}}%WIDTH-LIMITED CIRCUMFLEX
  }{\textheight}% 
}{2.4ex}}%
\stackon[-6.9pt]{#1}{\tmpbox}%
}
\parskip 1ex

\newcommand\pstar[0]{\p\star}

\newcommand\edeltavrule[5]{\deltavrule{#1 \mid #2}{#3}{#4}{#5}}

\newcommand\subeffecttreep[4]{\ntreeruleII{\teffect}{
    #1\subeffectp #3}{#2 \subtypep #4
}{\M{#1}{
    #2
}\subtypep\M{#3}{#4}}}
\newcommand\subeffecttree[4]{\ntreeruleII{\teffect}{
    #1\subeffect #3}{#2 \subtype #4
}{\M{#1}{
    #2
}\subtype\M{#3}{#4}}}


\newcommand{\edeltavruleprime}[5]{
        \deltavruleprime{#1\mid #2}{#3}{#4}{#5}}
    
\newcommand{\edeltavruleprimeprime}[5]{
        \deltavruleprimeprime{#1\mid #2}{#3}{#4}{#5}}
    
\newcommand{\edeltacrule}[6]{
            \ntreeruleII{\vsubtype}{
                \treeruleI{
                    \D
                }{
                    \typerelation{\P\mid#1}{#2}{\M{#3}{#4}}
                }
            }{
                \ntreeruleII{\teffect}{
                    #4 \subtypep #6
                    }{
                         #3 \subeffectp #5
                }{
                    \M{#3}{#4}\subtypep{\M{#5}{#6}}
                }
            }{
                \typerelation{\P\mid #1}{#2}{\M{#5}{#6}}
            }
        }
        

        \newcommand{\edeltacruleprime}[6]{
            \ntreeruleII{\vsubtype}{
                \treeruleI{
                    \D'
                }{
                    \typerelation{\P\mid #1}{#2}{\M{#3}{#4}}
                }
            }{
                \ntreeruleII{\teffect}{
                    #4 \subtypep #6
                }{#3 \subeffectp #5
                }{
                    \M{#3}{#4}\subtypep{\M{#5}{#6}}
                }
            }{
                \typerelation{\P\mid #1}{#2}{\M{#5}{#6}}
            }
        }
                   

        \newcommand{\edeltacruleprimeprime}[6]{
            \ntreeruleII{\vsubtype}{
                \treeruleI{
                    \D''
                }{
                    \typerelation{\P\mid #1}{#2}{\M{#3}{#4}}
                }
            }{
                \ntreeruleII{\teffect}{
                    #4 \subtypep #6
                    }{ #3 \subeffectp #5
                }{
                    \M{#3}{#4}\subtypep{\M{#5}{#6}}
                }
            }{
                \typerelation{\P\mid #1}{#2}{\M{#5}{#6}}
            }
        }

        \newcommand\obj[0]{\texttt{obj }}


        \newcommand{\Tz}[2]{\texttt{T}^0_{#1}#2}
        \newcommand{\Tn}[2]{\texttt{T}^n_{#1}#2}
        \newcommand{\Tm}[2]{\texttt{T}^m_{#1}#2}
        
        \newcommand{\pointz}[1]{\point{#1}^0}
        \newcommand{\pointn}[1]{\point{#1}^n}
        \newcommand{\pointm}[1]{\point{#1}^m}
        
        \newcommand{\bindz}[3]{\bind{#1}{#2}{#3}^0}
        \newcommand{\bindn}[3]{\bind{#1}{#2}{#3}^n}
        \newcommand{\bindm}[3]{\bind{#1}{#2}{#3}^m}
        
        \newcommand\tstrengthz[3]{\tstrength{#1}{#2}{#3}^0}
        \newcommand\tstrengthn[3]{\tstrength{#1}{#2}{#3}^n}
        \newcommand\tstrengthm[3]{\tstrength{#1}{#2}{#3}^m}
        
        \newcommand\set[0]{\texttt{Set}}
        \newcommand\cccat[0]{\textit{CCCat}}

        \newcommand\ev[0]{\vec{\e}}
        \newcommand\emv[0]{\vec{\e_m}}
        \newcommand\env[0]{\vec{\e_n}}
        
        \newcommand\subeffectm[0]{\subeffect_m}
        
        \newcommand\dsem[2]{\db{#1 \subeffectm #2}}
        \newcommand\dsen[2]{\db{#1 \subeffectn #2}}
        \newcommand\dsez[2]{\db{#1 \subeffectz #2}}
        \newcommand\dsep[2]{\db{#1 \subeffectp #2}}
        \newcommand\dsepp[2]{\db{#1 \subeffectpp #2}}
        
        \newcommand\allEn[0]{\forall_{E^n}}
        \newcommand\allEm[0]{\forall_{E^m}}
        
        \newcommand\counit[1]{\boldsymbol{\epsilon}_{#1}}
        \newcommand\unit[1]{\boldsymbol{\eta}_{#1}}


        
%% Adequacy shorthands
\newcommand{\relates}[0]{\lhd}
\newcommand{\logRel}[3]{#1 \relates_{#2} #3}
\newcommand{\plogRel}[4]{#1 \relates_{\wellformed{#2}{#3}} #4}

\newcommand{\zberelation}[3]{\berelation{}{#1}{#2}{#3}}
\newcommand\ztyperelation[2]{\typerelation{}{#1}{#2}}

\newcommand{\N}[0]{\mathbb{N}}
\renewcommand\put[0]{\texttt{put}}
\newcommand\ecput[0]{\texttt{EC}_\put}
\newcommand\ecputA[0]{\texttt{EC}_\put^A}
\newcommand\ecputG[0]{\texttt{EC}_\put^G}

\newcommand\mna[0]{\M{n}{A}}
\newcommand\mmb[0]{\M{m}{B}}
\newcommand\mnb[0]{\M{n}{B}}
\newcommand\mma[0]{\M{m}{A}}

\newcommand{\setcomp}[2]{\{#1 \mid #2 \}}
\fi

\begin{document}


\todo{E-, T- prepend to rule names}
\todo{Check rule names in textit}
\todo{Check rule names in case}

\abstract

This (online only) document contains full proofs and derivations of all of the theorems for the soundness of PEC semantics. It is not intended to have the same polish as the dissertation.

\tableofcontents

\chapter{Language Definition}

\section{Terms}

\begin{align*}
        v \gens & x \\
        & \mid \lam{x}{A}{v} \\
        & \mid \const{A} \\
        & \mid \u \\
        & \mid \t \mid \f \\
        & \mid \elam{\a}{v} \\
        & \mid \eapp{v}{\e} \\
        & \mid \pifthenelse{A}{v}{v_1}{v_2} \\
        & \mid \apply{v_1}{v_2} \\
        & \mid \doin{x}{v_1}{v_2} \\
        & \mid \return{v}
\end{align*}



\section{Type System}
\subsection{Ground Effects}
The effects should form a monotonous, partially-ordered monoid $(E, \dot, \1, \subeffect)$ with ground elements $e$.

\subsection{Effect Po-Monoid Under an Effect Environment}

Derive a new Po-Monoid for each $\P$:
\begin{equation}
    (E_\P, \dotp, \1, \subeffectp)
\end{equation}
Where meta-variables, $\e$, range over $E_\P$
Where

\begin{equation}
    E_\P = E \cup \left\{\a \mid \a \in \P \right\}
\end{equation}

And 
\begin{equation}
    \treeruleI{\e_3 = \e_1\dot\e_2}{\e_3 = \e_1\dotp\e_2}
\end{equation}

Otherwise, $\dotp$ is symbolic in nature.

\begin{equation}
    \e_1 \subeffectp \e_2 \Leftrightarrow \forall \si\downarrow. \e_1\sub{\si\downarrow} \subeffect \e_2\sub{\si\downarrow}
\end{equation}

Where $\si\downarrow$ denotes any ground-substitution of $\P$. That is any substitution of all effect variables in $\P$ to ground effects. Where it is obvious from the context, I shall use $\subeffect$ instead of $\subeffectp$.


\subsection{Types}
    \paragraph{Ground Types}
        There exists a set $\ground$ of ground types, including \U, \B
    \paragraph{Term Types}
    $$ A, B, C \gens \ground \mid \ab \mid \mea \mid \all{\a}{A}$$
  
\subsection{Type and Effect Environments}
A type environment is a snoc-list of term-variable, type pairs, $G \gens \nil \mid \gax$ .
An effect environment is a snoc-list of effect variables.

$\P\gens\nil\mid\P, \a$
\paragraph{Domain Function on Type Environments}
\begin{itemize}
    \item $\dom{\nil} = \emptyset$
    \item $\dom{\gax} =  \dom{\G}  \cup \left\{x \right\}$
\end{itemize}

\paragraph{Membership of Effect Environments}
Informally, $\a \in \P$ if $\a$ appears in the list represented by $\P$.

\paragraph{$\okt$ Predicate On Effect Environments}
\begin{itemize}
    \item $\ntreerulez{Nil}{\ok{\nil}}$
    \item $\condtreeruleI{Extend}{\ok{\P}}{\ok{\P, \a}}{ \a\notin \P}$
\end{itemize}

\paragraph{Wellformedness of effects}
We define a relation $\wellformed{\P}{\e}$.

\begin{itemize}
    \item $\ntreeruleI{Ground}{\ok{\P}}{\wellformed{\P}{e}}$
    \item $\ntreeruleI{Var}{\ok{\P,\a}}{\wellformed{\P,\a}{\a}}$
    \item $\condtreeruleI{Weaken}{\wellformed{\P}{\a}}{\wellformed{\P,\b}{\a}}{\a\neq\b, b\notin\P}$
    \item $\ntreeruleII{Compose}{\wellformed{\P}{\e_1}}{\wellformed{\P}{\e_2}}{\wellformed{\P}{\e_1\dot\e_2}}$
\end{itemize}

\paragraph{Wellformedness of Types}
We define a relation $\wellformed{\P}{A}$ on types.

\begin{itemize}
    \item $\ntreerulez{Ground}{\wellformed{\P}{\g}}$
    \item $\ntreeruleII{Fn}{\wellformed{\P}{A}}{\wellformed{\P}{B}}{\wellformed{\P}{\ab}}$
    \item $\ntreeruleII{Effect}{\wellformed{\P}{A}}{\wellformed{\P}{\e}}{\wellformed{\P}{\mea}}$
    \item $\ntreeruleI{Quantification}{\wellformed{\P,\a}{A}}{\wellformed{\P}{\all{\a}{A}}}$
\end{itemize}

\paragraph{Ok Predicate on Type Environments}
We now define a predicate on type environments and effect environments: $\oke{\P}{\G}$

\begin{itemize}
    \item $\ntreerulez{Nil}{\oke{\P}{\nil}}$
    \item $\ntreeruleIII{Extend}{\oke{\P}{\G}}{ x\notin\dom{\G}}{ \wellformed{\P}{A}}{\oke{\P}{\gax}}$
\end{itemize}



\subsection{Subtyping}
    There exists a subtyping partial-order relation $\subtype_{\ground}$ over ground types that is:
    \begin{itemize}
        \item $\ntreerulez{Reflexive}{A \subtype_{\ground} A}$
        \item $\ntreeruleII{Transitive}{A \subtype_{\ground} B }{ B \subtype_{\ground} C}{A \subtype_{\ground} C}$
    \end{itemize}

    We extend this relation with the (Fn) and (Quantification) subtyping rules to yield the full subtyping relation under an effect environment, $\P$, $\subtypep$

    \begin{itemize}
        \item $\ntreeruleI{Ground}{A \subtype_{\ground} B}{A \subtypep B}$
        \item $\ntreeruleII{Fn}{A \subtypep A' }{ B' \subtypep B }{\fntype{A'}{B'} \subtypep \ab}$
        \item $\ntreeruleII{Quantification}{A\subtypep A'}{a\notin\P}{\all{\a}{A}\subtypep\all{a}{A'}}$
        \item $\ntreeruleII{Effect}{A\subtypep B}{ \e_1\subeffectp\e_2}{\M{\e_1}{A}\subtypep\M{\e_2}{B}}$
    \end{itemize}

\subsection{Type Rules}
\begin{itemize}
    \item $\ntreeruleII{Const}{\oke{\P}{\G}}{\wellformed{\P}{A}}{\gpetyperelation{\const{A}}{A}}$
    \item $\ntreeruleI{Unit}{\oke{\P}{\G}}{\gpetyperelation{\u}{\U}}$
    \item $\ntreeruleI{True}{\oke{\P}{\G}}{\gpetyperelation{\t}{\B}}$
    \item $\ntreeruleI{False}{\oke{\P}{\G}}{\gpetyperelation{\f}{\B}}$
    \item $\ntreeruleI{Var}{\oke{\P}{\gax}}{\etyperelation{\P}{\gax}{x}{A}}$
    \item $\condtreeruleII{Weaken}{\etyperelation{\P}{\G}{x}{A}}{\wellformed{\P}{B}}{\etyperelation{\P}{\gby}{x}{A}}{x \neq y, y\notin \dom{\G}}$
    \item $\ntreeruleI{Fn}{\etyperelation{\P}{\gax}{v}{B}}{\etyperelation{\P}{\G}{\lam{x}{A}{v}}{\ab}}$
    \item $\ntreeruleII{Sub}{\etyperelation{\P}{\G}{v}{A}}{ A \subtypep B}{\etyperelation{\P}{\G}{v}{B}}$
    \item $\ntreeruleI{Effect-Gen}{\etyperelation{\P,\a}{\G}{v}{A}}{\gpetyperelation{\elam{\a}{v}}{\all{\a}{A}}}$
    \item $\ntreeruleII{Effect-Spec}{\gpetyperelation{v}{\all{\a}A}}{\wellformed{\P}{\e}}{\gpetyperelation{\eapp{v}{\e}}{A\ssub{\a}{\e}}}$
    \item $\ntreeruleI{Return}{\gpetyperelation{v}{A}}{\gpetyperelation{\return{v}}{\moa}}$
    \item $\ntreeruleII{Apply}{\gpetyperelation{v_1}{\aeb}}{\gpetyperelation{v_2}{A}}{\gpetyperelation{\apply{v_1}{v_2}}{\meb}}$
    \item $\ntreeruleIII{If}{\gpetyperelation{v}{\B} }{ \gpetyperelation{v_1}{A}}{\gpetyperelation{v_2}{A}}{\gpetyperelation{\pifthenelse{A}{v}{v_1}{v_2}}{A}}$
    \item $\ntreeruleII{Do}{\gpetyperelation{v_1}{\M{\e_1}{A}} }{ \etyperelation{\P}{\gax}{v_2}{\M{\e_2}{B}}}{\gpetyperelation{\doin{x}{v_1}{v_2}}{\M{\e_1 \dot \e_2}{B}}}$
\end{itemize}

\subsection{Ok Lemma}
If $\gpetyperelation{v}{A}$ then $\oke{\P}{\G}$.
\paragraph{Proof}
If $\ok{\gax}$ then by inversion. $\ok{\G}$.
Only the type rule \texttt{Weaken} adds terms to the environment from its preconditions to its post-condition and it does so in an $\okt$ preserving way. Any type derivation tree has at least one leaf. All leaves are axioms which require $\oke{\P}{\G}$. And all non axiom derivations preserve the $\okt$ property.



\chapter{Preliminaries}

\section{Base Category Requirements}
There are 2 distinct objects in the base category, $\C$:

\begin{itemize}
    \item $U$ - The kind of \effect
    \item $\1$ - A terminal object
\end{itemize}

And we have finite products on $U$.

\begin{itemize}
    \item $U^0 = \1$
    \item $U^{n+1} = U^n \times U$
\end{itemize}

We also have the following natural operations on morphisms in $\C$.

Let $I = U^n$.

\begin{itemize}
    \item $\Mul_I:\ciu\times\ciu\rightarrow\ciu$ - Generates multiplication of effects.
\end{itemize}

With naturality conditions which mean, for $\theta: U^m \rightarrow U^n (I' \rightarrow I)$,
\begin{itemize}
    \item $\Mul_{I}(\phi,\psi)\after\theta = \Mul_{I'}(\phi\after\theta,\psi\after\theta)$
\end{itemize}


\section{Wellformedness}

Each instance of the wellformedness relation on effects, $\wellformed{\P}{\e}$ has a denotation in $\C$: \begin{equation}
    \deno{\typerelation{\P}{\e}{\effect}}: I \rightarrow U
\end{equation}

It should also be the case that \begin{equation}
    \Mul_I(\deno{\typerelation{\P}{\e_1}{\effect}}, \deno{\typerelation{\P}{\e_2}{\effect}}) = \deno{\typerelation{\P}{\e_1\dot\e_2}{\effect}} \in \ciu
\end{equation}

That is, $\Mul_I$ should be have identity $\deno{\typerelation{\P}{\1}{\effect}}$ and be associative.


\section{Substitution and Weakening of the Effect Environment}


\subsection{Denotations}


For each instance of the wellformedness relation on substitution of effects $\typerelation{\P'}{\si}{\P}$, there exists a denotation in $\C$:

\begin{equation}
    \deno{\typerelation{\P'}{\si}{\P}}: I'\rightarrow I
\end{equation}

For each instance of the well formed weakening relation on effect-variable environments, $\wrelw{\P'}{\P}$
 there exists a denotation in $\C$:

\begin{equation}
     \deno{\wrelw{\P'}{\P}}: I'\rightarrow I
\end{equation}.

\section{Fibre Categories}
Each set of morphisms $\C(I)$ corresponds to a semantic category (S-category). That is, a category satisfying all the properties needed for the non-polymorphic language:

\begin{itemize}
    \item Cartesian Closed
    \item Co-product of the terminal object with itself ($\1 + \1$)
    \item Ground morphisms for each ground constant ($\const{A}: \1\rightarrow A$)
    \item Partial order morphisms on ground types ($\deno{A\subtypeg} B$)
    \item A strong, monad, graded by the po-monoid $(\ciu, \Mul_I, \subeffectp, \deno{\1})$.
\end{itemize}


\section{Re-Indexing Functors}
   \todo{Use this section for the S-Preservation definition appendix}
For each morphism $f: I' \rightarrow I$ in $\C$, there should be a co-variant, re-indexing functor  $f\star: \C(I) \rightarrow \C(I')$, which is S-preserving. That is, it preserves the S-category structure of $\C(I)$. (See below).

$(-)\star$ should be a contra-variant functor in its $\C$ argument and co-variant in its right argument.

\begin{itemize}
    \item $(g\after f)\star(a) = f\star(\g\star(a))$
    \item $\Id{I}\star(a) = a$
    \item $f\star(\Id{A}) = \Id{f\star(A)}$
    \item $f\star(a\after b) = f\star(a)\after f\star(b)$
\end{itemize}
\subsection{Preserves Ground Types}
If $\deno{\g}\in\obj\C(I)$ then $f\star\deno{\g} = \deno{\g}\in\obj\C(I')$
\subsection{$f\star$ Preserves Products}
If $\pr{g}{h}:\C(I)(Z, X\times Y)$
Then 
\begin{align*}
    f\star(X\times Y) & = f\star(X)\times f\star(Y)\\
    f\star(\pr{g}{h}) & = \pr{f\star(g)}{f\star{h}}&:\C(I')(f\star Z, f\star(X)\times f\star(Y))\\
    f\star(\p) & = \p&:\C(I')(f\star(X)\times f\star(Y), f\star(X)) \\
    f\star(\pp) &= \pp&:\C(I')(f\star(X)\times f\star(Y), f\star(Y))
\end{align*}

\subsection{$f\star$ Preserves Terminal Object}
If $\Id{A}:\C(I)(A, \1)$
Then 
\begin{align*}
    f\star(\1) & = \1 \\
    f\star(\term{A}) & = \term{f\star(A)}&:\C(I')(f\star A, \1)\\
\end{align*}

\subsection{$f\star$ Preserves Exponentials}
\begin{align*}
    f\star(Z^X) & = (f\star(Z))^{(f\star(X))}\\
     f\star(\app) &= \app&:\C(I')(f\star(Z^X)\times f\star(X), f\star(Z))\\
     f\star(\cur{g}) &= \cur{f\star(g)}&:\C(I')(f\star(Y)\times f\star(X), f\star(Z)^{f\star(X)})
\end{align*}

\subsection{$f\star$ Preserves Co-product on Terminal}

\begin{align*}
    f\star(\1+\1) &= \1+\1\\
    f\star(\inl)  &= \inl&:\C(I')(\1, \1+\1) \\
    f\star(\inr) &= \inr&:\C(I')(\1, \1+\1) \\
    f\star([g, h]) &= [f\star(g), f\star(h)]&:\C(I')(\1+\1, f\star(Z))
\end{align*}

\subsection{$f\star$ Preserves Graded Monad}
\begin{align*}
    f\star(\tea) &= \T{f\star(\e)}{f\star(A)}&:\C(I')\\    
    f\star(\point{A}) &= \point{f\star(A)}&:\C(I')(f\star(A), f\star(\toa))\\
    f\star(\bind{\e_1}{\e_2}{A}) &= \bind{f\star(\e_1)}{f\star(\e_2)}{f\star(A)}&:\C(I')(f\star(\T{\e_1}{\T{\e_2}{A}}), f\star(\T{\e_1\dot\e_2}{A}))\\
\end{align*}

\subsection{$f\star$ and Effects}
\begin{align*}
    f\star(\1) &= \1 \qt{The unit effect}\\
    f\star(\e_1\dot\e_2) &= f\star(\e_1)\dot f\star(\e_2)\qt{Multiplication}\\
\end{align*}

This is done By
\begin{align*}
    f\star\deno{\typerelation{\P}{\e}{\effect}} & = \deno{\typerelation{\P}{\e}{\effect}}\after f
\end{align*}

\subsection{$f\star$ Preserves Tensor Strength}
\begin{align*}
    f\star(\tstrength{\e}{A}{B}) &= \tstrength{f\star(\e)}{f\star(A)}{f\star(B)} &: \C(I')(f\star(A\times\teb), f\star(\T{\e}{(A\times B)}))
\end{align*}
\subsection{$f\star$ Preserves Ground Constants}
For each ground constant $\deno{\const{A}}$ in $\C(I)$,

\begin{align*}
    f\star(\deno{\const{A}}) = \const{f\star(A)} : \C(I')(\1, f\star(A))
\end{align*}
\subsection{$f\star$ Preserves Ground Subeffecting}
For ground effects $e_1, e_2$ such that $e_1\subeffect e_2$



\begin{align*}
    f\star(e) & = e: \C(I')\\
    f\star\db{\e_1\subeffect e_2}_A = \db{e_1 \subeffect e_2}_{f\star(A)} &:\C(I'){f\star(\T{e_1}{A}), f\star(\T{e_2}{A})} \\
\end{align*}
\subsection{$f\star$ Preserves Ground Subtyping}
For ground types $\g_1, \g_2$ such that $\g_1\subtypeg\g_2$

\begin{align*}
    f\star{\g} = \g: \C(I')(\1, \g)\\
    f\star(\deno{\g_1 \subtypeg \g_2}) & = \deno{\g_1 \subtypeg \g_2} &: \C(I')(\g_1, \g_2)\\
\end{align*}

\section{The $\allI$ functor}

We expand $\allI: \C(I\times U) \rightarrow \C(I)$ to be a functor which is right adjoint to the re-indexing functor $\pstar$.

\begin{equation}
    \bar{(\_)}: \C(I\times U)(\pstar A, B) \leftrightarrow \C(I)(A, \allI B) : \widehat{(\_)}
\end{equation}

For $A\in\obj\C(I)$, $B\in\obj\C(I\times U)$.

Hence the action of $\allI$ on a morphism $l : A\rightarrow A'$ is as follows:
\begin{eqnarray}
    \allI(l) = \bar{l\after\counit{A}}
\end{eqnarray}
Where $\e_A: \C(I\times U)(\pstar\allI A \rightarrow A)$ is the co-unit of the adjunction.

\subsection{Beck Chevalley Condition}
We need to be able to commute the $\allI$ functor with re-indexing functors. A natural way to do this is:
\begin{align*}
    \theta\star\after\allI & = \allII\after(\theta\times\Id{U})\star
\end{align*}

We shall also require that the canonical natural-transformation between these functors is the identity.

That is, $\bar{(\theta\times\Id{U})\star(\counit{})} = \Id{}: \theta\star\after\allI \rightarrow \allII\after(\theta\times\Id{U})\star\in \C(I')$

This shall be called the Beck-Chevalley condition.


\section{Naturality Corollaries}
Here are some simple corollaries of the adjunction between $\pstar$ and $\allI$.
    
    \subsection{Naturality}
    By the definition of an adjunction:
    
    \begin{equation}
        \bar{f\after\pstar(n)} = \bar{f}\after n
    \end{equation}
    
    \subsection{$\bar{(-)}$ and Re-indexing Functors}
    By assuming the Beck-Chevalley condition that:
    
    \begin{equation}
        \bar{(\theta\times\Id{U})\star(\counit{})} = \Id{}: \theta\star\after\allI \rightarrow \allII\after(\theta\times\Id{U})\star
    \end{equation}
    
    We then have:
    
    \begin{align*}
        \theta\star\unit{A}:\quad&\theta\star A \rightarrow \theta\star\after\allI\after\pstar A\\
        \theta\star\unit{} =& \bar{(\theta\times\Id{U})\star(\counit{\pstar})}\after\theta\star\unit{}\\
        =& (\allII\after(\theta\times\Id{U})\star)(\counit{\pstar})\after\unit{(\allII\after(\theta\times\Id{U})\star)\after\pstar}\after\theta\star\unit{}\\
        = & (\allII\after(\theta\times\Id{U})\star)(\counit{\pstar})\after\unit{\theta\star\after\allI\after\pstar}\after\theta\star\unit{}\\
        = & (\allII\after(\theta\times\Id{U})\star)(\counit{\pstar}) \after(\theta\star\after\allI\after\pstar)\unit{} \after\unit{(\theta\times\Id{U})\star}\\
        = & (\theta\star\after\allI)(\counit{\pstar}
        \after\pstar\unit{})\after\unit{(\theta\times\Id{U})\star}\\
        = & (\theta\star\after\allI)(\Id{})\after\unit{(\theta\times\Id{U})\star}\\
        = & \unit{(\theta\times\Id{U})\star}
    \end{align*}
    \begin{align*}
        \theta\star(\bar{f}) & = \theta\star(\allI(f)\after\eta_A)\\
        & = \theta\star(\allI(f))\after\theta\star(\eta_A)\\
        & =  (\allII\after(\theta\times\Id{U})\star)f \after\unit{(\theta\times\Id{U})\star A}\\
        & = \bar{(\theta\times\Id{U})\star f}\\
    \end{align*}
    
    \subsection{$\hat{(-)}$ and Re-Indexing Functors}
    \begin{align*}
        \theta\star(\pr{\Id{I}}{\rho}\star(\widehat{m})) &= (\pr{\Id{I}}{\rho}\after\theta)\star(\widehat{m})\\
        & = ((\theta\times\Id{U})\after\pr{\Id{I}}{\rho})\star(\widehat{m})\\
        & = \pr{\Id{I}}{\rho\after\theta}\star(\theta\times\Id{U})\star(\widehat{m}) \\
        & = \pr{\Id{I}}{\theta\star\rho}\star(\theta\star(\widehat{m}))
    \end{align*}

\subsection{Pushing Morphisms into $f\star$}

\begin{align*}
    \pr{\Id{I}}{\rho}\star(\widehat{m})\after n &= \pr{\Id{I}}{\rho}\star(\widehat{m})\after\pr{\Id{I}}{\rho}\star\p\star(n)\\
    & = \pr{\Id{I}}{\rho}\star(\widehat{m}\after\pstar(n))\\
    &= \pr{\Id{I}}{\rho}\star(\widehat{m\after n})
\end{align*}


\chapter{Weakenings and Substitutions}
\section{Effect-Environment Weakenings}
        Introduce a relation $\wrel{\w}{\P'}{\P}$ relating effect-variable environments.
        \subsection{Relation}

        \begin{itemize}
            \item $\ntreeruleI{Id}{\ok{\P}}{\wrel{\i}{\P}{\P}}$
            \item $\ntreeruleI{Project}{\wrel{\w}{\P'}{\P}}{\wrel{\w\pi}{(\P', \a)}{\P}}$
            \item $\ntreeruleI{Extend}{\wrel{\w}{\P'}{\P}}{\wrel{\w\x}{(\P', \a)}{(\P, \a)}}$
        \end{itemize}
        \subsection{Weakening Properties}
        \begin{property}[Weakening Preserves Ok]

        \begin{equation}
            \wrelw{\P'}{\P} \wedge \ok{\P}\Rightarrow \ok{\P'}
        \end{equation}
        \end{property}
        \begin{framed}
            \begin{proof}
    
            \subcase{$\i$}
            $$\ok{\P} \wedge\wrel{\i}{\P}{\P} \Rightarrow \ok{\P}$$ 
            \subcase{$\w\pi$}
            \bi,
            \begin{equation}
                \wrelw{\P'}{\P} \wedge \a\notin\P'
            \end{equation}
    
            So, by induction, $\ok{\P'}$ and hence $\ok{(\P', \a)}$
    
            \subcase{$\w\x$}
            \bi, 
            \begin{equation}
                \wrelw{\P'}{\P} \wedge \a\notin\P'
            \end{equation}
    
            So
            \begin{align*}
                \ok{(\P, \a)} & \Rightarrow \ok{\P} \\
                & \Rightarrow \ok{\P'}\\
                & \Rightarrow \ok{(\P', \a)}\\
            \end{align*}
    
            \end{proof}
        \end{framed}

        \begin{property}[Domain Lemma]

        \subsection{Domain Lemma}
        $$\wrelw{\P'}{\P} \Rightarrow (\a\notin\P' \Rightarrow \a\notin\P)$$
        \end{property}

        \begin{proof}
            By trivial Induction.
        \end{proof}

\section{Effect-Environment Substitutions}
    \subsection{Snoc Lists}
        Effect-Environment substitutions may be represented as a snoc-list of variable-effect pairs.

        \[
            \si \gens \nil \mid \si, \a \setto \e
        \]

    
    \subsection{Wellformedness}    
        For any two effect-variable environments, and a substitution, define the wellformedness relation:

        \begin{equation}
            \typerelation{\P'}{\si}{\P}
        \end{equation}

        \begin{itemize}
            \item $\ntreeruleI{Nil}{\ok{\P'}}{\typerelation{\P'}{\nil}{\nil}}$
            \item $\ntreeruleIII{Extend}{\typerelation{\P'}{\si}{\P}}{\wellformed{\P'}{\e}}{\a\notin\P}{\typerelation{\P'}{\si, \a \setto\e}{(\P, \a)}}$
        \end{itemize}

    \subsection{Actions}
        \subsubsection{On Effects}
            Define the action of applying an effect substitution to an effect symbol:

            \begin{equation}
               \si(\e)
            \end{equation}

            \begin{align*}
                \si(e) & = e \\
                \si(\e_1\dot\e_2) & = (\si(\e_1))\dot(\si(\e_2))\\
                \nil(\a) & = \a\\
                (\si, \b\setto \e)(\a) & = \si(\a) \\
                (\si, \a\setto \e)(\a) & =\e
            \end{align*}
        \subsubsection{On Types}
            
            Define the action of applying an effect substitution, $\si$ to a type $A$ as:

            $$A\sub{\si}$$

            Defined as so

            \begin{align*}
                \g\sub{\si} &= \g \\
                (\fntype{A}{B})\sub{\si} &= \fntype{(A\sub{\si})}{(B\sub{\si})} \\
                (\M{\e}{A})\ssi &= \M{\si(\e)}{(A\ssi)}\\
                (\all{\a}{A})\ssi &= \all{\a}{(A\ssi)}\qt{If $\a\#\si$}
            \end{align*}

        \subsubsection{On Term Environments}
            

            Define the action of effect substitution on type environments:
                    
            $$\G\ssi$$
                    
            Defined as so:
                    
            \begin{align*}
                \nil\ssi & = \nil \\
                (\gax)\ssi &= (\G\ssi, x:(A\ssi))\\
            \end{align*}

        \subsubsection{On Terms}
            Define the action of effect-environment substitution on terms:

            \begin{align*}
                x\ssi & = x \\
                \const{A}\ssi & = \const{(A\ssi)} \\
                (\lam{x}{A}{v})\ssi &= \lam{x}{(A\ssi)}{(v\ssi)}\\
                (\pifthenelse{A}{v}{v_1}{v_2})\ssi &= \pifthenelse{(A\ssi)}{v\ssi}{v_1\ssi}{v_2\ssi}\\
                (\apply{v_1}{v_2})\ssi&= \apply{(v_1\ssi)}{v_2\ssi}\\
                (\doin{x}{v_1}{v_2})\ssi&= \doin{x}{(v_1\ssi)}{(v_2\ssi)}\\
                (\elam{\a}{v})\ssi & = \elam{\a}{(v\ssi)}\qt{If $\a\#\si$}\\
                (\eapp{v}{\e})\ssi & = \eapp{(v\ssi)}{\si(\e)}\\
            \end{align*}
    \subsection{Properties}
        
        \begin{property}[Wellformedness]\label{EffectSubsPropertyOne}
            If $\typerelation{\P'}{\si}{\P}$ then $\ok{\P'}$ (By  the 
            Nil case) and $\ok{\P}$ Since each use of the extend case preserves $\okt$.
        \end{property}

        \begin{property}[Weakening]\label{EffectSubsPropertyTwo}
            If $\typerelation{\P'}{\si}{\P}$ then $\wrelw{\P'}{\P'} \implies \typerelation{\P''}{\si}{\P}$ since $\wellformed{\P'}{\e}\implies \wellformed{\P''}{\e}$ and $\ok{\P'}\implies \ok{\P''}$
        \end{property}

        \begin{property}[Extension]\label{EffectSubsPropertyThree}
            If $\typerelation{\P'}{\si}{\P}$ then
        
            \begin{equation}
                \a\notin\P\wedge\a\notin\P' \implies \typerelation{(\P', \a)}{(\si, \a\setto\a)}{(\P,\a)}
            \end{equation}

            Since $\wrel{\i\pi}{\P',\a}{\P'}$ so $\typerelation{\P',\a}{\si}{\P}$ and $\wellformed{\P',\a}{\a}$        
        \end{property}

\section{Term-Environment Weakenings}
Type environment weakenings are inductively defined with respect to an effect environment.


\[
    \ntreeruleI{Id}{\wellformedok{\P}{\G}}{\pewrel{\i}{\G}{\G}}
    \quad  
    \condtreeruleII{Project}{\pewrel{\w}{\G'}{\G}}{\typerelation{\P}{A}{\type}}{\pewrel{\w \pi}{\G, x: A}{\G}}{x \notin \dom{\G'}}
\]

\[
    \condtreeruleII{Extend}{\pewrel{\w}{\G'}{\G}}{ A \subtype B}{\pewrel{\w \x}{\G', x: A}{\G, x: B}}{x \notin \dom{\G'}}
\]

\section{Term-Environment Substitutions}
    \subsection{Snoc Lists}
        Term-Environment substitutions may be represented as a snoc-list of variable-term pairs.

        \[
            \si \gens \nil \mid \si, x \setto v
        \]

    \subsection{Wellformedness}
    The relation instance $\typerelation{\P'}{\si}{\P}$ means that $\si$ is a substitution from $\P'$ to $\P$. It is defined inductively using the following rules.

    \[
        \ntreeruleI{Nil}{\ok{\P'}}{\typerelation{\P'}{\nil}{\nil}}
        \quad\quad
        \condtreeruleII{Extend}{\typerelation{\P'}{\si}{\P}}{\wellformed{\P'}{\e}}{\typerelation{\P'}{\si, \a \setto\e}{(\P, \a)}}{\a\notin\P}
    \]

    \subsection{Action on Terms}
         We define the action of applying a term substitution $\si$ as 
         $$v\ssi$$
         
         
         \begin{align*}
             x\sub{\nil} & = x \\
             x\sub{\si,x\setto v} & = v \\
             x\sub{\si,x'\setto v'} & = x\ssi\qt{If }x \neq x'\\
             \const{A}\ssi & = \const{A} \\
             (\lam{x}{A}{v})\ssi &= \lam{x}{A}{(v\ssi)}\qt{If }x\#\si\\
             (\pifthenelse{A}{v}{v_1}{v_2})\ssi &= \pifthenelse{A}{v\ssi}{v_1\ssi}{v_2\ssi}\\
             (\apply{v_1}{v_2})\ssi&= \apply{(v_1\ssi)}{v_2\ssi}\\
             (\doin{x}{v_1}{v_2})&= \doin{x}{(v_1\ssi)}{(v_2\ssi)}\qt{If } x\#\si\\
             (\elam{\a}{v})\ssi & = \elam{\a}{(v\ssi)}\\
             (\eapp{v}{\e})\ssi & = \eapp{(v\ssi)}{\e}\\
         \end{align*}    

    \subsection{Properties}
        \begin{property}[Wellformedness]\label{TermSubsPropOne}
            If $\etyperelation{\P}{\G'}{\si}{\G}$ then $\wellformedok{\P}{\G}$ and $\wellformedok{\P}{\G'}$. 
            Since $\wellformedok{\P}{\G'}$ holds by the Nil axiom. $\wellformedok{\P}{\G}$ holds by induction on the wellformedness relation.
        \end{property}

        \begin{property}[Weakening]\label{TermSubsPropTwo}
            If $\etyperelation{\P}{\G'}{\si}{\G}$ then $\wrel{\w}{\G''}{\G'}$ implies $\etyperelation{\P}{\G''}{\si}{\G}$. 
            By induction over wellformedness relation. For each $x\setto v$ in $\si$, $\etyperelation{\P}{\G''}{v}{A}$ holds if $\etyperelation{\P}{\G'}{v}{A}$ holds.
        \end{property}

        \begin{property}[Extension]\label{TermSubsPropThree}
            If $\etyperelation{\P}{\G'}{\si}{\G}$ then $x \notin (\dom{\G} \cup \dom{\G''})$ implies $\etyperelation{\P}{(\G', x: A)}{(\si, x \setto x)}{(\gax)}$. Since $\wrel{\i\pi}{\G',x: A}{\G'}$, so by property \ref{TermSubsPropTwo}, 
            $$\etyperelation{\P}{\G', x: A}{\si}{\G}$$
            In addition, $\etyperelation{\P}{\G', x: A}{x}{A}$ trivially, so by the rule \textbf{Extend}, wellformedness holds for
            \begin{equation}
                \etyperelation{\P}{(\G', x:A)}{(\si, x\setto v)}{(\gax)}
            \end{equation}
        \end{property}


\chapter{Denotations}
\section{Effects}
For each instance of the wellformedness relation on effects, we define a morphism $\deno{\typerelation{\P}{\e}{\effect}}: \ciu$

\begin{itemize}
    \item $\deno{\typerelation{\P}{e}{\effect}} = \deno{\e} \after \term{I}: \rightarrow U$
    \item $\deno{\typerelation{\P,\a}{\a}{\effect}} = \pp: I\times U \rightarrow U$
    
    \item $\deno{\typerelation{\P, \b}{\a}{\effect}} = \deno{\typerelation{\P}{\a}{\effect}}\after\p: I\times U\rightarrow U$
    
    \item $\deno{\typerelation{\P}{\e_1\dot \e_2}{\effect}} = \Mul_I(\deno{\typerelation{\P}{\e_2}{\effect}},\deno{\typerelation{\P}{\e_1}{\effect}}): I \rightarrow U$
\end{itemize}

\section{Types}
  
For each instance of the wellformedness relation on types, we derive an object $\deno{\typerelation{\P}{A}{\type}}\in\obj\C(I)$.

Since the fibre category $\C(I)$ is S-Closed, it has objects for all ground types, a terminal object, graded monad $\T{}{}$, exponentials, products, and co-product over $\1+\1$.

\begin{itemize}
    \item $\deno{\typerelation{\P}{\U}{\type}} = \1$
    
    \item $\deno{\typerelation{\P}{\B}{\type}} = \1+\1$
    
    \item $\deno{\typerelation{\P}{\g}{\type}} = \deno{\g} $
    
    \item $\deno{\typerelation{\P}{\ab}{\type}} = (\deno{\typerelation{\P}{B}{\type}})^{(\deno{\typerelation{\P}{A}{\type}})}$
    
    \item $\deno{\typerelation{\P}{\mea}{\type}} =\T{\deno{\typerelation{\P}{\e}{\effect}}}{\deno{\typerelation{\P}{A}{\type}}}$
    \item $\deno{\typerelation{\P}{\all{\a}{A}}{\type}} =\allI(\deno{\typerelation{\P,\a}{A}{\type}})$
\end{itemize}

\section{Effect Substitution}
For each effect-environment substitution wellformedness-relation, define a denotation morphism, $\deno{\typerelation{\P'}{\si}{\P}}: \cii$

\begin{itemize}
    \item $\deno{\typerelation{\P'}{\nil}{\nil}} = \term{I}: \C(I', \1)$
    \item $\deno{\typerelation{\P'}{(\si, \a\setto\e)}{\P,\a}} = \pr{\deno{\typerelation{\P'}{\si}{\P}}}{\deno{\typerelation{\P}{\e}{\effect}}}: \C(I', I\times U)$
\end{itemize}

\section{Effect Weakening}

For each instance of the effect-environment weakening relation, define a denotation morphism: $\deno{\wrelw{\P'}{P}}: \cii$

\begin{itemize}
    \item $\deno{\wrel{\i}{\P}{\P}} = \Id{I}: I \rightarrow I$
    \item $\deno{\wrel{w\pi}{\P',\a}{\P}} = \deno{\wrelw{\P'}{\P}}\after \p: I'\times U\rightarrow I$
    \item $\deno{\wrel{w\x}{\P',\a}{\P,\a}} = (\deno{\wrelw{\P'}{\P}}\times \Id{U}): I'\times U\rightarrow I\times U$
\end{itemize}

\section{Subtyping}

For each instance of the subtyping relation with respect  to an effect environment, there exists a denotation, $\deno{A\subtypep B}: \C(I)(A, B)$.

\begin{itemize}
    \item $\deno{\g_1\subtypep \g_2} = \deno{\g_1\subtypeg \g_2} : \C(I)(\g_1, \g_2)$
    \item $\deno{\ab \subtypep \fntype{A'}{B'}} = \deno{B\subtypep B'}^{A'}\after B^{\deno{A'\subtypep A}}$
    \item $\deno{\M{\e_1}{A}\subtypep\M{\e_2}{B}} = \deno{\e_1\subeffectp\e_2}\after\T{\e_1}{\deno{A\subtypep B}}$
    \item $\deno{\all{\a}{A}\subtypep\all{\a}{B}} = \allI{\deno{A\subtypepa B}}$
\end{itemize}

\section{Term-Environments}
For each instance of the well formed relation on type environments, define an object $\deno{\wellformedok{\P}{\G}}\in\obj\C(I)$.

\begin{itemize}
    \item $\deno{\wellformedok{\P}{\nil}} = \1: \C(I)$
    \item $\deno{\wellformedok{\P}{\gax}} = (\deno{\wellformedok{\P}{\G}} \times \deno{\typerelation{\P}{A}{\type}})$
\end{itemize}

\section{Terms}

For each instance of the typing relation, define a denotation morphism: $\deno{\gpetyperelation{v}{A}}: \C(I)(\G_I, A_I)$. Writing $\G_I$ and $A_I$ for $\deno{\wellformedok{\P}{\G}}$ and $\deno{\typerelation{\P}{A}{\type}}$.

For each ground constant, $\const{A}$, there exists $c: \1 \rightarrow A_I$ in $\C(I)$.

\begin{itemize}
    \item $\ntreeruleI{Unit}{\wellformedok{\P}{\G}}{\deno{\etyperelation{\P}{\G}{\u}{\U}} = \term{\G} : \G_I \rightarrow \1}$
        
    \item $\ntreeruleI{Const}{\wellformedok{\P}{\G}}{\deno{\etyperelation{\P}{\G}{\const{A}}{A}} = \deno{\const{A}} \after \term{\G} : \G \rightarrow \deno{A}}$
         
    \item $\ntreeruleI{True}{\wellformedok{\P}{\G}}{\deno{\etyperelation{\P}{\G}{\t}{\B}} = \inl \after \term{\G} : \G \rightarrow \deno{\B} = \1+\1}$
        
    \item $\ntreeruleI{False}{\wellformedok{\P}{\G}}{\deno{\etyperelation{\P}{\G}{\f}{\B}} = \inr \after \term{\G} : \G \rightarrow \deno{\B} = \1+\1}$
        
    \item $\ntreeruleI{Var}{\wellformedok{\P}{\G}}{\deno{\etyperelation{\P}{\gax}{x}{A}} = \pp: \G \times A \rightarrow A}$
    \item $\ntreeruleI{Weaken}{f = \deno{\gpetyperelation{x}{A}}: \G \rightarrow A}{\deno{\etyperelation{\P}{\gby}{x}{A}} = f \after \p: \G \times B \rightarrow A}$
    \item $\ntreeruleI{Fn}{f = \deno{\etyperelation{\P}{\gax}{v}{B}} : \G \times A \rightarrow B}{\deno{\etyperelation{\P}{\G}{\lam{x}{A}{v}}{\ab}} = \cur{f} : \G \rightarrow (B)^A}$
    
    \item $\ntreeruleII{Subtype}{f = \deno{\etyperelation{\P}{\G}{v}{A}} : \G \rightarrow A}{ g = \deno{A \subtypep B}}{\deno{\etyperelation{\P}{\G}{v}{B}} = g \after f : \G \rightarrow B}$
  
    \item $\ntreeruleI{Return}{f = \deno{\etyperelation{\P}{\G}{v}{A}}}{\deno{\etyperelation{\P}{\G}{\return{v}}{\moa}} = \point{A} \after f}$
        
 
    \item \scalebox{.9}{$\ntreeruleIII{If}{f = \deno{\etyperelation{\P}{\G}{v}{\B}}: \G\rightarrow\1+\1 }{ g = \deno{\etyperelation{\P}{\G}{v_1}{\mea}}}{ h = \deno{\etyperelation{\P}{\G}{v_2}{\mea}}}{\deno{{\etyperelation{\P}{\G}{\ifthenelse{\e}{A}{v}{v_1}{v_2}}{\mea}}} = \app\after((\fld{\cur{g\after\pp}}{\cur{h\after\pp}}\after f)\times \idg)\after \diag{\G} : \G \rightarrow \tea}$}
        
    \item $\ntreeruleII{Bind}{f = \deno{\etyperelation{\P}{\G}{v_1}{\M{\e_1}{A}} : \G \rightarrow \T{\e_1}{A}}}{{ g = \deno{\etyperelation{\P}{\gax}{v_2}{\M{\e_2}{B}}}}: \G \times A \rightarrow \T{\e_2}{B}}{\deno{\etyperelation{\P}{\G}{\doin{x}{v_1}{v_2}}{\M{\e_1 \dot \e_2}{B}}} = \bind{\e_1}{\e_2}{B} \after \T{\e_1}{g} \after \tstrength{\G}{A}{\e_1} \after \pr{\idg}{f}: \G \rightarrow \T{\e_1 \dot \e_2}{B}}$ 
   
    \item $\ntreeruleII{Apply}{f = \deno{\gpetyperelation{v_1}{\ab}}: \G \rightarrow (B)^{A} }{ g=\deno{\gpetyperelation{v_2}{A}}: \G \rightarrow A}{\deno{\gpetyperelation{\apply{v_1}{v_2}}{B}}= \app\after\pr{f}{g}: \G \rightarrow B }$
    \item $\ntreeruleI{Effect-Gen}{f = \deno{\etyperelation{\P,\a}{\G}{v}{A}}: \ciuw(\G, A)}{\deno{\gpetyperelation{\elam{\a}{A}}{\all{\e}{A}}} = \bar{f}: \C(I)(\G, \allI(A))}$
    
    \item $\ntreeruleII{Effect-Spec}{g=\deno{\gpetyperelation{v}{\all{\a}{A}}}: \C(I)(\G, \allI(A))}{ h = \deno{\typerelation{\P}{\e}{\effect}}: \ciu}{\deno{\gpetyperelation{\eapp{v}{\e}}{A\ssub{\a}{\e}}} = \pr{\Id{I}}{h}\star(\e_{\deno{\typerelation{\P,\b}{A\ssub{\a}{\b}}{\type}}})\after g: \C(I)(\G, A\ssub{\a}{\e})}$
\end{itemize}

\section{Term Weakening}

For each instance of the term-environment weakening relation, define a morphism $\deno{\pewrel{\w}{\G'}{\G}}: \G' \rightarrow \G \in \C(I)$

\begin{itemize}
    \item $\deno{\pewrel{\i}{\G}{\G}} = \idg: \G \rightarrow \G \in \C(I)$
    \item $\deno{\pewrel{\w\pi}{\G', ax}{\G}} = \deno{\pewrel{\w}{\G'}{\G}}\after\p: \G'\times A \rightarrow \G$
    \item $\deno{\pewrel{\w\x}{\G', x: A}{\G, x: B}} = \deno{\pewrel{\w}{\G'}{\G}}\times \deno{A\subtypep B}: \G'\times A \rightarrow \G\times B$
\end{itemize}

\section{Term Substitutions}
For each instance of the term-environment substitution relation, define a denotation morphism: $\deno{\etyperelation{\P}{\G'}{\si}{\G}}: \G' \rightarrow \G \in \C(I)$

\begin{itemize}
    \item $\deno{\etyperelation{\P}{\G'}{\nil}{\nil}} = \term{\G'}: \G' \rightarrow \1$
    \item $\deno{\etyperelation{\P}{\G'}{(\si, x \setto v)}{\gax}} = \pr{\deno{\etyperelation{\P}{\G'}{\G}}}{\deno{\etyperelation{\P}{\G'}{v}{A}}}: \G' \rightarrow \G\times \1$
\end{itemize}

\chapter{Effect Substitution Theorem}
In this section, we state and prove a theorem that the action of a simultaneous effect-variable substitution upon a structure in the language has a consistent effect upon the denotation of the language. More formally, for the denotation morphism $\D$ of some relation, the denotation of the substituted relation, $\D' = \si\star(\D)$.


\section{Effects}

\begin{theorem}[Effect Substitution Preserves Effect Wellformedness]
    If $\wellformed{\P}{\e}$ and $\typerelation{\P'}{\i}{\P}$ then $\wellformed{\P'}{\si(\e)}$   
\end{theorem}

\begin{framed}
    \begin{proof}
        \case{Ground}
        $\si(e) = e$, so $\wellformed{\P'}{\si(\e)}$ holds.
        \case{Compose}
        \bi, $\wellformed{\P}{\e_1}$ and $\wellformed{\P}{\e_2}$ so $\wellformed{\P'}{\si(\e_1)}$ and $\wellformed{\P'}{\si(\e_2)}$ by induction and hence $\wellformed{\P'}{\si(\e_1\dot\e_2)}$
        \case{Var}
        \bi, $\P=\P'',\a$ and $\ok{\P'',\a}$. Hence by case splitting on $\i$, we see that $\si=\si', \a\setto\e$.
        
        So by inversion, $\wellformed{\si}{\e}$ so $\wellformed{\P'}{\si(\a)=\e}$
        \case{Weaken}
        By inversion,  $\P=\P'', \b$ and $\wellformed{\P''}{\a}$, so $\si=\si'\b\setto\e$.
        
        So $\typerelation{\P'}{\si'}{\P''}$.
        
        hence by induction, $\wellformed{\P'}{\si'(a)}$, so $\wellformed{\P'}{\si(\a)}$ since $\a\neq\b)$
    \end{proof}
    
\end{framed}
\begin{theorem}[Effect Substitution Preserves the Subeffect Relation]
    If $\typerelation{\P'}{\si}{\P}$ and $\e_1\subeffectp\e_2$, then $\e_1\ssi\subeffect_{\P'}\e_2\ssi$.
\end{theorem}


\begin{framed}  
    \begin{proof}
        For any ground substitution $\si'$ of $\P'$, then $\si\si'$ (the substitution $\si'$ applied after $\si$) is also a ground substitution.
        
        So $\e_1\ssi\sub{\si'}\subeffect\e_2\ssi\sub{\si'}$. 
        
        So $\e_1\ssi\subeffect_{\P'}\e_2\ssi$.
        
    \end{proof}
\end{framed}

\begin{theorem}[Effect Substitution and Effect Denotation]
    If $\si = \deno{\typerelation{\P'}{\si}{\P}}$ then 
    
    $\deno{\typerelation{\P'}{\si(\e)}{\effect}} = \si\star\deno{\typerelation{\P}{\e}{\effect}} = \deno{\typerelation{\P}{\e}{\effect}}\after\si$.
\end{theorem}



\begin{framed}
    \begin{proof}
        By induction on the derivation on $\deno{\typerelation{\P}{\e}{\effect}}$
        
        \case{Ground}
        \begin{align*}
            \deno{\typerelation{\P}{e}{\effect}}\after\si & = \deno{e}\after\term{I}\after\si \\
            & = \deno{e}\after\term{I'} \\
            & = \deno{\typerelation{\P'}{e}{\type}}\\
        \end{align*}
        
        \case{Var}
        \begin{align*}
            \deno{\typerelation{\P,\a}{\a}{\effect}}\after\si' &= \pp\after\pr{\si}{\deno{\typerelation{\P'}{\e}{\effect}}}\qt{By inversion, $\si' = (\si, \a\setto\e)$}\\
            & =\deno{\typerelation{\P'}{\e}{\effect}} \\
            &= \deno{\typerelation{\P'}{\si'(\a)}{\effect}}\\
        \end{align*}
        
        \case{Weaken}
        \begin{align*}
            \deno{\typerelation{\P, \b}{\a}{\type}}\after\si' &= \deno{\typerelation{\P}{\a}{\type}} \after \p\after\pr{\si}{\deno{\typerelation{\P'}{\e}{\effect}}}\qt{By inversion, $\si' = (\si, \b\setto\e)$}\\
            & = \deno{\typerelation{\P}{\a}{\type}}\after\si\\
            & = \deno{\typerelation{\P'}{\si(\a)}{\type}}\\
            & = \deno{\typerelation{\P'}{\si'(\a)}{\type}}\\
        \end{align*}
        
        \case{Compose}
        \begin{align*}
            \deno{\typerelation{\P}{\e_1\dot\e_2}{\type}} \after\si &=
            \Mul_{I}(\deno{\typerelation{\P}{\e_1}{\effect}}, \deno{\typerelation{\P}{\e_2}{\effect}})\after \si \\
            & = \Mul_{I'}(\deno{\typerelation{\P}{\e_1}{\effect}}\after \si, \deno{\typerelation{\P}{\e_2}{\effect}}\after \si)\qt{By Naturality}\\
            & = \Mul_{I'}(\deno{\typerelation{\P'}{\si(\e_1)}{\effect}}, \deno{\typerelation{\P}{\si(\e_2)}{\effect}})\\
            & = \deno{\typerelation{\P'}{\si(\e_1)\dot\si(\e_2)}{\effect}}\\
            & = \deno{\typerelation{\P'}{\si(\e_1\dot\e_2)}{\effect}}\\
        \end{align*}
        
    \end{proof}
    
\end{framed}\section{Types}

\begin{theorem}[Effect Substitution Preserves Type Wellformedness]
    If $\typerelation{\P'}{\si}{\P}$ and $\wellformed{\P}{A}$ then $\wellformed{\P'}{A\ssi}$
\end{theorem}


\begin{framed}    
    \begin{proof}
        \case{Ground}
        
        $\ok{\P'}$ so $\wellformed{\P'}{\g}$ and $\g\ssi=\g$.
        
        Hence $\wellformed{\P'}{\g\ssi}$.
        
        \case {Fn}
        
        By inversion,  $\wellformed{\P}{A}$ and $\wellformed{\P}{B}$.
        
        So by induction, $\wellformed{\P'}{A\ssi}$ and $\wellformed{\P'}{B\ssi}$.
        
        So \begin{equation}
            \wellformed{\P'}{\fntype{(A\ssi)}{(B\ssi)}}
        \end{equation}
        
        So \begin{equation}
            \wellformed{\P'}{(\ab)\ssi}
        \end{equation}
        
        
        
        \case{Effect}
        
        \bi, $\wellformed{\P}{\e}$ and $\wellformed{\P}{A}$ so by induction and substitution of effect preserving effect-wellformedness, 
        
        $\wellformed{\P'}{\si(\e)}$ and $\wellformed{\P'}{A\ssi}$ so $\wellformed{\P}{\M{\si(\e)}{A\ssi}}$ so $\wellformed{\P'}{(\mea)\ssi}$
        
        \case{Quantification}
        \bi, $\wellformed{\P,\a}{A}$. So by picking $\a\notin\P\wedge\a\notin\P'$ using $\a$-equivalence, we have $\typerelation{(\P',\a)}{(\si\a\setto\a)}{(\P,\a)}$.
        
        So by induction $\wellformed{(\P, \a)}{A\sub{\si,\a\setto\a}}$
        
        So $\wellformed{(\P',\a)}{A\ssi}$
        
        So $\wellformed{\P'}{(\all{\a}{A})\ssi}$
        
    \end{proof}
    
\end{framed}\begin{theorem}[Effect Substitution and Type Denotations]
    If $\si = \deno{\typerelation{\P'}{\si}{\P}}$ then 
    
    $\deno{\typerelation{\P'}{A\ssi}{\type}} = \si\star\deno{\typerelation{\P}{A}{\type}}$.
\end{theorem}

\begin{framed}
    \begin{proof}
        By induction on the derivation on $\deno{\typerelation{\P}{A}{\type}}$. Making use of the S-Closure of the re-indexing functor.
        
        \case{Ground}
        \begin{align*}
            \si\star\deno{\typerelation{\P}{\g}{\type}} &= \si\star\deno{\g}\\
            & =  \deno{\g}\qt{By S-Closure}\\
            & = \deno{\typerelation{\P'}{\g\ssi}{\type}}
        \end{align*}
        
        \case{Effect}
        \begin{align*}
            \si\star\deno{\typerelation{\P}{\mea}{\type}} & =  \si\star(\T{\deno{\typerelation{\P}{\e}{\effect}}}{\deno{\typerelation{\P}{A}{\type}}})\\
            &= \T{\si\star(\deno{\typerelation{\P}{\e}{\effect}})}{\si\star(\deno{\typerelation{\P}{A}{\type}})}\\
            & = \deno{\typerelation{\P'}{(\mea)\ssi}{\type}}
        \end{align*}
        \case{Quantification}
            \begin{align*}
                \si\star\deno{\typerelation{\P}{\all{\a}A}{\type}} & = \si\star(\allI(\deno{\typerelation{\P,\a}{A}{\type}}))\\
                & = \allI((\si\times\Id{U})\star\deno{\typerelation{\P,\a}{A}{\type}})\qt{By Beck-Chevalley}\\
                & = \allI(\deno{\typerelation{\P',\a}{A\sub{\si, \a\setto\a}}{\type}})\\
                & = \allI(\deno{\typerelation{\P',\a}{A\ssi}{\type}})\\
                & = \deno{\typerelation{\P'}{\all{\a}{A\ssi}}{\type}}\\
                & = \deno{\typerelation{\P'}{(\all{\a}{A})\ssi}{\type}}\\
            \end{align*}
        
        \case{Fn}
        \begin{align*}
            \si\star\deno{\typerelation{\P}{\ab}{\type}} &= \si\star(\deno{\typerelation{\P}{B}{\type}}^{\deno{\typerelation{\P}{A}{\type}}})\\
            &=\si\star(\deno{\typerelation{\P}{B}{\type}})^{\si\star(\deno{\typerelation{\P}{A}{\type}})}\\
            & = \deno{\typerelation{\P'}{B\ssi}{\type}}^{\deno{\typerelation{\P'}{A\ssi}{\type}}}\\
            & = \deno{\typerelation{\P'}{\fntype{(A\ssi)}{(B\ssi)}}{\type}}\\
            & = \deno{\typerelation{\P'}{(\ab)\ssi}{\type}}\\
        \end{align*}
    \end{proof}
\end{framed}

\section{Subtyping}

\begin{theorem}[Effect Substitution Preserves the Subtyping Relation] 
    If $\typerelation{\P'}{\si}{\P}$ and $A\subtypep B$ then $A\ssi\subtypepp B\ssi$    
\end{theorem}
\begin{framed}
    \begin{proof}
        By induction on the subtyping relation
        
        \case{Ground}
        \bi, $A\subtypeg B$, so $A, B$ are ground types.
        Hence $A\ssi = A$ and $B\ssi = B$.
        So $A\ssi\subtypepp B\ssi$
        
        \case{Fn}
        \bi, $A'\subtypep A$ and $B\subtypep B'$.
        
        So by induction, $A'\ssi\subtypepp A\ssi$ and $B\ssi\subtypepp B'\ssi$.
        
        So $\fntype{(A\ssi)}{(B\ssi)}\subtypepp \fntype{(A'\ssi)}{(B'\ssi)}$
        
        So $(\ab)\ssi\subtypepp (\fntype{A'}{B'})\ssi$
        
        \case{Effect}
        
        \bi, $A\subtypep B$, $\e_1\subeffectp\e_2$.
        
        So by induction and substitution preserving the subeffect relation, 
        
        $A\ssi\subtypepp B\ssi$ and $\si(\e_1)\subeffectpp\si(\e_2)$
        
        So $\M{\si(\e_1)}{(A\ssi)} \subtypepp\M{\si(\e_2)}{(B\ssi)}$
        
        So $(\M{\e_1}{A})\ssi \subtypepp(\M{\e_2}{B})\ssi$
        
        \case{Quantification}
        By inversion, $A\subtypepa B$ and $\ok{\P,\a}$. Picking $\a\notin\P'$ by $\a$ equivalence, and by the extension lemma \ref{EffectSubsPropertyThree}, $\typerelation{(\P,\a)}{(\si, \a\setto\a)}{(\P,\a)}$. Hence, by induction $A\sub{\si, \a\setto\a}\subtypeppa B\sub{\si, \a\setto\a}$, so $\all{\a}{A}\subtypepp\all{\a}{B}$.
        $$\square$$
    \end{proof}
    
\end{framed}

\begin{theorem}[Effect Substitution and Subtyping Denotations]
    If $\si = \deno{\typerelation{\P'}{\si}{\P}}$ then $\deno{A\ssi\subtypepp B\ssi} = \si\star\deno{A\subtypep B}: \C(I')(A, B)$.    
\end{theorem}

\begin{framed}
    \begin{proof}
        
        By induction on the derivation on $\deno{A\subtypep B}$. Using S-preserving property of $\si\star$ 
        
        \case{Ground}
        \begin{align*}
            \si\star(\g_1\subtypeg\g_2) &= (\g_1\subtypeg\g_2)
        \end{align*}
        
        Since $\si\star$ is s-closed.
        
        \case{Effect}
        \begin{align*}
            \si\star\deno{\M{\e_1}{A} \subtypep \M{\e_2}{B}} & = \si\star(\deno{\e_1\subeffectp\e_2})\after\si\star(\T{\e_1}(\deno{A\subtypep B})) \\ 
             &= \deno{\si(\e_1)\subeffectpp\si(\e_2)} \after \T{\si(\e_1)}{\deno{A\ssi\subtypepp B\ssi}}\qt{By S-Closure}\\
             &= \deno{\M{\si(\e_1)}{A\ssi}\subtypepp\M{\si(\e_2)}{B\ssi}}\\
             &= \deno{(\M{\e_1}{A})\ssi\subtypepp\M{\e_2}{B}\ssi}\\
        \end{align*}
        
        \case{Quantification}
            \begin{align*}
                \si\star\deno{\all{\a}{A}\subtypep\all{\a}{B}} &= \si\star(\allI(\deno{A\subtypepa B})) \\
                &=\allII((\si\times\Id{U})\star(\deno{A\subtypepa B}))\\
                &=\allII(\deno{A\sub{\si,\a\setto\a}\subtypeppa B\sub{\si,\a\setto\a}})\\
                &= \deno{(\all{\a}{A})\ssi \subtypepp(\all{\a}{B})\ssi}\\
            \end{align*}
        
        \case{Fn}
        
        \begin{align*}
            \si\star&\deno{(\ab)\subtypep\fntype{A'}{B'}} \\
             &= \si\star(\deno{B\subtypep B'}^{A'}\after B^{\deno{A'\subtypep A}})\\
            &= \si\star(\cur{\deno{B\subtypep B'}\after\app})\after\si\star(\cur{\app\after(\Id{B}\times\deno{A'\subtypep A})})\\
            & = \cur{\si\star(\deno{B\subtypep B'})\after\app}\after\cur{\app\after(\Id{B}\times\si\star(\deno{A'\subtypep A}))}\\
            & = \cur{\deno{B\ssi\subtypepp B'\ssi}\after\app}\after\cur{\app\after(\Id{B\ssi}\times\deno{A'\ssi\subtypepp A\ssi})}\\
            &= \deno{\fntype{(A\ssi)}{(B\ssi)}\subtypepp\fntype{(A'\ssi)}{(B'\ssi)}}\\
            &= \deno{(\ab)\ssi\subtypepp(\fntype{A'}{B'})\ssi}
        \end{align*}
        
    \end{proof}
\end{framed}

\section{Term Environments}

\begin{theorem}[Effect Substitution Preserves Term-Environment wellformedness]
    If $\wellformedok{\P}{\G}$ and $\typerelation{\P'}{\si}{\P}$ then $\wellformedok{\P'}{\G\ssi}$
    
\end{theorem}

\begin{framed}   
    \begin{proof}
        \case{Nil}
        $\ok{\P}\implies\ok{\P'}$ so $\wellformedok{\P'}{\nil}$
        and $\nil\ssi = \nil$
        \case{Extend}
        \bi, $\wellformedok{\P}{\G}$ and $\wellformed{\P}{A}$.
        
        By induction and substitution preserving wellformedness of types, $\wellformedok{\P'}{\G'\ssi}$ and $\wellformed{\P'}{A\ssi}$.
        
        
        So $\wellformedok{\P'}{(\G'\ssi, x: A\ssi)}$.
        
        Hence $\wellformedok{\P'}{\gax\ssi}$.
    \end{proof}
    
\end{framed}


\begin{theorem}[Effect Substitution and Term Environment Denotations]
    If $\si = \deno{\typerelation{\P'}{\si}{\P}}$ then $\deno{\wellformedok{\P'}{\G\ssi}} = \si\star\deno{\wellformedok{\P}{\G}}\in\obj\C(I')$.
\end{theorem}

\begin{framed}
    \begin{proof}
        By induction on the derivation on $\deno{\wellformedok{\P}{\G}}$. Using the S-Closure of the re-indexing functor.
        
        \case{Nil}
        \begin{align*}
            \si\star\deno{\wellformedok{\P}{\nil}} &=\si\star\1\\
            & = \1\qt{By S-preservation}\\
            &= \deno{\wellformedok{\P'}{\nil}}\\
            &= \deno{\wellformedok{\P'}{\nil\ssi}}\\
        \end{align*}
        
        \case{Extend}
        \begin{align*}
           \si\star\deno{\wellformedok{\P}{\gax}} &= \si\star(\deno{\wellformedok{\P}{\G}} \times \deno{\typerelation{\P}{A}{\type}}) \\
           & = (\si\star\deno{\wellformedok{\P}{\G}} \times \si\star\deno{\typerelation{\P}{A}{\type}})\\
            & = (\deno{\wellformedok{\P'}{\G\ssi}} \times \deno{\typerelation{\P'}{A\ssi}{\type}})\\
            & = \deno{\wellformedok{\P'}{\G\ssi, x: A\ssi}}\\
            & = \deno{\wellformedok{\P'}{(\gax)\ssi}}\\
        \end{align*}
    \end{proof}
\end{framed}



\section{Terms}
\begin{theorem}[Effect-Environment Substitution Preserves the Typing Relation]
    If $\typerelation{\P'}{\si}{\P}$ and $\gpetyperelation{v}{A}$, then $\etyperelation{\P'}{\G\ssi}{v\ssi}{A\ssi}$ 
\end{theorem}


\begin{framed}
    \begin{proof}
        
    \case{Const}
    \bi, $\wellformedok{\P}{\G}$.
    
    So $\wellformedok{\P'}{\G}$
    
    So $\etyperelation{\P'}{\G\ssi}{\const{A\ssi}}{A\ssi}$
    \case{True, False, Unit}
    The logic is the same for each of these cases, so we look at the case $\t$ only.
    
    \bi, $\wellformedok{\P}{\G}$.
    
    So $\wellformedok{\P'}{\G}$
    
    So $\etyperelation{\P'}{\G\ssi}{\t}{\B}$
    
    Since $\t\ssi=\t$ and $\B\ssi=\B$.
    
    \case{Var}
    By inversion,  $\G = \G', x:A$ and $\wellformedok{\P}{\G', x:A}$.
    
    So since substitution preserves wellformedness of type environments, $\wellformedok{\P'}{\G'\ssi, x:A\ssi}$
    
    So $\etyperelation{\P'}{\G\ssi}{x}{A\ssi}$
    
    Since $x\ssi = x$
    
    \case{Weaken}
    By inversion,  $\G = \G', y:B$, $\wellformed{\P}{B}$, and $\etyperelation{\P}{\G'}{x}{A}$. $x \neq y$
    
    By induction and the theorem that effect-environment substitution preserves type wellformedness, we have: $\etyperelation{\P'}{\G'\ssi}{x}{A\ssi}$ and $\wellformed{\P'}{B\ssi}$
    
    So $\etyperelation{\P'}{\G\ssi}{x\ssi}{A\ssi}$
    
    Since $x\ssi = x$, $\G\ssi = (\G'\ssi, y:B\ssi)$
    
    
    \case{Fn}
    By inversion,  $\etyperelation{\P}{\gax}{v}{B}$.
    
    So, by induction $\etyperelation{\P'}{(\gax)\ssi}{v\ssi}{B\ssi}$.
    
    So, $\etyperelation{\P}{\G\ssi, x:A\ssi}{v\ssi}{B\ssi}$.
    
    Hence by the lambda type rule,
    
    $\etyperelation{\P'}{\G\ssi}{\lam{x}{A\ssi}{v\ssi}}{\fntype{(A\ssi)}{(B\ssi)}}$
    
    So
    
    $\etyperelation{\P'}{\G\ssi}{(\lam{x}{A}{v})\ssi}{(\ab)\ssi)}$
    
    \case{Apply}
    \bi, $\gpetyperelation{v_1}{\ab}$, $\gpetyperelation{V_2}{A}$.
    
    So by induction, $\etyperelation{\P'}{\G\ssi}{v_1\ssi}{\fntype{(A\ssi)}{(B\ssi)}}$.
    
    So $\etyperelation{\P'}{\G\ssi}{\apply{(v_1\ssi)}{(v_2\ssi)}}{B\ssi}$.
    
    So $\etyperelation{\P'}{\G\ssi}{(\apply{v_1}{v_2})\ssi}{(\ab)\ssi}$
    \case{Subtype}
    
    \bi, $\gpetyperelation{v}{A}$ and $\wellformed{\P}{A\subtype B}$
    
    So by induction and effect-environment substitution preserving subtyping, $\etyperelation{\P'}{\G\ssi}{v\ssi}{A\ssi}$ and $\wellformed{\P'}{A\ssi\subtype B\ssi}$
    
    So $\etyperelation{\P'}{\G\ssi}{v\ssi}{B\ssi}$
    \case{Return}
    \bi, $\gpetyperelation{v}{A}$
    
    So by induction, $\etyperelation{\P'}{\G\ssi}{v\ssi}{A\ssi}$
    
    So $\etyperelation{\P'}{\G\ssi}{\return{(v\ssi)}}{\M{\1}{(A\ssi)}}$
    
    Hence $\etyperelation{\P'}{\G\ssi}{(\return{v})\ssi}{(\moa)\ssi}$
    \case{Bind}
    
    \bi, $\gpetyperelation{v_1}{\M{\e_1}{A}}$ and $\etyperelation{\P}{\gax}{v_2}{\M{\e_2}{B}}$.
    
    So by induction: $\etyperelation{\P'}{\G\ssi}{v_1\ssi}{\M{\si(\e_1)}{(A\ssi)}}$, and $\etyperelation{\P'}{\G\ssi, x:A\ssi}{v_2}{\M{\si(\e_2)}{(B\ssi)}}$.
    
    And so $\etyperelation{\P'}{\G\ssi}{\doin{x}{(v_1\ssi)}{(v_2\ssi)}}{\M{\si(\e_1)\dot(\e_2\ssi)}{B\ssi}}$
    
    
    \case{If}
    
    \bi, $\gpetyperelation{v}{\B}$, $\gpetyperelation{v_1}{A}$, and $\gpetyperelation{v_2}{A}$
    
    So by induction $\etyperelation{\P'}{\G\ssi}{v\ssi}{\B}$, $\etyperelation{\P'}{\G\ssi}{v_1}{A\ssi}$, and $\etyperelation{\P'}{\G\ssi}{v\ssi}{\B}$, $\etyperelation{\P'}{\G\ssi}{v_2}{A\ssi}$.
    (Since $\B\ssi = \B$)
    
    Hence:
    
    $\etyperelation{\P'}{\G\ssi}{\pifthenelse{A\ssi}{v\ssi}{v_1\ssi}{v_2\ssi}}{A\ssi}$
    
    So $\etyperelation{\P'}{\G\ssi}{(\pifthenelse{A}{v}{v_1}{v_2})\ssi}{A\ssi}$
    
    \case{Effect-Gen}
    \bi, $\etyperelation{\P,\a}{\G}{v}{A}$.
    
    So by the substitution property \ref{EffectSubsPropertyThree}, pick $\a\notin\P'\wedge\a\notin\P$ so we have:
    
    $$\typerelation{(\P',\a)}{(\si, \a\setto\a)}{(\P,\a)}$$
    
    So by induction, $\etyperelation{\P', \a}{\G\sub{\si, \a\setto\a}}{v\sub{\si,\a\setto\a}}{A\sub{\si, \a\setto\a}}$
    
    So $\etyperelation{\P',\a}{\G\ssi}{v\ssi}{A\ssi}$ since $\a\notin\P' \wedge\a\notin\P$.
    
    So $\etyperelation{\P'}{\G\ssi}{v\ssi}{(\all{\a}{A})\ssi}$
    
    \case{Effect-Spec}
    \bi, $\gpetyperelation{v}{\all{\a}{A}}$, $\wellformed{\P}{\e}$.
    
    So by induction and effect-environment substitution preserving wellformedness of effects: $\etyperelation{\P'}{\G\ssi}{v\ssi}{(\all{\a}{A})\ssi}$ and $\wellformed{\P'}{\si(\e)}$
    
    So $\etyperelation{\P'}{\G\ssi}{\eapp{(v\ssi)}{(\si(\e))}}{A\ssi\ssub{\a}{\si(\e)}}$.
    
    Since $\a\#\si$, we can commute the applications of substitution.
    
    So, $\etyperelation{\P'}{\G\ssi}{(\eapp{v}{\e})\ssi}{A\ssub{\a}{\e}\ssi}$
    
    \end{proof}
\end{framed}

\begin{theorem}[Effect Substitution and Term Denotations]
    
If 
\begin{align*}
    \si &= \deno{\typerelation{\P'}{\si}{\P}}\\
    \D &= \deno{\gpetyperelation{v}{A}}\\
    \D' &= \deno{\etyperelation{\P'}{\G\ssi}{v\ssi}{A\ssi}}\\
\end{align*}

Then \begin{eqnarray}
    \D' = \si\star(\D)
\end{eqnarray}
\end{theorem}


\begin{framed}
    \begin{proof}
        
    By induction over the derivation of $\D$. Using the S-Closure of $\si\star$. We use $\G_I$ to indicate $\deno{\wellformedok{\P}{\G}}$, an $A_I$ to indicate $\deno{\typerelation{\P}{A}{\type}}$
    
    \case{Unit}
    
    \begin{equation}
        \D = \term{\G_I}
    \end{equation}
    
    So
    
    \begin{equation}
        \si\star(\D) = \term{\G_I\ssi} = \D'
    \end{equation}
    
    \case{True, False}
    Giving the case for true as false is the same but using $\inr$
    \begin{equation}
        \D = \inl\after\term{\G_I}
    \end{equation}
    
    So
    
    \begin{equation}
        \si\star(\D) = \inl\after \term{\G_I\ssi} = \D'
    \end{equation}
    
    Since $\si\star$ is S-preserving.
    
    \case{Constant}
    
    
    \begin{equation}
        \D = \deno{\const{A}}\after\term{\G_I}
    \end{equation}
    
    So
    
    \begin{equation}
        \si\star(\D) = \si\star\deno{\const{A}}\after \term{\G_I\ssi}=\deno{\const{A\ssi}}\after \term{\G_I\ssi}  = \D'
    \end{equation}
    
    Since $\si\star$ is S-preserving.
    
    \case{Subtype}
    
    Let \begin{equation}
        \D_1 = \deno{\gpetyperelation{v}{A}}
    \end{equation}
    
    Then
    
    \begin{equation}
        \D = \deno{A\subtypep B}\after \D_1\\
    \end{equation}
    
    So 
    \begin{align*}
        \si\star(\D) & = \si\star{\deno{A\subtypep B}}\after\si\star\D_1 \\
        & = \deno{A\ssi\subtypepp B\ssi}\after\D_1'\qt{By induction}\\
        & = D'
    \end{align*}
    
    \case{Fn}
    Let \begin{equation}
        \D_1 = \deno{\etyperelation{\P}{\gax}{v}{B}}
    \end{equation}
    
    Then
    
    \begin{equation}
        \D = \cur(\D_1)\\
    \end{equation}
    
    So
    \begin{align*}
        \si\star(\D) & = \si\star(\cur{\D_1})\\
        & = \cur{\si\star(\D_1)}\qt{By S-preservation}\\
        & = \cur{\D_1'}\qt{By induction}\\
        & = \D'
    \end{align*}
    
    \case{Apply}
    Let \begin{align*}
        \D_1 &= \deno{\gpetyperelation{v_1}{\ab}}\\
        \D_2 &= \deno{\gpetyperelation{v_2}{A}}
    \end{align*}
    
    Then
    
    \begin{equation}
        \D = \app\after\pr{\D_1}{\D_2}\\
    \end{equation}
    
    So
    
    \begin{align*}
        \si\star\D & = \si\star(\app\after\pr{\D_1}{\D_2})\\
        & = \app\after\pr{\si\star(\D_1)}{\si\star(\D_2)}\qt{By S-preservation}\\
        & = \app\after\pr{\D_1'}{\D_2'}\qt{By Induction}\\
        & = \D'
    \end{align*}
    
    \case{Return}
    Let \begin{equation}
        \D_1 = \deno{\gpetyperelation{v}{A}}
    \end{equation}
    
    Then
    
    \begin{equation}
        \D = \point{A_I}\after \D_1\\
    \end{equation}
    
    So
    
    \begin{align*}
        \si\star(\D) &= \si\star(\point{A_I}\after \D_1)\\
                & = \point{A_{I'}} \after\si\star(\D_1)\qt{By S-preservation}\\
                & = \point{A_{I'}} \after\D_1'\\
                & = \D'
    \end{align*}
    
    \case{Bind}
    Let \begin{align*}
        \D_1 &= \deno{\gpetyperelation{v_1}{\M{\e_1}{A}}}\\
        \D_2 &= \deno{\etyperelation{\P}{\gax}{v_2}{\M{\e_2}{B}}}
    \end{align*}
    
    Then
    
    \begin{equation}
        \D = \M{\e_1}{\e_2}{A_I}\after\T{\e_1}{\D_2}\after\tstrength{\e_1}{\G_I}{A_I}\after\pr{\Id{\G_{I}}}{\D_1}\\
    \end{equation}
    
    So
    
    \begin{align*}
        \si\star(\D) &= \si\star(\bind{\e_1}{\e_2}{A}\after\T{\e_1}{\D_2}\after\tstrength{\e_1}{\G}{A}\after\pr{\Id{\G}}{\D_1})\\
        & = \si\star(\bind{\e_1}{\e_2}{A})\after\si\star(\T{\e_1}{\D_2})\after\si\star(\tstrength{\e_1}{\G}{A})\after\pr{\si\star(\Id{\G_{I})}}{\si\star(\D_1)}\qt{By S-Closure}\\
        &= \bind{\si(\e_1)}{\si(\e_2)}{A\ssi'}\after\T{\si(\e_1)}{\si\star(\D_2)}\after\tstrength{\si(\e_1)}{\G\ssi}{A\ssi}\after\pr{\si\star(\Id{\G_{I})}}{\si\star(\D_1)}\qt{By S-Closure}\\
        &= \bind{\si(\e_1)}{\si(\e_2)}{A\ssi'}\after\T{\si(\e_1)}{\D_2'}\after\tstrength{\si(\e_1)}{\G\ssi}{A\ssi}\after\pr{\si\star(\Id{\G_{I})}}{\D_1'}\qt{By Induction}\\
        &= \D'\\
    \end{align*}
    
    \case{If}
    
    Let \begin{align*}
        \D_1 &= \deno{\gpetyperelation{v}{\B}}\\
        \D_2 &= \deno{\gpetyperelation{v_1}{A}}\\
        \D_3 &= \deno{\gpetyperelation{v_2}{A}}\\
    \end{align*}
    
    Then
    
    \begin{equation}
        \D = \app\after(([\cur{\D_2\after\pp}, \cur{\D_3\after\pp}]\after\D_1)\times\Id{\G})\after\diag{\G}\\
    \end{equation}
    
    So
    
    \begin{align*}
        \si\star(\D) &= \si\star(\app\after(([\cur{\D_2\after\pp}, \cur{\D_3\after\pp}]\after\D_1)\times\Id{\G})\after\diag{\G})\\
        &= \app\after(([\cur{\si\star(\D_2)\after\pp}, \cur{\si\star(\D_3)\after\pp}]\after\si\star(\D_1))\times\Id{\G\ssi})\after\diag{\G\ssi}\qt{By S-Closure}\\
        &= \app\after(([\cur{\D_2'\after\pp}, \cur{\D_3'\after\pp}]\after\D_1')\times\Id{\G\ssi})\after\diag{\G\ssi}\qt{By Induction}\\
        & = \D'\\
    \end{align*}
    
    
    \case{Effect-Gen}
    
    Let \begin{equation}
        \D_1 = \deno{\etyperelation{\P,\a}{\G}{v}{A}}
    \end{equation}
    
    Then
    
    \begin{equation}
        \D = \widehat{\D_1}\\
    \end{equation}
    
    And also
    
    \begin{equation}
        \si\times\Id{} = \deno{\typerelation{(\P',\a)}{(\si, \a\setto\e)}{(\P,\a)}}
    \end{equation}
    
    So
    \begin{align*}
        \si\star\D &= \si\star(\widehat{\D_1})\\
        & = \widehat{(\si\times\Id{U})\star\D_1}\qt{By naturality}\\
        & = \widehat{\D_1'}\qt{By induction}\\
        & = \D'
    \end{align*}
    
    \case{Effect-Spec}
    
    Let \begin{align*}
        \D_1 &= \deno{\gpetyperelation{v}{\all{\a}{A}}}\\
        h &= \deno{\typerelation{\P}{\e}{\effect}}\\
    \end{align*}
    
    Then
    
    \begin{equation}
        \D = \pr{\Id{\G}}{h}\star(\e_{\deno{\typerelation{\P,\b}{A\ssub{\a}{\b}}{\type}}})\after\D_1\\
    \end{equation}
    
    So
    Due to the substitution theorem on effects
    \begin{equation}
        h\after\si = \deno{\typerelation{\P}{\e}{\effect}}\after\si = \deno{\typerelation{\P'}{\si(\e)}{\effect}} = h'
    \end{equation}
    
    \begin{align*}
        \si\star{\D} & = \si\star(\pr{\Id{\G}}{h}\star(\e_{\deno{\typerelation{\P,\b}{A\ssub{\a}{\b}}{\type}}})\after\D_1)\\
        & = (\pr{\Id{\G}}{h}\after\si)\star(\e_{\deno{\typerelation{\P,\b}{A\ssub{\a}{\b}}{\type}}})\after\si\star(\D_1)\\
        & = ((\si\times\Id{U})\after\pr{\Id{\G}}{h\after\si})\star(\e_{\deno{\typerelation{\P,\b}{A\ssub{\a}{\b}}{\type}}})\after\D_1')\\
        & = (\pr{\Id{\G}}{h'})\star((\si\times\Id{U})\star\e_{\deno{\typerelation{\P,\b}{A\ssub{\a}{\b}}{\type}}})\after\D_1')\\
    \end{align*}
    
    Looking at the inner part of the functor application:
    Let \begin{align*}
        A & = \deno{\typerelation{\P,\b}{A\ssub{\a}{\b}}{\type}}\\
    \end{align*}
    \begin{align*}
        (\si\times\Id{U})\star\e_{\deno{\typerelation{\P,\b}{A\ssub{\a}{\b}}{\type}}} &= (\si\times\Id{U})\star\e_{A}\\
        & = (\si\times\Id{U})\star(\widehat{\Id{\allI(A)}})\\
        & = \widehat{\bar{(\si\times\Id{U})\star(\widehat{\Id{\allI(A)}})}}\qt{By bijection}\\
        & = \widehat{\si\star(\bar{\widehat{\Id{\allI(A)}}})}\qt{By naturality}\\
        & = \widehat{\si\star(\Id{\allI(A)})}\qt{By bijection}\\
        & = \widehat{\Id{\allII(A\after(\si\times\Id{U}))}}\qt{By S-Closure, naturality}\\
        & = \widehat{\Id{\allII(A\sub{\si,\a\setto\a})}}\qt{By Substitution theorem}\\
        & = \e_{A\ssi}
    \end{align*}
    
    Going back to the original expression:
    
    \begin{align*}
        \si\star{\D} & = (\pr{\Id{\G}}{h'})\star(\e_{A\ssi})\after\D_1')\\
        & = \D'\\
    \end{align*}
    
    \end{proof}
    
\end{framed}

\chapter{Effect Weakening Theorem}
In this section, we state and prove a theorem that the action of a simultaneous effect-weakening upon a structure in the language has a consistent effect upon the denotation of the language. More formally, for the denotation morphism $\D$ of some relation, the denotation of the weakened relation, $\D' = \w\star(\D)$.
   
\section{Effects}

\begin{theorem}[Effect Weakening Preserves Effect Wellformedness]
    If $\wrelw{\P'}{\P}$ then $\wellformed{\P}{\e}\implies\wellformed{\P'}{\e}$ 
\end{theorem}


\begin{framed}
        \begin{proof}
            
            By induction over the wellformedness of effects
            \case{Ground}
            \bi, $\ok{\P}\wedge\e\in E$.
            Hence by the ok-property, $\ok{\P'}$
            So $\wellformed{\P'}{\e}$
        
            \case{Var}
        
            $\P = \P'', \a$
        
            So either:
        
            \subcase{$\P' = \P''', \a$}
                So $\w = \w'\x$
                So $\wrel{\w'}{\P'''}{\P''}$, and hence:
                \begin{equation}
                    \ntreeruleI{Var}{\ok{\P''', \a}}{\wellformed{\P''',\a}{\a}}
                \end{equation}
            \subcase{$\P' = \P''', \b$} and $\b \neq\a$
        
            So $\w = \w'\pi$
        
            By induction, $\wrel{\w'}{\P'''}{\P}$
            so \begin{equation}
                \ntreeruleI{Weaken}{\wellformed{\P'''}{\a}}{\wellformed{\P'}{\a}}
            \end{equation}
        
            \case{Weaken}
        
            \bi, $\P = \P'', \b$.
        
            So $\w = \w'\x$
        
            And, $\P' = \P''', \b$
            So By inversion,  $\wrel{\w'}{\P'''}{\p''}$
        
            So by induction
            \begin{equation}
                \ntreeruleI{weak}{\wellformed{\P'''}{\a}}{\wellformed{\P'}{\a}}
            \end{equation}
        
        
            \case{Effect}
        
            \bi, $\wellformed{\P}{\e_1}$ and $\wellformed{\P}{\e_2}$. So by induction,
            $\wellformed{\P'}{\e_1}$ and $\wellformed{\P'}{\e_2}$, and so:
        
            \begin{equation}
                \wellformed{\P'}{\e_1\dot\e_2}
            \end{equation}
        \end{proof}
    
\end{framed}

\begin{theorem}[Effect-Environment Weakening and Effect Denotations]
    If $\w = \deno{\wrelw{\P'}{\P}}$ then $\typerelation{\P'}{\e}{\effect} = \w\star\deno{\typerelation{\P}{\e}{\effect}} = \deno{\typerelation{\P}{\e}{\effect}}\after\w$
    
\end{theorem}

\begin{framed}
    \begin{proof}
        By induction on the derivation on $\deno{\typerelation{\P}{\e}{\effect}}$
        
        \case{Ground}
        \begin{align*}
            \deno{\typerelation{\P}{e}{\effect}}\after\w & = \deno{e}\after\term{I}\after\w \\
            & = \deno{e}\after\term{I'} \\
            & = \deno{\typerelation{\P'}{e}{\type}}\\
        \end{align*}
        
        \case{Var}
        Case split on $\w$.
        \subcase{$\w = \i$}
        Then $\P' = \P$ and $\w = \Id{I}$. So the theorem holds trivially.
        \subcase{$\w = \w'\x$}
        Then
        
        \begin{align*}
            \deno{\typerelation{\P,\a}{\a}{\effect}}\after\w &= \pp\after(\w'\times \Id{U}) \\
            & = \pp\\
            & = \deno{\typerelation{\P',\a}{\a}{\effect}}
        \end{align*}
        
        \subcase{$\w = \w'\p$}
        Then \begin{equation}
            \deno{\typerelation{\P,\a}{\a}{\effect}} = \pp\after\w'\after\p
        \end{equation}
        
        Where $\P' = \P,\b$ and $\wrel{\w'}{\P''}{\P}$.
        
        So\begin{align*}
            \pp\after\w' & = \deno{\typerelation{\P''}{\a}{\effect}}
            \\
            \pp\after\w'\after\p & =\deno{\typerelation{\P'',\b}{\a}{\effect}}
            &= \deno{\typerelation{\P'}{\a}{\effect}}
        \end{align*}
        
        \case{Weaken}
        \begin{equation}
            \deno{\typerelation{\P,\b}{\a}{\effect}}\after\w = \deno{\typerelation{\P}{\a}{\effect}}\after\p\after\w
        \end{equation}
        
        Case split of structure of $w$
        
        \subcase{$\w=\i$}
        Then $\P' = \P,\b$ so $\w=\Id{I}$
        So $\deno{\typerelation{\P,\b}{\a}{\effect}}\after\w = \deno{\typerelation{\P'}{\a}{\effect}}$
        
        \subcase{$\w=\w'\p$}
        Then $\P' = \P'',\g$ and $\w=\w'\after\p$
        Where $\wrel{\w'}{\P''}{\P,\b}$.
        So
        \begin{align*}
            \deno{\typerelation{\P,\b}{\a}{\effect}}\after\w & = \deno{\typerelation{\P,\b}{\a}{\effect}}\after\w'\after\p\\
            & = {\typerelation{\P''}{\a}{\effect}}\after\p\\
            & = {\typerelation{\P'',\g}{\a}{\effect}}\\
            & = {\typerelation{\P'}{\a}{\effect}}\\
        \end{align*}
        
        \subcase{$\w=\w'\x$}
        Then $\P'=\P'',\b$ and $\wrel{\w'}{\P'}{\P}$
        
        So \begin{align*}
            \deno{\typerelation{\P,\b}{\a}{\effect}}\after\w &= \deno{\typerelation{\P}{\a}{\effect}}\after\p\after(\w'\times\Id{U})\\
            &=\deno{\typerelation{\P}{\a}{\effect}}\after\w'\after\p\\
            & = \deno{\typerelation{\P''}{\a}{\effect}}\after\p\\
            & = \deno{\typerelation{\P'}{\a}{\effect}}\\
        \end{align*}
        
        \case{Compose}
        \begin{align*}
            \deno{\typerelation{\P}{\e_1\dot\e_2}{\type}} \after\w &=
            \Mul_I(\deno{\typerelation{\P}{\e_1}{\effect}}, \deno{\typerelation{\P}{\e_2}{\effect}})\after \w \\
            & = \Mul_{I'}(\deno{\typerelation{\P}{\e_1}{\effect}}\after \w, \deno{\typerelation{\P}{\e_2}{\effect}}\after \w)\qt{By Naturality}\\
            & = \Mul_{I'}(\deno{\typerelation{\P'}{\e_1}{\effect}}, \deno{\typerelation{\P}{\e_2}{\effect}})\\
            & = \deno{\typerelation{\P}{\e_1\dot\e_2}{\effect}}
        \end{align*}
    \end{proof}
    
\end{framed}
\section{Types}

\begin{theorem}[Effect-Environment Weakening Preserves Type Wellformedness]
        If $\wrelw{\P'}{\P}$ and $\wellformed{\P}{A}$ then $\wellformed{\P'}{A}$.
    
\end{theorem}


\begin{framed}
    \begin{proof}
            \case{Ground}
            \bi, $\ok{\P}$, hence by property 1 of weakening, $\ok{\P'}$. Hence $\wellformed{\P'}{\g}$.
            \case{Fn}
            \bi, $\wellformed{\P}{A}$, $\wellformed{\P}{B}$.
            So by induction $\wellformed{\P'}{A}$, $\wellformed{\P'}{B}$, hence,
        
            $$\wellformed{\P'}{\ab}$$
        
            \case{Effect}
        
            By inversion,  $\wellformed{\P}{A}$, and $\wellformed{\P}{\e}$.
        
            So by induction and the effect-wellformedness theorem, 
        
            $\wellformed{\P'}{A}$ and $\wellformed{\P'}{\e}$
        
            So $$\wellformed{\P'}{\mea}$$
        
            \case{Quantification}
            \bi, $\wellformed{\P,\a}{A}$
            Picking $\a\notin\P'$ using $\a$-conversion.
        
            So $\wrel{\w\x}{(\P', \a)}{(\P, \a)}$
        
            So $\wellformed{(\P', \a)}{A}$
        
            So $\wellformed{\P}{\all{\a}{A}}$
        
        
    \end{proof}
        
\end{framed}

\begin{theorem}[Effect-Environment Weakening and Type Denotations]
    If $\w = \deno{\typerelation{\P'}{\w}{\P}}$ then $\deno{\typerelation{\P'}{A}{\type}} = \w\star\deno{\typerelation{\P}{A}{\type}}$.
\end{theorem}

\begin{framed}
    \begin{proof}
            By induction on the derivation on $\deno{\typerelation{\P}{A}{\type}}$. Making use of the S-Closure of the re-indexing functor.
            
            \case{Ground}
            \begin{align*}
                \w\star\deno{\typerelation{\P}{\g}{\type}} &= \w\star\deno{\g}\\
                & =  \deno{\g}\qt{By S-Closure}\\
                & = \deno{\typerelation{\P'}{\g}{\type}}
            \end{align*}
            
            \case{Effect}
            \begin{align*}
                \w\star\deno{\typerelation{\P}{\mea}{\type}}\after\w & =  \w\star(\T{\deno{\typerelation{\P}{\e}{\effect}}}{\deno{\typerelation{\P}{A}{\type}}})\\
                &= \T{\w\star(\deno{\typerelation{\P}{\e}{\effect}})}{\w\star(\deno{\typerelation{\P}{A}{\type}})}\\
                & = \deno{\typerelation{\P'}{\mea}{\type}}
            \end{align*}
            \case{Quantification}
                \begin{align*}
                    \w\star\deno{\typerelation{\P}{\all{\a}A}{\type}} & = \w\star(\allI(\deno{\typerelation{\P,\a}{A}{\type}}))\\
                    & = \allI((\w\times\Id{U})\star\deno{\typerelation{\P,\a}{A}{\type}})\qt{By Beck-Chevalley}\\
                    & = \allI(\deno{\typerelation{\P',\a}{A}{\type}})\\
                    & = \allI(\deno{\typerelation{\P',\a}{A}{\type}})\\
                    & = \deno{\typerelation{\P'}{\all{\a}{A}}{\type}}\\
                \end{align*}
    \end{proof}
        
\end{framed}
    \case{Fn}
    \begin{align*}
        \w\star\deno{\typerelation{\P}{\ab}{\type}} &= \w\star(\deno{\typerelation{\P}{B}{\type}}^{\deno{\typerelation{\P}{A}{\type}}})\\
        &=\w\star(\deno{\typerelation{\P}{B}{\type}})^{\w\star(\deno{\typerelation{\P}{A}{\type}})}\\
        & = \deno{\typerelation{\P'}{B}{\type}}^{\deno{\typerelation{\P'}{A}{\type}}}\\
        & = \deno{\typerelation{\P'}{\ab}{\type}}\\
    \end{align*}

\section{Subtyping}

\begin{theorem}[Effect-Environment Weakening Preserves Subtyping Relations]
    If $\wrelw{\P'}{\P}$ and $A\subtypep B$ then $A\subtypepp B$.
\end{theorem}


\begin{framed}
    \begin{proof}
        \case{Ground}
            By inversion, $A\subtypeg B$, so $A\subtypepp B$
    
        \case{Fn}
            By inversion, $A'\subtypep A$ and $B\subtypep B'$, so by induction $A'\subtypepp A$ and $B\subtypepp B'$, so $\fntype{A}{B}\subtypepp \fntype{A'}{B'}$.
    
        \case{Quantification}
            By inversion, $A\subtypepa A'$. So since $\wrel{\w\x}{\P',\a}{\P,\a}$, by induction, $A\subtypeppa A'$, so $\all{\a}{A}\subtypepp \all{\a}{A'}$.   
    
        \case{Effect}
            By inversion, $A \subtype A'$ and $e\subeffectp e'$. By induction $A\subtypepp A'$ and by the weakening-subeffecting theorem, $e\subeffectpp e'$, so $\mea\subtypepp \M{\e'}{A'}$
    \end{proof}
\end{framed}

\begin{theorem}[Effect-Environment Weakening and Subtype Denotations]
    If $\w = \deno{\wrelw{\P'}{\P}}$ then $\deno{A\subtypepp B} = \w\star\deno{A\subtypep B}: \C(I')(A, B)$.
\end{theorem}


\begin{framed}
    
    \begin{proof}
        
        By induction on the derivation on $\deno{A\subtypep B}$. Using S-preserving property of $\w\star$ 
        
        \case{Ground}
        \begin{align*}
            \w\star(\g_1\subtypeg\g_2) &= (\g_1\subtypeg\g_2)
        \end{align*}
        
        Since $\w\star$ is s-closed.
        
        \case{Effect}
        \begin{align*}
            \w\star\deno{\M{\e_1}{A} \subtypep \M{\e_2}{B}} & = \w\star(\deno{\e_1\subeffectp\e_2})\after\w\star(\T{\e_1}(\deno{A\subtypep B})) \\ 
             &= \deno{\e_1\subeffectpp\e_2} \after \T{\e_1}{\deno{A\subtypepp B}}\qt{By S-Closure}\\
             &= \deno{\M{\e_1}{A}\subtypepp\M{\e_2}{B}}\\
             &= \deno{(\M{\e_1}{A})\subtypepp\M{\e_2}{B}}\\
        \end{align*}
        
        \case{Quantification}
        Note $\deno{\wrel{\w\x}{\P',\a}{\P,\a}} = (\w\times\Id{U})$
            \begin{align*}
                \w\star\deno{\all{\a}{A}\subtypep\all{\a}{B}} &= \w\star(\allI(\deno{A\subtypepa B})) \\
                &=\allII((\w\times\Id{U})\star(\deno{A\subtypepa B}))\\
                &=\allII(\deno{A\subtypeppa B})\\
                &= \deno{(\all{\a}{A}) \subtypepp(\all{\a}{B})}\\
            \end{align*}
        
        \case{Fn}
        \begin{align*}
            \w\star\deno{(\ab)\subtypep\fntype{A'}{B'}} &= \w\star(\deno{B\subtypep B'}^{A'}\after B^{\deno{A'\subtypep A}})\\
            &= \w\star(\cur{\deno{B\subtypep B'}\after\app})\after\w\star(\cur{\app\after(\Id{B}\times\deno{A'\subtypep A})})\\
            & = \cur{\w\star(\deno{B\subtypep B'})\after\app}\after\cur{\app\after(\Id{B}\times\w\star(\deno{A'\subtypep A}))}\\
            & = \cur{\deno{B\subtypepp B'}\after\app}\after\cur{\app\after(\Id{B}\times\deno{A'\subtypepp A})}\\
            &= \deno{(\ab)\subtypepp(\fntype{A'}{B'})}
        \end{align*}
    \end{proof}
    
\end{framed}

\section{Term Environments}
 
\begin{theorem}[Effect-Environment Weakening Preserves Term-Environment Wellformedness]
If $\wrel{\w}{\P'}{\P}$ and $\wellformedok{\P}{\G}$  then $\wellformedok{\P'}{\G}$.
\end{theorem}

\begin{framed}
     \begin{proof}
            \case{Nil}
            By inversion,  $\ok{\P}$ so $\wellformedok{\P}{\nil}$
            \case{Extend}
        
            By inversion,  $\wellformedok{\P}{\G}$, $x\in\dom{\G}$, $\wellformed{\P}{A}$
        
            So by induction $\wellformedok{\P'}{\G}$, and $\wellformedok{\p'}{\G}$
        
            So $\wellformedok{\P'}{(\gax)}$
     \end{proof}
        
    
\end{framed}
\begin{theorem}[Effect-Environment Weakening and Term-Environment Denotations]
      
    If $\w = \deno{\typerelation{\P'}{\w}{\P}}$ then $\deno{\wellformedok{\P'}{\G}} = \w\star\deno{\wellformedok{\P}{\G}}\in\obj\C(I')$.
\end{theorem}

\begin{framed}
    \begin{proof}
        By induction on the derivation on $\deno{\wellformedok{\P}{\G}}$. Using the S-Closure of the re-indexing functor.
        
        \case{Nil}
        \begin{align*}
            \w\star\deno{\wellformedok{\P}{\nil}} &=\w\star\1\\
            & = \1\qt{By S-preservation}\\
            &= \deno{\wellformedok{\P'}{\nil}}\\
        \end{align*}
        
        \case{Extend}
        \begin{align*}
           \w\star\deno{\wellformedok{\P}{\gax}} &= \w\star(\deno{\wellformedok{\P}{\G}} \times \deno{\typerelation{\P}{A}{\type}}) \\
           & = (\w\star\deno{\wellformedok{\P}{\G}} \times \w\star\deno{\typerelation{\P}{A}{\type}})\\
            & = (\deno{\wellformedok{\P'}{\G}} \times \deno{\typerelation{\P'}{A}{\type}})\\
            & = \deno{\wellformedok{\P'}{\G, x: A}}\\
        \end{align*}
        
    \end{proof}
\end{framed}
\section{Terms}


\begin{theorem}[Effect-Environment Weakening Preserves Typing Relation]
    If $\gpetyperelation{v}{A}$ and $\wrelw{\P'}{\P}$ then $\etyperelation{\P'}{\G}{v}{A}$
\end{theorem}



\begin{framed}
    \begin{proof}
            \case{Constants}
            If $\wellformedok{\P}{\G}$ then $\wellformedok{\P'}{\G}$ so:
        
            \begin{equation}
                \ntreeruleI{Const}{\wellformedok{\P'}{\G}}{\etyperelation{\P'}{\G}{\const{A}}{A}}
            \end{equation}
        
            \case{Variables}
            If $\wellformedok{\P}{\G}$ then $\wellformedok{\P'}{\G}$ so:
            So, $\etyperelation{\P'}{G}{x}{A}$, if $\etyperelation{\P}{G}{x}{A}$
            \case{Fn}
            \bi, $\etyperelation{\P}{\gax}{v}{B}$, so by induction $\etyperelation{\P'}{\gax}{v}{B}$.
        
            So,
        
            \begin{equation}
                \etyperelation{\P'}{\G}{\lam{x}{A}{v}}{\ab}
            \end{equation}
        
            \case{Apply}
            By inversion,  $\gpetyperelation{v_1}{\ab}$ and $\gpetyperelation{v_2}{A}$.
        
            Hence by induction,
            $\gppetyperelation{v_1}{\ab}$ and $\gppetyperelation{v_2}{A}$.
        
            So $$\gppetyperelation{\app{v_1}{v_2}}{B}$$
        
            \case{Return}
        
            By inversion,  $\gpetyperelation{v}{A}$ 
            
            So by induction $\gppetyperelation{v}{A}$
        
            Hence $\gppetyperelation{\return{v}}{\moa}$
        
            \case{Bind}
        
            By inversion,  $\gpetyperelation{v_1}{\M{\e_1}{A}}$ and $\etyperelation{\P}{\gax}{\e_2}{\M{\e_2}{A}}$.
        
            Hence by induction $\gppetyperelation{v_1}{\M{\e_1}{A}}$ and $\etyperelation{\P'}{\gax}{v_2}{\M{\e_2}{A}}$.
        
            So
        
            \begin{equation}
                \gppetyperelation{\doin{x}{v_1}{v_2}}{\M{\e_1\dot\e_2}{B}}
            \end{equation}
        
            \case{If}
        
            By inversion,  $\gpetyperelation{v}{\B}$, $\gpetyperelation{v_1}{A}$, and $\gpetyperelation{v_2}{A}$.
        
            Hence by induction
            $\gppetyperelation{v}{\B}$, $\gppetyperelation{v_1}{A}$, and $\gppetyperelation{v_2}{A}$.
        
            So 
        
            \begin{equation}
                \gppetyperelation{\pifthenelse{A}{v}{v_1}{v_2}}{A}
            \end{equation}
        
            \case{Subtype}
            By inversion,  $\gpetyperelation{v}{A}$, and $A\subtype B$.
        
            So by induction:
            $\gppetyperelation{v}{A}$, and $A\subtype B$.
        
            So \begin{equation}
                \gppetyperelation{v}{B}
            \end{equation}
        
            \case{Effect-Gen}
            By inversion,  $\etyperelation{\P, \a}{\G}{v}{A}$
        
            By picking $\a\notin\P'$ using $\a$-conversion.
        
            \begin{equation}
                \wrel{\w\x}{\P',\a}{\P,\a}
            \end{equation}
        
            So by induction, $\etyperelation{\P', \a}{\G}{v}{A}$
        
            Hence,
        
            \begin{equation}
                \gppetyperelation{\elam{\a}{v}}{\all{a}{A}}
            \end{equation}
        
            \case{Effect-Spec}
        
            \bi, $\gpetyperelation{v}{\all{\a}{A}}$, and $\wellformed{\P}{\e}$.
        
            So by induction, $\gppetyperelation{v}{\all{\a}{A}}$
        
            And by the wellformedness-theorem $\wellformed{\P'}{\e}$
        
            Hence, \begin{equation}
                \gppetyperelation{\eapp{v}{\e}}{A\ssub{\a}{\e}}
            \end{equation}
        
    \end{proof}
    
    
\end{framed}
\begin{theorem}[Effect Environment Weakening]
    If 
    \begin{align*}
        \w &= \deno{\wrelw{\P'}{\P}}\\
        \D &= \deno{\gpetyperelation{v}{A}}\\
        \D' &= \deno{\etyperelation{\P'}{\G}{v}{A}}\\
    \end{align*}
    
    Then \begin{eqnarray}
        \D' = \w\star(\D)
    \end{eqnarray}
\end{theorem}



\begin{framed}
    \begin{proof}
        By induction over the derivation of $\D$. Using the S-Closure of $\w\star$. We use $\G_I$ to indicate $\deno{\wellformedok{\P}{\G}}$, an $A_I$ to indicate $\deno{\typerelation{\P}{A}{\type}}$
        
        \case{Unit}
        
        \begin{equation}
            \D = \term{\G_I}
        \end{equation}
        
        So
        
        \begin{equation}
            \w\star(\D) = \term{\G_{I'}} = \D'
        \end{equation}
        
        \case{True, False}
        Giving the case for true as false is the same but using $\inr$
        \begin{equation}
            \D = \inl\after\term{\G_I}
        \end{equation}
        
        So
        
        \begin{equation}
            \w\star(\D) = \inl\after \term{\G_{I'}} = \D'
        \end{equation}
        
        Since $\w\star$ is S-preserving.
        
        \case{Constant}
        
        
        \begin{equation}
            \D = \deno{\const{A}}\after\term{\G_I}
        \end{equation}
        
        So
        
        \begin{equation}
            \w\star(\D) = \w\star\deno{\const{A}}\after \term{\G_{I'}}=\deno{\const{A_{I'}}}\after \term{\G_{I'}}  = \D'
        \end{equation}
        
        Since $\w\star$ is S-preserving.
        
        \case{Subtype}
        
        Let \begin{equation}
            \D_1 = \deno{\gpetyperelation{v}{A}}
        \end{equation}
        
        Then
        
        \begin{equation}
            \D = \deno{A\subtypep B}\after \D_1\\
        \end{equation}
        
        So 
        \begin{align*}
            \w\star(\D) & = \w\star{\deno{A\subtypep B}}\after\w\star\D_1 \\
            & = \deno{A_{I'}\subtypepp B_{I'}}\after\D_1'\qt{By induction}\\
            & = D'
        \end{align*}
        
        \case{Fn}
        Let \begin{equation}
            \D_1 = \deno{\etyperelation{\P}{\gax}{v}{B}}
        \end{equation}
        
        Then
        
        \begin{equation}
            \D = \cur{\D_1}\\
        \end{equation}
        
        So
        \begin{align*}
            \w\star(\D) & = \w\star(\cur{\D_1})\\
            & = \cur{\w\star(\D_1)}\qt{By S-preservation}\\
            & = \cur{\D_1'}\qt{By induction}\\
            & = \D'
        \end{align*}
        
        \case{Apply}
        Let \begin{align*}
            \D_1 &= \deno{\gpetyperelation{v_1}{\ab}}\\
            \D_2 &= \deno{\gpetyperelation{v_2}{A}}
        \end{align*}
        
        Then
        
        \begin{equation}
            \D = \app\after\pr{\D_1}{\D_2}\\
        \end{equation}
        
        So
        
        \begin{align*}
            \w\star\D & = \w\star(\app\after\pr{\D_1}{\D_2})\\
            & = \app\after\pr{\w\star(\D_1)}{\w\star(\D_2)}\qt{By S-preservation}\\
            & = \app\after\pr{\D_1'}{\D_2'}\qt{By Induction}\\
            & = \D'
        \end{align*}
        
        \case{Return}
        Let \begin{equation}
            \D_1 = \deno{\gpetyperelation{v}{A}}
        \end{equation}
        
        Then
        
        \begin{equation}
            \D = \point{A_I}\after \D_1\\
        \end{equation}
        
        So
        
        \begin{align*}
            \w\star(\D) &= \w\star(\point{A_I}\after \D_1)\\
                    & = \point{A_{I'}} \after\w\star(\D_1)\qt{By S-preservation}\\
                    & = \point{A_{I'}} \after\D_1'\\
                    & = \D'
        \end{align*}
        
        \case{Bind}
        Let \begin{align*}
            \D_1 &= \deno{\gpetyperelation{v_1}{\M{\e_1}{A}}}\\
            \D_2 &= \deno{\etyperelation{\P}{\gax}{v_2}{\M{\e_2}{B}}}
        \end{align*}
        
        Then
        
        \begin{equation}
            \D = \M{\e_1}{\e_2}{A_I}\after\T{\e_1}{\D_2}\after\tstrength{\e_1}{\G_I}{A_I}\after\pr{\Id{\G_{I}}}{\D_1}\\
        \end{equation}
        
        So
        
        \begin{align*}
            \w\star(\D) &= \w\star(\bind{\e_1}{\e_2}{A}\after\T{\e_1}{\D_2}\after\tstrength{\e_1}{\G}{A}\after\pr{\Id{\G}}{\D_1})\\
            & = \w\star(\bind{\e_1}{\e_2}{A})\after\w\star(\T{\e_1}{\D_2})\after\w\star(\tstrength{\e_1}{\G}{A})\after\pr{\w\star(\Id{\G_{I})}}{\w\star(\D_1)}\qt{By S-Closure}\\
            &= \bind{\e_1}{\e_2}{A_{I'}}\after\T{\e_1}{\w\star(\D_2)}\after\tstrength{\e_1}{\G_{I'}}{A_{I'}}\after\pr{\w\star(\Id{\G_{I})}}{\w\star(\D_1)}\qt{By S-Closure}\\
            &= \bind{\e_1}{\e_2}{A_{I'}}\after\T{\e_1}{\D_2'}\after\tstrength{\e_1}{\G_{I'}}{A_{I'}}\after\pr{\w\star(\Id{\G_{I})}}{\D_1'}\qt{By Induction}\\
            &= \D'\\
        \end{align*}
        
        \case{If}
        
        Let \begin{align*}
            \D_1 &= \deno{\gpetyperelation{v}{\B}}\\
            \D_2 &= \deno{\gpetyperelation{v_1}{A}}\\
            \D_3 &= \deno{\gpetyperelation{v_2}{A}}\\
        \end{align*}
        
        Then
        
        \begin{equation}
            \D = \app\after(([\cur{\D_2\after\pp}, \cur{\D_3\after\pp}]\after\D_1)\times\Id{\G})\after\diag{\G}\\
        \end{equation}
        
        So
        
        \begin{align*}
            \w\star(\D) &= \w\star(\app\after(([\cur{\D_2\after\pp}, \cur{\D_3\after\pp}]\after\D_1)\times\Id{\G})\after\diag{\G})\\
            &= \app\after(([\cur{\w\star(\D_2)\after\pp}, \cur{\w\star(\D_3)\after\pp}]\after\w\star(\D_1))\times\Id{\G_{I'}})\after\diag{\G_{I'}}\qt{By S-Closure}\\
            &= \app\after(([\cur{\D_2'\after\pp}, \cur{\D_3'\after\pp}]\after\D_1')\times\Id{\G_{I'}})\after\diag{\G_{I'}}\qt{By Induction}\\
            & = \D'\\
        \end{align*}
        
        
        \case{Effect-Gen}
        
        Let \begin{equation}
            \D_1 = \deno{\etyperelation{\P,\a}{\G}{v}{A}}
        \end{equation}
        
        Then
        
        \begin{equation}
            \D = \bar{\D_1}\\
        \end{equation}
        
        And also
        
        \begin{equation}
            \w\times\Id{} = \deno{\wrel{\w\x}{(\P',\a)}{(\P,\a)}}
        \end{equation}
        
        So
        \begin{align*}
            \w\star\D &= \w\star(\bar{\D_1})\\
            & = \bar{(\w\times\Id{U})\star\D_1}\qt{By naturality}\\
            & = \bar{\D_1'}\qt{By induction}\\
            & = \D'
        \end{align*}
        
        \case{Effect-Spec}
        
        Let \begin{align*}
            \D_1 &= \deno{\gpetyperelation{v}{\all{\a}{A}}}\\
            h &= \deno{\typerelation{\P}{\e}{\effect}}\\
        \end{align*}
        
        Then
        
        \begin{equation}
            \D = \pr{\Id{\G}}{h}\star(\e_{\deno{\typerelation{\P,\b}{A\ssub{\a}{\b}}{\type}}})\after\D_1\\
        \end{equation}
        
        So due to the substitution theorem on effects
        \begin{equation}
            h\after\w = \deno{\typerelation{\P}{\e}{\effect}}\after\w = \deno{\typerelation{\P'}{\e}{\effect}} = h'
        \end{equation}
        
        Also note $(\w\times\Id{U}) = \deno{\wrel{\w\x}{\P',\a}{\P\a}}$
        
        \begin{align*}
            \w\star{\D} & = \w\star(\pr{\Id{\G}}{h}\star(\e_{\deno{\typerelation{\P,\b}{A\ssub{\a}{\b}}{\type}}})\after\D_1)\\
            & = (\pr{\Id{\G}}{h}\after\w)\star(\e_{\deno{\typerelation{\P,\b}{A\ssub{\a}{\b}}{\type}}})\after\w\star(\D_1)\\
            & = ((\w\times\Id{U})\after\pr{\Id{\G}}{h\after\w})\star(\e_{\deno{\typerelation{\P,\b}{A\ssub{\a}{\b}}{\type}}})\after\D_1')\\
            & = (\pr{\Id{\G}}{h'})\star((\w\times\Id{U})\star\e_{\deno{\typerelation{\P,\b}{A\ssub{\a}{\b}}{\type}}})\after\D_1')\\
        \end{align*}
        
        Looking at the inner part of the functor application:
        Let \begin{align*}
            A & = \deno{\typerelation{\P,\b}{A\ssub{\a}{\b}}{\type}}\\
        \end{align*}
        \begin{align*}
            (\w\times\Id{U})\star\e_{\deno{\typerelation{\P,\b}{A\ssub{\a}{\b}}{\type}}} &= (\w\times\Id{U})\star\e_{A}\\
            & = (\w\times\Id{U})\star(\widehat{\Id{\allI(A)}})\\
            & = \widehat{\bar{(\w\times\Id{U})\star(\widehat{\Id{\allI(A)}})}}\qt{By bijection}\\
            & = \widehat{\w\star(\bar{\widehat{\Id{\allI(A)}}})}\qt{By naturality}\\
            & = \widehat{\w\star(\Id{\allI(A)})}\qt{By bijection}\\
            & = \widehat{\Id{\allII(A\after(\w\times\Id{U}))}}\qt{By S-Closure, naturality}\\
            & = \widehat{\Id{\allII(A)}}\qt{By Substitution theorem}\\
            & = \e_{A_{I'}}
        \end{align*}
        
        Going back to the original expression:
        
        \begin{align*}
            \w\star{\D} & = (\pr{\Id{\G}}{h'})\star(\e_{A_{I'}})\after\D_1')\\
            & = \D'\\
        \end{align*}
    \end{proof}
    
\end{framed}
\section{Term-Substitution}

\begin{theorem}[Effect-Environment Weakening Preserves Term Substitutions]
    If $\wrelw{\P'}{\P}$ and $\etyperelation{\P}{\G'}{\si}{\G}$ then $\etyperelation{\P'}{\G'}{\si}{\G}$.
\end{theorem}

\begin{proof}
    If $\oke{\P}{\G'}$ then $\oke{\P'}{\G'}$
    For each term $v$ in $\si$, $\etyperelation{\P}{\G'}{v}{A}$, which means $\etyperelation{\P'}{}{\G'}{v}{A}$. So $\etyperelation{\P'}{\G'}{\si}{\G}$ holds.
\end{proof}


\begin{theorem}[Effect-Environment Weakening and Term-Substitution Denotations]
    If $\w = \deno{\wrelw{\P'}{\P}}$, then $\deno{\etyperelation{\P'}{\G'}{\si}{\G}} = \w\star\deno{\etyperelation{\P}{\G'}{\si}{\G}}$.
    
\end{theorem}


\begin{framed}
    \begin{proof}
        By induction on the structure of $\si$, making use of the weakening of term denotations above.
        
        \case{Nil}
        Then $\si = \term{\G'_{I}}$, so $\w\star(\si) = \term{\G'_{I'}} = \deno{\etyperelation{\P'}{\G'}{\si}{\G}}$
        
        \case{Extend}
        Then $\si = (\si',x\setto v)$
        
        \begin{align*}
            \w\star\si & = \w*\pr{\si'}{\deno{\gpetyperelation{v}{A}}}\\
            & = \pr{\w\star\si'}{\w\star\deno{\gpetyperelation{v}{A}}}\\
            &=\pr{\deno{\etyperelation{\P'}{\G'}{\si'}{\G}}}{\deno{\etyperelation{\G'}{\P'}{v}{A}}}\\
            &=\deno{\etyperelation{\P'}{\G'}{\si}{\gax}}
        \end{align*}
    \end{proof}
\end{framed}


\section{Term-Weakening}


\begin{theorem}[Effect-Environment Weakening Preserves Term-Environment Weakening]
    If $\wrel{\w_1}{\P'}{\P}$ and $\ewrel{\P}{\w}{\G'}{\G}$ then $\ewrel{\P'}{\w}{\G'}{\G}$.
\end{theorem}


\begin{framed}
    \begin{proof}
        \case{Id}
        By inversion, $\oke{\P}{\G}$, so $\oke{\P'}{\G}$, so $\ewrel{\P'}{\w}{\G}{\G}$.

        \case{Project}
        By inversion, $\oke{\P}{A}$ and $\ewrel{\P}{\w}{\G'}{\G}$, so by induction and the preservation of wellformedness of types, $\oke{\P'}{A}$ and $\ewrel{\P'}{\w}{\G'}{\G}$. Hence $\ewrel{\P'}{\w\p}{\G', x: A}{\G}$
        
        \case{Extend}
        By inversion, $\oke{\P}{A}$ and $\ewrel{\P}{\w}{\G'}{\G}$, $A\subtypep B$. So by induction and the preservation of wellformedness of types, $\oke{\P'}{A}$ and $\ewrel{\P'}{\w}{\G'}{\G}$, $A\subtypepp B$. Since $A\subtypepp B$, $\wellformed{\P'}{B}$. Hence $\ewrel{\P'}{\w\x}{\G', x: A}{\G: B}$.
    \end{proof}
\end{framed}


\begin{theorem}[Effect Weakening and Term Weakening Denotations]
    If $\ewrel{\P'}{\w_1}{\G'}{\G}$ and $\w = \deno{\wrelw{\P'}{\P}}$, then $\ewrel{\P'}{\w_1}{\G'}{\G}$ and $\deno{\ewrel{\P'}{\w_1}{\G'}{\G}} = \w\star\deno{\ewrel{\P}{\w_1}{\G'}{\G}}$.
\end{theorem}




\begin{framed}
    \begin{proof}
        
        By induction on the structure of $\w_1$.
        
        \case{Id}
        Then $\w_1 = \i$, so its denotation is $\w_1 = \Id{\G_I}$
        
        So
        \begin{equation}
          \w\star(\Id{\G_I}) = \Id{\G_{I'}} = \deno{\ewrel{\P'}{\i}{\G}{\G}}  
        \end{equation}
        
        \case{Project}
        Then $\w_1 = \w_1'\pi$
        
        \begin{equation}
            \ntreeruleI{Project}{\ewrel{\P}{\w_1'}{\G'}{\G}}{\ewrel{\P}{\w_1\pi}{\G',x:A}{\G}}
        \end{equation}
        
        So $\w_1 = \w_1'\after\p$
        
        Hence
        \begin{align*}
            \w\star(\w_1) &= \w\star(\w_1')\after\w\star(\p)\\
            & = \deno{\ewrel{\P'}{\w_1'}{\G'}{\G}}\after\p\\
            & = \deno{\ewrel{\P'}{\w_1'\pi}{\G',x:A}{\G}}\\
            & = \deno{\ewrel{\P'}{\w_1}{\G', x:A}{\G}}
        \end{align*}
        
        \case{Extend}
        Then $\w_1 = \w_1'\x$
        
        \begin{equation}
            \ntreeruleII{Extend}{\ewrel{\P}{\w_1'}{\G'}{\G}}{ A\subtypep B}{\ewrel{\P}{\w_1\x}{\G',x:A}{\G, x:B}}
        \end{equation}
        
        So $\w_1 = \w_1'\times\deno{A\subtypep B}$
        
        Hence
        \begin{align*}
            \w\star(\w_1) &=(\w\star(\w_1')\times\w\star(\deno{A\subtypep B})\\
            & = (\deno{\ewrel{\P'}{\w_1'}{\G'}{\G}}\times\deno{A\subtypepp B})\\
            & = \deno{\ewrel{\P'}{\w_1}{\G',x:A}{\G,x:B}}
        \end{align*}
    \end{proof}
\end{framed}

\chapter{Term Substitution Theorem}

\begin{theorem}[Term Substitution]
    If $\D$ derives $\gpetyperelation{v}{A}$ and $\etyperelation{\P}{\G'}{\si}{\G}$ then the derivation $\D'$ deriving $\etyperelation{\P}{\G'}{v\ssi}{A}$ satisfies:
    
    \begin{equation}
        \D' = \D \after\deno{\etyperelation{\P}{\G'}{\si}{\G}}
    \end{equation}
    
\end{theorem}

\begin{framed}
    \begin{proof}
        
        This is proved by induction over the derivation of $\etyperelation{\P}{\G}{v}{A}$.
        We shall use $\si$ to denote $\deno{\etyperelation{\P}{\G'}{\si}{\G}}$ where it is clear from the context.
        
        \case{Var}
        By inversion, $\G = \G'', x:A$
        \begin{equation}
            \ntreeruleI{Var}{\wellformedok{\P}{\G}}{\etyperelation{\P}{\G'', x:A}{x}{A}}
        \end{equation}
        By inversion, $\si = \si', x\setto v$ and $\etyperelation{\P}{\G'}{v}{A}$.
        
        Let 
        \begin{align*}
            \si &=\deno{\etyperelation{\P}{\G'}{\si}{\G}} = \pr{\si'}{\D'}\\
            \D &=\deno{\etyperelation{\P}{\G'', x:A}{x}{A}} = \pp\\
        \end{align*}
        
        \begin{align*}
            \D\after\si &= \pp\after\pr{\si'}{\D'}\qt{By definition}\\
            &= \D'\qt{By product property}
        \end{align*}
        
        
        \case{Weaken}
        By inversion, $\G = \G', y:B$ and $\si = \si', y\setto v$
        and we have $\D_1$ deriving:
        
        \begin{equation}
            \ntreeruleI{Weaken}{\treeruleI{\D_1}{\etyperelation{\P}{\G''}{x}{A}}}{\etyperelation{\P}{\G'',y: B}{x}{A}}
        \end{equation}
        
        Also by inversion of the wellformedness of $\etyperelation{\P}{\G'}{\si}{\G}$, we have $\etyperelation{\P}{\G'}{\si'}{\G''}$ and 
        
        \begin{equation}
            \deno{\etyperelation{\P}{\G'}{\si}{\G}} = \pr{\deno{\etyperelation{\P}{\G'}{\si}{\G''}}}{\deno{\etyperelation{\P}{\G'}{v}{B}}}
        \end{equation}
        
        Hence by induction on $\D_1$ we have $\D_1'$ such that
        
        \begin{equation}
            \treeruleI{\D_1'}{\etyperelation{\P}{\G'}{x\ssi}{A}}
        \end{equation}
        
        
        
        Hence
        \begin{align*}
            \D' & = \D_1' \qt{By definition}\\
                & = \D_1\after\si'\qt{By induction}\\
                & = \D_1\after\p\after\pr{\si'}{\deno{\etyperelation{\P}{\G'}{v}{B}}}\qt{By product property}\\
                & = \D_1\after\p\after\si\qt{By defintion of the denotation of $\si$}\\
                & = \D\after\si\qt{By defintion.}
        \end{align*}
        
        \case{Constants}
        The logic for all constant terms ($\t,\f,\u,\const{A}$) is the same.
        Let
        \begin{equation}
            c = \deno{\const{A}}
        \end{equation}
        \begin{align*}
            \D' & = c\after\term{\G'}\qt{By Definition}\\
                & = c\after\term{G}\after\si\qt{Terminal property}\\
                & = \D\after\si\qt{By definition}
        \end{align*}
        
        
        \case{Fn}
        
        By inversion, we have $\D_1$ such that
        \begin{equation}
            \D = \ntreeruleI{Fn}{
                \treeruleI{\D_1}{\etyperelation{\P}{\G, x:A}{v}{B}}
            }{\etyperelation{\P}{\G}{\lam{x}{A}{v}}{\ab}}
        \end{equation}
        
        By induction of $\D_1$ we have $\D_1'$ such that
        \begin{equation}
            \D' = \ntreeruleI{Fn}{
                \treeruleI{\D_1'}{\etyperelation{\P}{\G', x:A}{(v\ssi)}{B}}
            }{\etyperelation{\P}{\G}{(\lam{x}{A}{v})\ssi}{\ab}}
        \end{equation}
        By induction and the extension lemma, we have:
        \begin{equation}
            \D_1' = \D_1\after(\si\times\Id{A})
        \end{equation}
        
        Hence:
        
        \begin{align*}
            \D' &= \cur{\D_1'}\qt{By definition}\\
                &= \cur{\D_1\after(\si\times\Id{A})}\qt{By induction and extension lemma.}\\
                & = \cur{\D_1}\after\si\qt{By the exponential property (Uniqueness)}\\
                &= \D\after\si\qt{By Definition}\\
        \end{align*}
        \case{Subtype}
        By inversion, there exists derivation $\D_1$ such that:
        
        \begin{equation}
            \D = \ntreeruleII{Subtype}{\treeruleI{\D_1}{\etyperelation{\P}{\G}{v}{A}}}{ A\subtypep B}{\etyperelation{\P}{\G}{v}{B}}
        \end{equation}
        
        By induction on $\D_1$, we find $\D_1'$ such that $\D_1' = \D_1\after\si$ and:
        
        \begin{equation}
            \D' = \ntreeruleII{Subtype}{\treeruleI{\D_1'}{\etyperelation{\P}{\G'}{v\ssi}{A}}}{ A\subtypep B}{\etyperelation{\P}{\G'}{v\ssi}{B}}
        \end{equation}
        
        Hence,
        
        \begin{align*}
            \D' &= \deno{A\subtypep B}\after\D_1'\qt{By definition}\\
                &= \deno{A\subtypep B}\after\D_1\after\si\qt{By induction}\\
                &= \D\after\si\qt{By definition}\\
        \end{align*}
        
        \case{Return}
        
        By inversion, we have $\D_1$ such that:
        \begin{equation}
            \D = \ntreeruleI{Return}{\treeruleI{\D_1}{\etyperelation{\P}{\G}{v}{A}}}{\etyperelation{\P}{\G}{\return{v}}{\moa}}
        \end{equation}
        
        By induction on $\D_1$, we find $\D_1'$ such that $\D_1' = \D_1\after\si$ and:
        
        \begin{equation}
            \D' = \ntreeruleI{Return}{\treeruleI{\D_1'}{\etyperelation{\P}{\G'}{v\ssi}{A}}}{\etyperelation{\P}{\G'}{(\return{v})\ssi}{\moa}}
        \end{equation}
        
        Hence,
        
        \begin{align*}
            \D' & = \point{A}\after\D_1'\qt{By Definition}\\
            & = \point{A}\after\D_1\after\si\qt{By induction}\\
            & = \D\after\si\qt{By Definition}\\
        \end{align*}
        \case{Apply}
        By inversion, we find $\D_1, \D_2$ such that
        \begin{equation}
            \D = \ntreeruleII{Apply}{\treeruleI{\D_1}{\etyperelation{\P}{\G}{v_1}{\ab}}}{\treeruleI{\D_2}{\etyperelation{\P}{\G}{v_2}{A}}}{\etyperelation{\P}{\G}{\apply{v_1}{v_2}}{B}}
        \end{equation}
        
        By induction we find $\D_1', \D_2'$ such that 
        \begin{align*}
            \D_1' &= \D_1\after\si\\
            \D_2' &= \D_2\after\si\\
        \end{align*}
        
        And
        \begin{equation}
            \D' = \ntreeruleII{Apply}{\treeruleI{\D_1'}{\etyperelation{\P}{\G'}{v_1\ssi}{\ab}}}{\treeruleI{\D_2'}{\etyperelation{\P}{\G'}{v_2\ssi}{A}}}{\etyperelation{\P}{\G'}{(\apply{v_1}{v_2})\ssi}{B}}
        \end{equation}
        
        Hence
        \begin{align*}
            \D' &= \app\after\pr{\D_1'}{\D_2'}\qt{By Definition}\\
                &= \app\after\pr{\D_1\after\si}{\D_2\after\si}\qt{By induction}\\
                &= \app\after\pr{\D_1}{\D_2}\after\si\qt{By Product Property}\\
                & = \D\after\si\qt{By Definition}\\
        \end{align*}
        
        \case{If}
        
        
        By inversion, we find $\D_1, \D_2, \D_3$ such that
        \begin{equation}
            \D = \ntreeruleIII{If}{\treeruleI{\D_1}{\etyperelation{\P}{\G}{v}{\B}}}{\treeruleI{\D_2}{\etyperelation{\P}{\G}{v_1}{A}}}{\treeruleI{\D_3}{\etyperelation{\P}{\G}{v_2}{A}}}{\etyperelation{\P}{\G}{\pifthenelse{A}{v}{v_1}{v_2}}{A}}
        \end{equation}
        
        By induction we find $\D_1', \D_2', \D_3'$ such that 
        \begin{align*}
            \D_1' &= \D_1\after\si\\
            \D_2' &= \D_2\after\si\\
            \D_3' &= \D_3\after\si\\
        \end{align*}
        
        And
        \begin{equation}
            \D' = \ntreeruleIII{If}{\treeruleI{\D_1'}{\etyperelation{\P}{\G'}{v\ssi}{\B}}}{\treeruleI{\D_2'}{\etyperelation{\P}{\G'}{v_1\ssi}{A}}}{\treeruleI{\D_3'}{\etyperelation{\P}{\G'}{v_2\ssi}{A}}}{\etyperelation{\P}{\G'}{(\pifthenelse{A}{v}{v_1}{v_2})\ssi}{A}}
        \end{equation}
        
        Since $\si: \G' \rightarrow \G$, let $(\tea)^{\si}: \tea^{\G}\rightarrow\tea^{\G'}$ be as defined in ExSh 3 (\footnote{https://www.cl.cam.ac.uk/teaching/1819/L108/exercises/L108-exercise-sheet-3.pdf})
        That is:
        
        \begin{align*}
            (\tea)^{\si} & = \cur{\app\after(\Id{\tea}\times w)}
        \end{align*}.
        And hence, we have:
        
        \begin{align*}
            \cur{f\after(\Id{}\times \si)} & = (\tea)^{\si} \after\cur{f}
        \end{align*}
        
        And so:
        
        \scalebox{.8}{\parbox{1.2\linewidth}{
        \begin{align*}
            \D' & =\app\after((\fld{\cur{\D_2'\after\pp}}{\cur{\D_3'\after\pp}}\after\D_1')\times \Id{\G'})\after\diag{\G'}\qt{By Definition}\\
            & =\app\after((\fld{\cur{\D_2\after\si\after\pp}}{\cur{\D_3\after\si\after\pp}}\after\D_1')\times \Id{\G'})\after\diag{\G'}\qt{By Induction}\\
            & = \app\after((\fld{\cur{\D_2\after\pp\after(\Id{\1}\times \si)}}{\cur{\D_3\after\pp\after(\Id{\1}\times \si)}}\after\D_1\after\si)\times \Id{\G'})\after\diag{\G'}\qt{By product property}\\
            & = \app\after((\fld{(\tea)^{\si}\after\cur{\D_2\after\pp}}{(\tea)^{\si}\after\cur{\D_3\after\pp}}\after\D_1\after\si)\times \Id{\G'})\after\diag{\G'}\qt{By $(\tea)^{\si}$ property}\\
            & = \app\after(((\tea)^{\si}\after\fld{\cur{\D_2\after\pp}}{\cur{\D_3\after\pp}}\after\D_1\after\si)\times \Id{\G'})\after\diag{\G'}\qt{Factor out transformation}\\
            & = \app\after((\tea)^{\si}\times\Id{\G'})\after((\fld{\cur{\D_2\after\pp}}{\cur{\D_3\after\pp}}\after\D_1)\times \Id{\G'})\after(\si \times \Id{\G'})\after\diag{\G'}\qt{Factor out Identity pairs}\\
            & = \app\after(\Id{(\tea)}\times\si)\after((\fld{\cur{\D_2\after\pp}}{\cur{\D_3\after\pp}}\after\D_1) \times \Id{\G'})\after(\si \times \Id{\G'})\after\diag{\G'}\qt{By defintion of $\app, (\tea)^{\si}$}\\
            & = \app\after((\fld{\cur{\D_2\after\pp}}{\cur{\D_3\after\pp}}\after\D_1)\times \idg)\after(\si \times \si)\after\diag{\G'}\qt{Push through pairs}\\
            & = \app\after((\fld{\cur{\D_2\after\pp}}{\cur{\D_3\after\pp}}\after\D_1)\times \idg)\after\diag{\G}\after\si\qt{By Definition of the diagonal morphism.}\\
            & = \D\after\si
        \end{align*}
        }}
        
        
        \case{Bind}
        
        By inversion, we have $\D_1, \D_2$ such that:
        
        \begin{equation}
            \D = \ntreeruleII{Bind}{
                \treeruleI{\D_1}{\etyperelation{\P}{\G}{v_1}{A}}
                }{
                \treeruleI{\D_2}{\etyperelation{\P}{\G, x:A }{v_1}{B}}
            }{
                \etyperelation{\P}{\G}{\doin{x}{v_1}{v_2}}{\M{\e_1\dot\e_2}{B}}
            }
        \end{equation}
        
        By property \ref{TermSubsPropThree},
        
        \begin{equation}
            \etyperelation{\P}{(\G', x:A)}{(\si, x\setto x}{(\gax)}
        \end{equation}
        
        With denotation (extension lemma)
        
        \begin{equation}
            \deno{\etyperelation{\P}{(\G', x:A)}{(\si, x\setto x}{(\gax)}} = \si\times\Id{A}
        \end{equation}
        
        By induction, we derive $\D_1', \D_2'$ such that:
        
        \begin{align*}
            \D_1' & = \D_1\after \si\\
            \D_2' & = \D_2\after (\si\times\Id{A})\qt{By Extension Lemma}
        \end{align*}
        
        And:
        
        \begin{equation}
            \D' = \ntreeruleII{Bind}{
                \treeruleI{\D_1'}{\etyperelation{\P}{\G'}{v_1\ssi}{A}}
                }{
                \treeruleI{\D_2'}{\etyperelation{\P}{\G', x:A }{v_1\ssi}{B}}
            }{
                \etyperelation{\P}{\G'}{(\doin{x}{v_1}{v_2})\ssi}{\M{\e_1\dot\e_2}{B}}
            }
        \end{equation}
        
        Hence:
        
        \scalebox{.8}{\parbox{1.2\linewidth}{        \begin{align*}
            \D' & = \bind{\e_1}{\e_2}{B}\after\T{\e_1}{\D_2'}\after\tstrength{\e_1}{\G'}{A}\after\pr{\Id{\G'}}{\D_1'}\qt{By Definition}\\
            & = \bind{\e_1}{\e_2}{B}\after\T{\e_1}{(\D_2\after(\si\times\Id{A}))}\after\tstrength{\e_1}{\G'}{A}\after\pr{\Id{\G'}}{\D_1\after\si}\qt{By Induction using the extension lemma}\\
            & = \bind{\e_1}{\e_2}{B}\after\T{\e_1}{(\D_2)}\after\tstrength{\e_1}{\G}{A}\after(\si\times\Id{\T{\e_1}{A}})\after\pr{\Id{\G'}}{\D_1\after\si}\qt{By Tensor Strength}\\
            & = \bind{\e_1}{\e_2}{B}\after\T{\e_1}{(\D_2)}\after\tstrength{\e_1}{\G}{A}\after\pr{\si}{\D_1\after\si}\qt{By Product rule}\\
            & = \bind{\e_1}{\e_2}{B}\after\T{\e_1}{(\D_2)}\after\tstrength{\e_1}{\G}{A}\after\pr{\idg}{\D_1}\after\si\qt{By Product rule}\\
            &= \D\after\si\qt{By Defintion}\\
        \end{align*}
        }}
        
        \case{Effect-Gen}
        
        By inversion, we have $\D_1$ such that
        
        \begin{equation}
            \D = \ntreeruleI{Effect-Gen}{\treeruleI{\D_1}{\etyperelation{\P,\a}{\G}{v}{A}}}{\gpetyperelation{\elam{\a}{v}}{\all{\e}{A}}}
        \end{equation}
        
        By induction, we derive $\D_1'$ such that
        
        \begin{equation}
            \D' = \ntreeruleI{Effect-Gen}{\treeruleI{\D_1'}{\etyperelation{\P,\a}{\G'}{v\ssi}{A}}}{\etyperelation{\P}{\G'}{(\elam{\a}{v})\ssi}{\all{\e}{A}}}
        \end{equation}
        
        Where 
        \begin{align*}
            \D_1' & = \D_1\after\deno{\etyperelation{\P,\a}{\G'}{\si}{\G}}\\
            & = \D_1\after\deno{\wrel{\i\pi}{\P,a}{\P}}\star(\si)\\
            & = \D_1\after\pstar(\si)
        \end{align*}
        
        Hence \begin{align*}
            \D\after\si & = \bar{\D_1}\after\si\\
            & = \bar{\D_1\after\pstar(\si)}\\
            & = \bar{\D_1'}\\
            & = \D'
        \end{align*}
        
        \case{Effect-Spec}
        By inversion, we derive $\D_1$ such that
        \begin{equation}
            \D = \ntreeruleII{Effect-Spec}{\treeruleI{\D_1}{\etyperelation{\P}{\G}{v}{\all{\a}{A}}}}{\wellformed{\P}{\e}}{\etyperelation{\P}{\G}{\eapp{v}{\e}}{A\ssub{\a}{\e}}}
        \end{equation}
        
        By induction, we derive $\D_1'$ such that
        
        \begin{equation}
            \D' = \ntreeruleII{Effect-Spec}{\treeruleI{\D_1'}{\etyperelation{\P}{\G'}{v\ssi}{\all{\a}{A}}}}{\wellformed{\P}{\e}}{\etyperelation{\P}{\G'}{(\eapp{v}{\e})\ssi}{A\ssub{\a}{\e}}}
        \end{equation}
        
        Where 
        
        \begin{align*}
            \D_1' = \D\after\si
        \end{align*}
        
        
        
        Hence, if $h = \deno{\typerelation{\P}{\e}{\effect}}$
        \begin{align*}
            \D\after\si &= \pr{\Id{I}}{h}\star(\e{\deno{\typerelation{\P,\b}{A\ssub{\b}{\a}}{\effect}}})\after\D_1\after\si\\
            &= \pr{\Id{I}}{h}\star(\e{\deno{\typerelation{\P,\b}{A\ssub{\b}{\a}}{\effect}}})\after\D_1'\\
            & = \D'
        \end{align*}
    \end{proof}
    
\end{framed}

\chapter{Term-Environment Weakening Theorem}

\begin{theorem}[Term-Environment Weakening]
    If $w = \deno{\ewrel{\P}{\w}{\G'}{G}}$ and $\D$ derives  $\gpetyperelation{v}{A}$ then there exists $\D'$ deriving $\etyperelation{\P}{\G'}{v}{A}$ such that $\D' = \D\after\w$
\end{theorem}




\begin{framed}
    \begin{proof}
        
        We induct over the structure of typing derivations of $\gpetyperelation{v}{A}$, assuming $\ewrel{\P}{\w}{\G'}{\G}$ holds. In each case, we construct the new derivation $\D'$ from the derivation $\D$ giving $\gpetyperelation{v}{A}$ and show that $\D\after\deno{\ewrel{\P}{\w}{\G'}{\G}} = \D'$
        
            \case{Var and Weaken}
                We case split on the weakening $\w$.
                \subcase{$\w = \i$}
                Then $\G' = \G$, and so $\etyperelation{\P}{\G'}{x}{A}$ holds and the derivation $\D'$ is the same as $\D$
        
                \begin{equation}
                    \D' = \D = \D\after\idg = \D\after\deno{\ewrel{\P}{\i}{\G}{\G}} 
                \end{equation}
                \subcase{$\w = \w'\pi$}
                Then  $\G' = (\G'',x': A')$ and $\ewrel{\P}{\w'}{\G''}{\G}$. So by induction, there is a tree, $\D_1$ deriving $\etyperelation{\P}{\G''}{x}{A}$,  such that 
                \begin{equation}
                    \D_1 = \D\after\deno{\ewrel{\P}{\w'}{\G''}{\G}} \qt{By Induction}
                \end{equation}
                
                , and hence by the weaken rule, we have 
                \begin{equation}
                    \ntreeruleI{Weaken}{\etyperelation{\P}{\G''}{x}{A}}{\etyperelation{\P}{\G'', x':A' }{x}{A}}
                \end{equation}
        
                This preserves denotations:
                \begin{align*}
                    \D' & = \D_1\after\p\qt{By Definition} \\
                    & = \D\after\deno{\ewrel{\P}{\w'}{\G''}{\G}}\after\p\qt{By induction}\\
                    & = \D\after\deno{\ewrel{\P}{\w'\p}{\G'}{\G}}\qt{By denotation of weakening}
                \end{align*}
        
                \subcase{$\w = \w'\x$} 
                Then 
                \begin{align*}
                    \G' & = \G''', x': B\\
                    \G &= \G'', x': A'\\
                    B & \subtypep A
                \end{align*}
        
                \subcase{$x = x'$}
        
                Then $A = A'$.
        
                Then we derive the new derivation, $\D'$ as so:
        
                \begin{equation}
                    \ntreeruleII{Subtype}{
                        \ntreeruleI{Var}{\oke{\P}{\G}}{\etyperelation{\P}{\G''', x: B}{x}{B}}
                        }{
                        B \subtypep A
                    }{
                        \etyperelation{\P}{\G'}{x}{A}
                    }
                \end{equation}
        
                This preserves denotations:
        
                \begin{align*}
                    \D' & = \deno{B\subtypep A}\after\pp\qt{By Definition} \\
                     & = \pp\after (\deno{\ewrel{\P}{\w'}{\G'''}{\G''}}\times \deno{B\subtypep A}) \qt{By the properties of binary products}\\
                     & = \D\after\deno{\ewrel{\P}{\w}{\G'}{\G}}\qt{By Definition}
                \end{align*}
        
                \subcase{$x \neq x'$}
                Then 
                \begin{equation}
                    \D = \ntreeruleI{Weaken}{\treeruleI{\D_1}{\etyperelation{\P}{\G''}{x}{A}}}{\gpetyperelation{x}{A}}
                \end{equation}
        
                By induction with $\ewrel{\P}{\w}{\G'''}{\G''}$,
                 we have a derivation $\D_1$ of $\etyperelation{\P}{\G'''}{x}{A}$
        
                We have the weakened derivation:
        
                \begin{equation}
                    \D' = \ntreeruleI{Weaken}{\treeruleI{\D_1'}{\etyperelation{\P}{\G'''}{x}{A}}}{\etyperelation{\P}{\G'}{x}{A}}
                \end{equation}
        
                This preserves denotations:
        
                By induction, we have
                \begin{equation}
                    \D_1' = \D_1 \after \deno{\ewrel{\P}{\w}{\G'''}{\G''}}
                \end{equation}
                So we have:
                \begin{align*}
                    \D' &= \D_1' \after \p\qt{By denotation definition}\\
                    & = \D_1\after\deno{\ewrel{\P}{\w'}{\G'''}{\G''}}\qt{By induction}\after\p \\
                    & = \D_1\after\p\after(\deno{\ewrel{\P}{\w'}{\G'''}{\G''}}\times\deno{A' \subtypep B})\qt{By product properties}\\
                    & = \D\after\deno{\ewrel{\P}{\w}{\G'}{\G}}\qt{By definition}
                \end{align*}
        
        
            From this point onwards, since we no-longer case split over the weakening relations, we write the denotation $\deno{\ewrel{\P}{\w}{\G'}{\G'}}$, simply as $\w$.
        
        
            \case{Constant}
            The constant typing rules, $\u$, $\t$, $\f$, $\const{A}$, all proceed by the same logic. Hence I shall only prove the theorems for the case $\const{A}$.
        
            \begin{equation}
                \ntreeruleI{Const}{\wellformedok{\P}{\G}}{\gpetyperelation{\const{A}}{A}}
            \end{equation}
        
            By inversion, we have $\wellformedok{\P}{\G}$, so we have $\wellformedok{\P}{\G'}$.
        
            Hence
        
            \begin{equation}
                \ntreeruleI{Const}{\wellformedok{\P}{\G'}}{\etyperelation{\P}{\G'}{\const{A}}{A}}
            \end{equation}
            Holds.
        
            This preserves denotations:
        
        
            \begin{align*}
                \D' & = \deno{\const{A}} \after \term{\G'}\qt{By definition}\\
                & = \deno{\const{A}} \after \term{\G}\after \w \qt{By the terminal property}\\
                & = \D\qt{By Definition}\\
            \end{align*}
        
        
            \case{Fn}
            By inversion, we have a derivation $\D_1$ giving
        
            \begin{equation}
                \D = \ntreeruleI{Fn}{
                    \treeruleI{\D_1}{\etyperelation{\P}{\gax}{v}{B}}
                }{\gpetyperelation{\lam{x}{A}{v}}{\ab}}
            \end{equation}
        
            Since $\ewrel{\P}{\w}{\G'}{\G}$, we have:
        
            \begin{equation}
                \ewrel{\P}{\w\x}{(\G,x:  A)}{(\gax)}
            \end{equation}
        
            Hence, by induction, using $\ewrel{\P}{\w\x}{(\G,x:  A)}{(\gax)}$, we derive $\D_1'$:
        
            \begin{equation}
                \D' = \ntreeruleI{Fn}{
                    \treeruleI{\D_1'}{\etyperelation{\P}{\G',x: A}{v}{B}}
                }{\etyperelation{\P}{\G',x: A}{\lam{x}{A}{v}}{\ab}}
            \end{equation}
        
            This preserves denotations:
        
        
            \begin{align*}
            \D' & = \cur{\D_1'} \qt{By Definition}\\
            & = \cur{\D_1\after(\w\times \idg)}\qt{By the denotation of $\w\x$} \\
            &= \cur{\D_1}\after\w\qt{By the exponential property}\\
            &= \D\after \w \qt{By Definition}
            \end{align*}
        
        
            \case{Subtyping}
        
            \begin{equation}
                \ntreeruleII{Subtype}{\gpetyperelation{v}{A}}{ A\subtypep B}{\gpetyperelation{v}{B}}
            \end{equation}
        
            by inversion, we have a derivation $\D_1$
            \begin{equation}
                \treeruleI{\D_1}{\gpetyperelation{v}{A}}
            \end{equation}
        
            So by induction, we have a derivation $\D_1'$ such that:
            \begin{equation}
                \ntreeruleII{Subtype}{\treeruleI{\D_1'}{\etyperelation{\P}{\G'}{v}{a}}}{ A \subtypep B}{\etyperelation{\P}{\G'}{v}{B}}
            \end{equation}
        
            This preserves denotations:
        
            \begin{align*}
                \D' & = \deno{A\subtypep B}\after \D_1' \qt{By Definition} \\
                & = \deno{A\subtypep B}\after \D_1\after\w \qt{By induction}\\
                & = \D\after\w \qt{By Definition}\\
            \end{align*}
        
            \case{Return}
            We have the Subderivation $\D_1$ such that
            \begin{equation}
                \D = \ntreeruleI{Return}{\treeruleI{\D_1}{\gpetyperelation{v}{A}}}{\gpetyperelation{\return{v}}{\moa}}
            \end{equation}
        
            Hence, by induction, with $\ewrel{\P}{\w}{\G'}{\G}$, we find the derivation $\D_1'$ such that:
            \begin{equation}
                \D' = \ntreeruleI{Return}{\treeruleI{\D_1'}{\etyperelation{\P}{\G'}{v}{A}}}{\etyperelation{\P}{\G'}{\return{v}}{\moa}}
            \end{equation}
        
            This preserves denotations:
        
            \begin{align*}
                \D' & = \point{A}\after\D_1' \qt{By definition}\\
                    & = \point{A}\after\D_1\after\w\qt{By induction of $\D_1, \D_1'$}\\
                    & = \D\after\w\qt{By Definition}
            \end{align*}
        
            \case{Apply}
                By inversion, we have derivations $\D_1$, $\D_2$ such that
        
                \begin{equation}
                    \D = 
                    \ntreeruleII{Apply}{
                        \treeruleI{\D_1}{\gpetyperelation{v_1}{\ab}}
                        }{
                        \treeruleI{\D_2}{\gpetyperelation{v_2}{A}}
                    } {
                        \gpetyperelation{\apply{v_1}{v_2}}{B}
                    }
                \end{equation}
        
                By induction, this gives us the respective derivations: $\D_1',\D_2'$ such that
        
                
                \begin{equation}
                    \D' = 
                    \ntreeruleII{Apply}{
                        \treeruleI{\D_1'}{\etyperelation{\P}{\G'}{v_1}{\ab}}
                        }{
                        \treeruleI{\D_2'}{\etyperelation{\P}{\G'}{v_2}{A}}
                    } {
                        \etyperelation{\P}{\G'}{\apply{v_1}{v_2}}{B}
                    }
                \end{equation}
        
                This preserves denotations:
        
                \begin{align*}
                    \D' &= \app\after\pr{\D_1'}{\D_2'} \qt{By Definition}\\
                    &= \app\after\pr{\D_1\after\w}{\D_2\after\w} \qt{By induction on $\D_1, \D_2$}\\
                    &= \app\after\pr{\D_1}{\D_2}\after\w\\
                    &= \D\after\w\qt{By Definition}
                \end{align*}
            \case{If}
            By inversion, we have the Subderivations $\D_1,\D_2,\D_3$, such that:
        
        
            \begin{equation}
                \D = \ntreeruleIII{If}{
                    \treeruleI{\D_1}{\etyperelation{\P}{\G}{v}{\B}}
                    }{
                    \treeruleI{\D_2}{\etyperelation{\P}{\G}{v_1}{A}}
                    }{
                    \treeruleI{\D_3}{\etyperelation{\P}{\G}{v_2}{A}}
                }{
                    \etyperelation{\P}{\G}{\pifthenelse{A}{v}{v_1}{v_2}}{A}
                }
            \end{equation}
        
            By induction, this gives us the Subderivations $\D_1', \D_2', \D_3'$ such that
        
            \begin{equation}
                \D' = \ntreeruleIII{If}{
                    \treeruleI{\D_1'}{\etyperelation{\P}{\G'}{v}{\B}}
                    }{
                    \treeruleI{\D_2'}{\etyperelation{\P}{\G'}{v_1}{A}}
                    }{
                    \treeruleI{\D_3'}{\etyperelation{\P}{\G'}{v_2}{A}}
                }{
                    \etyperelation{\P}{\G'}{\pifthenelse{A}{v}{v_1}{v_2}}{A}
                }
            \end{equation}
        
            And 
            \begin{align*}
                \D_1' & =\D_1 \after \w\\
                \D_3' & =\D_2 \after \w\\
                \D_3' & =\D_3 \after \w 
            \end{align*}
        
        
            This preserves denotations.
            Since $\w: \G' \rightarrow \G$, \\
            Let $(\tea)^{\w}: \tea^{\G}\rightarrow\tea^{\G'}$ be as defined in ExSh 3 (\footnote{https://www.cl.cam.ac.uk/teaching/1819/L108/exercises/L108-exercise-sheet-3.pdf})
            That is:
        
            \begin{align*}
                (\tea)^{\w} & = \cur{\app\after(\Id{\tea}\times w)}
            \end{align*}.
            And hence, we have:
        
            \begin{align*}
                \cur{f\after(\Id{}\times \w)} & = (\tea)^{\w} \after\cur{f}
            \end{align*}
        
            \scalebox{.8}{\parbox{1.2\linewidth}{
            \begin{align*}
                \D' & =\app\after((\fld{\cur{\D_2'\after\pp}}{\cur{\D_3'\after\pp}}\after\D_1')\times \Id{\G'})\after\diag{\G'}\qt{By Definition}\\
                & =\app\after((\fld{\cur{\D_2\after\w\after\pp}}{\cur{\D_3\after\w\after\pp}}\after\D_1')\times \Id{\G'})\after\diag{\G'}\qt{By Induction}\\
                & = \app\after((\fld{\cur{\D_2\after\pp\after(\Id{\1}\times \w)}}{\cur{\D_3\after\pp\after(\Id{\1}\times \w)}}\after\D_1\after\w)\times \Id{\G'})\after\diag{\G'}\qt{By product property}\\
                & = \app\after((\fld{(\tea)^{\w}\after\cur{\D_2\after\pp}}{(\tea)^{\w}\after\cur{\D_3\after\pp}}\after\D_1\after\w)\times \Id{\G'})\after\diag{\G'}\qt{By $(\tea)^{\w}$ property}\\
                & = \app\after(((\tea)^{\w}\after\fld{\cur{\D_2\after\pp}}{\cur{\D_3\after\pp}}\after\D_1\after\w)\times \Id{\G'})\after\diag{\G'}\qt{Factor out transformation}\\
                & = \app\after((\tea)^{\w}\times\Id{\G'})\after((\fld{\cur{\D_2\after\pp}}{\cur{\D_3\after\pp}}\after\D_1)\times \Id{\G'})\after(\w \times \Id{\G'})\after\diag{\G'}\qt{Factor out Identity pairs}\\
                & = \app\after(\Id{(\tea)}\times\w)\after((\fld{\cur{\D_2\after\pp}}{\cur{\D_3\after\pp}}\after\D_1) \times \Id{\G'})\after(\w \times \Id{\G'})\after\diag{\G'}\qt{By defintion of $\app, (\tea)^{\w}$}\\
                & = \app\after((\fld{\cur{\D_2\after\pp}}{\cur{\D_3\after\pp}}\after\D_1)\times \idg)\after(\w \times \w)\after\diag{\G'}\qt{Push through pairs}\\
                & = \app\after((\fld{\cur{\D_2\after\pp}}{\cur{\D_3\after\pp}}\after\D_1)\times \idg)\after\diag{\G}\after\w\qt{By Definition of the diagonal morphism.}\\
                & = \D\after\w
            \end{align*}
            }}
        
            \case{Bind}
            By inversion, we have derivations $\D_1, \D_2$ such that:
        
        
            \begin{equation}
                \D = \ntreeruleII{Bind}{
                    \treeruleI{\D_1}{\etyperelation{\P}{\G}{v_1}{\M{\E_1}{A}}}
                    }{
                    \treeruleI{\D_2}{\etyperelation{\P}{\G,x: A}{v_2}{\M{\e_2}{B}}}
                }{
                    \etyperelation{\P}{\G}{\doin{x}{v_1}{v_2}}{\M{\e_1\dot\e_2}{B}}
                }
            \end{equation}
        
            If $\ewrel{\P}{\w}{\G'}{\G}$ then $\ewrel{\P}{\w\x}{\G',x:A}{\gax}$, so by induction, we can derive $\D_1'$, $\D_2'$ such that:
        
            \begin{equation}
                \D' = \ntreeruleII{Bind}{
                    \treeruleI{\D_1'}{\etyperelation{\P}{\G'}{v_1}{\M{\e_1}{A}}}
                    }{
                    \treeruleI{\D_2'}{\etyperelation{\P}{\G',x: A}{v_2}{\M{\e_2}{B}}}
                }{
                    \etyperelation{\P}{\G'}{\doin{x}{v_1}{v_2}}{\M{\e_1\dot\e_2}{B}}
                }
            \end{equation}
        
            This preserves denotations:
        
            \begin{align*}
                \D' & = \bind{\e_1}{\e_2}{B}\after\T{\e_1}{\D_2'}\after\tstrength{\e_1}{\G'}{A}\after\pr{\Id{G'}}{\D_1'}\qt{By definition}\\
                & = \bind{\e_1}{\e_2}{B}\after\T{\e_1}{(\D_2\after(\w\times\Id{A}))}\after\tstrength{\e_1}{\G'}{A}\after\pr{\Id{G'}}{\D_1\after\w}\qt{By induction on $\D_1', \D_2'$}\\
                & = \bind{\e_1}{\e_2}{B}\after\T{\e_1}{\D_2}\after\tstrength{\e_1}{\G}{A}\after\pr{\w}{\D_1\after\w}\qt{By tensor strength}\\
                & = \bind{\e_1}{\e_2}{B}\after\T{\e_1}{\D_2}\after\tstrength{\e_1}{\G}{A}\after\pr{\idg}{\D_1}\after\w\qt{By product property}\\
                & = \D \qt{By definition}
            \end{align*}
        
        
        
            \case{Effect-Gen}
        
            By inversion, we have $\D_1$ such that
            
            \begin{equation}
                \D = \ntreeruleI{Effect-Gen}{\treeruleI{\D_1}{\etyperelation{\P,\a}{\G}{v}{A}}}{\gpetyperelation{\elam{\a}{v}}{\all{\e}{A}}}
            \end{equation}
            
            By induction, we derive $\D_1'$ such that
            
            \begin{equation}
                \D' = \ntreeruleI{Effect-Gen}{\treeruleI{\D_1'}{\etyperelation{\P,\a}{\G'}{v}{A}}}{\etyperelation{\P}{\G'}{(\elam{\a}{v})}{\all{\e}{A}}}
            \end{equation}
            
            Where 
            \begin{align*}
                \D_1' & = \D_1\after\deno{\ewrel{\P,\a}{\w}{\G'}{\G}}\\
                & = \D_1\after\deno{\wrel{\i\pi}{\P,a}{\P}}\star(\w)\\
                & = \D_1\after\pstar(\w)
            \end{align*}
            
            Hence \begin{align*}
                \D\after\w & = \bar{\D_1}\after\w\\
                & = \bar{\D_1\after\pstar(\w)}\\
                & = \bar{\D_1'}\\
                & = \D'
            \end{align*}
            
            \case{Effect-Spec}
            By inversion, we derive $\D_1$ such that
            \begin{equation}
                \D = \ntreeruleII{Effect-Spec}{\treeruleI{\D_1}{\etyperelation{\P}{\G}{v}{\all{\a}{A}}}}{\wellformed{\P}{\e}}{\etyperelation{\P}{\G}{\eapp{v}{\e}}{A\ssub{\a}{\e}}}
            \end{equation}
            
            By induction, we derive $\D_1'$ such that
            
            \begin{equation}
                \D' = \ntreeruleII{Effect-Spec}{\treeruleI{\D_1'}{\etyperelation{\P}{\G'}{v}{\all{\a}{A}}}}{\wellformed{\P}{\e}}{\etyperelation{\P}{\G'}{\eapp{v}{\e}}{A\ssub{\a}{\e}}}
            \end{equation}
            
            Where 
            
            \begin{align*}
                \D_1' = \D\after\w
            \end{align*}
            
            
            
            Hence, if $h = \deno{\typerelation{\P}{\e}{\effect}}$
            \begin{align*}
                \D\after\w &= \pr{\Id{I}}{h}\star(\e{\deno{\typerelation{\P,\b}{A\ssub{\b}{\a}}{\effect}}})\after\D_1\after\w\\
                &= \pr{\Id{I}}{h}\star(\e{\deno{\typerelation{\P,\b}{A\ssub{\b}{\a}}{\effect}}})\after\D_1'\\
                & = \D'
            \end{align*}
    \end{proof} 
\end{framed}   

\chapter{Unique Denotation Theorem}

\section{Reduced Type Derivation}
    A reduced type derivation is one where subtype and subeffect rules must, and may only, occur at the root or directly above an \textbf{if}, or \textbf{apply} rule.
    
    In this section, I shall prove that there is at most one reduced derivation of $\gpetyperelation{v}{A}$. Secondly, I shall present a function for generating reduced derivations from arbitrary typing derivations, in a way that does not change the denotations. These imply that all typing derivations of a type-relation have the same denotation.
    
    \section{Reduced Type Derivations are Unique}

\begin{theorem}[Reduced Type Derivations are Unique]
        For each instance of the relation $\gpetyperelation{v}{A}$,there exists at most one reduced derivation of  $\gpetyperelation{v}{A}$. 
\end{theorem}
    
 
 \begin{framed}
        \begin{proof}
            This is proved by induction over the typing rules on the bottom rule used in each derivation.

            \case{Variables}
            To find the unique derivation of $\gpetyperelation{x}{A}$, we case split on the term-environment, $\G$.
            
            \subcase{$\G = \G', x: A'$}
            Then the unique reduced derivation of $\gpetyperelation{x}{A}$ is, if $A' \subtypep A$, as below:
            
            \begin{equation}
                \ntreeruleII{Subtype}{\ntreeruleI{Var}{\wellformedok{\P}{\G', x: A'}}{\etyperelation{\P}{\G,x: A'}{x}{A'}}}{ A' \subtypep A}{\etyperelation{\P}{\G', x: A'}{x}{A}}
            \end{equation}
            
            \subcase{$\G = \G', y: B$} with $y \neq x$.
            
            Hence, if $\gpetyperelation{x}{A}$ holds, then so must $\etyperelation{\P}{\G'}{x}{A}$.
            
            Let 
            \begin{equation}
                \edeltavrule{\P}{\G'}{x}{A'}{A}
            \end{equation}
            Be the  unique reduced derivation of $\etyperelation{\P}{\G'}{x}{A}$.
            
            Then the unique reduced derivation of $\gpetyperelation{x}{A}$ is:
            
            
            \begin{equation}
                \ntreeruleII{Subtype}{
                    \ntreeruleI{Weaken}{
                        \treeruleI{\D}{\etyperelation{\P}{\G, x: A'}{x}{A'}}
                    }{\gpetyperelation{x}{A'}}   
                }{ A' \subtypep A
                }{\gpetyperelation{x}{A}}
            \end{equation}
            
            \case{Constants}
            For each of the constants, ($\const{A}$, $\t$, $\f$, $\u$), there is exactly one possible derivation for $\etyperelation{\P}{\G}{c}{A}$ for a given A. I shall give examples using the case $\const{A}$
            
            
            $$
                \ntreeruleII{Subtype}{\truleconst}{ A \subtypep B}{\gpetyperelation{\const{A}}{B}}
            $$
            
            If $A = B$, then the subtype relation is the identity subtype ($A \subtypep A$).
            
            \case{Fn}
            The reduced derivation of $\gpetyperelation{\lam{x}{A}{v}}{\fntype{A'}{B'}}$ is:
            
            
            $$
            \ntreeruleII{Subtype}
            {\ntreeruleI{Fn}{\treeruleI{\D}{\etyperelation{\P}{\gax}{v}{B}}}
            {
                \gpetyperelation{\lam{x}{A}{B}}{\ab}}
                }{
                \ab \subtypep \fntype{A'}{B'}
            }{
               \gpetyperelation{\lam{x}{A}{v}}{\fntype{A'}{B'}} 
            }
            $$
            
            Where 
            \begin{equation}
                \ntreeruleII{Subtype}{\treeruleI{\D}{\etyperelation{\P}{\gax}{v}{B}}}{ B\subtypep B'}{\etyperelation{\P}{\gax}{v}{B'}}
            \end{equation}
            is the reduced derivation of $\etyperelation{\P}{\gax}{v}{B}$ if it exists.
            
            \case{Return}
            The reduced denotation of $\gpetyperelation{\return{v}}{B}$ is 
            $$
                \ntreeruleII{Subtype}
                {
                    \ntreeruleI{Return}
                    {\treeruleI{\D}{\gpetyperelation{v}{A}}}
                    {\gpetyperelation{\return{v}}{\moa}}
                    }{
                    \subeffecttreep{\1}{A}{\e}{B}
                }
                {\gpetyperelation{\return{v}}{\meb}}
            $$
            
            Where $$\edeltavrule{\P}{\G}{v}{A}{B}$$ is the reduced derivation of $\gpetyperelation{v}{B}$
            
            \case{Apply}
            If 
            $$
                \edeltavrule{\P}{\G}{v_1}{\ab}{\fntype{A'}{B'}}
            $$ and $$
                \edeltavruleprime{\P}{\G}{v_2}{A''}{A'}
            $$
            
            Are the reduced type derivations of $\gpetyperelation{v_1}{\fntype{A'}{B'}}$ and $\gpetyperelation{v_2}{A'}$
            
            
            
            Then we can construct the reduced derivation of $\gpetyperelation{\apply{v_1}{v_2}}{B}$ as
            
            $$
                \ntreeruleII{Subtype}{
                    \ntreeruleII{Apply}{
                        \treeruleI{\D}{
                            \gpetyperelation{v_1}{\ab}
                        }
                        }{
                        \ntreeruleII{Subtype}{
                            \treeruleI{\D'}{
                                \gpetyperelation{v}{A''}
                            } }{ A'' \subtypep A
                        }
                        {\gpetyperelation{v}{A}}
                    }{
                        \gpetyperelation{\apply{v_1}{v_2}}{B}
                    }
                    }{
                    B \subtypep B'
                }{
                    \gpetyperelation{\apply{v_1}{v_2}}{B'}
                }
            $$
            \case{If}
            Let
            
            \begin{equation}
                \edeltavrule{\P}{\G}{v}{B'}{\B}
            \end{equation}
            
            \begin{equation}
                \edeltavruleprime{\P}{\G}{v_1}{A'}{A}
            \end{equation}
            
            \begin{equation}
                \edeltavruleprimeprime{\P}{\G}{v_2}{A''}{A}
            \end{equation}
            
            Be the unique reduced reduced derivations of $\gpetyperelation{v}{\B}$, $\gpetyperelation{v_1}{A}$, $\gpetyperelation{v_2}{A}$.
            
            Then the only reduced derivation of $\gpetyperelation{\pifthenelse{A}{v}{v_1}{v_2}}{B}$ is:
            
            \begin{equation}
                \ntreeruleII{Subtype}{
                    \scalebox{.8}{$
                    \ntreeruleIII{If}{
                        \scalebox{.5}{$
                            \edeltavrule{\P}{\G}{v}{B'}{\B}
                        $}}{\scalebox{.5}{$
                            \edeltavruleprime{\P}{\G}{v_1}{A'}{A}
                        $}}{\scalebox{.5}{$
                                \edeltavruleprimeprime{\P}{\G}{v_2}{A''}{A}
                    $}}{\gpetyperelation{\pifthenelse{A}{v}{v_1}{v_2}}{A}}
                    $}
                }{ A \subtypep B}
                {\gpetyperelation{\pifthenelse{A}{v}{v_1}{v_2}}{B}}
            \end{equation}
            
            \case{Bind}
            This case makes use of the weakening theorem on term environments. Let the trees in equations \ref{UniqueBindOne}, \ref{UniqueBindTwo} be the respective unique reduced type derivations of the subterms. By weakening, $\ewrel{\P}{\i\x}{(\G, x:A)}{(\G, x: A')}$ so if there's a derivation of $\etyperelation{\P}{(\G, x:A')}{v_2}{B}$, there's also one of $\etyperelation{\P}{\gax}{v_2}{B}$ (equation \ref{UniqueBindThree}). 

    \begin{equation}\label{UniqueBindOne}
        \edeltacrule{\G}{v_1}{\e_1}{A}{\e_1'}{A'}
    \end{equation}

    \begin{equation}\label{UniqueBindTwo}
        \edeltacruleprime{\G, x:A'}{v_2}{\e_2}{B}{\e_2'}{B'}
    \end{equation}

    \begin{equation}\label{UniqueBindThree}
        \edeltacruleprimeprime{(\G, x:A)}{v_2}{\e_2}{B}{\e_2'}{B'}
    \end{equation}

    Since the effects monoid operation is monotone, if $\e_1\subeffectp\e_1'$ and $\e_2\subeffectp\e_2'$ then $\e_1\dot\e_2 \subeffectp \e_1'\dot\e_2'$. Hence the reduced type derivation of $\gpetyperelation{\doin{x}{v_1}{v_2}}{\M{\e_1'\dot\e_2'}{B'}}$ can be seen in equation \ref{UniqueBindResult}.

    \begin{equation}\label{UniqueBindResult}
        \resizebox{.9\hsize}{!}{
        \ntreeruleII{Subtype}{
            \ntreeruleII{Bind}{
                \treeruleI{\D}{\gpetyperelation{v_1}{\M{\e_1}{A}}}
            }{
                \treeruleI{\D''}{\etyperelation{\P}{\G, x:A}{v_2}{\M{\e_2}{B}}}
            } {
                \gpetyperelation{\doin{x}{v_1}{v_2}}{\M{\e_1\dot\e_2}{B}}
            }
        }{
            \subeffecttreep{\e_1\dot\e_2}{B}{\e_1'\dot\e_2'}{B'}
        } {
            \gpetyperelation{\doin{x}{v_1}{v_2}}{\M{\e_1'\dot\e_2'}{B'}}
        }
    }
    \end{equation}
    
            \case{Effect-Gen}
            
            The unique reduced derivation of $\gpetyperelation{\elam{\a}{A}}{\all{\a}{B}}$
            
            is 
            
            \begin{equation}
                \ntreeruleII{Subtype}{
                    \ntreeruleI{Effect-Gen}{
                        \treeruleI{\D}{\etyperelation{\P,\a}{\G}{v}{A}}
                    }{
                        \gpetyperelation{\elam{\a}{v}}{\all{\a}{A}}
                    }
                    }{
                    \all{\a}{A}\subeffectp\all{\a}{B}
                }{
                    \gpetyperelation{\elam{\a}{B}}{\all{\a}{B}}
                }
            \end{equation}
            
            Where
            \begin{eqnarray}
                \ntreeruleII{Subtype}{
                    \treeruleI{\D}{\etyperelation{\P,\a}{\G}{v}{A}}
                    }{
                    A\subtypepa B
                }{
                    \etyperelation{\P,\a}{\G}{v}{B}
                }
            \end{eqnarray}
            
            Is the unique reduced derivation of $\etyperelation{\P,\a}{\G}{v}{B}$
            
            \case{Effect-Spec}
            The unique reduced derivation of $\gpetyperelation{\eapp{v}{\a}}{B'}$
            
            is 
            
            \begin{equation}
                \ntreeruleII{Subtype}{
                    \ntreeruleII{Effect-Spec}{
                        \treeruleI{\D}{\gpetyperelation{v}{\all{\a}{A}}}}{\wellformed{\P}{\e}
                    }{\gpetyperelation{\eapp{v}{\e}}{A\ssub{\a}{\e}}}
                    }{            
                    A\ssub{\a}{\e}\subtypep B'
                }{\gpetyperelation{\eapp{v}{\a}}{B'}}
            \end{equation}
            
            Where $B\ssub{\a}{\e}\subtypep B'$
            and
            \begin{equation}
                \ntreeruleII{Subtype}{
                    \treeruleI{\D}{\gpetyperelation{v}{\all{\a}{B}}}}{ \ntreeruleI{Quantification}{A\subtypepa B}{\all{\a}{A}\subtypep\all{\a}{B}}
                }{\gpetyperelation{v}{\all{\a}{B}}}
            \end{equation}
        \end{proof}
    
 \end{framed}

    \section{Each type derivation has a reduced equivalent with the same denotation.}
    We introduce a function, $\reduce$ that maps each valid type derivation of $\gpetyperelation{v}{A}$ to a reduced equivalent with the same denotation. To do this, we do case analysis over the root type rule of a derivation and prove that the denotation is not changed.
\begin{theorem}[Reduction Function]
    There exists a function $reduce$ that maps each typing derivation to a reduced derivation of the same term and type and the same denotation.
\end{theorem}
\begin{framed}
    \begin{proof}
        
                \case{Constants}
                    For the constants $\t, \f, \const{A}$, etc, $\reduce$ simply returns the derivation, as it is already reduced. This trivially preserves the denotation.
        
                    $\reduce(\truleconst) = \truleconst$
        
                    \case{Var}
                    \begin{equation}
                        \reduce(\ntreeruleI{Var}{\wellformedok{\P}{\G}}{\etyperelation{\P}{\gax}{x}{A}}) =  \ntreeruleI{Var}{\wellformedok{\P}{\G}}{\etyperelation{\P}{\gax}{x}{A}}
                    \end{equation}
        
                    Preserves denotation trivially.
        
                    \case{Weaken}
                    \subparagraph{$\reduce$ definition}
                    To find:
                    \begin{equation}
                        \reduce(\ntreeruleI{Weaken}{\treeruleI{\D}{\gpetyperelation{x}{A}}}{\etyperelation{\P}{\gby}{x}{A}})
                    \end{equation}
        
                    Let 
                    \begin{equation}\label{WeakenDeltaReduction}
                        \ntreeruleII{Subtype}{\treeruleI{\D'}{\gpetyperelation{x}{A}}}{ A'\subtypep A}{\gpetyperelation{x}{A}} = \reduce(\D)
                    \end{equation}
        
                    In 
                    \begin{equation}
                        \ntreeruleII{Subtype}{
                        \ntreeruleI{Weaken}{
                            \treeruleI{\D'}{\gpetyperelation{x}{A'}}
                        } {
                            \etyperelation{\P}{\gby}{x}{A'}
                        }
                        }{
                        A' \subtypep A
                        }{\etyperelation{\P}{\gby}{x}{A}}
                    \end{equation}
        
                    \subparagraph{Preserves Denotation}
                    Using the construction of denotations, we can find the denotation of the original derivation to be:
                    \begin{equation}
                        \deno{\ntreeruleI{Weaken}{\treeruleI{\D}{\gpetyperelation{x}{A}}}{\etyperelation{\P}{\gby}{x}{A}}} = \D \after \p
                    \end{equation}
        
                    Similarly, the denotation of the reduced denotation is:
                    \begin{equation}
                        \deno{\ntreeruleII{Subtype}{
                            \ntreeruleI{Weaken}{
                                \treeruleI{\D'}{\gpetyperelation{x}{A'}}
                            } {
                                \etyperelation{\P}{\gby}{x}{A'}
                            }
                            }{
                            A' \subtypep A
                            }{\etyperelation{\P}{\gby}{x}{A}}} = \deno{A'\subtypep A}\after \D' \after \p
                    \end{equation}
        
        
                    By induction on $\reduce$ preserving denotations and the reduction of $\D$ (\ref{WeakenDeltaReduction}), we have:
        
                    \begin{equation}
                        \D = \deno{A' \subtypep A}\after\D'
                    \end{equation}
        
                    So the denotations of the un-reduced and reduced derivations are equal.
        
                    \case{Fn}
                    \subparagraph{$\reduce$ definition}
                        To find:
                    
                        \begin{equation}
                            \reduce(\ntreeruleI{Fn}{
                                \treeruleI{\D}{\etyperelation{\P}{\gax}{v}{B}}
                            }{\gpetyperelation{\lam{x}{A}{v}}{\ab}})
                        \end{equation}
        
                        Let 
        
                        \begin{equation}
                            \ntreeruleII{Subtype}{
                                \treeruleI{\D'}{\etyperelation{\P}{\gax}{v}{B'}}
                                B' \subtypep B
                            }{
                                \etyperelation{\P}{\gax}{v}{\M{\e_2}{B}}
                            } = \reduce(\D)
                        \end{equation}
        
                        In
        
                        \begin{equation}
                            \ntreeruleII{Subtype}{
                                \ntreeruleI{Fn}{
                                    \treeruleI{
                                        \D'
                                    }{
                                        \etyperelation{\P}{\gax}{v}{B'}
                                    }    
                                }{
                                    \gpetyperelation{\lam{x}{A}{v}}{\fntype{A}{B'}}
                                }
                                }{
                                \fntype{A}{B'}\subtypep\fntype{A}{B}
                            } {
                                \gpetyperelation{\lam{x}{A}{v}}{\fntype{A}{B}}
                            }
                        \end{equation}
        
                    \subparagraph{Preserves Denotation}
                        Let
                        \begin{align*}
                            f &= \deno{B'\subtypep B'} \\
                            \deno{\fntype{A}{B'}\subtypep\fntype{A}{B}} & = f^A = \cur{f\after \app}
                        \end{align*}
        
                        Then
        
                        \begin{align*}
                            before & = \cur{\D} \qt{By definition}\\
                                    & = \cur{f\after\D'} \qt{By reduction of $\D$} \\
                                    & = f^A\after\cur{\D'}\qt{By the property of $f^X\after\cur{g} = \cur{f\after g}$} \\
                                    & = after \qt{By definition}\\
                        \end{align*}
        
                    \case{Subtype}
                    \subparagraph{$\reduce$ definition}
                    To find:
                    \begin{equation}
                        \reduce(\ntreeruleII{Subtype}{\treeruleI{\D}{\gpetyperelation{v}{A}}}{ A \subtypep B}{\gpetyperelation{v}{B}})
                    \end{equation}
        
                    Let 
                    \begin{equation}\label{SubtypeDeltaReduction}
                        \ntreeruleII{Subtype}{\treeruleI{\D'}{\gpetyperelation{x}{A}}}{ A'\subtypep A}{\gpetyperelation{x}{A}} = \reduce(\D)
                    \end{equation}
        
                    In 
                    \begin{equation}
                        \ntreeruleII{Subtype}{
                            \treeruleI{\D'}{\gpetyperelation{v}{A'}}
                        }{
                        A' \subtypep A \subtypep B
                        }{\gpetyperelation{v}{B}}
                    \end{equation}
        
                    \subparagraph{Preserves Denotation}
                        \begin{align*}
                            before & = \deno{A \subtypep B} \after \D \\
                            & = \deno{A \subtypep B} \after (\deno{A' \subtypep A} \after \D') \qt{ by Denotation of reduction of $\D$.}\\
                            & = \deno{A' \subtypep B} \after \D'\qt{Subtyping relations are unique} \\
                            & = after \\
                        \end{align*}
                    \case{Return}
                    \subparagraph{$\reduce$ definition}
                    To find:
                    
                        \begin{equation}
                            \reduce(\ntreeruleI{Return}{
                                \treeruleI{\D}{\gpetyperelation{v}{A}}
                            }{\gpetyperelation{\return{v}}{\moa}})
                        \end{equation}
        
                        Let 
        
                        \begin{equation}
                            \ntreeruleII{Subtype}{
                                \treeruleI{\D'}{\gpetyperelation{v}{A'}}
                                }{
                                A' \subtypep A
                            }{
                                \gpetyperelation{v}{A}
                            } = \reduce(\D)
                        \end{equation}
        
                        In
        
                        \begin{equation}
                            \ntreeruleII{Subtype}{
                                \ntreeruleI{Return}{
                                    \treeruleI{
                                    \D'
                                }{
                                    \gpetyperelation{v}{A}
                                }}{\gpetyperelation{\return{v}}{\M{\1}{A'}}}
                                }{
                                \subeffecttreep{\1}{A'}{\1}{A}
                            } {
                                \gpetyperelation{\return{v}}{\moa}
                            }
                        \end{equation}
        
                        \subparagraph{Preserves Denotation}
                        Then
                        \begin{align*}
                            before &= \point{A}\after\D\qt{By definition}\qt{By defintion}\\
                            &{}= \point{A}\after\deno{A'\subtypep A}\after \D'\qt{By reduction of $\D$}\\
                            &{}=\T{\1}{\deno{A'\subtypep A}}\after \point{A'}\after\D'\qt{By naturality of $\point{}$}\\
                            &{}=\db{\1\subeffectp\1}_{M, A} \after \T{\1}{\deno{A'\subtypep A}}\after\point{A'}\after\D'\qt{Since $\deno{\1\subeffectp\1}$ is the identity Nat-Trans }\\
                            &{}=after\qt{By definition}\\
                        \end{align*}
                    \case{Apply}
                    \subparagraph{$\reduce$ definition}
                    To find:
                    \begin{equation}
                        \reduce(\ntreeruleII{Apply}{
                            \treeruleI{\D_1}{
                                \gpetyperelation{v_1}{\ab}
                            }
                            }{
                            \treeruleI{\D_2}{
                                \gpetyperelation{v_2}{A}
                            }
                        }{
                            \gpetyperelation{\apply{v_1}{v_2}}{B}
                        })
                    \end{equation}
        
                    Let
                    \begin{align*}
                        \ntreeruleII{Subtype}{
                            \treeruleI{\D'_1}{\gpetyperelation{v_1}{\fntype{A'}{B'}}}
                            }{
                            \fntype{A'}{B'}\subtypep\fntype{A}{B}
                        }{
                            \gpetyperelation{v_1}{\ab}
                        } & = \reduce(\D_1)\\
                        \ntreeruleII{Subtype}{
                            \treeruleI{\D'_2}{\gpetyperelation{v}{A'}}
                            }{
                            A'\subtypep A
                        } {
                            \gpetyperelation{v_1}{A}
                        } & = \reduce(\D_2)
                    \end{align*}
        
                    In
                    \begin{equation}
                        \scalebox{.9}{$
                        \ntreeruleII{Subtype}{
                            \ntreeruleII{Apply}{
                                \treeruleI{
                                    \D'_1
                                }{
                                    \gpetyperelation{v_1}{\fntype{A'}{B'}}
                                }
                            }{
                                \ntreeruleII{Subtype}{
                                    \treeruleI{\D'_2}{\gpetyperelation{v_2}{A''}}
                                    }{
                                    A'' \subtypep A \subtypep A'
                                } {
                                    \gpetyperelation{v_2}{A'}
                                }
                            }{
                                \gpetyperelation{\apply{v_1}{v_2}}{B'}
                            }
                            }{
                            B' \subtypep B
                        }{
                            \gpetyperelation{\apply{v_1}{v_2}}{B}
                        }
                        $}
                    \end{equation}

                    \subparagraph{Preserves Denotation}
                        Let
                        \begin{align*}
                            f & = \deno{A\subtypep A'}: A\rightarrow A' \\
                            f' & = \deno{A''\subtypep A}: A'' \rightarrow A \\
                            g & = \deno{B' \subtypep B}: B' \rightarrow B \\
                        \end{align*}
        
                        Hence 
                        \begin{align*}
                            \deno{\fntype{A'}{B'}\subtypep \ab} & = (g)^A \after (B')^f \\
                            & = \cur{app\after \app}\after\cur{\app\after(\Id{}\times f)}\\
                            & = \cur {g\after\app\after(\Id{}\times f)}
                        \end{align*}
        
                        Then 
                        \begin{align*}
                            before & = \app\after\pr{\D_1}{\D_2}\qt{By definition}\\
                            & = \app\after\pr{\cur {g\after\app\after(\Id{}\times f)}\after\D'_1}{f'\after\D'_2}\qt{By reductions of $\D_1, \D_2$}\\
                            & = \app\after(\cur {g\after\app\after(\Id{}\times f)}\times\Id{A})\after\pr{\D'_1}{f'\after\D'_2} \qt{Factoring out}\\
                            & = g\after\app\after(\Id{}\times f)\after\pr{\D'_1}{f'\after\D'_2}\qt{By the exponential property}\\
                            & = g\after\app\after\pr{\D'_1}{f\after f'\after \D'_2}\\
                            & = after\qt{By defintion}
                        \end{align*}
                    \case{If}
                   
                    \subparagraph{$\reduce$ definition}
                        \begin{align*}
                            \reduce(&\ntreeruleIII{If}{
                                \treeruleI{\D_1}{\gpetyperelation{v}{\B}}
                                }{
                                \treeruleI{\D_2}{\gpetyperelation{v_1}{A}}
                                }{
                                \treeruleI{\D_3}{\gpetyperelation{v_2}{A}}
                            }{
                                \gpetyperelation{\pifthenelse{A}{v}{v_1}{v_2}}{A}
                            }) \\
                            & = \ntreeruleIII{If}{
                                \treeruleI{\reduce(\D_1)}{\gpetyperelation{v}{\B}}
                                }{
                                \treeruleI{\reduce(\D_2)}{\gpetyperelation{v_1}{A}}
                                }{
                                \treeruleI{\reduce(\D_3)}{\gpetyperelation{v_2}{A}}
                            }{
                                \gpetyperelation{\pifthenelse{A}{v}{v_1}{v_2}}{A}
                            }
                        \end{align*}
        
                    \subparagraph{Preserves Denotation}
                        Since calling $\reduce{}$ on the Subderivations preserves their denotations, this definition trivially preserves the denotation of the derivation.
        
        
                        \case{Bind}
                        \subparagraph{$\reduce$ definition}
        
                        To find
                        \begin{equation}
                            \reduce(
                                \ntreeruleII{Bind}{
                                    \treeruleI{
                                        \D_1
                                    }{
                                        \gpetyperelation{v_1}{\M{\e_1}{A}}
                                    }
                                    }{
                                    \treeruleI{
                                        \D_2
                                    }{
                                        \etyperelation{\P}{\gax}{v_2}{\M{\e_2}{B}}
                                    }
                                } {
                                    \gpetyperelation{\doin{x}{v_1}{v_2}}{\M{\e_1\dot\e_2}{B}}
                                }
                            )
                        \end{equation}
        
        
                        Let \begin{equation}
                            \ntreeruleII{Subtype}{
                                \treeruleI{\D'_1}{\gpetyperelation{v_1}{\M{\e'_1}{A'}}}
                            }{
                            \subeffecttreep{\e_1'}{A'}{\e_1}{A}
                            }{
                                \gpetyperelation{v_1}{\M{\e_1}{A}}
                            } = \reduce(\D_1)
                        \end{equation}
        
                        Since $\ewrel{\P}{(i,\x)}{(\G, x: A')}{(\gax)}$ if $A' \subtypep A$, and by $\D_2 = \etyperelation{\P}{(\gax)}{v_2}{\M{\e_2}{B}}$, there also exists a derivation $\D_3$ of $\etyperelation{\P}{(\G, x: A')}{v_2}{\M{\e_2}{B}}$. $\D_3$ is derived from $\D_2$ simply by inserting a (Subtype) rule below all instances of the (Var) rule.
        
                        Let \begin{equation}
                            \ntreeruleII{Subtype}{
                                \treeruleI{\D'_3}{\etyperelation{\P}{\G, x: A'}{v_2}{\M{\e'_2}{B'}}}
                            }{
                            \subeffecttreep{\e_1'}{B'}{\e_2}{B}
                            }{
                                \etyperelation{\P}{\G, x: A'}{v_2}{\M{\e_2}{B}}
                            } = \reduce(\D_3)
                        \end{equation}
                        
        
                        Since the effects monoid operation is monotone, if $\e_1\subeffectp\e_1'$ and $\e_2\subeffectp\e_2'$ then $\e_1\dot\e_2 \subeffectp \e_1'\dot\e_2'$
                        
        
        
                        Then the result of reduction of the whole bind expression is:
        
        
                        \begin{equation}
                            \scalebox{.9}{$
                            \ntreeruleII{Subtype}{
                                \ntreeruleII{Bind}{
                                    \treeruleI{
                                        \D'_1
                                    }{
                                        \gpetyperelation{v_1}{\M{\e_1'}{A'}}
                                    }
                                    }{
                                    \treeruleI{
                                        \D'_3
                                    }{
                                        \etyperelation{\P}{\G, x: A'}{v_2}{\M{\e_2'}{B'}}
                                    }
                                }{
                                \gpetyperelation{\doin{x}{v_1}{v_2}}{\M{\e_1'\dot\e_2'}{B}}
                                }
                                }{
                                \subeffecttreep{\e_1'\dot\e_2'}{B'}{\e_1\dot\e_2}{B}
                            }{
                                \gpetyperelation{\doin{x}{v_1}{v_2}}{\M{\e_1\dot\e_2}{B}}
                            }
                            $}
                        \end{equation}
                        \subparagraph{Preserves Denotation}
        
                        Let \begin{align*}
                            f & = \deno{A' \subtypep A}: A' \rightarrow A\\
                            g & = \deno{B' \subtypep B}: B' \rightarrow B\\
                            h_1 & = \deno{\e_1' \subeffectp \e_1} : \T{\e_1'}{} \rightarrow \T{\e_1}{} \\
                            h_2 & = \deno{\e_2'\subeffectp \e_2}:\T{\e_2'}{} \rightarrow \T{\e_2}{}\\
                            h & = \deno{\e_1'\dot\e_2'\subeffectp\e_1\dot\e_2}: \T{\e_1'\dot\e_2'}{}\rightarrow \T{\e_1\dot\e_2}{}
                        \end{align*}
        
                        Due to the denotation of the weakening used to derive $\D_3$ from $\D_2$, we have 
                        \begin{equation}
                            \D_3 = \D_2\after(\idg\times f)
                        \end{equation}
        
                        And due to the reduction of $\D_3$,
                        we have 
                        \begin{equation}
                            \D_3 = h_{2, B} \after \T{\e_2'}{g}\after \D_3'
                        \end{equation}
        
                        So:
        
                        \scalebox{.85}{\parbox{1.1\linewidth}{
                        \begin{align*}
                            before &= \bind{\e_1}{\e_2}{B}\after \T{\e_1}{\D_2}\after\tstrength{\e_1}{\G}{A}\after\pr{\idg}{\D_1}\qt{By definition.}\\
                            &= \bind{\e_1}{\e_2}{B}\after \T{\e_1}{\D_2}\after\tstrength{\e_1}{\G}{A}\after\pr{\idg}{h_{1, A}\after\T{\e_1'}{f}\after\D_1'}\qt{By reduction of $\D_1$.}\\
                            &= \bind{\e_1}{\e_2}{B}\after \T{\e_1}{\D_2}\after\tstrength{\e_1}{\G}{A}\after(\idg\times h_{1, A})\after\pr{\idg}{\T{\e_1'}{f}\after\D_1'}\qt{Factor out $h_1$}\\
                            &= \bind{\e_1}{\e_2}{B}\after \T{\e_1}{\D_2}\after
                            h_{1, (\G\times A)}\after
                            \tstrength{\e_1'}{\G}{A}\after\pr{\idg}{\T{\e_1'}{f}\after\D_1'}\qt{Tensor strength and subeffecting $h_1$}\\
                            &= \bind{\e_1}{\e_2}{B}\after 
                            h_{1, B}\after\T{\e_1'}{\D_2}\after
                            \tstrength{\e_1'}{\G}{A}\after\pr{\idg}{\T{\e_1'}{f}\after\D_1'}\qt{Naturality of $h_1$}\\
                            &= \bind{\e_1}{\e_2}{B}\after 
                            h_{1, B}\after\T{\e_1'}{\D_2}\after
                            \tstrength{\e_1'}{\G}{A}\after(\idg\times \T{\e_1'}{f})\after\pr{\idg}{\D_1'}\qt{Factor out pairing again}\\
                            &= \bind{\e_1}{\e_2}{B}\after 
                            h_{1, B}\after\T{\e_1'}{(\D_2\after(\idg\times f))}\after
                            \tstrength{\e_1'}{\G}{A'}\after\pr{\idg}{\D_1'}\qt{Tensorstrength}\\
                            &= \bind{\e_1}{\e_2}{B}\after 
                            h_{1, B}\after\T{\e_1'}{(\D_3)}\after
                            \tstrength{\e_1'}{\G}{A'}\after\pr{\idg}{\D_1'}\qt{By the definition of $\D_3$}\\
                            &= \bind{\e_1}{\e_2}{B}\after 
                            h_{1, B}\after\T{\e_1'}{(h_{2, B}\after\T{\e_2'}{g}\after \D_3')}\after
                            \tstrength{\e_1'}{\G}{A'}\after\pr{\idg}{\D_1'}\qt{By the reduction of $\D_3$}\\
                            &= \bind{\e_1}{\e_2}{B}\after 
                            h_{1, B}\after\T{\e_1'}{h_{2, B}}\after\T{\e_1'}{\T{\e_2'}{g}}\after \T{\e_1'}{\D_3'}\after
                            \tstrength{\e_1'}{\G}{A'}\after\pr{\idg}{\D_1'}\qt{Factor out the functor}\\
                            &= h_B\after\bind{\e_1'}{\e_2'}{B}\after\T{\e_1'}{\T{\e_2'}{g}}\after \T{\e_1'}{\D_3'}\after
                            \tstrength{\e_1'}{\G}{A'}\after\pr{\idg}{\D_1'}\qt{By the $\mu$ and Subtype rule }\\
                            & = h_B\after\T{\e_1'\dot\e_2'}{g}\after\bind{\e_1'}{\e_2'}{B'}\after \T{\e_1'}{\D_3'}\after
                            \tstrength{\e_1'}{\G}{A'}\after\pr{\idg}{\D_1'}\qt{By naturality of $\bind{}{}{}$ }\\
                            & = after \qt{By definition}
                        \end{align*}
                        }}

                \case{Effect-Gen}
                \subparagraph{$\reduce$ definition}
        
                
                To find 
                \begin{equation}
                    \reduce(\ntreeruleI{Effect-Gen}{\treeruleI{\D_1}{\etyperelation{\P,\a}{\G}{v}{A}}}{\gpetyperelation{\elam{\a}{v}}{\all{\a}{A}}})
                \end{equation}
        
                Let
                \begin{equation}
                    \ntreeruleII{Subtype}{\treeruleI{\D_1'}{\etyperelation{\P,\a}{\G}{v}{A'}}}{ A'\subtypep A}{\etyperelation{\P,\a}{\G}{v}{A}} = \reduce(\D_1)
                \end{equation}
        
                in 
        
                \begin{equation}
                    \ntreeruleII{Subtype}{
                        \ntreeruleI{Effect-Gen}{
                            \treeruleI{\D_1'}{\etyperelation{\P,\a}{\G}{v}{A'}}
                        }{
                            \gpetyperelation{\elam{\a}{v}}{\all{\a}{A'}}
                        }
                    }{ 
                    \ntreeruleI{Quantification}{A' \subtypepa}{\all{\a}{A'} \subtypep \all{\a}{A}}
                    }{
                        \gpetyperelation{\elam{\a}{v}}{\all{\a}{A}}
                    }
                \end{equation}
        
                \subparagraph{Preserves Denotation}
        
                \begin{align*}
                    before &= \bar{\D_1}\\
                    & = \bar{\deno{A'\subtypepa A}\after \D_1'}\qt{By induction}\\
                    & = \allI(\deno{A'\subtypepa A})\after\bar{\D_1'}\\
                    &= \deno{\all{\a}{A'}\subtypep\all{\a}{A}}\after\bar{\D_1'}\qt{By definition}\\
                    & = after\qt{By definition}     
                \end{align*}
        
                \case{Effect-Spec}
                \subparagraph{$\reduce$ definition}
        
                
                To find 
                \begin{equation}
                    \reduce(\ntreeruleII{Effect-Spec}{\treeruleI{\D_1}{\gpetyperelation{v}{\all{\a}{A}}}}{\wellformed{\P}{\e}}{\gpetyperelation{\eapp{v}{\e}}{A\ssub{\a}{\e}}})
                \end{equation}
        
                Let
        
                \begin{eqnarray}
                    \ntreeruleII{Subtype}{
                        \treeruleI{\D_1'}{\gpetyperelation{v}{\all{\a}{A'}}}
                        }{ 
                        \ntreeruleI{Quantification}{A'\subtypepa A}{\all{\a}{A'}\subtypep\all{\a}{A}}
                    }{
                        \gpetyperelation{v}{\all{\a}{A}}
                    }= \reduce(\D_1)
                \end{eqnarray}
        
                In
        
                \begin{equation}
                    \ntreeruleII{Subtype}{
                        \ntreeruleII{E-app}{
                            \treeruleI{\D_1'}{\gpetyperelation{v}{\all{\a}{A}}}}{\wellformed{\P}{\e}
                        }{\gpetyperelation{\eapp{v}{\e}}{A\ssub{\a}{\e}}}
                    }{
                    A'\ssub{\a}{\e}\subtypep A\ssub{\a}{\e}
                    }{
                        \gpetyperelation{\eapp{v}{\e}}{A\ssub{\a}{\e}}
                    }
                \end{equation}
        
                \subparagraph{Preserves Denotation}
                Let
                \begin{align*}
                    h &= \deno{\typerelation{\P}{\e}{\effect}}\\
                    A & = \deno{\typerelation{\P,\b}{A\ssub{\a}{\b}}{\effect}}\\
                    A' & = \deno{\typerelation{\P,\b}{A'\ssub{\a}{\b}}{\effect}}
                \end{align*}
        
                Note that 
        
                \begin{equation}
                    \pr{\Id{I}}{h}\star(\pstar(f)) = (\p\after\pr{\Id{I}}{h})\star(f) = \Id{I}\star(f) = f
                \end{equation}
        
                And that 
                \begin{equation}
                    \pr{\Id{I}}{h} = \deno{\typerelation{\P}{\ssub{\a}{\e}}{\P,\a}}
                \end{equation}
        
                With lemma:
                \begin{align*}
                    \deno{\all{\a}{A'}\subtypep{\all{\a}{A}}} & = \allI(\deno{A' \subtypepa A})\\
                    & = \pr{\Id{I}}{h}\star(\pstar(\allI(\deno{A' \subtypepa A})))
                \end{align*}
        
                In
        
                \begin{align*}
                    before &= \pr{\Id{I}}{h}\star(\e_A)\after\D_1\\
                    &= \pr{\Id{I}}{h}\star(\e_A)\after\deno{\all{\a}{A'}\subtypep{\all{\a}{A}}}\after\D_1'\qt{By induction}\\
                    &= \pr{\Id{I}}{h}\star(\e_A)\after\pr{\Id{I}}{h}\star(\pstar(\allI(\deno{A' \subtypepa A})))\after\D_1'\qt{By lemma}\\
                    &= \pr{\Id{I}}{h}\star(\e_A\after \pstar(\allI(\deno{A' \subtypepa A})))\after\D_1'\qt{By functorality}\\
                    & = \pr{\Id{I}}{h}\star(\deno{A' \subtypepa A}\after\e_{A'})\after\D_1'\qt{By Naturality}\\
                    & =  \pr{\Id{I}}{h}\star(\deno{A' \subtypepa A})\after\pr{\Id{I}}{h}\star(\e_{A'})\after\D_1'\\
                    & =  \deno{A'\ssub{\a}{\e} \subtypepa A\ssub{\a}{\e}}\after\pr{\Id{I}}{h}\star(\e_{A'})\after\D_1'\qt{By substitution of subtyping}\\
                    & = after
                \end{align*}
                
            $$\square$$   
    \end{proof}
\end{framed}


\begin{theorem}[Denotations are Equivalent]
    Any two derivations of the type relation instance $\gpetyperelation{v}{A}$ have the same denotation.
\end{theorem}

\begin{framed}
    \begin{proof}
            For each type relation instance $\gpetyperelation{v}{A}$ there exists a unique reduced derivation of the relation instance. For all derivations $\D$, $\D'$ of the type relation instance, $\deno{\D} = \deno{\reduce{\D}} = \deno{\reduce{\D'}} = \deno{\D'} $, hence the denotation $\deno{\gpetyperelation{v}{A}}$ is unique.
    \end{proof}
\end{framed}


\chapter{Equational-Equivalence Theorem (Soundness)}

\section{Equational Equivalence Relation}

The equational equivalence relation is a rule based relation with three main flavours of rules. These are a set of rules that model a monadic reduction system of the language, a set of congruence relation rules, and rules that extend the system into an equivalence relation.


Reduction-based rules
    \begin{framed}
        \[
            \scalebox{0.8}{$
            \ntreeruleII{Lambda-Beta}{\etyperelation{\P}{\gax}{v_2}{B}}{\gpetyperelation{v_1}{A}}{\gpeberelation{\apply{(\lam{x}{A}{v_1})}{v_2} }{ v_1\ssub{x}{v_2}}{B}}
            \quad
            \ntreeruleI{Lambda-Eta}{\gpetyperelation{v}{\ab}}{\gpeberelation{\lam{x}{A}{(\apply{v}{x}})}{v}{\ab}}
            $}
        \]
    
        \[\scalebox{0.8}{$
            \ntreeruleII{Left Unit}{\gpetyperelation{v_1}{A} }{ \etyperelation{\P}{\gax}{v_2}{\meb}}{\gpeberelation{\doin{x}{\return{v_1}}{v_2}}{v_2\ssub{x}{v_1}}{\meb}}
            \quad
            \ntreeruleI{Right Unit}{\gpetyperelation{v}{\mea}}{\gpeberelation{\doin{x}{v}{\return{x}} }{v}{\mea}}
        $}\]
    
        \[\scalebox{0.8}{$
            \ntreeruleIII{Associativity}{\gpetyperelation{v_1}{\M{\e_1}{A}} }{\etyperelation{\P}{\gax}{v_2}{\M{\e_2}{B}}}{ \etyperelation{\P}{\gby}{v_3}{\M{\e_3}{C}}}{
                \gpeberelation{\doin{x}{v_1}{(\doin{y}{v_2}{v_3})}}{\doin{y}{(\doin{x}{v_1}{v_2})}{v_3}}{\M{\e_1 \dot \e_2 \dot \e_3}{C}}
            }
        $}\]
    
        \[\scalebox{0.8}{$
            \ntreeruleI{Unit}{\gpetyperelation{v}{\U}}{\gpeberelation{v}{\u}{\U}}
        $}\]
    
        \[\scalebox{0.8}{$
            \ntreeruleII{If-True}{\gpetyperelation{v_1}{A}}{\gpetyperelation{v_2}{A}}{\gpeberelation{\pifthenelse{A}{\t}{v_1}{v_2}}{v_1}{A}}
            \quad
            \ntreeruleII{If-False}{\gpetyperelation{v_2}{A}}{\gpetyperelation{v_1}{A}}{\gpeberelation{\pifthenelse{A}{\f}{v_1}{v_2}}{v_2}{A}}    
        $}\]
    
        \[\scalebox{0.8}{$
            \ntreeruleII{If-Eta}{\etyperelation{\P}{\G, x: \B}{v_2}{A}}{\gpetyperelation{v_1}{\B}}{\gpeberelation{\pifthenelse{A}{v_1}{v_2\ssub{x}{\t}}{v_2\ssub{x}{\f}}}{v_2\ssub{x}{v_1}}{A}}
        $}\]
    
        \[\scalebox{0.8}{$
            \ntreeruleII{Effect-Beta}{\wellformed{\P}{\e}}{\etyperelation{\P, \a}{\G}{v}{A}}{\gpeberelation{\eapp{(\elam{\a}{v}}{\e})}{v\ssub{\a}{\e}}{A\ssub{\a}{\e}}}
            \quad 
            \\ntreeruleI{Effect-Eta}{\etyperelation{\P}{\G}{v}{\all{\a}{A}}}{\gpeberelation{\elam{\a}{(\eapp{v}{\a})}}{v}{\all{\a}{A}}}
        $}\]
    \end{framed}

Congruence rules
   \begin{framed}
        \[\scalebox{0.8}{$
            \ntreeruleI{Effect-Spec}{\eberelation{\P, \a}{\G}{v_1}{v_2}{A}}{\gpeberelation{\elam{\a}{v_1}}{\elam{\a}{v_2}}{\all{\a}{A}}}
            \quad
            \ntreeruleII{Effect-Gen}{\gpeberelation{v_1}{v_2}{\all{\a}{A}}}{\wellformed{\P}{\e}}{\gpeberelation{\eapp{v_1}{\e}}{\eapp{v_2}{\e}}{A\ssub{\a}{\e}}}
        $}\]
    
        \[\scalebox{0.8}{$
            \ntreeruleI{Fn}{\eberelation{\P}{\gax}{v_1}{v_2}{B}}{\gpeberelation{\lam{x}{A}{v_1}}{\lam{x}{A}{v_2}}{\ab}}
            \quad
            \ntreeruleI{Return}{\gpeberelation{v_1}{v_2}{A}}{\gpeberelation{\return{v_1}}{\return{v_2}}{\moa}}
        $}\]
    
        \[\scalebox{0.8}{$
            \ntreeruleII{Apply}{\gpeberelation{v_1}{v_1'}{\ab}}{\gpeberelation{v_2}{v_2'}{A}}{\gpeberelation{\apply{v_1}{v_2}}{\apply{v_1'}{v_2'}}{B}}
            \quad   
            \ntreeruleII{Bind}{\gpeberelation{v_1}{v_1'}{\M{\e_1}{A}} }{\eberelation{\P}{\gax}{v_2}{v_2'}{\M{\e_2}{B}}}{\gpeberelation{\doin{x}{v_1}{v_2}}{\doin{x}{v_1'}{v_2'}}{\M{\e_1 \dot \e_2}{B}}} 
        $}\]
    
        \[\scalebox{0.8}{$
            \ntreeruleIII{If}{\gpeberelation{v}{v'}{\B} }{ \gpeberelation{v_1}{v_1'}{A}}{\gpeberelation{v_2}{v_2'}{A}}{\gpeberelation{\pifthenelse{A}{v}{v_1}{v_2}}{\pifthenelse{A}{v'}{v_1'}{v_2'}}{A}}
            \quad    
            \ntreeruleII{Subtype}{\gpeberelation{v}{v'}{A}}{A \subtypep B}{\gpeberelation{v}{v'}{B}}
        $}\]
   \end{framed}


We extend the relation to an equivalence relation as so:
    \begin{framed}
        \[
            \ntreeruleI{Reflexive}{\gpetyperelation{v}{A}}{\gpeberelation{v}{v}{A}}
            \quad
            \ntreeruleI{Symmetric}{\gpeberelation{v_1}{v_2}{A}}{\gpeberelation{v_2}{v_1}{A}}
        \]
    
        \[
            \ntreeruleII{Transitive}{\gpeberelation{v_1}{v_2}{A}}{\gpeberelation{v_2}{v_3}{A}}{\gpeberelation{v_1}{v_3}{A}}
        \]
    \end{framed}



\section{Soundness}

\begin{theorem}[Soundness]
    If $\eberelation{\P}{\G}{v}{v'}{A}$ then $\gdenoequality{v}{v'}{A}$
\end{theorem}

\begin{framed}
    \begin{proof}
        By induction over equational equivalence relation.
        \subsection{Equivalence Relation}
        The cases over the equivalence relation laws hold by the uniqueness of denotations and the fact that equality over morphisms is an equivalence relation.
        \case{Reflexive}
        Equality is reflexive, so if $\gpetyperelation{v}{A}$ then $\deno{\gpetyperelation{v}{A}}$ is equal to itself.
        \case{Symmetric}
        By inversion, if $\gpeberelation{v}{v'}{A}$ then $\gpeberelation{v'}{v}{A}$, so by induction $\gdenoequality{v'}{v}{A}$ and hence $\gdenoequality{v}{v'}{A}$
        \case{Transitive}
        There must exist $v_2$ such that $\gpeberelation{v_1}{v_2}{A}$ and $\gpeberelation{v_2}{v_3}{A}$, so by induction,
        $\gdenoequality{v_1}{v_2}{A}$ and $\gdenoequality{v_2}{v_3}{A}$. Hence by transitivity of equality, $\gdenoequality{v_1}{v_3}{A}$
        
        \subsection{Reduction Conversions}
        These cases are typically proved using the properties of a cartesian closed category with a strong graded monad.
        
        \case{Lambda-Beta}
            Let $f = \deno{\etyperelation{\P}{\gax}{v_1}{B}}: (\G \times A) \rightarrow B$
        
            Let $g = \deno{\gpetyperelation{v_2}{A}}: \G \rightarrow A$
        
            By the substitution denotation, $$\deno{\etyperelation{\P}{\G}{\ssub{x}{v_2}}{\gax}}: \G \rightarrow (\G \times A) = \pr{\idg}{g}$$
        
            We have $$\deno{\gpetyperelation{v_1\ssub{x}{_2v}}{B}} = f \after \pr{\idg}{g}$$
        
            and hence
            \begin{align*}
                    \deno{\gpetyperelation{\apply{(\lam{x}{A}{v})}{v}}{B}} & = \app\after\pr{\cur{f}}{g} \\
                    & = \app\after(\cur{f}\times\Id{A})\after\pr{\idg}{g}\\
                    & = f \after \pr{\idg}{g} \\
                    & = \deno{\gpetyperelation{v_1\ssub{x}{v_2}}{B}}  
            \end{align*}
           
        
        \case{Left Unit}
        Let $f = \deno{\etyperelation{\P}{\gax}{v_1}{\meb}}$
        
        Let $g = \deno{\gpetyperelation{v_2}{A}}: \G \rightarrow A$
        
        By the substitution denotation, $$\deno{\etyperelation{\P}{\G}{\ssub{x}{v_2}}{\gax}}: \G \rightarrow (\G \times A) = \pr{\idg}{g}$$
        
        We have $$\deno{\gpetyperelation{v_1\ssub{x}{v_2}}{\meb}} = f \after \pr{\idg}{g}$$
        
        And hence
        
        \scalebox{.95}{\parbox{\linewidth}{
        \begin{align*}
                \deno{&\gpetyperelation{\doin{x}{\return{v_2}}{v_1}}{\meb}} 
                \\
                = &  \bind{\1}{\e}{B} \after \T{\1}{f} \after \tstrength{1}{\G}{A} \after \pr{\idg}{\point{A}\after g} \\
                = & \bind{\1}{\e}{B} \after \T{\1}{f} \after \tstrength{1}{\G}{A} \after (\idg\times \point{A}) \after \pr{\idg}{g} \\
                = & \bind{\1}{\e}{B} \after \T{\1}{f} \after \point{(\G \times A)} \after \pr{\idg}{g} \qt{By Tensor strength + unit}\\
                = & \bind{\1}{\e}{B}\after\point{\teb}\after f\after\pr{\idg}{g} \qt{By Naturality of $\point{}$}\\
                = & f\after\pr{\idg}{g} \qt{By left unit law}\\
                = & \deno{\gpetyperelation{v_1\ssub{x}{v_2}}{\meb}}\\
        \end{align*}
        }}
        
        
        
        
        \case{Right Unit}
        
        Let $f = \deno{\gpetyperelation{v}{\mea}}$ 
            \begin{align*}
                \deno{\gpetyperelation{\doin{x}{v}{\return{x}}}{\mea}}  & = \bind{\e}{\1}{A} \after \T{\e}{(\point{A} \after \pp)} \after \tstrength{\e}{\G}{A}\after \pr{\idg}{f} \\
                & = \T{\e}{\pp} \after \tstrength{\e}{\G}{A} \after \pr {\idg}{f} \\
                & = \pp \after \pr{\idg}{f}\\
                & = f
        \end{align*}
        
        
        
        \case{Associative}
        Let
        \begin{align*}
            f ={}& \deno{\gpetyperelation{v_1}{\M{\e_1}{A}}}\\
            g ={}& \deno{\etyperelation{\P}{\gax}{v_2}{\M{\e_2}{B}}}\\
            h ={}& \deno{\etyperelation{\P}{\gby}{v_3}{\M{\e_3}{C}}}\\
        \end{align*}
        
        We also have the weakening:
        \begin{equation}
            \ewrel{\P}{\i\pi\times}{\gax, y: B}{\gby}
        \end{equation}
        
        With denotation:
        
        \begin{equation}
            \deno{\ewrel{\P}{\i\pi\times}{\gax, y: B}{\gby}} = (\p \times \Id{B})
        \end{equation}
        
        We need to prove that the following are equal
        
        \begin{align*}
            &lhs =  \deno{\gpetyperelation{\doin{x}{v_1}{(\doin{y}{v_2}{v_3})}}{\M{\e_1\dot\e_2\dot\e_2}{C}}} \\
            & = \bind{\e_1}{\e_2\dot\e_3}{C}\after\T{
                \e_1}{(\bind{\e_2}{\e_3}{C}\after\T{\e_2}{h\after(\p \times \Id{B})}\after\tstrength{\e_2}{(\G\times A)}{B}\after\pr{\Id{(\G\times A)}}{g}
                )}\after\tstrength{\e_1}{\G}{A}\after\pr{\idg}{f}\\
            & rhs = \deno{\gpetyperelation{\doin{y}{(\doin{x}{v_1}{v_2})}{v_3}}{\M{\e_1\dot\e_2\dot\e_2}{C}}}  \\
            & = \bind{\e_1\dot\e_2}{\e_3}{C}\after\T{\e_1\dot\e_2}{(h)}\after\tstrength{\e_1\dot\e_2}{\G}{B}\after\pr{\idg}{(\bind{\e_1}{\e_2}{B}\after\T{\e_1}{g}\after\tstrength{\e_1}{\G}{A}\after\pr{\idg}{f})} \\
        \end{align*}
        Let's look at fragment $F$ of $rhs$.
        \begin{equation}
            F = \tstrength{\e_1\dot\e_2}{\G}{B}\after\pr{\idg}{(\bind{\e_1}{\e_2}{B}\after\T{\e_1}{g}\after\tstrength{\e_1}{\G}{A}\after\pr{\idg}{f})}
        \end{equation}
        
        So 
        \begin{equation}
            rhs = \bind{\e_1\dot\e_2}{\e_3}{C}\after\T{\e_1\dot\e_2}{(h)}\after F
        \end{equation}
        
        \scalebox{.65}{\parbox{1.4\linewidth}{
        \begin{align*}
                F & = \tstrength{\e_1\dot\e_2}{\G}{B}\after(\idg\times\bind{\e_1}{\e_2}{B})\after(\idg\times\T{\e_1}{g})\after\pr{\idg}{\tstrength{\e_1}{\G}{A}\after\pr{\idg}{f}}\\
                &  = \bind{\e_1}{\e_2}{(\G\times B)} \after \T{\e_1}{\tstrength{\e_2}{\G}{B}} \after\tstrength{\e_1}{\G}{(\T{\e_2}{B})} \after (\idg \after\T{\e_1}{g})\after\pr{\idg}{\tstrength{\e_1}{\G}{A}\after\pr{\idg}{f}}\qt{By tensor strength commuting with the bind N-T}
                \\ & = \bind{\e_1}{\e_2}{(\G \times B))} \after \T{\e_1}{(\tstrength{\e_2}{\G}{B} \after (\idg \times g))} \after \tstrength{\e_1}{\G}{(\G \times A)} \after \pr{\idg}{\tstrength{\e_1}{\G}{A}\after\pr{\idg}{f}}\qt{By naturality of t-strength}
        \end{align*}
        }}

        Since
        $
            rhs = \bind{\e_1\dot\e_2}{\e_3}{C}\after\T{\e_1\dot\e_2}{(h)}\after F
        $, 
        
        \scalebox{.8}{\parbox{1.2\linewidth}{
        \begin{align*}
                rhs = &\bind{\e_1\dot\e_2}{\e_3}{C}\after\T{\e_1\dot\e_2}{(h)}\after\bind{\e_1}{\e_2}{(\G \times B))} \after \T{\e_1}{(\tstrength{\e_2}{\G}{B} \after (\idg \times g))} \after \tstrength{\e_1}{\G}{(\G \times A)} \after \pr{\idg}{\tstrength{\e_1}{\G}{A}\after\pr{\idg}{f}}\\
                = &\bind{\e_1\dot\e_2}{\e_3}{C}\after\bind{\e_1}{\e_2}{(\T{\e_3}{C})}\after\T{\e_1}{(\T{\e_2}{(h)} \after \tstrength{\e_2}{\G}{B} \after (\idg \times g))} \after \tstrength{\e_1}{\G}{(\G \times A)} \after \pr{\idg}{\tstrength{\e_1}{\G}{A}\after\pr{\idg}{f}}\qt{Naturality of $\mu$}\\
                = & \bind{\e_1}{\e_2\dot\e_3}{C}\after\T{\e_1}{(\bind{\e_2}{\e_3}{C}\after \T{\e_2}{(h)} \after \tstrength{\e_2}{\G}{B} \after (\idg \times g))} \after \tstrength{\e_1}{\G}{(\G \times A)} \after \pr{\idg}{\tstrength{\e_1}{\G}{A}\after\pr{\idg}{f}}        
        \end{align*}
        }}
        
        Let's now look at the fragment $G$ of $rhs$
        \begin{equation}
            G = \T{\e_1}{(\idg\times g)}\after\tstrength{\e_1}{\G}{(\G\times A)}\after\pr{\idg}{\tstrength{\e_1}{\G}{A}\after\pr{\idg}{f}}
        \end{equation}
        
        So
        \begin{equation}
            rhs = \bind{\e_1}{\e_2\dot\e_3}{C}\after\T{\e_1}{(\bind{\e_2}{\e_3}{C}\after \T{\e_2}{(h)} \after \tstrength{\e_2}{\G}{B})}\after G
        \end{equation}
        
        
        By folding out the $\pr{...}{...}$, we have
        \begin{equation}
            G = \T{\e_1}{(\idg\times g)}\after\tstrength{\e_1}{\G}{\G \times A}\after(\idg\times\tstrength{\e_1}{\G}{A})\after\pr{\idg}{\pr{\idg}{f}}
        \end{equation}
        
        Since tensor strength commutes with, $\alpha$, the associativity natural transformation on binary products, the following commutes:


        {
            \centering
            \begin{tikzcd}[ampersand replacement=\&]
                \G 
                \arrow [r, "\pr{\idg}{\pr{\idg}{f}}"]
                \&
                \G\times(\G \times \T{\e_1}{A}) 
                \arrow [d, "\idg\times\tstrength{\e_1}{\G}{A}"]
                \&
                (\G \times \G) \times\T{\e_1}{A} 
                \arrow [l, "\alpha_{\G, \G, (\T{\e_1}{A})}"]
                \arrow [d, "\tstrength{\e_1}{(\G \times \G)}{A}"]
                \\
                \&
                \G \times \T{\e_1}{(\G \times A)}
                \arrow [d, "\tstrength{\e_1}{\G}{\G \times A}"]
                \& \T{\e_1}{((\G \times \G)\times A)} 
                \arrow [ld, "\T{\e_1}{\alpha_{\G, \G, A}}"]
                \\
                \& \T{\e_1}{(\G \times (\G \times A))}
            \end{tikzcd}
        }
        
        
        Where $\alpha: ((\_ \times \_) \times \_) \rightarrow (\_ \times (\_ \times \_))$ is a natural isomorphism.
        
        \begin{align*}
            \alpha & = \pr{\p\after\p}{\pr{\pp\after\p}{\pp}}\\
            \alpha^{-1} &= \pr{\pr{\p}{\p\after\pp}}{\pp\after\pp}
        \end{align*}
        
        So:
        
        \scalebox{.8}{\parbox{1.2\linewidth}{
        \begin{align*}
                G = & \T{\e_1}{((\idg\times g)\after \alpha_{\G, \G, A})} \after \tstrength{\e_1}{(\G\times\G)}{A}\after\alpha^{-1}_{\G, \G, (\T{\e_1}{A})}\after\pr{\idg}{\pr{\idg}{f}} \\
                = & \T{\e_1}{((\idg\times g)\after \alpha_{\G, \G, A})} \after \tstrength{\e_1}{(\G\times\G)}{A}\after(\pr{\idg}{\idg}\times \Id{\T{\e_1}{A}})\after\pr{\idg}{f}\qt{By definition of $\alpha$ and products}\\
                = & \T{\e_1}{((\idg\times g)\after \alpha_{\G, \G, A}\after(\pr{\idg}{\idg}\times \Id{A}))}\after\tstrength{\e_1}{\G}{A}\after\pr{\idg}{f}\qt{By tensor strength's left-naturality}\\
                = & \T{\e_1}{((\p\times\Id{\T{\e_2}{B}})\after\pr{\Id{(\G\times A)}}{g})}\after\tstrength{\e_1}{\G}{A}\after\pr{\idg}{f}
        \end{align*}
        }}

        Since 

        \scalebox{.8}{\parbox{1.2\linewidth}{
        \begin{equation}
        rhs = \bind{\e_1}{\e_2\dot\e_3}{C}\after\T{\e_1}{(\bind{\e_2}{\e_3}{C}\after \T{\e_2}{(h)} \after \tstrength{\e_2}{\G}{B})}\after G
        \end{equation}
        }}
        
        We have

        \scalebox{.7}{\parbox{1.3\linewidth}{
            \begin{align*}
                rhs ={} & \bind{\e_1}{\e_2\dot\e_3}{C}\after\T{\e_1}{(\bind{\e_2}{\e_3}{C}\after \T{\e_2}{(h)} \after \tstrength{\e_2}{\G}{B}\after(\p\times\Id{\T{\e_2}{B}})\after\pr{\Id{(\G\times A)}}{g})}\after\tstrength{\e_1}{\G}{A}\after\pr{\idg}{f}
            \\
                ={} & \bind{\e_1}{\e_2\dot\e_3}{C}\after\T{\e_1}{(\bind{\e_2}{\e_3}{C}\after \T{\e_2}{(h \after (\p\times\Id{B}))}\after\tstrength{\e_2}{(\G\times A)}{B}\after\pr{\Id{(\G\times A)}}{g})}\after\tstrength{\e_1}{\G}{A}\after\pr{\idg}{f}\qt{By Left-Tensor Strength}
            \\
            ={} & lhs \qt{Woohoo!}
        \end{align*}
        }}
        
        
        %%%%%% END CASE ASSOCIATIVE %%%%%%%
        
        \case{Lambda-Eta}
        Let 
        \begin{equation}
            f = \deno{\gpetyperelation{v}{\ab}} : \G \rightarrow (B)^{A}
        \end{equation}
        
        By weakening, we have
        
        \begin{align*}
            \deno{\etyperelation{\P}{\gax}{v}{\ab}} & = f \after \p : \G \times A \rightarrow (B)^A \\
            \deno{\etyperelation{\P}{\gax}{\apply{v}{x}}{B}} & = \app\after\pr{f \after \p}{\pp} \\
        \end{align*}
        
        Hence, we have 
        \begin{align*}
                \deno{\gpetyperelation{\lam{x}{A}{(\apply{v}{x})}}{\ab}} & = \cur{\app\after\pr{f \after \p}{\pp}}\\
                \app\after(\deno{\gpetyperelation{\lam{x}{A}{(\apply{v}{x})}}{\ab}} \times \Id{A}) & = \app \after(\cur{\app\after\pr{f \after \p}{\pp}}\times \Id{A})\\
                & = \app \after\pr{f\after\p}{\pp}\\
                & = \app\after(f\times\Id{A})
        \end{align*}
        
        Hence, by the fact that $\cur{f}$ is unique in a cartesian closed category, 
        
        \begin{equation}
            \deno{\gpetyperelation{\lam{x}{A}{(\apply{v}{x})}}{\ab}} = f =\deno{\gpetyperelation{v}{\ab}}
        \end{equation}
        
        \case{If-True}
        Let
        \begin{align*}
            f = &\deno{\gpetyperelation{v_1}{A}}\\
            g = &\deno{\gpetyperelation{v_2}{A}}\\
        \end{align*}
        
        Then
        \begin{align*}
                \deno{\gpetyperelation{\pifthenelse{A}{v}{v_1}{v_2}}{A}} & = \ifMorph{\inl\after\term{\G}}{f}{g} \\
                & = \app\after((\cur{f\after\pp}\after\term{\G})\times\idg)\after\diag{\G}\\
                & = \app\after(\cur{f\after\pp}\times\idg)\after(\term{\G}\times\idg)\after\diag{\G}\\
                & = f\after\pp\after\pr{\term{\G}}{\idg}\\
                & = f \\
                & = \deno{\gpetyperelation{v_1}{A}}\\
        \end{align*}
        
        
        \case{If-False}
        Let
        \begin{align*}
            f = &\deno{\gpetyperelation{v_1}{A}}\\
            g = &\deno{\gpetyperelation{v_2}{A}}\\
        \end{align*}
        
        Then
        \begin{align*}
                \deno{\gpetyperelation{\pifthenelse{A}{v}{v_1}{v_2}}{A}} & = \ifMorph{\inr\after\term{\G}}{f}{g} \\
                & = \app\after((\cur{g\after\pp}\after\term{\G})\times\idg)\after\diag{\G}\\
                & = \app\after(\cur{g\after\pp}\times\idg)\after(\term{\G}\times\idg)\after\diag{\G}\\
                & = g\after\pp\after\pr{\term{\G}}{\idg}\\
                & = g \\
                & = \deno{\gpetyperelation{v_2}{A}}\\
        \end{align*}
        
        \case{If-Eta}
            Let 
            \begin{align*}
                f & = \deno{\etyperelation{\P}{\G}{v_1}{\B}} \\
                g & = \deno{\etyperelation{\P}{\G, x: \B}{v_2}{A}} \\
            \end{align*}
        
        
            Then by the substitution theorem,
            \begin{align*}
                \deno{\etyperelation{\P}{\G}{v_2\ssub{x}{\t}}{A}} & = g\after\pr{\idg}{\inl_\1\after{\term{\G}}}\\
                \deno{\etyperelation{\P}{\G}{v_2\ssub{x}{\f}}{A}} & = g\after\pr{\idg}{\inr_\1\after{\term{\G}}}\\
                \deno{\gpetyperelation{v_2\ssub{x}{v_1}}{A}} & = g\after\pr{\idg}{f}
            \end{align*}
        
            Hence we have (Using the diagonal and twist morphisms):
         
            \scalebox{.65}{\parbox{1.45\linewidth}{
            \begin{align*}
                \deno{&\gpetyperelation{\pifthenelse{A}{v_1}{v_2\ssub{x}{\t}}{v_2\ssub{x}{\f}}}{A}}  \\ 
                & = \ifMorph{f}{g\after\pr{\idg}{\inl_\1\after{\term{\G}}}}{g\after\pr{\idg}{\inr_\1\after{\term{\G}}}}\\
                & = \app\after((\fld{
                    \cur{g\after\pr{\pp}{\inl_1\after\term{\G}\after\pp}}
                }{
                    \cur{g\after\pr{\pp}{\inr_1\after\term{\G}\after\pp}}
                }\after f)\times \idg)\after\diag{\G}\qt{Pairing property}\\
                &= \app\after((\fld{
                    \cur{g\after\pr{\pp}{\inl_1\after\term{\G}\after\p}}
                }{
                    \cur{g\after\pr{\pp}{\inr_1\after\term{\G}\after\p}}
                }\after f)\times \idg)\after\diag{\G}\qt{Terminal is unique}\\
                &= \app\after((\fld{
                    \cur{g\after(\idg\times (\inl_\1\after\term{\1}))\after\twist{\1}{\G}}
                }{
                    \cur{g\after(\idg\times (\inr_\1\after\term{\1}))\after\twist{\1}{\G}}
                }\after f)\times \idg)\after\diag{\G}\qt{Twist definition}\\
                & = \app\after((\fld{
                    \cur{g\after(\idg\times \inl_\1)\after\twist{\1}{\G}}
                }{
                    \cur{g\after(\idg\times \inr_\1)\after\twist{\1}{\G}}
                }\after f)\times \idg)\after\diag{\G}\qt{Identity = $\Id{\1}$}\\
                & = \app\after((\fld{
                    \cur{g\after\twist{\1+\1}{\G}\after(\inl_\1\times \idg)}
                }{
                    \cur{g\after\twist{\1+\1}{\G}\after(\inr_\1\times \idg)}
                }\after f)\times \idg)\after\diag{\G}\qt{Twist commutivity}\\
                & = \app\after((\fld{
                    \cur{g\after\twist{\1+\1}{\G}}\after\inl_\1
                }{
                    \cur{g\after\twist{\1+\1}{\G}}\after\inr_\1
                }\after f)\times \idg)\after\diag{\G}\qt{Exponential property}\\
                & = \app\after((\cur{g\after\twist{\1+\1}{\G}}\after\fld{
                    \inl_\1
                }{
                    \inr_\1
                }\after f)\times \idg)\after\diag{\G}\qt{Factoring out $\cur{..}$}\\
                & = \app\after((\cur{g\after\twist{\1+\1}{\G}}\after f)\times \idg)\after\diag{\G}\qt{Since $\fld{\inl}{\inr}$ is the identity}\\
                & = \app\after(\cur{g\after\twist{\1+\1}{\G}}\times \idg)\after (f\times \idg)\after\diag{\G}\qt{Factoring}\\
               & = g\after\twist{\1+\1}{\G} \after (f\times \idg)\after\diag{\G} \qt{Definition of $\app,\cur{..}$}\\
               & = g\after(\idg\times f)\after \twist{\G}{\G} \after\diag{\G} \qt{Twist commutivity}\\
               & = g\after(\idg\times f) \after\pr{\idg}{\idg} \qt{Twist, diagonal defintions} \\
               & = g\after\pr{\idg}{f} \\
               & = \deno{\gpetyperelation{v_2\ssub{x}{v_1}}{A}} \\
            \end{align*}
            }}
        
            \case{Effect-Beta}
            let 
            \begin{align*}
                h & = \deno{\typerelation{\P}{\e}{\effect}}
                \\
                f & = \deno{\etyperelation{\P,\a}{\G}{v}{A}}
                \\
                A &= \deno{\typerelation{\P,\a}{A\ssub{\a}{\a}}{\type}}
            \end{align*}
        
            Then
        
            \begin{equation}
                \deno{\gpetyperelation{\elam{\a}{v}}{\all{\a}{A}}} = \bar{f}\\
            \end{equation}
        
            So
            \begin{align*}
                \deno{\gpetyperelation{\eapp{(\elam{\a}{v})}{\e}}{\all{\a}{A}}} & = \pr{\Id{I}}{h}\star(\e_A)\after\bar{f}\\
                & = \pr{\Id{I}}{h}\star(\e_A)\after\pr{\Id{I}}{h}\star(\pstar(\bar{f}))\qt{Identity functor}\\
                &= \pr{\Id{I}}{h}\star(\e_A\after\pstar(\bar{f}))\\
                &= \pr{\Id{I}}{h}\star(f)\qt{By adjunction}\\
                &= \deno{\gpetyperelation{v\ssub{\a}{\e}}{A\ssub{\a}{e}}}\qt{By substitution theorem}\\
            \end{align*}
        
            \case{Effect-Eta}

            Let \begin{align*}
                f & = \deno{\gpetyperelation{v}{\all{\a}{A}}}\\
                A & = \deno{\typerelation{\P,\a}{A}{\type}}
            \end{align*}
        
            so
            \begin{align*}
                \deno{\gpetyperelation{\elam{\a}{(\eapp{v}{\a})}}{\all{\a}{A}}} & = \bar{\deno{\etyperelation{\P,\a}{\G}{\eapp{v}{\a}}{{A}}}} \\
                & = \bar{\pr{\Id{I\times U}}{\pp}\star(\e_{\deno{\typerelation{\P,\a,\b}{A\ssub{\a}{\b}}{\type}}})\after\pstar(f)}
            \end{align*}
        
            Let's look at $\deno{\typerelation{\P,\a,\b}{A\ssub{\a}{B}}{\type}}$.
        
            We have the weakening:
            \begin{equation}
                \wrel{\i\pi\x}{\P,\a,\b}{\P,\b}
            \end{equation}
        
            So by the weakening theorem on type denotations:
        
            \begin{align*}
                \deno{\typerelation{\P,\a,\b}{A\ssub{\a}{\b}}{\type}} & = (\p\times\Id{U})\star\deno{\typerelation{\P,\b}{A\ssub{\a}{\b}}{\type}}\\
                \allIU(\deno{\typerelation{\P,\a,\b}{A\ssub{\a}{\b}}{\type}}) & = \allI(\p\times\Id{U})\star(\deno{\typerelation{\P,\b}{A\ssub{\a}{\b}}{\type}})\\
                & = \pstar\allI(\deno{\typerelation{\P,\b}{A\ssub{\a}{\b}}{\type}})\\
                \e_{\deno{\typerelation{\P,\a,\b}{A\ssub{\a}{\b}}{\type}}} & = \widehat{\Id{\pstar\allI(\deno{\typerelation{\P,\b}{A\ssub{\a}{\b}}{\type}})}}\\
                & =\widehat{\Id{\pstar\allI A}}\\
                & = \widehat{\pstar(\Id{\allI A})}\\
                & = \widehat{\pstar(\bar{\e_{A}})}\\
                & = \widehat{\bar{(\p\times\Id{U})\star(\e_{A})}}\\
                & = (\p\times\Id{U})\star(\e_A) 
            \end{align*}
        
            So
            \begin{align*}
                \deno{\gpetyperelation{\elam{\a}{(\eapp{v}{\a})}}{\all{\a}{A}}} & = \bar{\pr{\Id{I\times U}}{\pp}\star(\e_{\deno{\typerelation{\P,\a,\b}{A\ssub{\a}{\b}}{\type}}})\after\pstar(f)}\\
                & = \bar{\pr{\Id{I\times U}}{\pp}\star((\p\times\Id{U})\star(\e_A))\after\pstar(f)}\\
                & = \bar{\pr{\p}{\pp}\star(\e_A)\after\pstar(f)}\\
                &= \bar{\Id{I\times U}\star(\e_A)\after\pstar(f)}\\
                & = \bar{\e_A\after \pstar(f)}\qt{By adjunction}\\
                & = f
            \end{align*}
        
        \subsection{Congruences}
        These cases can be proved fairly mechanically by assuming the preconditions, using induction to prove that the matching pairs of Subexpressions have equal denotations, then constructing the denotations of the expressions using the equal denotations which gives trivially equal denotations.
        
        \case{Fn}
            By inversion, we have $\eberelation{\P}{\gax}{v_1}{v_2}{B}$
            By induction, we therefore have $\deno{\etyperelation{\P}{\gax}{v_1}{B}} = \deno{\etyperelation{\P}{\gax}{v_2}{B}}$
        
            Then let
            \begin{equation}
                f = \deno{\etyperelation{\P}{\gax}{v_1}{B}} = \deno{\etyperelation{\P}{\gax}{v_2}{B}}
            \end{equation}
        
            And so
            \begin{equation}
                \deno{\gpetyperelation{\lam{x}{A}{v_1}}{\ab}} = \cur{f} = \deno{\gpetyperelation{\lam{x}{A}{v_2}}{\ab}}
            \end{equation}
        
        
        \case{Return}
        By inversion, we have $\gpeberelation{v_1}{v_2}{A}$
        By induction, we therefore have $\deno{\gpetyperelation{v_1}{A}} = \deno{\gpetyperelation{v_2}{A}}$
        
        Then let
        \begin{equation}
            f = \deno{\gpetyperelation{v_1}{A}} = \deno{\gpetyperelation{v_2}{A}}
        \end{equation}
        
        And so
        \begin{equation}
            \deno{\gpetyperelation{\return{v_1}}{\M{\1}{A}}} = \point{A}\after f = \deno{\gpetyperelation{\return{v_2}}{\M{\1}{A}}}
        \end{equation}
        
        
        \case{Apply}
        By inversion, we have $\gpeberelation{v_1}{v_1'}{\ab}$ and $\gpeberelation{v_2}{v_2'}{A}$
        By induction, we therefore have $\deno{\gpetyperelation{v_1}{\ab}}= \deno{\gpetyperelation{v_1'}{\ab}}$ and $\deno{\gpetyperelation{v_2}{A}} = \deno{\gpetyperelation{v_2'}{A}}$
        
        Then let
        \begin{align*}
            f &{}= \deno{\gpetyperelation{v_1}{\ab}} = \deno{\gpetyperelation{v_1'}{\ab}}\\
            g &{}= \deno{\gpetyperelation{v_2}{A}} = \deno{\gpetyperelation{v_2'}{A}}
        \end{align*}
        
        
        
        And so
        \begin{equation}
            \deno{\gpetyperelation{\apply{v_1}{v_2}}{B}} = \app\after\pr{f}{g} = \deno{\gpetyperelation{\apply{v_1'}{v_2'}}{B}}
        \end{equation}
        
        
        \case{Bind}
        By inversion, we have $\gpeberelation{v_1}{v_1'}{\M{e_1}{A}}$ and $\eberelation{\P}{\gax}{v_2}{v_2'}{\M{\e_2}{B}}$
        By induction, we therefore have $\deno{\gpetyperelation{v_1}{\M{e_1}{A}}}= \deno{\gpetyperelation{v_1'}{\M{e_1}{A}}}$ and $\deno{\etyperelation{\P}{\gax}{v_2}{\M{\e_2}{B}}} = \deno{\etyperelation{\P}{\gax}{v_2'}{\M{\e_2}{B}}}$
        
        Then let
        \begin{align*}
            f &{}=\deno{\gpetyperelation{v_1}{\M{\e_1}{A}}}= \deno{\gpetyperelation{v_1'}{\M{\e_1}{A}}}\\
            g &{}=\deno{\etyperelation{\P}{\gax}{v_2}{\M{\e_2}{B}}} = \deno{\etyperelation{\P}{\gax}{v_2'}{\M{\e_2}{B}}}
        \end{align*}
        
        
        
        And so
        \begin{align*}
                \deno{\gpetyperelation{\doin{x}{v_1}{v_2}}{\M{\e_1\dot\e_2}{A}}} & = \bind{\e_1}{\e_2}{B} \after \T{\e_1}{g} \after \tstrength{\e_1}{\G}{A} \after \pr{\idg}{f}\\
                 & = \deno{\gpetyperelation{\doin{x}{v_1}{v_2}}{\M{\e_1\dot\e_2}{A}}}
        \end{align*}

        \case{If}
        By inversion, we have $\gpeberelation{v}{v'}{\B}$, $\gpeberelation{v_1}{v_1'}{A}$ and $\gpeberelation{v_2}{v_2'}{A}$
        By induction, we therefore have $\deno{\gpetyperelation{v}{\B}} = \deno{\gpetyperelation{v'}{\B}}$, $\deno{\gpetyperelation{v_1}{A}}= \deno{\gpetyperelation{v_1'}{A}}$ and $\deno{\etyperelation{\P}{\gax}{v_2}{A}} = \deno{\etyperelation{\P}{\gax}{v_2'}{A}}$
        
        Then let
        \begin{align*}
            f &{}=\deno{\gpetyperelation{v}{\B}} = \deno{\gpetyperelation{v'}{\B}}\\
            g &{}=\deno{\gpetyperelation{v_1}{A}}= \deno{\gpetyperelation{v_1'}{A}}\\
            h &{}=\deno{\etyperelation{\P}{\gax}{v_2}{A}} = \deno{\etyperelation{\P}{\gax}{v_2'}{A}}
        \end{align*}
        
        
        
        And so
        \begin{align*}
                \deno{\gpetyperelation{\pifthenelse{A}{v}{v_1}{v_2}}{A}} & = \ifMorph{f}{g}{h}\\
                 & = \deno{\gpetyperelation{\pifthenelse{A}{v'}{v_1'}{v_2'}}{A}}
        \end{align*}

        \case{Subtype}
        By inversion, we have $\gpeberelation{v_1}{v_2}{A}$, and $A \subtypep B$ 
        By induction, we therefore have $\deno{\gpetyperelation{v_1}{A}} = \deno{\gpetyperelation{v_2}{A}}$
        
        Then let
        \begin{align*}
            f &{}=\deno{\gpetyperelation{v_1}{A}} = \deno{\gpetyperelation{v_2}{B}}\\
            g &{}=\deno{A \subtypep B}
        \end{align*}
        
        
        
        And so
        \begin{equation}
                \deno{\gpetyperelation{v_1}{B}} = g \after f = \deno{\gpetyperelation{v_1}{B}}
        \end{equation}
        
        \case{Effect-Gen}
        By inversion, we have $\eberelation{\P,\a}{\G}{v_1}{v_2}{A}$. So by induction, $\deno{\etyperelation{\P,\a}{\G}{v_1}{A}} = \deno{\etyperelation{\P,\a}{\G}{v_2}{A}}$
        
        So \begin{align*}
            \deno{\gpetyperelation{\elam{\a}{v_1}}{\all{\a}{A}}} & = \bar{\deno{\etyperelation{\P,\a}{\G}{v_1}{A}}}\\
            & = \bar{\deno{\etyperelation{\P,\a}{\G}{v_2}{A}}}\\
            & = \deno{\gpetyperelation{\elam{\a}{v_2}}{\all{\a}{A}}} 
        \end{align*}
        
        \case{Effect-Spec}
        
        By inversion, we have $\gpeberelation{v_1}{v_2}{\all{\a}{A}}$ and $\typerelation{\P}{\e}{\effect}$.
        
        So by induction, we have $\deno{\gpetyperelation{v_1}{\all{\a}{A}}} = \deno{\gpetyperelation{v_2}{\all{\a}{A}}}$
        
        So
        \begin{align*}
            \deno{\gpetyperelation{\eapp{v_1}{\e}}{A\ssub{\a}{\e}}} & = \pr{\Id{I}}{\deno{\typerelation{\P}{\e}{\effect}}}\star(\e_{A})\after\deno{\gpetyperelation{v_1}{\all{\a}{A}}}
            \\
            & = \pr{\Id{I}}{\deno{\typerelation{\P}{\e}{\effect}}}\star(\e_{A})\after\deno{\gpetyperelation{v_2}{\all{\a}{A}}}
            \\
            & = \deno{\gpetyperelation{\eapp{v_2}{\e}}{A\ssub{\a}{\e}}}
        \end{align*}
        $$\square$$
    \end{proof}
    
\end{framed}
    
\end{document}