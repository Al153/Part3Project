\documentclass{Report}

%% Don't import the header multiple times

\ifdefined\HEADERIMPORTED
\else
\newcommand\HEADERIMPORTED[0]{This file is HEADERIMPORTED}
\usepackage{amssymb}

\usepackage{amsmath}
\usepackage[a4paper,includeheadfoot,margin=2.54cm]{geometry}

% For typesetting tree rules
\usepackage{mathpartir}

% For colouring code
\usepackage{xcolor}


\usepackage{array}   % for \newcolumntype macro
\usepackage{tikz-cd}
\usepackage{tabstackengine}
\usepackage{breqn}
\usepackage{stmaryrd}

\usepackage{framed}
\usepackage{listings}

\usepackage{amsthm}
%Theorems
\usepackage[utf8]{inputenc}
\usepackage[english]{babel}
\newtheorem{theorem}{Theorem}[section]
\newtheorem{corollary}{Corollary}[theorem]
\newtheorem{lemma}[theorem]{Lemma}
\newtheorem{property}[theorem]{Property}
\theoremstyle{definition}
\newtheorem{definition}{Definition}[section]

\usepackage{tikz}

\definecolor{grey}{rgb}{0.75, 0.75, 0.75}
\definecolor{DarkGreen}{rgb}{0.1, 0.6, 0.1}

\usetikzlibrary{shapes.geometric,fit}
\usetikzlibrary{arrows,automata,positioning}
\usetikzlibrary{decorations.pathreplacing,calc}



\setstackEOL{\cr}
\setstackgap{L}{\normalbaselineskip}

\newcommand\todo[1]{\textbf{TODO: #1}}
\newcommand\needsRef[1]{\textbf{Reference Needed: (#1)}}
\newcommand\fixLayout[1]{\textbf{Fix Layout: #1}}

\newcommand{\s}{\;}
\newcommand{\doin}[3]{\texttt{do}\s #1 \leftarrow #2 \s\texttt{in}\s #3\s}
\newcommand\apply[2]{#1\s#2}
\newcommand{\pifthenelse}[4]{\texttt{if}_{\textcolor{grey}{#1}}\s#2\s \texttt{then}\s #3 \s\texttt{else} \s#4\s}
\newcommand\ifthenelse[5]{\pifthenelse{#1, #2}{#3}{#4}{#5}}
\newcommand\const[1]{\texttt{C}^{\color{grey} #1}}
\newcommand\return[1]{\texttt{return} \s#1\s}


\newcommand\lam[3]{\lambda #1 \colon {\color{grey}#2}. #3\s}
\renewcommand\u[0]{\texttt{()}}
\newcommand{\U}[0]{\texttt{Unit}}
\renewcommand\t[0]{\texttt{true}}
\newcommand\f[0]{\texttt{false}}
\newcommand{\B}[0]{\texttt{Bool}}
\newcommand{\G}[0]{\Gamma}
\newcommand\D{\Delta}


% draw type relations
\newcommand{\typerelation}[3]{{\color{DarkGreen}#1} \vdash #2 \colon {\color{blue}#3}}
\newcommand\wellformed[2]{{\color{DarkGreen}#1}\vdash {\color{blue}#2}}
\newcommand\wellformedok[2]{\ok{{\color{DarkGreen}#1}\vdash {\color{blue} #2}}}



\newcommand{\gtyperelation}[2]{\typerelation{\G}{#1}{#2}}
 

\newcommand\treerulez[1]{\inferrule{ }{#1}}
\newcommand\treeruleI[2]{\inferrule{#1}{#2}}
\newcommand\treeruleII[3]{\inferrule{#1 \\ #2}{#3}}
\newcommand\treeruleIII[4]{\inferrule{#1 \\ #2 \\ #3}{#4}}
\newcommand\treeruleIV[5]{\inferrule{#1 \\ #2 \\ #3 \\ #4}{#5}}
\newcommand\treeruleV[6]{\inferrule{#1 \\ #2 \\ #3 \\ #4 \\ #5}{#6}}

\newcommand\ntreerulez[2]{(\text{#1})\inferrule{ }{#2}}
\newcommand\ntreeruleI[3]{(\text{#1})\inferrule{#2}{#3}}
\newcommand\ntreeruleII[4]{(\text{#1})\inferrule{#2 \\ #3}{#4}}
\newcommand\ntreeruleIII[5]{(\text{#1})\inferrule{#2 \\ #3 \\ #4}{#5}}
\newcommand\ntreeruleIV[6]{(\text{#1})\inferrule{#2 \\ #3 \\ #4 \\ #5}{#6}}
\newcommand\ntreeruleV[7]{(\text{#1})\inferrule{#2 \\ #3 \\ #4 \\ #5 \\ #6}{#7}}

\newcommand\condtreerulez[3]{(\text{#1})\inferrule{ }{#2}(\text{if } #3)}
\newcommand\condtreeruleI[4]{(\text{#1})\inferrule{#2}{#3}(\text{if } #4)}
\newcommand\condtreeruleII[5]{(\text{#1})\inferrule{#2 \\ #3}{#4}(\text{if } #5)}
\newcommand\condtreeruleIII[6]{(\text{#1})\inferrule{#2 \\ #3 \\ #4}{#5}(\text{if } #6)}
\newcommand\condtreeruleIV[7]{(\text{#1})\inferrule{#2 \\ #3 \\ #4 \\ #5}{#6}(\text{if } #7)}
\newcommand\condtreeruleV[8]{(\text{#1})\inferrule{ #2 \\ #3 \\ #4 \\ #5 \\ #6 }{#7}(\text{if } #8)}



\newcommand{\subtype}[0]{\leq\colon}
\newcommand\subeffect[0]{\leq}

\newcommand{\M}[2]{\texttt{M}_{#1}{#2}}

\newcommand\lamtype[3]{#1 \rightarrow \M{#2}{#3}}
\newcommand{\1}[0]{\texttt{1}}

\newcommand\e[0]{\epsilon}

\newcommand{\db}[1]{{\bf [\![}#1{\bf ]\!]}}
\newcommand{\deno}[1]{\db{#1}}
\newcommand\after\circ
\newcommand\term[1]{\left\langle\right\rangle_{#1}}

\newcommand\bindmu[0]{\mu}
\newcommand\point[1]{\eta_{#1}}
\newcommand\bind[3]{\bindmu_{#1, #2, #3}}

\newcommand\T[2]{T_{#1}{#2}}

\newcommand\pr[2]{\langle#1, #2\rangle}
\newcommand\finpr[2]{\langle #1\rangle_{#2}}

\newcommand\strengtht[0]{\texttt{t}}
% tensor strength Nat-tran
\newcommand\tstrength[3]{\strengtht_{#1, #2, #3}}

% Id morphism
\newcommand\Id[1]{\texttt{Id}_{#1}}

\newcommand\idg[0]{\Id{\G}}
% beta-eta equivalence
\newcommand\beequiv[0]{\approx}
% Substitutions
\newcommand\si{\sigma}

\newcommand{\sub}[1]{\left[#1\right]}
\newcommand{\ssub}[2]{\left[#2 / #1\right]}
\newcommand{\ssi}[0]{\sub{\si}}

% beta-eta equivalence relation
\newcommand{\berelation}[4]{\typerelation{#1}{#2 \beequiv #3}{#4}}
\newcommand{\gberelation}[3]{\gtyperelation{#1 \beequiv #2}{#3}}


% Shortcuts for denotational equality
\newcommand{\denoequality}[4]{\deno{\typerelation{#1}{#2}{#4}} = \deno{\typerelation{#1}{#3}{#4}}}
\newcommand{\gdenoequality}[3]{\denoequality{\G}{#1}{#2}{#3}}

% Shorthand for monad types
\newcommand\mea[0]{\M{\e}{A}}
\newcommand\meb[0]{\M{\e}{B}}
\newcommand\mec[0]{\M{\e}{C}}

\newcommand\tea[0]{\T{\e}{A}}
\newcommand\teb[0]{\T{\e}{B}}
\newcommand\tec[0]{\T{\e}{C}}


\newcommand\moa[0]{\M{\1}{A}}
\newcommand\mob[0]{\M{\1}{B}}
\newcommand\moc[0]{\M{\1}{C}}

\newcommand\toa[0]{\T{\1}{A}}
\newcommand\tob[0]{\T{\1}{B}}
\newcommand\toc[0]{\T{\1}{C}}

\newcommand\aeb[0]{\lamtype{A}{\e}{B}}

% Shorthand for Gammas
\newcommand{\gax}[0]{\G, x\colon A}
\newcommand{\gby}[0]{\G, y\colon B}

% reduction function
\newcommand{\reduce}[0]{reduce}



% Combinators for building delta-based tree proof terms
\newcommand{\deltavrule}[4]{
    \ntreeruleII{Subtype}{\treeruleI{\D}{\typerelation{#1}{#2}{#3}}}{#3 \subtype #4}{\typerelation{#1}{#2}{#4}}}

\newcommand{\deltavruleprime}[4]{
    \ntreeruleII{Subtype}{\treeruleI{\D'}{\typerelation{#1}{#2}{#3}}}{#3 \subtype #4}{\typerelation{#1}{#2}{#4}}}

\newcommand{\deltavruleprimeprime}[4]{
        \ntreeruleII{Subtype}{\treeruleI{\D'}{\typerelation{#1}{#2}{#3}}}{#3 \subtype #4}{\typerelation{#1}{#2}{#4}}}
    
\newcommand{\deltacrule}[6]{
            \ntreeruleII{Subeffect}{\treeruleI{\D}{\typerelation{#1}{#2}{\M{#3}{#4}}}}{\subeffecttree{#3}{#4}{#5}{#6}}{\typerelation{#1}{#2}{\M{#5}{#6}}}}
\newcommand{\deltacruleprime}[6]{
    \ntreeruleII{Subeffect}{\treeruleI{\D'}{\typerelation{#1}{#2}{\M{#3}{#4}}}}{
    \subeffecttree{#3}{#4}{#5}{#6}}{\typerelation{#1}{#2}{\M{#5}{#6}}}}
\newcommand{\deltacruleprimeprime}[6]{
    \ntreeruleII{Subtype}{\treeruleI{\D''}{\typerelation{#1}{#2}{\M{#3}{#4}}}}{
        \subeffecttree{#3}{#4}{#5}{#6}}{\typerelation{#1}{#2}{\M{#5}{#6}}}}
                            

\newcommand{\p}[0]{\pi_1}
\newcommand{\pp}[0]{\pi_2}

% short-hands for weakening
\newcommand{\wrel}[3]{#1 \colon {\color{blue}#2} \triangleright {\color {blue} #3}}
\newcommand{\ok}[1]{{\color{blue} #1} \texttt{ Ok}}
\renewcommand\i[0]{\iota}
\newcommand\w{\omega}
\newcommand\dom[1]{\texttt{dom}(#1)}
\newcommand\x{\times}


\newcommand\fev[1]{fev(#1)}
\newcommand\union[0]{\cup}


% Combinators to build tree proofs
\newcommand{\truleconst}[0]{\ntreeruleI{Const}{\ok{\G}}{\gtyperelation{\const{A}}{A}}}
\newcommand{\truleunit}[0]{\ntreeruleI{Unit}{\ok{\G}}{\typerelation{\G}{\u}{\U}}}
\newcommand{\truletrue}[0]{\ntreeruleI{True}{\ok{\G}}{\typerelation{\G}{\t}{\B}}}
\newcommand{\trulefalse}[0]{\ntreeruleI{False}{\ok{\G}}{\typerelation{\G}{\f}{\B}}}


\newcommand{\E}[0]{\mathbb{E}}
\renewcommand{\dot}{\cdot}
\newcommand{\gens}[0]{\colon\colon=}
\newcommand{\nil}[0]{\diamond}
\newcommand{\ground}[0]{\gamma}

% Terminal object of C
\newcommand{\terminal}[0]{\texttt{\1}}

% The category C
\newcommand{\C}[0]{\mathbb{C}}
\newcommand{\Cz}[0]{\C_0}
\newcommand\DC[0]{\mathbb{D}}

% The category of locally-small categories
\newcommand{\Cat}[0]{\texttt{Cat}}
% Sub-effect Nat-trans
\newcommand{\dse}[2]{\db{#1 \subeffect #2}}

\newcommand\app[0]{\texttt{app}}
\newcommand\cur[1]{\texttt{cur}(#1)}
\newcommand{\ifnt}[1]{\texttt{If}_{#1}}


\newcommand{\setto}{\colon=}
\newcommand{\fv}[1]{\texttt{fv}(#1)}

% shorthand for inserting text to equations
\newcommand\qt[1]{\quad\text{#1}}

% Co-product short-hands
\newcommand\inr[0]{\texttt{inr}}
\newcommand\inl[0]{\texttt{inl}}
    
\newcommand\fld[2]{\left[#1,#2\right]}
\newcommand{\diag}[1]{\delta_{#1}}
\newcommand{\twist}[2]{\tau_{#1, #2}}

\newcommand\ifMorph[3]{\app\after((\fld{\cur{#2\after\pp}}{\cur{#3\after\pp}}\after #1)\times \idg)\after \diag{\G}}


% Polymorphic short-hands
\newcommand\elam[2]{\Lambda #1. #2}
\newcommand{\eapp}[2]{#1\s#2}
\renewcommand{\a}[0]{\alpha}
\newcommand{\all}[2]{\forall #1. #2}
\renewcommand{\P}[0]{\Phi}

\renewcommand{\b}[0]{\beta}
\newcommand{\g}[0]{\gamma}
\renewcommand\d[0]{\delta}
\newcommand\oke[2]{\wellformedok{#1}{#2}}
\newcommand\etyperelation[4]{\typerelation{#1\mid#2}{#3}{#4}}
\newcommand{\gpetyperelation}[2]{\etyperelation{\P}{\G}{#1}{#2}}
\newcommand{\gppetyperelation}[2]{\etyperelation{\P'}{\G}{#1}{#2}}


\newcommand{\eberelation}[5]{\berelation{#1\mid#2}{#3}{#4}{#5}}
\newcommand{\gpeberelation}[3]{\berelation{\P\mid\G}{#1}{#2}{#3}}
\newcommand{\gppeberelation}[3]{\berelation{\P'\mid\G}{#1}{#2}{#3}}

\newcommand{\dotp}[0]{\dot_\P}
\newcommand{\fntype}[2]{#1\rightarrow #2}
\newcommand{\ab}[0]{\fntype{A}{B}}

\newcommand\wrelw[2]{\wrel{\w}{#1}{#2}}
\renewcommand\proof[0]{\paragraph{Proof:}}
\newcommand{\case}[1]{\paragraph{Case #1:}}
\newcommand{\subcase}[1]{\subparagraph{Case: #1}}
\newcommand\bi[0]{By inversion}

%pre-filled effect-weakening relations
\newcommand\ewrel[4]{\wellformed{#1}{\color{black}\wrel{#2}{#3}{#4}}}
\newcommand\pewrel[3]{\ewrel{\P}{#1}{#2}#3}
\newcommand\ppewrel[3]{\ewrel{\P'}{#1}{#2}#3}

\newcommand\subtypep[0]{\subtype_\P}
\newcommand\subtypepp[0]{\subtype_{\P'}}
\newcommand\subeffectp[0]{\subeffect_{\P}}
\newcommand\subeffectpp[0]{\subeffect_{\P'}}
\newcommand\subeffectn[0]{\subeffect_{n}}
\newcommand\subeffectz[0]{\subeffect_{0}}

\newcommand{\allI}[0]{\forall_I}
\newcommand{\allII}[0]{\forall_{I'}}
\newcommand\allIU[0]{\forall_{I\times U}}
\newcommand\type[0]{\texttt{Type}}
\newcommand\effect[0]{\texttt{Effect}}
\newcommand\ciw[0]{\C(I, W)}
\newcommand\ciu[0]{\C(I, U)}
\newcommand\ciuw[0]{\C(I\times U, W)}
\newcommand\cipw[0]{\C(I', W)}
\newcommand\cipu[0]{\C(I', U)}
\newcommand\ciuu[0]{\C(I\times U, U)}
\newcommand\cii[0]{\C(I', I)}
\newcommand\Eff[0]{\texttt{Eff}}
\newcommand\Mul[0]{\texttt{Mul}}
\newcommand\singleton[0]{\ast}
\renewcommand\star[0]{^*}
\renewcommand\bar[1]{\overline{#1}}

\newcommand\subtypeg[0]{\subtype_\g}
\newcommand\subtypepa[0]{\subtype_{\P, \a}}
\newcommand\subtypeppa[0]{\subtype_{\P', \a}}

\newcommand\subtypez[0]{\subtype_{0}}
\newcommand\subtypen[0]{\subtype_{n}}

\usepackage{scalerel,stackengine}
\stackMath
\renewcommand\widehat[1]{%
\savestack{\tmpbox}{\stretchto{%
  \scaleto{%
    \scalerel*[\widthof{\ensuremath{#1}}]{\kern.1pt\mathchar"0362\kern.1pt}%
    {\rule{0ex}{\textheight}}%WIDTH-LIMITED CIRCUMFLEX
  }{\textheight}% 
}{2.4ex}}%
\stackon[-6.9pt]{#1}{\tmpbox}%
}
\parskip 1ex

\newcommand\pstar[0]{\p\star}

\newcommand\edeltavrule[5]{\deltavrule{#1 \mid #2}{#3}{#4}{#5}}

\newcommand\subeffecttreep[4]{\ntreeruleII{Effect}{
    #1\subeffectp #3}{#2 \subtypep #4
}{\M{#1}{
    #2
}\subtypep\M{#3}{#4}}}
\newcommand\subeffecttree[4]{\ntreeruleII{Effect}{
    #1\subeffect #3}{#2 \subtype #4
}{\M{#1}{
    #2
}\subtype\M{#3}{#4}}}


\newcommand{\edeltavruleprime}[5]{
        \deltavruleprime{#1\mid #2}{#3}{#4}{#5}}
    
\newcommand{\edeltavruleprimeprime}[5]{
        \deltavruleprimeprime{#1\mid #2}{#3}{#4}{#5}}
    
\newcommand{\edeltacrule}[6]{
            \ntreeruleII{Subtype}{
                \treeruleI{
                    \D
                }{
                    \typerelation{\P\mid#1}{#2}{\M{#3}{#4}}
                }
            }{
                \ntreeruleII{Effect}{
                    #4 \subtypep #6
                    }{
                         #3 \subeffectp #5
                }{
                    \M{#3}{#4}\subtypep{\M{#5}{#6}}
                }
            }{
                \typerelation{\P\mid #1}{#2}{\M{#5}{#6}}
            }
        }
        

        \newcommand{\edeltacruleprime}[6]{
            \ntreeruleII{Subtype}{
                \treeruleI{
                    \D'
                }{
                    \typerelation{\P\mid #1}{#2}{\M{#3}{#4}}
                }
            }{
                \ntreeruleII{Effect}{
                    #4 \subtypep #6
                }{#3 \subeffectp #5
                }{
                    \M{#3}{#4}\subtypep{\M{#5}{#6}}
                }
            }{
                \typerelation{\P\mid #1}{#2}{\M{#5}{#6}}
            }
        }
                   

        \newcommand{\edeltacruleprimeprime}[6]{
            \ntreeruleII{Subtype}{
                \treeruleI{
                    \D''
                }{
                    \typerelation{\P\mid #1}{#2}{\M{#3}{#4}}
                }
            }{
                \ntreeruleII{Effect}{
                    #4 \subtypep #6
                    }{ #3 \subeffectp #5
                }{
                    \M{#3}{#4}\subtypep{\M{#5}{#6}}
                }
            }{
                \typerelation{\P\mid #1}{#2}{\M{#5}{#6}}
            }
        }

        \newcommand\obj[0]{\texttt{obj }}


        \newcommand{\Tz}[2]{\texttt{T}^0_{#1}#2}
        \newcommand{\Tn}[2]{\texttt{T}^n_{#1}#2}
        \newcommand{\Tm}[2]{\texttt{T}^m_{#1}#2}
        
        \newcommand{\pointz}[1]{\point{#1}^0}
        \newcommand{\pointn}[1]{\point{#1}^n}
        \newcommand{\pointm}[1]{\point{#1}^m}
        
        \newcommand{\bindz}[3]{\bind{#1}{#2}{#3}^0}
        \newcommand{\bindn}[3]{\bind{#1}{#2}{#3}^n}
        \newcommand{\bindm}[3]{\bind{#1}{#2}{#3}^m}
        
        \newcommand\tstrengthz[3]{\tstrength{#1}{#2}{#3}^0}
        \newcommand\tstrengthn[3]{\tstrength{#1}{#2}{#3}^n}
        \newcommand\tstrengthm[3]{\tstrength{#1}{#2}{#3}^m}
        
        \newcommand\set[0]{\texttt{Set}}
        \newcommand\cccat[0]{\textit{CCCat}}

        \newcommand\ev[0]{\vec{\e}}
        \newcommand\emv[0]{\vec{\e_m}}
        \newcommand\env[0]{\vec{\e_n}}
        
        \newcommand\subeffectm[0]{\subeffect_m}
        
        \newcommand\dsem[2]{\db{#1 \subeffectm #2}}
        \newcommand\dsen[2]{\db{#1 \subeffectn #2}}
        \newcommand\dsez[2]{\db{#1 \subeffectz #2}}
        \newcommand\dsep[2]{\db{#1 \subeffectp #2}}
        \newcommand\dsepp[2]{\db{#1 \subeffectpp #2}}
        
        \newcommand\allEn[0]{\forall_{E^n}}
        \newcommand\allEm[0]{\forall_{E^m}}
        
        \newcommand\counit[1]{\boldsymbol{\epsilon}_{#1}}
        \newcommand\unit[1]{\boldsymbol{\eta}_{#1}}
\fi



\usepackage[mathscr]{euscript}
\let\euscr\mathscr \let\mathscr\relax% just so we can load this and rsfs
\usepackage[scr]{rsfso}
\newcommand{\powerset}{\raisebox{.15\baselineskip}{\Large\ensuremath{\wp}}}
\newcommand\erelates[2]{\texttt{R}_{\wellformedt{#1}{#2}}}
\newcommand\erelatesg[2]{\texttt{R}_{\wellformedok{#1}{#2}}}
\newcommand\powset[1]{\powerset(#1)}
\newcommand{\av}[0]{\vec{\a}}

\newcommand\inLogRel[5]{(#1, #2)\in\erelates{#3}{#4}(#5)}
\newcommand\inLogRelE[4]{\inLogRel{#1}{#2}{#3}{#4}{\ev}}
\newcommand{\inLogRelPE}[3]{\inLogRelE{#1}{#2}{\P}{#3}}
\newcommand{\inLogRelPEG}[3]{({#1}{#2})\in\erelatesg{\P}{#3}(\ev)}
\newcommand\fundSub[0]{\ssub{\av}{\ev}\ssi}
\newcommand\fundRho[0]{(\ev)(\rho)}
\newcommand\fundRhoExt[1]{(\ev)(\rho#1)}

\begin{document}



\chapter{Adequacy of a Model of the Polymorphic Effect Calculus}
    \section{Instantiation of the Polymorphic Effect Calculus}
    Let us instantiate the polymorphic effect calculus to be a language in which one can write programs which can create an output signal. The effect system of EC is then used to count an upper bound on the number of outputs that a program can make. This language shall be called $\pecput$

    \subsection{Ground Types}
    We simply use the basic ground types.

    \begin{equation}
        \label{AdequacyGroundTypes}
        \g \gens \B \mid \U
    \end{equation}

    \subsection{Graded Monad}
    We grade index the base graded monad with a partially ordered monoid derived from the natural numbers.

    \begin{equation}
        E = (\N, 0, +, \leq)        
    \end{equation}
    This extended as described in the dissertation \todo{Ref}, to symbolically include variables $\a, \b, \g$ which range over the natural numbers.

    This means that the $\doin{x}{v}{v'}$ type rule adds together the upper bounds on the two expressions to give an upper bound on the number of outputs of the sequenced expression. The $\return{v}$ type rule acknowledges that a pure expression does not have any output.

    \subsection{Constants}
    We extend the set of constant, built in expressions to include a \put\s statement which makes a single output action.

    \begin{equation}
        \label{AdequacyConstants}
        \const{A} \gens \t^\B \mid \f^\B \mid \u^\U \mid \put^{\M{1}{\U}}        
    \end{equation}

    \subsection{Subtyping}
    The ground subtyping relation is the trivial identity relation. This is extended using the subeffect and function subtyping rules given in \todo{Ref}.


    \section{Instantiation of a Model of the Polymorphic Effect Calculus}

    Let us now instantiate a model of $\pecput$ in the indexed category derived as in chapter \todo{ref} from a model of the polymorphic version of $\pecput$ in $\set$, the category of sets and functions.

    \subsection{Cartesian Closed Category}
    Is given by the usage of $\set$.
    
    \subsection{Graded Monad}
    The strong graded monad on $\set$ is given by tagging values of the underlying type with the number of output operations required to compute that value.

    \begin{align}
        \label{AdequacyGradedMonad}
        \Tz{n}{A} & = \setcomp{(n', a)}{n'\leq n\wedge a\in A}\\
        \bindz{m}{n}{A} & = (m', (n', a))\mapsto (n' + m', a)\\
        \pointz{A} & = a\mapsto (0, a)\\\
        \tstrengthz{n}{A}{B} & = (a, (n', b)) \mapsto (n', (a, b))
    \end{align}


\subsection{Subeffecting Natural Transformations}
These natural transformations are given by inclusion functions (identities), since $n\leq m\wedge (n',a)\in\Tz{n}{A} \implies (n'\leq n \leq m, a\in A)\implies (n',a)\in\Tz{m}{A}$. Other subtyping morphisms are generated using the usual method according to the subtype derivation.

\subsection{Ground Denotations}
We define denotations for ground types as follows:

\begin{align}
    \label{AdequacyGroundTypeDenotations}
    \deno{\wellformedt{\P}\U} & = \ev\mapsto\{\singleton\}\\
    \deno{\wellformedt{\P}\B} & = \ev\mapsto \{\top, \bot\}
\end{align}

We then define denotations for the constant expressions, including the \put operation. 

\begin{align}
    \label{AdequacyGroundTermDenotations}
    \deno{\u} & = \ev\mapsto\singleton\mapsto \singleton\\
    \deno{\t} & = \ev\mapsto\singleton\mapsto \top\\
    \deno{\f} & = \ev\mapsto\singleton\mapsto \bot\\
    \deno{\put} & = \ev\mapsto\singleton\mapsto (1, \singleton)\\
\end{align}



\subsection{Soundness}
This category is now an Indexed-S-category and hence a sound model for $\pecput$ when completed using the techniques in chapter \todo{ref}.

\section{Programming With Put}

This simple language now has some extra properties which the general EC does not have.

\begin{definition}[Powers of Put as an Equational Equivalence Class]
Define $\put^n$ as follows:

\begin{align*}
    \label{PutnDefinition}
    \gpeberelation{\put^0}{&{\s}\return{\u}}{\M{0}{\U}}\\ 
    \gpeberelation{\put^{m+1}}{&{\s}\doin{\_}{\put^n}{\put}}{\M{m+1}{\U}}
\end{align*}
\end{definition}


\begin{lemma}[Denotations of Powers of Put]
    Powers of put have a simple denotation.
$\deno{\gpetyperelation{\put^m}{\M{m}{\U}}} = \ev\mapsto\rho\mapsto(m, \singleton)$    
\end{lemma}

\begin{proof}
    By induction on $m$.

    \case{0}
    
    \begin{equation}
        \deno{\gpetyperelation{\put^0}{\M{0}{\U}}}\fundRho = \pointn{}(\singleton) = (0, \singleton)
    \end{equation}
    
    \case{m+1}
    \begin{align*}
        \deno{\gpetyperelation{\put^{m+1}}{\M{0}{\U}}}\fundRho & = (\bindmu^n\after\Tn{m}{(\deno{\typerelation{\nil}{\put}{\M{1}{\U}}}\after\p)}\after \strengtht^n)(\singleton, \deno{\gpetyperelation{\put^m}{\M{n}{\U}}}\fundRho)\\
        & = (m+1, \singleton)
    \end{align*}
\end{proof}

\section{Logical Relations}

We define a logical relation between elements of the image of a type denotation and closed $\pecput$ terms.
\begin{equation}
    \label{LogicalRelationType}
    \erelates{\P}{A}(\ev: E^n) \in \powset{\deno{\wellformedt{\P}{A}}\ev\times \pecput^{A\ssub{\a}{\e}}}
\end{equation}

\subsection{Definition}
\begin{definition}[Logical Relation]

    \begin{align*}
        \label{LogicalRelationDefinition}
        (d, v)\in\erelates{\P}{\all{\a}{A}}(\ev) & \leftrightarrow \forall\e. (\pe(d), \eapp{v}{\e})\in\erelates{\P}{A\ssub{\a}{\e}}(\ev)\\
        (d, v)\in\erelates{\P}{\U}(\ev) & \Leftrightarrow (d=\singleton \wedge \zberelation{v}{()}{\U})
        \\
        (d, v)\in\erelates{\P}{\B}(\ev) & \Leftrightarrow ((d=\top \wedge \zberelation{v}{\t}{\B}) 
        \\& \vee (d=\bot \wedge \zberelation{v}{\f}{\B}))
        \\
        (d, v)\in\erelates{\P}{\ab}(\ev)  &\Leftrightarrow \forall e, u. (e, u)\in\erelates{\P}{A}(\ev) \implies (d(e), \apply{v}{u})\in \erelates{\P}{B}(\ev)
        \\
        (d, v)\in\erelates{\P}{\M{f}{A}}(\ev) &\Leftrightarrow (d= (n, d')
        \\ & \wedge \exists v'. ((d', v')\in\erelates{\P}{A}
        \\ & \wedge  \zberelation{v}{\doin{\_}{\put^{n'}}{\return{v'}}}{(\M{f}{A})\ssub{\av}{\ev}}))
    \end{align*}
\end{definition}
\subsection{Subtyping}

\begin{theorem}[Logical Relation and Subtyping]\label{LogRelSubtype}
    If $A\subtypep B$ and $\inLogRelPE{d}{v}{A}$ then $\inLogRelPE{d}{v}{B}$
\end{theorem}

\begin{proof}
    By induction on the derivation of $A\subtypep B$.

    \case{\sground}
        $A\subtypep B\implies A = B$, since ground subtyping is the identity relation.

    \case{\sfun}
        $A\subtypep B\implies A = \fntype{A_1}{A_2}, B = \fntype{B_1}{B_2}$ where $B_1\subtypep A_1$ and $A_2\subtypep B_2$.

        By the definition of the $\relates_{\wellformedt{\P}{\ab}}$ relation, $\inLogRelPE{d}{v}{\ab} \Leftrightarrow (\forall e, u. \inLogRelPE{e}{u}{A}\implies \inLogRelPE{d(e)}{\apply{v}{u}}{B})$.

        So 

        \begin{align*}
            \forall e, u. \inLogRelPE{e}{u}{B_1} & \implies \inLogRelPE{e}{u}{A_1} \qt{By induction $B_1\subtypep A_1$}\\
            & \implies \inLogRelPE{d(e)}{\apply{u}{v}}{A_2}\qt{By definition}\\
            & \implies \inLogRelPE{d(e)}{\apply{u}{v}}{B_2}\qt{By induction $A_2\subtypep B_2$}
        \end{align*}

        As required.
    \case{\seffect}

    $\M{n_1}{A_1} \subtypep \M{n_2}{A_2} \implies n_1\leq n_2, A_1\subtypep A_2$

    \begin{align*}
        \inLogRelPE{d}{v}{\M{n_1}{A_1}} & \implies  d=(n_1', d') \wedge n_1' \leq n_1\leq n_2 
        \\ & \quad\wedge \exists v'. (\inLogRelPE{d'}{v'}{A_1} \wedge\zberelation{v}{\doin{\_}{\put^{n'}}{\return{v'}}}{\M{n_1}{A_1}})
        \\
        & \implies \ztyperelation{v_1'}{A_2} \wedge \inLogRelPE{d'}{v'}{A_2} \wedge \zberelation{v}{\doin{\_}{\put^{n'}}{\return{v'}}}{\M{n_1}{A_2}}
        \\
        & \implies \inLogRelPE{d}{v}{\M{n_2}{A_2}}
    \end{align*}

    \case{\squant}
    $\all{\a}{A_1}\subtypep \all{\a}{A_2}\implies A_1\subtype A_2$

    So:

    \begin{align*}
        \inLogRelPE{d}{v}{\all{\a}{A_1}} & \implies \forall \e.\inLogRelPE{\pe(d)}{\eapp{v}{\e}}{A_1\ssub{\e}{\a}}\\
        & \implies \forall\e \inLogRelPE{\pe(d)}{\eapp{v}{\e}}{A_2\ssub{\e}{\a}}\\
        & \implies \inLogRelPE{d}{v}{\all{\a}{A_2}}
    \end{align*}
\end{proof}

\subsection{The Environment Lemma}

\todo{Write this up}

\begin{lemma}[Environment Lemma]
    $$\erelates{\P,\a}{A}(\ev, \e) = \erelates{\P}{A\ssub{\e}{\a}}(\ev)$$
\end{lemma}

\begin{proof}
    By induction over the type $A$, proving that $\inLogRel{d}{v}{\P,\a}{A}{\ev, \e} \Leftrightarrow \inLogRelPE{d}{v}{A\ssub{\a}{\e}}$

    \case{Bool}
        \begin{align*}
            \inLogRel{d}{v}{\P,\a}{\B}{\ev, \e} & \Leftrightarrow (d = \top \wedge \zberelation{v}{\t}{\B})
            \\ 
            & \quad\wedge(d = \bot \wedge \zberelation{v}{\f}{\B})
            \\
            & \Leftrightarrow  \inLogRelPE{d}{v}{\B}
        \end{align*}

    \case{Unit}
    \begin{align*}
        \inLogRel{d}{v}{\P,\a}{\U}{\ev, \e} & \Leftrightarrow (d = \singleton \wedge \zberelation{v}{\u}{\U})
        \\
        & \Leftrightarrow  \inLogRelPE{d}{v}{\U}
    \end{align*}

    \case{\tfun}
    \begin{align*}
        \inLogRel{d}{v}{\P,\a}{\ab}{\ev, \e} & \Leftrightarrow \forall\inLogRel{e}{u}{\P,\a}{A}{\ev, \e}. \inLogRel{d(e)}{\apply{v}{u}}{\P,\a}{B}{\ev, \e} 
        \\
        & \Leftrightarrow 
        \forall\inLogRelPE{e}{u}{A\ssub{\e}{\a}}. \inLogRelPE{d(e)}{\apply{v}{u}}{B\ssub{\a}{\e}}
        \\
        & \Leftrightarrow  \inLogRelPE{d}{v}{\fntype{A\ssub{\a}{\e}}{B\ssub{\a}{\e}}}
        \\
    \end{align*}

    \case{\teffect}

    \begin{align*}
        \inLogRel{d}{v}{\P,\a}{\M{f}{A}}{\ev, \e} & \Leftrightarrow d= (n, d')
        \\ & \wedge \exists v'. \inLogRel{d'}{v'}{\P,\a}{A}{\ev, \e}
        \\ & \quad\wedge  \zberelation{v}{\doin{\_}{\put^{n'}}{\return{v'}}}{\M{f}{A}\ssub{\av}{\ev}\ssub{\a}{\e}}
        \\ & \Leftrightarrow \exists v'. \inLogRelPE{d'}{v'}{A\ssub{\a}{\e}}
        \\ & \quad\wedge  \zberelation{v}{\doin{\_}{\put^{n'}}{\return{v'}}}{\M{f}{A}\ssub{\a}{\e}\ssub{\av}{\ev}}
        \\ & \Leftrightarrow \inLogRelPE{d}{v}{A\ssub{\a}{\e}}
    \end{align*}
\case{\tquant}
\begin{align*}
    \inLogRel{d}{v}{\P,\a}{\all{\b}{A}}{\ev, \e_1} & \Leftrightarrow \forall \e_2. \inLogRel{\pi_{\e_2}(d)}{\eapp{v}{\e_2}}{\P,\a}{A\ssub{\b}{\e_2}}{\ev, \e_1}
    \\
    & \Leftrightarrow \forall \e_2. \inLogRelPE{\pi_{\e_2}(d)}{\eapp{v}{\e_2}}{A\ssub{\b}{\e_2}\ssub{\a}{\e_1}}
    & \Leftrightarrow \inLogRelPE{d}{v}{\all{\b}{A\ssub{\a}{\e_1}}}
\end{align*}

\end{proof}



\subsection{Fundamental Property}\label{FundProp}
Let $\erelatesg{\P}{\G}(\ev) \in \powset{\deno{\G}\ev)\times \pecput^{\G\ssub{\av}{\ev}}}$ mean:

\begin{equation}
    \label{LogicalRelationTypeEnvironment}
    \inLogRelPEG{\rho}{\si}{\G} \Leftrightarrow \forall x. \inLogRelPE{\rho(x)}{\si(x)}{\G(x)}
\end{equation}


\begin{theorem}[Fundamental Theorem]
    If $\inLogRelPEG{\rho}{\si}{\G}$ then $\inLogRelPE{\deno{\gpetyperelation{v}{A}}\fundRho}{v\fundSub}{A}$ up to equational equivalence.    
\end{theorem}

\begin{proof}
    By induction over the derivation of $\gpetyperelation{v}{A}$

    \case{Variables}
    \begin{align*}
        \deno{\gpetyperelation{x}{\G(x)}}\fundRho & = \rho(x) \\
        x\ssi\ssub{\av}{\ev} & = \si(x)\\
    \end{align*}

    And $\inLogRelPE{\rho(x)}{\si(x)\ssub{\av}{\ev}}{\G(x)}$.

    \case{Constants}
    \begin{align*}
        \deno{\gpetyperelation{\t}{\B}}\fundRho & = \top \wedge \zberelation{\t\fundSub}{\t}{\B} \qt{ So } \inLogRelPE{\deno{\gpetyperelation{\t}{\B}}\fundRho}{\t\fundSub}{\B}\\
        \deno{\gpetyperelation{\f}{\B}}\fundRho & = \bot \wedge \zberelation{\f\fundSub}{\f}{\B} \qt{ So } \inLogRelPE{\deno{\gpetyperelation{\t}{\B}}\fundRho}{\f\fundSub}{\B}\\
        \deno{\gpetyperelation{\u}{\U}}\fundRho & = \singleton \wedge \zberelation{\u\fundSub}{\u}{\U} \qt{ So } \inLogRelPE{\deno{\gpetyperelation{\u}{\U}}\fundRho}{\u\fundSub}{\U}\\
        \deno{\gpetyperelation{\put}{\M{1}{\U}}}\fundRho & = (1, \singleton)\wedge \zberelation{\put}{\doin{\_}{\put^1}{\return{\u}}}{\s\M{\1}{\U}}\qt{So }\inLogRelPE{\deno{\gpetyperelation{\put}{\M{\1}{U}}}\fundRho}{\put\fundSub}{\M{\1}{\U}}\\ 
    \end{align*}


    \case{\vsubtype}

    \begin{equation}
        \deno{\gpetyperelation{v}{B}}\fundRho = \deno{\gpetyperelation{v}{A}}\fundRho
    \end{equation}

    By induction, $\inLogRelPE{\deno{\gpetyperelation{v}{A}}\fundRho}{v\fundSub}{A}$.
    Since, by the subtyping lemma (Lemma \ref{LogRelSubtype}) $A\subtypep B \wedge \inLogRelPE{d}{v}{A} \implies \inLogRelPE{d}{v}{B}$, we have that $\inLogRelPE{\deno{\gpetyperelation{v}{B}}}{v\fundSub}{B}$.

    \case{\vfun}

    For all $\inLogRelPE{d}{u}{A}$, 
    \begin{align*}
        (\deno{\gpetyperelation{\lam{x}{A}{v}}{\ab}}\fundRho) d & = (\cur{\deno{\typerelation{\gax}{v}{B}}}\fundRho) d\\
        & = \deno{\typerelation{\gax}{v}{B}} (\fundRhoExt{[x\mapsto d]})\\
    \end{align*}

    Since $\inLogRelPE{d}{u}{A}$, we have $\inLogRelPEG{(\rho[x\mapsto d])}{(\si, x\setto u)}{\gax}$, so by induction
    \begin{align*}
        \inLogRelPE{\deno{\typerelation{\gax}{v}{B}}(\ev)(\rho[x\mapsto d])&}{v\ssub{\av}{\ev}\sub{\si, x\setto u}}{B}
    \end{align*}

    We can see that $\zberelation{v\ssub{\av}{\ev}{\sub{\si, x\setto u}}}{\apply{((\lam{x}{A}{v})\fundSub)}{u}}{B\ssub{\av}{\ev}}$.

    Hence $\inLogRelPE{(\deno{\gpetyperelation{\lam{x}{A}{v}}{\ab}}\fundRho) }{(\lam{x}{A}{v})\fundSub}{\ab}$.
\end{proof}

\case{\vapply}
\begin{equation}
    \deno{\gpetyperelation{\apply{v}{u}}{B}}\fundRho = (\deno{\gpetyperelation{v}{\ab}}\fundRho)(\deno{\gpetyperelation{u}{A}}\fundRho)
\end{equation}

By induction $\inLogRelPE{\deno{\gpetyperelation{v}{\ab}}\fundRho}{v\fundSub}{\ab}$ and $\inLogRelPE{\deno{\gpetyperelation{u}{A}}\fundRho}{u\fundSub}{A}$. So by the definition of $\erelates{\P}{\ab}$, 

\begin{align*}
   \inLogRelPE{(\deno{\gpetyperelation{v}{\ab}}\fundRho)(\deno{\gpetyperelation{u}{A}}\fundRho)}{\apply{(v\fundSub)}{(u\fundSub)})}{B}
\end{align*}

Where $\zberelation{\apply{(v\fundSub)}{(u\fundSub)}}{(\apply{v}{u})\fundSub}{B\ssub{\av}{\ev}}$.

So $\inLogRelPE{\deno{\gpetyperelation{\apply{v}{u}}{B}}\fundRho}{(\apply{v}{u})\fundSub}{B}$.


\case{\vreturn}

\begin{equation}
    \deno{\gpetyperelation{v}{\M{0}{A}}}\fundRho = (0, \deno{\gpetyperelation{v}{A}}\fundRho)
\end{equation}

By induction, $\inLogRelPE{\deno{\gpetyperelation{v}{A}}\fundRho}{v\fundSub}{A}$. So by picking $v' = v\fundSub$, we have

\begin{equation}
    \zberelation{(\return{v})\fundSub}{\return{(v\fundSub)\beequiv \doin{\_}{\put^0}{\return{v'}}}}{\M{0}{A\ssub{\av}{\ev}}}
\end{equation}

So $\inLogRelPE{\deno{\gpetyperelation{\return{v}}{\M{0}{A}}}\fundRho}{(\return{v})\fundSub}{\M{0}{A}}$

\case{\vbind}

By inversion, $(\deno{\gpetyperelation{\doin{x}{v_1}{v_2}}{\M{m+n}{B}}}\fundRho) = (m'+n', d_2)$, where $(n', d_2) = (\deno{\typerelation{\gax}{v_2}{\mnb}}(\fundRhoExt{[x\mapsto d_1]}))$, and $(n', d_1) = (\deno{\gpetyperelation{v_1}{\mma}}\fundRho)$.

By induction, $\inLogRelPE{(m', d_1)}{v_1\fundSub}{\mma}$. So $\exists v_1'$ such that $\zberelation{v_1\fundSub}{\doin{\_}{\put^{m'}}{\return{v_1'}}}{\mma}$. So $\inLogRelPE{(\fundRhoExt{[x\mapsto d_1]})}{(\ssi, x\setto v_1')}{\gax}$.

So by induction $\inLogRelPE{\deno{\typerelation{\gax}{v_2}{\mnb}}(\fundRhoExt{[x\mapsto d_1]})}{v_2\sub{\si, x\setto v_1'}}{\mnb}$.

Hence, $\exists v_2'$ such that $\zberelation{v_2\sub{\si, x\setto v_1'}}{\doin{\_}{\put^{n'}{\return{v_2'}}}}{\M{m + n}{B}}$ and $\inLogRelPE{d_2}{v_2'}{\mnb}$.

Hence,

\begin{align*}
    \zberelation{\doin{x}{v_1\fundSub}{v_2\fundSub}&}{\doin{x}{(\doin{\_}{\put^{m'}}{\return{v_1'}})}{(v_2\fundSub)}}{\M{m+n}{B}}\\
    &\beequiv\doin{\_}{\put^{m'}}{v_2\sub{\si, x\setto{v_1'}}}\\
    &\beequiv\doin{\_}{\put^{m'+n'}}{\return{v_2'}}
\end{align*}

So $\inLogRelPE{\deno{\gpetyperelation{\doin{x}{v_1}{v_2}}{\M{m+n}{B}}}\fundRho}{(\doin{x}{v_1}{v_2})\fundSub}{\M{m+n}{B}}$.



\case{\vif}

By inversion, $$\deno{\gpetyperelation{\pifthenelse{A}{b}{v_1}{v_2}}{A}}\fundRho = \begin{cases}
    \deno{\gpetyperelation{v_1}{A}}\fundRho \qt{If }\deno{\gpetyperelation{b}{\B}}\fundRho = \top\\
    \deno{\gpetyperelation{v_2}{A}}\fundRho \qt{If }\deno{\gpetyperelation{b}{\B}}\fundRho = \bot
\end{cases}
$$.

By induction,

\begin{align*}
    \inLogRelPE{\deno{\gpetyperelation{b}{\B}&}\fundRho}{b\fundSub}{\B}\\
    \inLogRelPE{\deno{\gpetyperelation{v_1}{A}&}\fundRho}{v_1\fundSub}{A}\\
    \inLogRelPE{\deno{\gpetyperelation{v_2}{A}&}\fundRho}{v_2\fundSub}{A}
\end{align*}

\subcase{$\deno{\gpetyperelation{b}{\B}}\fundRho = \top$ and $\zberelation{b\fundSub}{\t}{\B}$}

\begin{equation}
    \deno{\gpetyperelation{\pifthenelse{A}{b}{v_1}{v_2}}{A}}\fundRho = \deno{\gpetyperelation{v_1}{A}}\fundRho
\end{equation}

And

\begin{equation}
   \zberelation{ v_1\fundSub}{(\pifthenelse{A}{b}{v_1}{v_2})\fundSub}{A\ssub{\av}{\ev}}
\end{equation}

So
\begin{equation}
        \inLogRelPE{\deno{\gpetyperelation{\pifthenelse{A}{b}{v_1}{v_2}}{A}}\fundRho}{(\pifthenelse{A}{b}{v_1}{v_2})\fundSub}{A}
    \end{equation}

\subcase{$\deno{\gpetyperelation{b}{\B}}\fundRho = \bot$ and $\zberelation{b\fundSub}{\f}{\B}$}

    \begin{equation}
        \deno{\gpetyperelation{\pifthenelse{A}{b}{v_1}{v_2}}{A}}\fundRho = \deno{\gpetyperelation{v_2}{A}}\fundRho
    \end{equation}
    
    And
    
    \begin{equation}
       \zberelation{v_2\fundSub}{(\pifthenelse{A}{b}{v_1}{v_2})\fundSub}{A\ssub{\av}{\ev}}
    \end{equation}
    
    So
    \begin{equation}
            \inLogRelPE{\deno{\gpetyperelation{\pifthenelse{A}{b}{v_1}{v_2}}{A}}\fundRho}{(\pifthenelse{A}{b}{v_1}{v_2})\fundSub}{A}
        \end{equation}
    

\case{\vgen}

For all $\e\in E$:

\begin{align}
    \pe(\deno{\gpetyperelation{\elam{\a}{v}}{\all{\a}{A}}}\fundRho) & = \pe(\finpr{\deno{\etyperelation{\P,\a}{\G}{v}{A}}(\ev, \e')}{\e'\in E}(\rho)) 
    \\
    & = \deno{\etyperelation{\P,\a}{\G}{v}{A}}(\ev, \e)(\rho)
\end{align}

By induction, we know that $\inLogRel{\deno{\etyperelation{\P,\a}{\G}{v}{A}}(\ev, \e)(\rho)}{v\ssub{\av}{\ev}\ssub{\a}{\e}\ssi}{\P,\a}{A}{\ev, \e}$.


By the environment lemma \todo{Ref}, $\erelates{\P, \a}{A}(\ev, \e) = \erelates{\P}{A\ssub{\a}{\e}}(\ev)$.


So $\inLogRelPE{\deno{\etyperelation{\P,\a}{\G}{v}{A}}(\ev, \e)(\rho)}{v\ssub{\av}{\ev}\ssub{\a}{\e}\ssi}{A\ssub{\a}{\e}}$

Since $\a$ does not appear in $\si$ (Since $\si$ is well formed under $\P = \nil$), we can commute the substitutions: $v\ssub{\av}{\ev}\ssub{\a}{\e}\ssi = v\ssub{\av}{\ev}\ssi\ssub{\a}{\e}$.

Hence: 
\begin{equation*}
    \zberelation{v\ssub{\av}{\ev}\ssub{\a}{\e}\ssi = v\ssub{\av}{\ev}\ssi\ssub{\a}{\e}}{\eapp{\elam{\a}{(v\fundSub)}}{\e}}{A\ssub{\a}{\e}\ssub{\av}{\ev}}
\end{equation*}

So $\inLogRelPE{    \pe(\deno{\gpetyperelation{\elam{\a}{v}}{\all{\a}{A}}}\fundRho)}{\eapp{\elam{\a}{(v\fundSub)}}{\e}}{A\ssub{\a}{\e}}$.

So $\inLogRelPE{\deno{\gpetyperelation{\elam{\a}{v}}{\all{\a}{A}}}}{(\elam{\a}{v})\fundSub}{\all{\a}{A}}$.


\case{\vspec}
Required to prove $\inLogRelPE{\deno{\gpetyperelation{\eapp{v}{\e}}{A\ssub{\a}{\e}}}\fundRho}{(\eapp{v}{\e})\fundSub\beequiv\eapp{(v\fundSub)}{\e}}{A\ssub{\a}{\e}}$.

By inversion and induction, $\inLogRelPE{\deno{\gpetyperelation{v}{\all{\a}{A}}}}{v\fundSub}{\all{\a}{A}}$. So, $$\forall\e. \inLogRelPE{\pe(\deno{\gpetyperelation{v}{\all{\a}{A}}}\fundRho)}{\eapp{(v\fundSub)}{\e}}{A\ssub{\a}{\e}}$$

Let $\e' = \deno{\wellformede{\P}{\e}}\ev$.

\begin{align*}
    \deno{\gpetyperelation{\eapp{v}{\e}}{A\ssub{\a}{\e}}}\fundRho 
    & = \pr{\Id{}}{\deno{\wellformede{\P}{\e}}}\star(\counit{\wellformedt{\P,\b}{A\ssub{\a}{\b}}})(\ev)(\deno{\gpetyperelation{v}{\all{\a}{A}}}\fundRho)\\
    & = \pi_{\e'}(\deno{\gpetyperelation{v}{\all{\a}{A}}}\fundRho)
\end{align*}

Its also the case that
\begin{equation}
    \eapp{(v\fundSub)}{\e'} = (\eapp{v}{\e})\fundSub
\end{equation}

So

\begin{equation}
    \inLogRelPE{\deno{\gpetyperelation{\eapp{v}{\e}}{A\ssub{\a}{\e}}}\fundRho }{(\eapp{v}{\e})\fundSub}{A\ssub{\a}{\e}}
\end{equation}



\section{Adequacy}
\begin{theorem}[Adequacy]
For $G$ defined as:

\begin{align*}
    G\gens \B \mid \U \mid \M{n}{G}
\end{align*}

Equality of denotations implies equational equality.

\begin{equation}
    \deno{\ztyperelation{v}{G}} = \deno{\ztyperelation{u}{G}} \implies \zberelation{v}{u}{G}
\end{equation}

\end{theorem}

\begin{proof}
    By induction on the structure of $G$, making use of the fundamental property \ref{FundProp}.

    \case{Boolean}
    Let $d=\deno{\ztyperelation{v}{\B}}=\deno{\ztyperelation{v}{\B}}\in \{ \top, \bot\}$. By the fundamental property, $\inLogRelPE{d}{v}{\B}$  and $\inLogRelPE{d}{v}{\B}$.

    \subcase{$d=\top$}
    Then $\zberelation{v\beequiv\t}{u}{\B}$

    
    \subcase{$d=\bot$}
    Then $\zberelation{v\beequiv\f}{u}{\B}$

    \case{Unit}
    Let $\singleton=\deno{\ztyperelation{v}{\U}}=\deno{\ztyperelation{v}{\U}}\in \{\singleton\}$. By the fundamental property, $\inLogRelPE{d}{v}{\U}$  and $\inLogRelPE{d}{v}{\U}$. Hence $\zberelation{v\beequiv\u}{u}{\U}$.

    \case{\teffect}

    Let $( n',  d) = \deno{\ztyperelation{v}{\M{n}{G}}} = \deno{\ztyperelation{u}{\M{n}{G}}}$. By the fundamental property, $\inLogRelPE{((n', d))}{v}{\M{n}{G}}$ and $\inLogRelPE{((n', d))}{u}{\M{n}{G}}$. So there exists $u', v'$ such that $\inLogRelPE{d'}{u'}{G}$ and $\inLogRelPE{d'}{u'}{G}$ and:

    \begin{align*}
        \zberelation{v&}{\doin{\_}{\put^{n'}}{\return{v'}}}{\M{n}{G}}\\
        &\beequiv \doin{\_}{\put^{n'}}{\return{u'}}\\
        & \beequiv u
    \end{align*}
$\square$
\end{proof}

\end{document}