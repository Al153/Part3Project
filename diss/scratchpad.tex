\documentclass{report}

%% Don't import the header multiple times

\ifdefined\HEADERIMPORTED
\else
\newcommand\HEADERIMPORTED[0]{This file is HEADERIMPORTED}
\usepackage{amssymb}

\usepackage{amsmath}
\usepackage[a4paper,includeheadfoot,margin=2.54cm]{geometry}
\usepackage{breqn}
\usepackage{amssymb}

\usepackage{amsmath}
\usepackage[a4paper,includeheadfoot,margin=2.54cm]{geometry}
\usepackage{breqn}
\usepackage{array}   % for \newcolumntype macro
\usepackage{tikz-cd}
\usepackage{tabstackengine}
\setstackEOL{\cr}
\setstackgap{L}{\normalbaselineskip}

\newcommand\todo[1]{\textbf{TODO: #1}}

\newcommand{\s}{\;}
\newcommand{\doin}[3]{\texttt{do}\s #1 \leftarrow #2 \s\texttt{in}\s #3}
\newcommand\apply[2]{#1\s#2}
\newcommand\ifthenelse[5]{\texttt{if}_{#1, #2}\s#3\s \texttt{then}\s #4 \s\texttt{else} \s#5}
\newcommand\const[1]{\texttt{C}^{#1}}
\newcommand\return[1]{\texttt{return} #1}
\newcommand\lam[3]{\lambda #1 : #2. #3}
\renewcommand\u[0]{\texttt{()}}
\newcommand{\U}[0]{\texttt{Unit}}
\renewcommand\t[0]{\texttt{true}}
\newcommand\f[0]{\texttt{false}}
\newcommand{\B}[0]{\texttt{Bool}}
\newcommand{\G}[0]{\Gamma}
\newcommand\D{\Delta}


% draw type relations
\newcommand{\typerelation}[3]{#1 \vdash #2 \colon #3}
\newcommand{\gtyperelation}[2]{\typerelation{\G}{#1}{#2}}

%draw tree rules
\newcommand{\treerule}[3]{(\text{#1}) \frac{#2}{#3}}
\newcommand{\condtreerule}[4]{\treerule{#1}{#2}{#3}(\text{if } #4)}

\newcommand{\subtype}[0]{\leq:}
\newcommand\subeffect[0]{\leq}

\newcommand{\M}[2]{\texttt{M}_{#1}{#2}}

\newcommand\lamtype[3]{#1 \rightarrow \M{#2}{#3}}
\newcommand{\1}[0]{\texttt{1}}

\newcommand\e[0]{\epsilon}

\newcommand{\db}[1]{{\bf [\![}#1{\bf ]\!]}}
\newcommand{\deno}[1]{\db{#1}_M}
\newcommand\after\circ
\newcommand\term[1]{\left\langle\right\rangle_{#1}}

\newcommand\point[1]{\eta_{#1}}
\newcommand\bind[3]{\mu_{#1, #2, #3}}

\newcommand\T[2]{T_{#1}{#2}}

\newcommand\pr[2]{\left\langle#1, #2\right\rangle}

% tensor strength Nat-tran
\newcommand\tstrength[3]{\texttt{t}_{#1, #2, #3}}

% Id morphism
\newcommand\Id[1]{\texttt{Id}_{#1}}

\newcommand\idg[0]{\Id{\G}}
% beta-eta equivalence
\newcommand\beequiv[0]{=_{\beta\eta}}
% Substitutions
\newcommand\si{\sigma}

\newcommand{\sub}[1]{\left[#1\right]}
\newcommand{\ssub}[2]{\left[#2 / #1\right]}
\newcommand{\ssi}[0]{\sub{\si}}

% beta-eta equivalence relation
\newcommand{\berelation}[4]{\typerelation{#1}{#2 \beequiv #3}{#4}}
\newcommand{\gberelation}[3]{\gtyperelation{#1 \beequiv #2}{#3}}


% Shortcuts for denotational equality
\newcommand{\denoequality}[4]{\deno{\typerelation{#1}{#2}{#4}} = \deno{\typerelation{#1}{#3}{#4}}}
\newcommand{\gdenoequality}[3]{\denoequality{\G}{#1}{#2}{#3}}

% Shorthand for monad types
\newcommand\mea[0]{\M{\e}{A}}
\newcommand\meb[0]{\M{\e}{B}}
\newcommand\mec[0]{\M{\e}{C}}

\newcommand\tea[0]{\T{\e}{A}}
\newcommand\teb[0]{\T{\e}{B}}
\newcommand\tec[0]{\T{\e}{C}}


\newcommand\moa[0]{\M{\1}{A}}
\newcommand\mob[0]{\M{\1}{B}}
\newcommand\moc[0]{\M{\1}{C}}

\newcommand\toa[0]{\T{\1}{A}}
\newcommand\tob[0]{\T{\1}{B}}
\newcommand\toc[0]{\T{\1}{C}}

\newcommand\aeb[0]{\lamtype{A}{\e}{B}}

% Shorthand for Gammas
\newcommand{\gax}[0]{\G, x: A}
\newcommand{\gby}[0]{\G, y: B}

% reduction function
\newcommand{\reduce}[0]{reduce}



% Combinators for building delta-based tree proof terms
\newcommand{\deltavrule}[4]{
    \treerule{Subtype}{\treerule{}{\D}{\typerelation{#1}{#2}{#3}}\s\s #3 \subtype #4}{\typerelation{#1}{#2}{#4}}}

\newcommand{\deltavruleprime}[4]{
    \treerule{Subtype}{\treerule{}{\D'}{\typerelation{#1}{#2}{#3}}\s\s #3 \subtype #4}{\typerelation{#1}{#2}{#4}}}

\newcommand{\deltavruleprimeprime}[4]{
        \treerule{Subtype}{\treerule{}{\D'}{\typerelation{#1}{#2}{#3}}\s\s #3 \subtype #4}{\typerelation{#1}{#2}{#4}}}
    
\newcommand{\deltacrule}[6]{
            \treerule{Subeffect}{\treerule{}{\D}{\typerelation{#1}{#2}{\M{#3}{#4}}}\s\s #4 \subtype #6\s\s #3 \subeffect #5}{\typerelation{#1}{#2}{\M{#5}{#6}}}}
\newcommand{\deltacruleprime}[6]{
    \treerule{Subeffect}{\treerule{}{\D'}{\typerelation{#1}{#2}{\M{#3}{#4}}}\s\s #4 \subtype #6\s\s #3 \subeffect #5}{\typerelation{#1}{#2}{\M{#5}{#6}}}}
\newcommand{\deltacruleprimeprime}[6]{
    \treerule{Subeffect}{\treerule{}{\D''}{\typerelation{#1}{#2}{\M{#3}{#4}}}\s\s #4 \subtype #6\s\s #3 \subeffect #5}{\typerelation{#1}{#2}{\M{#5}{#6}}}}
                            

\newcommand{\p}[0]{\pi_1}
\newcommand{\pp}[0]{\pi_2}

% short-hands for weakening
\newcommand{\wrel}[3]{#1 : #2 \triangleright #3}
\newcommand{\ok}[1]{#1 \texttt{Ok}}
\renewcommand\i[0]{\iota}
\newcommand\w{\omega}
\newcommand\dom[1]{\texttt{dom}(#1)}
\newcommand\x{\times}


% Combinators to build tree proofs
\newcommand{\truleconst}[0]{\treerule{Const}{\ok{\G}}{\gtyperelation{\const{A}}{A}}}
\newcommand{\truleunit}[0]{\treerule{Unit}{\ok{\G}}{\typerelation{\G}{\u}{\U}}}
\newcommand{\truletrue}[0]{\treerule{True}{\ok{\G}}{\typerelation{\G}{\t}{\B}}}
\newcommand{\trulefalse}[0]{\treerule{False}{\ok{\G}}{\typerelation{\G}{\f}{\B}}}


\newcommand{\E}[0]{\mathbb{E}}
\renewcommand{\dot}{\cdot}
\newcommand{\gens}[0]{::=}
\newcommand{\nil}[0]{\diamond}
\newcommand{\ground}[0]{\gamma}

% Terminal object of C
\newcommand{\terminal}[0]{\texttt{\1}}

% The category C
\newcommand{\C}[0]{\mathbb{C}}

% The category of locally-small categories
\newcommand{\Cat}[0]{\texttt{Cat}}
% Sub-effect Nat-trans
\newcommand{\dse}[2]{\db{#1 \subeffect #2}}

\newcommand\app[0]{\texttt{app}}
\newcommand\cur[1]{\texttt{cur}(#1)}
\newcommand{\ifnt}[1]{\texttt{If}_{#1}}


\newcommand{\setto}{:=}
\newcommand{\fv}[1]{\texttt{fv}(#1)}

% shorthand for inserting text to equations
\newcommand\qt[1]{\quad\text{#1}}


\fi

\begin{document}
We induct over the structure of typing derivations of $\gtyperelation{t}{\tau}$, assuming $\wrel{\w}{\G'}{\G}$ holds. In each case, we construct the new derivation $\D'$ from the derivation $\D$ giving $\gtyperelation{t}{\tau}$ and show that $\D\after\deno{\wrel{\w}{\G'}{\G}} = \D'$

    \subsection{Variable Terms}
    \paragraph{Case Var and Weaken}
        We case split on the weakening $\w$.
        \subparagraph{If $\w = \i$}
        Then $\G' = \G$, and so $\typerelation{\G'}{x}{A}$ holds and the derivation $\D'$ is the same as $\D$

        \begin{equation}
            \D' = \D = \D\after\idg = \D\after\deno{\wrel{\i}{\G}{\G}} 
        \end{equation}
        \subparagraph{If $\w = \w'\pi$}
        Then  $\G' = (\G'',x': A')$ and $\wrel{\w'}{\G''}{\G}$. So by induction, there is a tree, $\D_1$ deriving $\typerelation{\G''}{x}{A}$,  such that 
        \begin{equation}
            \D_1 = \D\after\deno{\wrel{\w'}{\G''}{\G}} \qt{By Induction}
        \end{equation}
        
        , and hence by the weaken rule, we have 
        \begin{equation}
            \treerule{Weaken}{\typerelation{\G''}{x}{A}}{\typerelation{\G'', x':A' }{x}{A}}
        \end{equation}

        This preserves denotations:
        \begin{align}
            \D' & = \D_1\after\p\qt{By Definition} \\
            & = \D\after\deno{\wrel{\w'}{\G''}{\G}}\after\p\qt{By induction}\\
            & = \D\after\deno{\wrel{\w'\p}{\G'}{\G}}\qt{By denotation of weakening}
        \end{align}

        \subparagraph{If $\w = \w'\x$} 
        Then 
        \begin{align}
            \G' & = \G''', x': B\\
            \G &= \G'', x': A'\\
            B & \subtype A
        \end{align}

        \subparagraph{If $x = x'$}

        Then $A = A'$.

        Then we derive the new derivation, $\D'$ as so:

        \begin{equation}
            \treerule{Sub-type}{
                \treerule{var}{}{\typerelation{\G''', x: B}{x}{B}}
                \s\s
                B \subtype A
            }{
                \typerelation{\G'}{x}{A}
            }
        \end{equation}

        This preserves denotations:

        \begin{align}
            \D' & = \deno{B\subtype A}\after\pp\qt{By Definition} \\
             & = \pp\after (\deno{\wrel{\w'}{\G'''}{\G''}}\times \deno{B\subtype A}) \qt{By the properties of binary products}\\
             & = \D\after\deno{\wrel{\w}{\G'}{\G}}\qt{By Definition}
        \end{align}

        \subparagraph{Case $x \neq x'$}
        Then 
        \begin{equation}
            \D = \treerule{Weaken}{\treerule{}{\D_1}{\typerelation{\G''}{x}{A}}}{\gtyperelation{x}{A}}
        \end{equation}

        By induction with $\wrel{\w}{\G'''}{\G''}$,
         we have a derivation $\D_1$ of $\typerelation{\G'''}{x}{A}$

        We have the weakened derivation:

        \begin{equation}
            \D' = \treerule{Weaken}{\treerule{}{\D_1'}{\typerelation{\G'''}{x}{A}}}{\typerelation{\G'}{x}{A}}
        \end{equation}

        This preserves denotations:

        By induction, we have
        \begin{equation}
            \D_1' = \D_1 \after \deno{\wrel{\w}{\G'''}{\G''}}
        \end{equation}
        So we have:
        \begin{align}
            \D' &= \D_1' \after \p\qt{By denotation definition}\\
            & = \D_1\after\deno{\wrel{\w'}{\G'''}{\G''}}\qt{By induction}\after\p \\
            & = \D_1\after\p\after(\deno{\wrel{\w'}{\G'''}{\G''}}\times\deno{A' \subtype B})\qt{By product properties}\\
            & = \D\after\deno{\wrel{\w}{\G'}{\G}}\qt{By definition}
        \end{align}

    \subsection{Value Terms}
    From this point onwards, since we no-longer case split over the weakening relations, we write the denotation $\deno{\wrel{\w}{\G'}{\G'}}$, simply as $\w$.


    \paragraph{Case Constant}
    The constant typing rules, $\u$, $\t$, $\f$, $\const{A}$, all proceed by the same logic. Hence I shall only prove the theorems for the case $\const{A}$.

    \begin{equation}
        \truleconst
    \end{equation}

    By inversion, we have $\ok{\G}$, so we have $\ok{\G'}$.

    Hence

    \begin{equation}
        \treerule{Const}{\ok{\G'}}{\typerelation{\G'}{\const{A}}{A}}
    \end{equation}
    Holds.

    This preserves denotations:


    \begin{align}
        \D' & = \deno{\const{A}} \after \term{\G'}\qt{By definition}\\
        & = \deno{\const{A}} \after \term{\G}\after \w \qt{By the terminal property}\\
        & = \D\qt{By Definition}\\
    \end{align}


    \paragraph{Case Lambda}
    By inversion, we have a derivation $\D_1$ giving

    \begin{equation}
        \D = \treerule{Fn}{
            \treerule{}{\D_1}{\typerelation{\gax}{C}{\meb}}
        }{\gtyperelation{\lam{x}{A}{C}}{\aeb}}
    \end{equation}

    Since $\wrel{\w}{\G'}{\G}$, we have:

    \begin{equation}
        \wrel{\w\x}{(\G,x:  A)}{(\gax)}
    \end{equation}

    Hence, by induction, using $\wrel{\w\x}{(\G,x:  A)}{(\gax)}$, we derive $\D_1'$:

    \begin{equation}
        \D' = \treerule{Fn}{
            \treerule{}{\D_1'}{\typerelation{\G',x: A}{C}{\meb}}
        }{\typerelation{\G',x: A}{\lam{x}{A}{C}}{\aeb}}
    \end{equation}

    This preserves denotations:


    \begin{align}
    \D' & = \cur{\D_1'} \qt{By Definition}\\
    & = \cur{\D_1\after(\w\times \idg)}\qt{By the denotation of $\w\x$} \\
    &= \cur{\D_1}\after\w\qt{By the exponential property}\\
    &= \D\after \w \qt{By Definition}
    \end{align}


    \paragraph{Case Sub-typing}

    \begin{equation}
        \treerule{Sub-type}{\gtyperelation{v}{A}\s\s A\subtype B}{\gtyperelation{v}{B}}
    \end{equation}

    by inversion, we have a derivation $\D_1$
    \begin{equation}
        \treerule{}{\D_1}{\gtyperelation{v}{A}}
    \end{equation}

    So by induction, we have a derivation $\D_1'$ such that:
    \begin{equation}
        \treerule{Sub-type}{\treerule{}{\D_1'}{\typerelation{\G'}{v}{a}}\s\s A \subtype B}{\typerelation{\G'}{v}{B}}
    \end{equation}

    This preserves denotations:

    \begin{align}
        \D' & = \deno{A\subtype B}\after \D_1' \qt{By Definition} \\
        & = \deno{A\subtype B}\after \D_1\after\w \qt{By induction}\\
        & = \D\after\w \qt{By Definition}\\
    \end{align}

    \subsection{Computation Terms}
    \paragraph{Case Return}
    We have the sub-derivation $\D_1$ such that
    \begin{equation}
        \D = \treerule{Return}{\treerule{}{\D_1}{\gtyperelation{v}{A}}}{\gtyperelation{\return{v}}{\moa}}
    \end{equation}

    Hence, by induction, with $\wrel{\w}{\G'}{\G}$, we find the derivation $\D_1'$ such that:
    \begin{equation}
        \D' = \treerule{Return}{\treerule{}{\D_1'}{\typerelation{\G'}{v}{A}}}{\typerelation{\G'}{\return{v}}{\moa}}
    \end{equation}

    This preserves denotations:

    \begin{align}
        \D' & = \point{A}\after\D_1' \qt{By definition}\\
            & = \point{A}\after\D_1\after\w\qt{By induction of $\D_1, \D_1'$}\\
            & = \D\after\w\qt{By Definition}
    \end{align}

    \paragraph{Case Apply}
        By inversion, we have derivations $\D_1$, $\D_2$ such that

        \begin{equation}
            \D = 
            \treerule{Apply}{
                \treerule{}{\D_1}{\gtyperelation{v_1}{\aeb}}
                \s\s
                \treerule{}{\D_2}{\gtyperelation{v_2}{A}}
            } {
                \gtyperelation{\apply{v_1}{v_2}}{\meb}
            }
        \end{equation}

        By induction, this gives us the respective derivations: $\D_1',\D_2'$ such that

        
        \begin{equation}
            \D' = 
            \treerule{Apply}{
                \treerule{}{\D_1'}{\typerelation{\G'}{v_1}{\aeb}}
                \s\s
                \treerule{}{\D_2'}{\typerelation{\G'}{v_2}{A}}
            } {
                \typerelation{\G'}{\apply{v_1}{v_2}}{\meb}
            }
        \end{equation}

        This preserves denotations:

        \begin{align}
            \D' &= \app\after\pr{\D_1'}{\D_2'} \qt{By Definition}\\
            &= \app\after\pr{\D_1\after\w}{\D_2\after\w} \qt{By induction on $\D_1, \D_2$}\\
            &= \app\after\pr{\D_1}{\D_2}\after\w\\
            &= \D\after\w\qt{By Definition}
        \end{align}
    \paragraph{Case If}
    By inversion, we have the sub-derivations $\D_1,\D_2,\D_3$, such that:


    \begin{equation}
        \D = \treerule{If}{
            \treerule{}{\D_1}{\typerelation{\G}{v}{\B}}
            \s\s
            \treerule{}{\D_2}{\typerelation{\G}{C_1}{\mea}}
            \s\s
            \treerule{}{\D_3}{\typerelation{\G}{C_2}{\mea}}
        }{
            \typerelation{\G}{\ifthenelse{\e}{A}{v}{C_1}{C_2}}{\mea}
        }
    \end{equation}

    By induction, this gives us the sub-derivations $\D_1', \D_2', \D_3'$ such that

    \begin{equation}
        \D' = \treerule{If}{
            \treerule{}{\D_1'}{\typerelation{\G'}{v}{\B}}
            \s\s
            \treerule{}{\D_2'}{\typerelation{\G'}{C_1}{\mea}}
            \s\s
            \treerule{}{\D_3'}{\typerelation{\G'}{C_2}{\mea}}
        }{
            \typerelation{\G'}{\ifthenelse{\e}{A}{v}{C_1}{C_2}}{\mea}
        }
    \end{equation}

    And 
    \begin{align}
        \D_1' & =\D_1 \after \w\\
        \D_3' & =\D_2 \after \w\\
        \D_3' & =\D_3 \after \w 
    \end{align}


    This preserves denotations.
    Since $\w: \G' \rightarrow \G$, \\
    Let $(\tea)^{\w}: \tea^{\G}\rightarrow\tea^{\G'}$ be as defined in ExSh 3 (\footnote{https://www.cl.cam.ac.uk/teaching/1819/L108/exercises/L108-exercise-sheet-3.pdf})
    That is:

    \begin{align}
        (\tea)^{\w} & = \cur{\app\after(\Id{\tea}\times w)}
    \end{align}.
    And hence, we have:

    \begin{align}
        \cur{f\after(\Id{}\times \w)} & = (\tea)^{\w} \after\cur{f}
    \end{align}

    \begin{align}
        \D' & =\app\after((\fld{\cur{\D_2'\after\pp}}{\cur{\D_3'\after\pp}}\after\D_1')\times \Id{\G'})\after\diag{\G'}\qt{By Definition}\\
        & =\app\after((\fld{\cur{\D_2\after\w\after\pp}}{\cur{\D_3\after\w\after\pp}}\after\D_1')\times \Id{\G'})\after\diag{\G'}\qt{By Induction}\\
        & = \app\after((\fld{\cur{\D_2\after\pp\after(\Id{\1}\times \w)}}{\cur{\D_3\after\pp\after(\Id{\1}\times \w)}}\after\D_1\after\w)\times \Id{\G'})\after\diag{\G'}\qt{By product property}\\
        & = \app\after((\fld{(\tea)^{\w}\after\cur{\D_2\after\pp}}{(\tea)^{\w}\after\cur{\D_3\after\pp}}\after\D_1\after\w)\times \Id{\G'})\after\diag{\G'}\qt{By $(\tea)^{\w}$ property}\\
        & = \app\after(((\tea)^{\w}\after\fld{\cur{\D_2\after\pp}}{\cur{\D_3\after\pp}}\after\D_1\after\w)\times \Id{\G'})\after\diag{\G'}\qt{Factor out transformation}\\
        & = \app\after((\tea)^{\w}\times\Id{\G'})\after((\fld{\cur{\D_2\after\pp}}{\cur{\D_3\after\pp}}\after\D_1)\times \Id{\G'})\after(\w \times \Id{\G'})\after\diag{\G'}\qt{Factor out Identity pairs}\\
        & = \app\after(\Id{(\tea)}\times\w)\after((\fld{\cur{\D_2\after\pp}}{\cur{\D_3\after\pp}}\after\D_1) \times \Id{\G'})\after(\w \times \Id{\G'})\after\diag{\G'}\qt{By defintion of $\app, (\tea)^{\w}$}\\
        & = \app\after((\fld{\cur{\D_2\after\pp}}{\cur{\D_3\after\pp}}\after\D_1)\times \idg)\after(\w \times \w)\after\diag{\G'}\qt{Push through pairs}\\
        & = \app\after((\fld{\cur{\D_2\after\pp}}{\cur{\D_3\after\pp}}\after\D_1)\times \idg)\after\diag{\G}\after\w\qt{By Definition of the diagonal morphism.}\\
        & = \D\after\w
    \end{align}


    \paragraph{Case Bind}
    By inversion, we have derivations $\D_1, \D_2$ such that:


    \begin{equation}
        \D = \treerule{Bind}{
            \treerule{}{\D_1}{\typerelation{\G}{C_1}{\M{\E_1}{A}}}
            \s\s
            \treerule{}{\D_2}{\typerelation{\G,x: A}{C_2}{\M{\e_2}{B}}}
        }{
            \typerelation{\G}{\doin{x}{C_1}{C_2}}{\M{\e_1\dot\e_2}{B}}
        }
    \end{equation}

    If $\wrel{\w}{\G'}{\G}$ then $\wrel{\w\x}{\G',x:A}{\gax}$, so by induction, we can derive $\D_1'$, $\D_2'$ such that:

    \begin{equation}
        \D' = \treerule{Bind}{
            \treerule{}{\D_1'}{\typerelation{\G'}{C_1}{\M{\E_1}{A}}}
            \s\s
            \treerule{}{\D_2'}{\typerelation{\G',x: A}{C_2}{\M{\e_2}{B}}}
        }{
            \typerelation{\G'}{\doin{x}{C_1}{C_2}}{\M{\e_1\dot\e_2}{B}}
        }
    \end{equation}

    This preserves denotations:

    \begin{align}
        \D' & = \bind{\e_1}{\e_2}{B}\after\T{\e_1}{\D_2'}\after\tstrength{\e_1}{\G'}{A}\after\pr{\Id{G'}}{\D_1'}\qt{By definition}\\
        & = \bind{\e_1}{\e_2}{B}\after\T{\e_1}{(\D_2\after(\w\times\Id{A}))}\after\tstrength{\e_1}{\G'}{A}\after\pr{\Id{G'}}{\D_1\after\w}\qt{By induction on $\D_1', \D_2'$}\\
        & = \bind{\e_1}{\e_2}{B}\after\T{\e_1}{\D_2}\after\tstrength{\e_1}{\G}{A}\after\pr{\w}{\D_1\after\w}\qt{By tensor strength}\\
        & = \bind{\e_1}{\e_2}{B}\after\T{\e_1}{\D_2}\after\tstrength{\e_1}{\G}{A}\after\pr{\idg}{\D_1}\after\w\qt{By product property}\\
        & = \D \qt{By definition}
    \end{align}





    \paragraph{Case Sub-effect}


    \begin{equation}
        \treerule{Sub-effect}{\gtyperelation{C}{\M{\e_1}{A}}\s\s A\subtype B\s\s \e_1\subeffect\e_2}{\gtyperelation{C}{\M{\e_2}{B}}}
    \end{equation}

    by inversion, we have a derivation $\D_1$
    \begin{equation}
        \treerule{}{\D_1}{\gtyperelation{C}{\M{\e_1}{A}}}
    \end{equation}

    So by induction, we have a derivation $\D_1'$ such that:
    \begin{equation}
        \treerule{Sub-effect}{\treerule{}{\D_1'}{\typerelation{\G'}{C}{\M{\e_1}{A}}}\s\s A \subtype B\s\s \e_1\subeffect\e_2}{\typerelation{\G'}{C}{\M{\e_2}{B}}}
    \end{equation}

    This preserves denotations:

    Let
    \begin{align}
        g &= \deno{A \subtype B}: A \rightarrow B\\
        h &= \deno{\e_1\subeffect\e_2}: \T{\e_1}{}\rightarrow\T{\e_2}{}
    \end{align}
    Then
    \begin{align}
        \D' & = h_B\after\T{\e_1}{g}\after \D_1' \qt{By Definition}\\
        & = h_B\after\T{\e_1}{g}\after \D_1\after\w\qt{By Induction}\\
        & = \D\after\w\qt{By Definition}
    \end{align}

\end{document}
