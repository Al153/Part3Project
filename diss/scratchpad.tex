\documentclass{report}

%% Don't import the header multiple times

\ifdefined\HEADERIMPORTED
\else
\newcommand\HEADERIMPORTED[0]{This file is HEADERIMPORTED}
\usepackage{amssymb}

\usepackage{amsmath}
\usepackage[a4paper,includeheadfoot,margin=2.54cm]{geometry}
\usepackage{breqn}
\usepackage{amssymb}

\usepackage{amsmath}
\usepackage[a4paper,includeheadfoot,margin=2.54cm]{geometry}
\usepackage{breqn}
\usepackage{array}   % for \newcolumntype macro
\usepackage{tikz-cd}
\usepackage{tabstackengine}
\setstackEOL{\cr}
\setstackgap{L}{\normalbaselineskip}

\newcommand\todo[1]{\textbf{TODO: #1}}

\newcommand{\s}{\indent}
\newcommand{\doin}[3]{\texttt{do}\s #1 \leftarrow #2 \s\texttt{in}\s #3}
\newcommand\apply[2]{#1\s#2}
\newcommand\ifthenelse[5]{\texttt{if}_{#1, #2}\s#3\s \texttt{then}\s #4 \s\texttt{else} \s#5}
\newcommand\const[1]{\texttt{C}^{#1}}
\newcommand\return[1]{\texttt{return} #1}
\newcommand\lam[3]{\lambda #1 : #2. #3}
\renewcommand\u[0]{\texttt{()}}
\newcommand{\U}[0]{\texttt{Unit}}
\renewcommand\t[0]{\texttt{true}}
\newcommand\f[0]{\texttt{false}}
\newcommand{\B}[0]{\texttt{Bool}}
\newcommand{\G}[0]{\Gamma}
\newcommand\D{\Delta}


% draw type relations
\newcommand{\typerelation}[3]{#1 \vdash #2 \colon #3}
\newcommand{\gtyperelation}[2]{\typerelation{\G}{#1}{#2}}

%draw tree rules
\newcommand{\treerule}[3]{(\text{#1}) \frac{#2}{#3}}
\newcommand{\condtreerule}[4]{\treerule{#1}{#2}{#3}(\text{if } #4)}

\newcommand{\subtype}[0]{\leq:}
\newcommand\subeffect[0]{\leq}

\newcommand{\M}[2]{\texttt{M}_{#1}{#2}}

\newcommand\lamtype[3]{#1 \rightarrow \M{#2}{#3}}
\newcommand{\1}[0]{\texttt{1}}

\newcommand\e[0]{\epsilon}

\newcommand{\db}[1]{{\bf [\![}#1{\bf ]\!]}}
\newcommand{\deno}[1]{\db{#1}_M}
\newcommand\after\circ
\newcommand\term[1]{\left\langle\right\rangle_{#1}}

\newcommand\point[1]{\eta_{#1}}
\newcommand\bind[3]{\mu_{#1, #2, #3}}

\newcommand\T[2]{T_{#1}{#2}}

\newcommand\pr[2]{\left\langle#1, #2\right\rangle}

% tensor strength Nat-tran
\newcommand\tstrength[3]{\texttt{t}_{#1, #2, #3}}

% Id morphism
\newcommand\Id[1]{\texttt{Id}_{#1}}

\newcommand\idg[0]{\Id{\G}}
% beta-eta equivalence
\newcommand\beequiv[0]{=_{\beta\eta}}
% Substitutions
\newcommand\si{\sigma}

\newcommand{\sub}[1]{\left[#1\right]}
\newcommand{\ssub}[2]{\left[#2 / #1\right]}
\newcommand{\ssi}[0]{\sub{\si}}

% beta-eta equivalence relation
\newcommand{\berelation}[4]{\typerelation{#1}{#2 \beequiv #3}{#4}}
\newcommand{\gberelation}[3]{\gtyperelation{#1 \beequiv #2}{#3}}


% Shortcuts for denotational equality
\newcommand{\denoequality}[4]{\deno{\typerelation{#1}{#2}{#4}} = \deno{\typerelation{#1}{#3}{#4}}}
\newcommand{\gdenoequality}[3]{\denoequality{\G}{#1}{#2}{#3}}

% Shorthand for monad types
\newcommand\mea[0]{\M{\e}{A}}
\newcommand\meb[0]{\M{\e}{B}}
\newcommand\mec[0]{\M{\e}{C}}

\newcommand\tea[0]{\T{\e}{A}}
\newcommand\teb[0]{\T{\e}{B}}
\newcommand\tec[0]{\T{\e}{C}}


\newcommand\moa[0]{\M{\1}{A}}
\newcommand\mob[0]{\M{\1}{B}}
\newcommand\moc[0]{\M{\1}{C}}

\newcommand\toa[0]{\T{\1}{A}}
\newcommand\tob[0]{\T{\1}{B}}
\newcommand\toc[0]{\T{\1}{C}}

\newcommand\aeb[0]{\lamtype{A}{\e}{B}}

% Shorthand for Gammas
\newcommand{\gax}[0]{\G, x: A}
\newcommand{\gby}[0]{\G, y: B}

% reduction function
\newcommand{\reduce}[0]{reduce}



% Combinators for building delta-based tree proof terms
\newcommand{\deltavrule}[4]{
    \treerule{Subtype}{\treerule{}{\D}{\typerelation{#1}{#2}{#3}}\s\s #3 \subtype #4}{\typerelation{#1}{#2}{#4}}}

\newcommand{\deltavruleprime}[4]{
    \treerule{Subtype}{\treerule{}{\D'}{\typerelation{#1}{#2}{#3}}\s\s #3 \subtype #4}{\typerelation{#1}{#2}{#4}}}

\newcommand{\p}[0]{\pi_1}
\newcommand{\pp}[0]{\pi_2}

% short-hands for weakening
\newcommand{\wrel}[3]{#1 : #2 \triangleright #3}
\newcommand{\ok}[1]{#1 \texttt{Ok}}
\renewcommand\i[0]{\iota}
\newcommand\w{\omega}
\newcommand\dom[1]{\texttt{dom}(#1)}
\newcommand\x{\times}


% Combinators to build tree proofs
\newcommand{\truleconst}[0]{\treerule{Const}{\ok{\G}}{\gtyperelation{\const{A}}{A}}}
\newcommand{\truleunit}[0]{\treerule{Unit}{\ok{\G}}{\typerelation{\G}{\u}{\U}}}
\newcommand{\truletrue}[0]{\treerule{True}{\ok{\G}}{\typerelation{\G}{\t}{\B}}}
\newcommand{\trulefalse}[0]{\treerule{False}{\ok{\G}}{\typerelation{\G}{\f}{\B}}}


\newcommand{\E}[0]{\mathbb{E}}
\renewcommand{\dot}{\cdot}
\newcommand{\gens}[0]{::=}
\newcommand{\nil}[0]{\diamond}
\newcommand{\ground}[0]{\gamma}

% Terminal object of C
\newcommand{\terminal}[0]{\texttt{\1}}

% The category C
\newcommand{\C}[0]{\mathbb{C}}

% The category of locally-small categories
\newcommand{\Cat}[0]{\texttt{Cat}}
% Sub-effect Nat-trans
\newcommand{\dse}[2]{\db{#1 \subeffect #2}}

\newcommand\app[0]{\texttt{app}}
\newcommand\cur[1]{\texttt{cur}(#1)}
\newcommand{\ifnt}[1]{\texttt{If}_{#1}}


\newcommand{\setto}{:=}
\newcommand{\fv}[1]{\texttt{fv}(#1)}



\fi

\newcommand\edeltavrule[5]{\deltavrule{#1 \mid #2}{#3}{#4}{#5}}

\newcommand{\edeltavruleprime}[5]{
        \deltavruleprime{#1\mid #2}{#3}{#4}{#5}}
    
\newcommand{\edeltavruleprimeprime}[5]{
        \deltavruleprimeprime{#1\mid #2}{#3}{#4}{#5}}
    
\newcommand{\edeltacrule}[6]{
            \treerule{Subtype}{
                \treerule{}{
                    \D
                }{
                    \typerelation{\P\mid#1}{#2}{\M{#3}{#4}}
                }
                \s\s
                \treerule{Computation}{
                    #4 \subtypep #6
                    \s\s #3 \subeffectp #5
                }{
                    \M{#3}{#4}\subtypep{\M{#5}{#6}}
                }
            }{
                \typerelation{\P\mid #1}{#2}{\M{#5}{#6}}
            }
        }
        

        \newcommand{\edeltacruleprime}[6]{
            \treerule{Subtype}{
                \treerule{}{
                    \D'
                }{
                    \typerelation{\P\mid #1}{#2}{\M{#3}{#4}}
                }
                \s\s
                \treerule{Computation}{
                    #4 \subtypep #6
                    \s\s #3 \subeffectp #5
                }{
                    \M{#3}{#4}\subtypep{\M{#5}{#6}}
                }
            }{
                \typerelation{\P\mid #1}{#2}{\M{#5}{#6}}
            }
        }
                  

\begin{document}

\section{Reduced Type Derivation}
A reduced type derivation is one where subtype and sub-effect rules must, and may only, occur at the root or directly above an \textbf{if}, or \textbf{apply} rule.

In this section, I shall prove that there is at most one reduced derivation of $\gpetyperelation{v}{A}$. Secondly, I shall present a function for generating reduced derivations from arbitrary typing derivations, in a way that does not change the denotations. These imply that all typing derivations of a type-relation have the same denotation.

\section{Reduced Type Derivations are Unique}
For each instance of the relation $\gpetyperelation{v}{A}$,there exists at most one reduced derivation of  $\gpetyperelation{v}{A}$. This is proved by induction over the typing rules on the bottom rule used in each derivation.

\proof
We induct on the structure of terms.

\paragraph{Case Variables}
To find the unique derivation of $\gpetyperelation{x}{A}$, we case split on the type-environment, $\G$.

\paragraph{Case $\G = \G', x: A'$}
Then the unique reduced derivation of $\gpetyperelation{x}{A}$ is, if $A' \subtypep A$, as below:

\begin{equation}
    \treerule{Subtype}{\treerule{Var}{\ok{\G', x: A'}}{\etyperelation{\P}{\G,x: A'}{x}{A'}}\s\s A' \subtype A}{\etyperelation{\P}{\G', x: A'}{x}{A}}
\end{equation}

\paragraph{Case $\G = \G', y: B$} with $y \neq x$.

Hence, if $\gpetyperelation{x}{A}$ holds, then so must $\etyperelation{\P}{\G'}{x}{A}$.

Let 
\begin{equation}
    \edeltavrule{\P}{\G'}{x}{A'}{A}
\end{equation}
Be the  unique reduced derivation of $\etyperelation{\P}{\G'}{x}{A}$.

Then the unique reduced derivation of $\gpetyperelation{x}{A}$ is:


\begin{equation}
    \treerule{Subtype}{
        \treerule{Weaken}{
            \treerule{}{\D}{\etyperelation{\P}{\G, x: A'}{x}{A'}}
        }{\gpetyperelation{x}{A'}}   
    \s\s A' \subtypep A
    }{\gpetyperelation{x}{A}}
\end{equation}

\paragraph{Case Constants}
For each of the constants, ($\const{A}$, $\t$, $\f$, $\u$), there is exactly one possible derivation for $\etyperelation{\P}{\G}{c}{A}$ for a given A. I shall give examples using the case $\const{A}$


$$
    \treerule{Subtype}{\truleconst\s\s A \subtypep B}{\gpetyperelation{\const{A}}{B}}
$$

If $A = B$, then the subtype relation is the identity subtype ($A \subtypep A$).



\paragraph{Case Lambda}
The reduced derivation of $\gpetyperelation{\lam{x}{A}{v}}{\fntype{A'}{B'}}$ is:


$$
\treerule{Subtype}
{\treerule{Lambda}{\treerule{}{\D}{\etyperelation{\P}{\gax}{v}{B}}}
{
    \gpetyperelation{\lam{x}{A}{B}}{\ab}}
    \s\s
    \ab \subtypep \fntype{A'}{B'}
}{
   \gpetyperelation{\lam{x}{A}{v}}{\fntype{A'}{B'}} 
}
$$

Where 
\begin{equation}
    \treerule{Sub-Type}{\treerule{}{\D}{\etyperelation{\P}{\gax}{v}{B}}\s\s B\subtypep B'}{\etyperelation{\P}{\gax}{v}{B'}}
\end{equation}
is the reduced derivation of $\etyperelation{\P}{\gax}{v}{B'}$ if it exists.

\subsection{Computation Terms}
\paragraph{Case Return}
The reduced derivation of $\gpetyperelation{\return{v}}{\meb}$ is 
$$
    \treerule{Subtype}
    {
        \treerule{Return}
        {\treerule{}{\D}{\gpetyperelation{v}{A}}}
        {\gpetyperelation{\return{v}}{\moa}}
        \s\s
        \treerule{Computation}{            
            A \subtypep B
        \s\s
        \1 \subeffectp \e}{\moa\subeffectp\meb}
    }
    {\gpetyperelation{\return{v}}{B}}
$$

Where $$\edeltavrule{\P}{\G}{v}{A}{B}$$ is the reduced derivation of $\gpetyperelation{v}{B}$

\paragraph{Case Apply}
If 
$$
    \edeltavrule{\P}{\G}{v_1}{\ab}{\fntype{A'}{B'}}
$$ and $$
    \edeltavruleprime{\P}{\G}{v_2}{A''}{A'}
$$

Are the reduced type derivations of $\gpetyperelation{v_1}{\fntype{A'}{B'}}$ and $\gpetyperelation{v_2}{A'}$



Then we can construct the reduced derivation of $\gpetyperelation{\apply{v_1}{v_2}}{\M{\e'}{B'}}$ as

$$
    \treerule{Subeffect}{
        \treerule{Apply}{
            \treerule{}{\D}{
                \gpetyperelation{v_1}{\ab}
            }
            \s\s
            \treerule{Subtype}{
                \treerule{}{\D'}{
                    \gpetyperelation{v}{A''}
                } \s\s A'' \subtypep A
            }
            {\gpetyperelation{v}{A}}
        }{
            \gpetyperelation{\apply{v_1}{v_2}}{B}
        }
        \s\s
        B \subtypep B'
        \s\s
        \e \subeffectp \e'
    }{
        \gpetyperelation{\apply{v_1}{v_2}}{\M{\e'}{B'}}
    }
$$
\paragraph{Case If}
Let

\begin{equation}
    \edeltavrule{\P}{\G}{v}{B}{\B}
\end{equation}

\begin{equation}
    \edeltavruleprime{\P}{\G}{v_1}{A'}{A}
\end{equation}

\begin{equation}
    \edeltavruleprimeprime{\P}{\G}{v_2}{A''}{A}
\end{equation}

Be the unique reduced reduced derivations of $\gpetyperelation{v}{\B}$, $\gpetyperelation{v_1}{A}$, $\gpetyperelation{v_2}{A}$.

Then the only reduced derivation of $\gpetyperelation{\ifthenelse{\e}{A}{v}{v_1}{v_2}}{A}$ is:

\todo{Scale this properly}
\begin{equation}
    \treerule{Subtype}{
        \treerule{If}{
            \edeltavrule{\P}{\G}{v}{B}{\B}
            \s\s
            \edeltavruleprime{\P}{\G}{v_1}{A'}{A}
            \s\s
            \edeltavruleprimeprime{\P}{\G}{v_2}{A''}{A}
        }{\gpetyperelation{\ifthenelse{\e}{A}{v}{v_1}{v_2}}{A}\s\s \e\subeffectp \e\s\s A \subtypep A}
    }{\gpetyperelation{\ifthenelse{\e}{A}{v}{v_1}{v_2}}{A}}
\end{equation}

\paragraph{Case Bind}

Let 

\begin{equation}
    \edeltacrule{\G}{v_1}{\e_1}{A}{\e_1'}{A'}
\end{equation}

\begin{equation}
    \edeltacruleprime{\G, x:A}{v_2}{\e_2}{B}{\e_2'}{B'}
\end{equation}

Be the respective unique reduced type derivations of the sub-terms]

By weakening, $\wrel{\i\x}{\G, x:A}{\G, x: A'}$ so if there's a derivation of $\etyperelation{\P}{\G, x:A'}{v_2}{B}$, there's also one of   $\etyperelation{\P}{\gax}{v_2}{B}$.

Since the effects monoid operation is monotone, if $\e_1\subeffectp\e_1'$ and $\e_2\subeffectp\e_2'$ then $\e_1\dot\e_2 \subeffectp \e_1'\dot\e_2'$

Hence the reduced type derivation of $\gpetyperelation{\doin{x}{v_1}{v-2}}{\M{\e_1'\dot\e_2'}{B'}}$ is the following:

\todo{Make this and the other smaller}
\begin{equation}
    \treerule{Subeffect} {
        \treerule{Bind}{
            \deltacrule{\G}{v_1}{\e_1}{A}{\e_1'}{A'}
            \s\s
            \deltacruleprime{\G, x:A}{v_2}{\e_2}{B}{\e_2'}{B'}
        } {
            \gpetyperelation{\doin{x}{v_1}{v_2}}{\M{\e_1\dot\e_2}{B}}
        }
        \s\s
        B\subtypep B'
        \s\s
        \e_1\dot\e_2\subeffectp\e_1'\dot\e_2'
    } {
        \gpetyperelation{\doin{x}{v_1}{v-2}}{\M{\e_1'\dot\e_2'}{B'}}
    }
\end{equation}

\section{Each type derivation has a reduced equivalent with the same denotation.}
We introduce a function, $\reduce$ that maps each valid type derivation of $\gpetyperelation{v}{A}$ to a reduced equivalent with the same denotation. To do this, we do case analysis over the root type rule of a derivation and prove that the denotation is not changed.

    \subsection{Constants}
        For the constants $\t, \f, \const{A}$, etc, $\reduce$ simply returns the derivation, as it is already reduced.

        $\reduce(\truleconst) = \truleconst$

    \subsection{Value Types}
        \paragraph{Var}
        \begin{equation}
            \reduce(\treerule{Var}{\ok{\G}}{\etyperelation{\P}{\gax}{x}{A}}) =  \treerule{Var}{\ok{\G}}{\etyperelation{\P}{\gax}{x}{A}}
        \end{equation}

        \paragraph{Weaken}
        \subparagraph{$\reduce$ definition}
        To find:
        \begin{equation}
            \reduce(\treerule{Weaken}{\treerule{}{\D}{\gpetyperelation{x}{A}}}{\etyperelation{\P}{\gby}{x}{A}})
        \end{equation}

        Let 
        \begin{equation}\label{WeakenDeltaReduction}
            \treerule{Subtype}{\treerule{}{\D'}{\gpetyperelation{x}{A}}\s\s A'\subtypep A}{\gpetyperelation{x}{A}} = \reduce(\D)
        \end{equation}

        In 
        \begin{equation}
            \treerule{Subtype}{
            \treerule{Weaken}{
                \treerule{}{\D'}{\gpetyperelation{x}{A'}}
            } {
                \etyperelation{\P}{\gby}{x}{A'}
            }
            \s\s
            A' \subtypep A
            }{\etyperelation{\P}{\gby}{x}{A}}
        \end{equation}

        \paragraph{Lambda}
        \subparagraph{$\reduce$ definition}
            To find:
        
            \begin{equation}
                \reduce(\treerule{Fn}{
                    \treerule{}{\D}{\etyperelation{\P}{\gax}{v}{\M{\e_2}{B}}}
                }{\gpetyperelation{\lam{x}{A}{v}}{\fntype{A}{\e_2}{B}}})
            \end{equation}

            Let 

            \begin{equation}
                \treerule{Sub-effect}{
                    \treerule{}{\D'}{\etyperelation{\P}{\gax}{v}{\M{\e_1}{B'}}}
                    \s\s
                    \e_1\subeffectp\e_2
                    \s\s
                    B' \subtypep B
                }{
                    \etyperelation{\P}{\gax}{v}{\M{\e_2}{B}}
                } = \reduce(\D)
            \end{equation}

            In

            \begin{equation}
                \treerule{Sub-type}{
                    \treerule{Fn}{
                        \D'
                    }{
                        \etyperelation{\P}{\gax}{v}{\M{\e_1}{B'}}
                    }
                    \s\s
                    \fntype{A}{\e_1}{B'}\subtypep\fntype{A}{\e_2}{B}
                } {
                    \gpetyperelation{\lam{x}{A}{v}}{\fntype{A}{\e_2}{B}}
                }
            \end{equation}

        \paragraph{Subtype}
        \subparagraph{$\reduce$ definition}
        To find:
        \begin{equation}
            \reduce(\treerule{Subtype}{\treerule{}{\D}{\gpetyperelation{v}{A}}\s\s A \subtypep B}{\gpetyperelation{v}{B}})
        \end{equation}

        Let 
        \begin{equation}\label{SubtypeDeltaReduction}
            \treerule{Subtype}{\treerule{}{\D'}{\gpetyperelation{x}{A}}\s\s A'\subtypep A}{\gpetyperelation{x}{A}} = \reduce(\D)
        \end{equation}

        In 
        \begin{equation}
            \treerule{Subtype}{
                \treerule{}{\D'}{\gpetyperelation{v}{A'}}
            \s\s
            A' \subtypep A \subtypep B
            }{\gpetyperelation{v}{B}}
        \end{equation}

    \subsection{Computation Types}
        \paragraph{Return}
        \subparagraph{$\reduce$ definition}
        To find:
        
            \begin{equation}
                \reduce(\treerule{Return}{
                    \treerule{}{\D}{\gpetyperelation{v}{A}}
                }{\gpetyperelation{\return{v}}{\moa}})
            \end{equation}

            Let 

            \begin{equation}
                \treerule{Sub-type}{
                    \treerule{}{\D'}{\gpetyperelation{v}{A'}}
                    \s\s
                    A' \subtypep A
                }{
                    \gpetyperelation{v}{A}
                } = \reduce(\D)
            \end{equation}

            In

            \begin{equation}
                \treerule{Sub-effect}{
                    \treerule{Return}{
                        \D'
                    }{
                        \gpetyperelation{v}{A}
                    }
                    \s\s
                    \1\subeffectp\1
                    \s\s
                    A'\subtypep A
                } {
                    \gpetyperelation{\return{v}}{\moa}
                }
            \end{equation}

        \paragraph{Apply}
        \subparagraph{$\reduce$ definition}
        To find:
        \begin{equation}
            \reduce(\treerule{Apply}{
                \treerule{}{\D_1}{
                    \gpetyperelation{v_1}{\ab}
                }
                \s\s
                \treerule{}{\D_2}{
                    \gpetyperelation{v_2}{A}
                }
            }{
                \gpetyperelation{\apply{v_1}{v_2}}{B}
            })
        \end{equation}

        Let
        \begin{align}
            \treerule{Subtype}{
                \treerule{}{\D'_1}{\gpetyperelation{v_1}{\fntype{A'}{B'}}}
                \s\s
                \fntype{A'}{B'}\subtypep\fntype{A}{\e}{B}
            }{
                \gpetyperelation{v_1}{\ab}
            } & = \reduce(\D_1)\\
            \treerule{Subtype}{
                \treerule{}{\D'_2}{\gpetyperelation{v}{A'}}
                \s\s
                A'\subtypep A
            } {
                \gpetyperelation{v_1}{A}
            } & = \reduce(\D_2)
        \end{align}

        In
        \begin{equation}
            \treerule{Sub-effect}{
                \treerule{Apply}{
                    \treerule{}{
                        \D'_1
                    }{
                        \gpetyperelation{v_1}{\fntype{A'}{B'}}
                    }
                \s\s
                    \treerule{Sub-type}{
                        \treerule{}{\D'_2}{\gpetyperelation{v_2}{A''}}
                        \s\s
                        A'' \subtypep A \subtypep A'
                    } {
                        \gpetyperelation{v_2}{A'}
                    }
                }{
                    \gpetyperelation{\apply{v_1}{v_2}}{\M{\e'}{B'}}
                }
                \s\s
                \e' \subeffectp \e
                \s\s
                B' \subtypep B
            }{
                \gpetyperelation{\apply{v_1}{v_2}}{B}
            }
        \end{equation}
        
        \paragraph{If}
       
        \subparagraph{$\reduce$ definition}
            \begin{equation}
                \reduce(\treerule{If}{
                    \treerule{}{\D_1}{\gpetyperelation{v}{\B}}
                    \s\s
                    \treerule{}{\D_2}{\gpetyperelation{v_1}{A}}
                    \s\s
                    \treerule{}{\D_3}{\gpetyperelation{v_2}{A}}
                }{
                    \gpetyperelation{\ifthenelse{\e}{A}{v}{v_1}{v_2}}{A}
                }) = \treerule{If}{
                    \treerule{}{\reduce(\D_1)}{\gpetyperelation{v}{\B}}
                    \s\s
                    \treerule{}{\reduce(\D_2)}{\gpetyperelation{v_1}{A}}
                    \s\s
                    \treerule{}{\reduce(\D_3)}{\gpetyperelation{v_2}{A}}
                }{
                    \gpetyperelation{\ifthenelse{\e}{A}{v}{v_1}{v_2}}{A}
                }
            \end{equation}

        
    
        \paragraph{Bind}
            \subparagraph{$\reduce$ definition}

            To find
            \begin{equation}
                \reduce(
                    \treerule{Bind}{
                        \treerule{}{
                            \D_1
                        }{
                            \gpetyperelation{v_1}{\M{\e_1}{A}}
                        }
                        \s\s
                        \treerule{}{
                            \D_2
                        }{
                            \etyperelation{\P}{\gax}{v_2}{\M{\e_2}{B}}
                        }
                    } {
                        \gpetyperelation{\doin{x}{v_1}{v_2}}{\M{\e_1\dot\e_2}{B}}
                    }
                )
            \end{equation}


            Let \begin{equation}
                \treerule{Sub-effect}{
                    \treerule{}{\D'_1}{\gpetyperelation{v_1}{\M{\e'_1}{A'}}}
                \s\s
                \e_1' \subtypep \e_1
                \s\s
                A' \subtypep A
                }{
                    \gpetyperelation{v_1}{\M{\e_1}{A}}
                } = \reduce(\D_1)
            \end{equation}

            Since $\wrel{i,\x}{\G, x: A'}{\gax}$ if $A' \subtypep A$, and by $\D_2$, $\etyperelation{\P}{(\gax)}{v_2}{\M{\e_2}{B}}$, there also exists a derivation $\D_3$ of $\etyperelation{\P}{(\G, x: A')}{v_2}{\M{\e_2}{B}}$. $\D_3$ is derived from $\D_2$ simply by inserting a (Sub-type) rule below all instances of the (Var) rule.

            Let \begin{equation}
                \treerule{Sub-effect}{
                    \treerule{}{\D'_3}{\etyperelation{\P}{\G, x: A'}{v_2}{\M{\e'_2}{B'}}}
                \s\s
                \e_2' \subtypep \e_2
                \s\s
                B' \subtypep B
                }{
                    \etyperelation{\P}{\G, x: A'}{v_2}{\M{\e_2}{B}}
                } = \reduce(\D_3)
            \end{equation}
            

            Since the effects monoid operation is monotone, if $\e_1\subeffectp\e_1'$ and $\e_2\subeffectp\e_2'$ then $\e_1\dot\e_2 \subeffectp \e_1'\dot\e_2'$
            


            Then the result of reduction of the whole bind expression is:


            \begin{equation}
                \treerule{Sub-effect}{
                    \treerule{Bind}{
                        \treerule{}{
                            \D'_1
                        }{
                            \gpetyperelation{v_1}{\M{\e_1'}{A'}}
                        }
                        \s\s
                        \treerule{}{
                            \D'_3
                        }{
                            \etyperelation{\P}{\G, x: A'}{v_2}{\M{\e_2'}{B'}}
                        }
                    }{
                    \gpetyperelation{\doin{x}{v_1}{v_2}}{\M{\e_1'\dot\e_2'}{B}}
                    }
                    \s\s
                    B' \subtypep B
                    \s\s
                    \e_1'\dot\e_2'\subeffectp\e_1\dot\e_2
                }{
                    \gpetyperelation{\doin{x}{v_1}{v_2}}{\M{\e_1\dot\e_2}{B}}
                }
            \end{equation}
        


\section{Denotations are Equivalent}
For each type relation instance $\gpetyperelation{v}{A}$ there exists a unique reduced derivation of the relation instance. For all derivations $\D$, $\D'$ of the type relation instance, $\deno{\D} = \deno{\reduce{\D}} = \deno{\reduce{\D'}} = \deno{\D'} $, hence the denotation $\deno{\gpetyperelation{v}{A}}$ is unique.

\end{document}
