
\chapter{The Semantics of PEC in an Strictly Indexed Category}
In this chapter, I  describe the category structure required to interpret an instance of PEC\@. I then present denotations for features of the language, such as types, effects, terms, substitutions, and environment weakenings. Finally, I provide outlines and interesting cases of the proofs of the lemmas leading up to and including soundness of the semantics with respect to a $\beta\eta$-conversion-based equational equivalence. However, I first give a brief treatment of the semantics of EC.

\section{Semantics for EC in an S-Category}\label{SCategoryIntro}

As suggested in Section \ref{LanguageFeatureRequirements}, since EC contains multiple effects, STLC terms, and \texttt{if} expressions, we should be able to interpret its semantics in a cartesian closed category with an appropriate strong graded monad and the co-product $\1 + \1$, which models the type of booleans. With the addition of some extra category structure to handle subtyping, which I explain shortly, it is indeed possible to interpret EC. This section contains a fairly high-level treatment of the semantics. This is because the concepts introduced are a subset of those required for the semantics of PEC, which I explain in more detail in Section \ref{PECDenotations}. Semantics for similar languages to EC have also been given using such a construct, such as by Katsumata \cite{Katsumata:2014}. 

In order to model a given instantiation of EC, that is an EC with a collection of effects, constants, and ground types, there should be a collection of objects in the category to represent the ground types. There should also be a \textit{point} morphism (a morphism from the terminal object to the appropriate type) for each constant. In addition to the previously stated requirements, we also require the category, $\C$, to be able to model subtyping. If a cartesian closed category has a morphism to represent each instance of  ground subtyping relation $A\subtypeg B$ and a natural transformation to represent each instance of the subeffect relation $\e_1\subeffect \e_2$, then using the cartesian closed structure of the category, all instances of the general subtyping relation can be modelled. This is explained in Section \ref{PECDenotations}.

From this point onwards, I refer to a category that fulfills these properties of having a strong graded monad, CCC, ground objects and points, subeffect natural transformations, and a co-product $\1 + \1$ as an \textit{S-Category} (Semantic Category).

\begin{framed}
    \begin{definition}[S-Category for an EC Instance]\label{SCategoryDefinition}   
        An S-category for a given instance of the effect calculus is a category that satisfies the following requirements.
        \begin{enumerate}[label=\roman*.]
            \item The category is cartesian closed.
            \item The category has a co-product for the terminal object ($\1 + \1$).
            \item The category has a graded monad indexed by the effect monoid of the EC instance.
            \item The category has an object $\deno{\g}$ for each ground type $\g$ in EC.
            \item The category has a \textit{point} morphism $\deno{\const{A}}: \1\rightarrow \deno{A}$ for each constant in EC. The object $\deno{A}$ is defined in Section \ref{PECDenotations}.
            \item For each instance of the ground subtyping relation, $A \subtypeg B$, there exists a morphism between the objects representing the ground types $A$ and $B$.
            \item For each instance of $\e_1 \subeffect \e_2$, there should exist a natural transformation $\db{\e_1\subeffect\e_2}: \T{\e_1}{}\rightarrow\T{\e_2}{}$ such that it has interactions with the graded monad as specified in Figures \ref{SubeffectTensorStrength}, \ref{SubeffectBind}.
        \end{enumerate}
    \end{definition}
\end{framed}

A full derivation and proof of soundness of the semantics of the Effect Calculus can be found in the online submission documents for this dissertation as it is too long to include here and many of the concepts are explained later in the remainder of this dissertation anyway. The categorical semantics of the Effect Calculus requires an S-category not only to simply model all of the features of EC but also to use the various commutivity diagrams and rules to manipulate the expressions encountered when proving properties of the semantics. Again, as all these manipulations are also applied when proving the semantics of PEC, I shall not go into further detail here.



\begin{figure}[H]
\centering
\begin{minipage}{0.45\linewidth}
    \begin{framed}
        \begin{tikzcd}[ampersand replacement=\&, column sep=huge]
            A \times \T{\e_1}{B} \arrow [r, "\Id{A} \times \dse{\e_1}{\e_2}_B"] \arrow [d, "\tstrength{\e_1}{A}{B}"] \&
            A \times \T{\e_2}{B} \arrow [d, "\tstrength{\e_2}{A}{B}"] \\
            \T{\e_1}{(A \times B)} \arrow [r, "\dse{\e_1}{\e_2}_{ A \times B}"] \&
            \T{\e_2}{(A \times B)} 
        \end{tikzcd}
    \end{framed}
    \caption{The interaction of the subeffect natural transformation with the tensor-strength natural transformation.}
    \label{SubeffectTensorStrength}
\end{minipage}  
\quad
\begin{minipage}{0.45\linewidth}
    \begin{framed}
        \scalebox{0.8}{    
\begin{tikzcd}[ampersand replacement=\&]
    \T{\e_1}{\T{\e_2}{}} 
    \arrow [rr, "\T{\e_1}{\deno{\e_2\subeffect\e_2'}
    }"]
    \arrow [d, "\bind{\e_1}{\e_2}{}"]
    \&  \&
    \T{\e_1}{\T{\e_2'}{}}
    \arrow [rr, "\db{\e_1 \subeffect\e_1'}_{M, \T{\e_2'}{}}"]
    \& \&
     \T{\e_1'}{\T{\e_2'}{}} 
     \arrow [d, "\bind{\e_1'}{\e_2'}{}"]
     \\
    \T{\e_1\dot\e_2}{}
    \arrow [rrrr, "\deno{\e_1\dot\e_2\subeffect\e_1'\dot\e_2'}"]
    \& \&
     \& \&
    \T{\e_1'\dot\e_2'}{}
\end{tikzcd}
}
    \end{framed}

\caption{The interaction of the subeffect natural transformation with the graded monad join natural transformation.}
\label{SubeffectBind}
\end{minipage}  
\end{figure}


\section{Required Category Structure}\label{PECRequirements}

In order to model the polymorphism of PEC, we need to now look at an indexed category. This consists of a base category, $\C$, in which we can interpret the possible effect-variable environments, and a mapping from objects in the base category to S-categories in the category of S-categories. This mapping is denoted from this point onwards as $\C(-)$ and the induced categories $\C(\deno{\P})$ are called \textit{fibres}. To extend our mapping $\C(-)$ to a contravariant functor, as required in Section \ref{IndexedCategoryDefinition}, it needs to map morphisms in the base category $\C$ to functors between fibres. In particular, all functors derived from $\C$ must be \textit{S-preserving} (Definition \ref{SPreservingDefinition}). Thus, each morphism $\theta: \deno{\P'}\rightarrow\deno{\P}$ in $\C$ should induce an S-preserving, re-indexing functor $\theta\star: \C(\deno{\P})\rightarrow\C(\deno{\P'})$ between the fibres.


\begin{framed}
    \begin{definition}[S-Preserving Functor]\label{SPreservingDefinition}
        A functor $F$ preserves the properties of S-categories if it preserves each of the features within an S-category (Definition \ref{SCategoryDefinition}).  For example $F(A\times B) = (FA)\times (FB)$. For a full list of properties that an S-preserving functor should preserve, please see Appendix \ref{AppendixReindexingFunctors}.
    \end{definition}
\end{framed}

The essential idea from this point on is that we have defined several relations of the form $\wellformed{\texttt{Env}}{\texttt{Conclusion}}$, such as the typing relation $\gpetyperelation{v}{A}$, and the well-formedness relation on effects $\wellformedeffect{\P}{\e}$. Each instance of such a relation has a denotation that is an object or morphism in a category. For example, $\deno{\typerelation{\P}{\e}{\effect}}$ is a morphism in the base category, $\deno{\typerelation{\P}{A}{\type}}$ is an object in the fibre (S-category) induced by $\P$, and $\deno{\gpetyperelation{v}{A}}$ is morphism between the objects which denote $\G$ and $A$ in the fibre induced by $\P$.

To form our base category, we pick the concrete category $\Eff$ (Definition \ref{EffDefinition}) of ground effects and monotone functions. The denotation of an effect-variable environment, $\deno{\P}$, is simply $E^n$ where $n$ is the number of variables in $\P$. The denotation of an effect, $\deno{\wellformedeffect{\P}{\e}}$ is a monotone function $E^n\rightarrow E$. For each ground effect $e$, we define $\deno{e}$ to be the constant function $\singleton \mapsto e$.


\begin{framed}
    
    \begin{definition}[The Category of Effects]
        \label{EffDefinition}
        We define $\Eff$ to be a subcategory of $\set$. Its objects are finite powers of $E$, the set of ground effects. That is the objects are of the form $E^n$, which indicates the set of $n$-tuples of ground effects. Morphisms in $\Eff$ are functions monotone with respect to the subeffect partial order ($\subeffect$).
    \end{definition}
\end{framed}

Next, we define a monoidal operator, $\Mul_n: (E^n \rightarrow E)\times (E^n \rightarrow E) \rightarrow (E^n \rightarrow E)$ (Definition \ref{MulDef}). $\Mul$ has the function $\ev\mapsto \1$ as its identity. Note that this makes the set of functions $(E^n \rightarrow E)$ into a partially-ordered monoid. We could use a more general category structure here; the monoidal, natural $\Mul$ operator is all that is needed. However, this extra generality is not essential and requires some more structure to ensure that the semantics of effects is sound with respect to the effect equational equivalence given in Section \ref{PECExtension}.



\begin{framed}
    \begin{definition}[$\Mul$ operator]\label{MulDef}
        $\Mul_n: (E^n \rightarrow E)\times (E^n \rightarrow E) \rightarrow (E^n \rightarrow E) = (f, g)\mapsto \ev\mapsto f(\ev)\dot g(\ev)$
    \end{definition}
    
    \begin{definition}[Naturality]\label{MulNaturality}
        $\Mul$ has the following property:

        \begin{equation*}
            (\Mul_n(f, g)\after \theta) = \ev\mapsto(f(\theta \ev))\dot(g(\theta\ev)) = \Mul_m(f\after\theta, g\after\theta)
        \end{equation*}
    \end{definition}
\end{framed}

There is also a requirement that the indexed category can model ground types and terms. In order to do this,  each fibre should contain an object $\deno{\g}$ for each ground type $\g$. Furthermore, for each constant, $\const{A}$, there should exist a morphism in each fibre: $\deno{\const{A}}: \1 \rightarrow A$. These requirements are satisfied by the fibres all being S-categories.

Our penultimate requirement is that the re-indexing functor induced by $\p: E^n\times E\rightarrow E^n$ (that is $\pstar: \C(E^n) \rightarrow \C(E^n\times E)$) has a right adjoint, denoted by $\allEn: \C(E^n\times E) \rightarrow \C(E^n)$. As the reader might be able to guess, this functor allows us to interpret quantification over effects. This quantification functor need not be S-preserving.

Finally, the adjunction should satisfy the Beck-Chevalley condition, as seen in Figure \ref{BeckChevalleyDiagram}. That is, the functors $\theta\star\after\allEn = \allEm\after(\theta\times\Id{E})\star$ are equal, and the natural transformation $\bar{(\theta\times\Id{E})\star(\counit{})}$ between these functors is equal to the identity natural transformation. This allows us to commute the re-indexing functors with the quantification functor.

\begin{figure}
    \begin{framed}
        \centering
        \begin{tikzpicture}[baseline= (a).base]
                    \node[scale=.7] (a) at (0,0){
                        \begin{tikzcd}[column sep=huge]
                            \C(E^n) 
                            \arrow[bend left=50]{rrrr}[name=U,label=above:$\scriptstyle \theta\star\after\allEn$]{}
                            \arrow[bend right=50]{rrrr}[name=D,label=below:$\scriptstyle \allEm\after(\theta\times\Id{E})\star$]{} 
                            &&&& 
                            \C(E^m\times E)
                            \arrow[name=Canon,shorten <=10pt,shorten >=10pt,equal,to path={(U) -- node[label=right:$\bar{(\theta\times\Id{E})\star(\counit{})}$] {} (D)}]{}
                        \end{tikzcd}
                    };
        \end{tikzpicture}
        
    \end{framed}
    \caption{A functor diagram of the Beck-Chevalley condition.}
    \label{BeckChevalleyDiagram}
\end{figure}

\begin{align*}
    \bar{(\theta\times\Id{E})\star(\counit{})} = \Id{}: \theta\star\after\allEn \rightarrow \allEm\after(\theta\times\Id{E})\star\in \C(E^m)
\end{align*}

From the Beck-Chevalley condition, we can derive some more naturality conditions that will become useful later.
Using the adjunction property, we have that: $\bar{f\after\pstar(n)} = \bar{f}\after n$. Secondly, using the Beck-Chevalley condition, we can derive some non-trivial interactions between re-indexing functors and the adjunction\footnote{This part of the proof comes from a stack-overflow answer: https://math.stackexchange.com/questions/188099/beck-chevalley-condition-and-maps-of-adjunctions}. 

    \begin{align*}
        \theta\star\unit{A}:\quad&\theta\star A \rightarrow (\theta\star\after\allEn\after\pstar) A\\
        \theta\star\unit{} =& \bar{(\theta\times\Id{E})\star(\counit{\pstar})}\after\theta\star\unit{}\\
        =& (\allEm\after(\theta\times\Id{E})\star)(\counit{\pstar})\after\unit{(\allEm\after(\theta\times\Id{E})\star)\after\pstar}\after\theta\star\unit{}\\
        = & (\allEm\after(\theta\times\Id{E})\star)(\counit{\pstar})\after\unit{\theta\star\after\allEn\after\pstar}\after\theta\star\unit{}\\
        = & (\allEm\after(\theta\times\Id{E})\star)(\counit{\pstar}) \after(\theta\star\after\allEn\after\pstar)\unit{} \after\unit{(\theta\times\Id{E})\star}\\
        = & (\theta\star\after\allEn)(\counit{\pstar}
        \after\pstar\unit{})\after\unit{(\theta\times\Id{E})\star}\\
        = & (\theta\star\after\allEn)(\Id{})\after\unit{(\theta\times\Id{E})\star}\\
        = & \unit{(\theta\times\Id{E})\star}
    \end{align*}

    Importantly, this gives us the useful naturality condition that allows us to push re-indexing functors into the adjunction terms.

    \begin{align*}\numberthis\label{NaturalityCondition}
        \theta\star(\bar{f}) & = \theta\star(\allEn(f)\after\eta_A)\\
        & = \theta\star(\allEn(f))\after\theta\star(\eta_A)\\
        & =  (\allEm\after(\theta\times\Id{E})\star)f \after\unit{(\theta\times\Id{E})\star A}\\
        & = \bar{(\theta\times\Id{E})\star f}
    \end{align*}

\section{Road Map}
In Figure \ref{RoadMap}, one can see a diagram of the collection of theorems that need to be proved to establish the soundness of a semantics for PEC.


The first pair of theorems (\ref{EffectSubstitutionOnEffects}, \ref{EffectWeakeningOnEffects}) is made up of the effect substitution and weakening theorems on effects. These theorems show that substitutions of effects have a well-behaved and easily defined action upon the denotations of effects. Using these theorems, we can then move on to characterize the action of effect substitutions and effect-environment weakening on the denotations of types in Theorems \ref{EffectSubstitutionOnTypes} and \ref{EffectWeakeningOnTypes}. From this, we can also look at the action of weakening and substituting effect-variable environments on the subtyping relation between types in Theorems \ref{EffectSubstitutionOnSubtyping}, \ref{EffectWeakeningOnSubtyping}.

The next step is to use these substitution theorems to formalise the action of substitution and weakening of the effect-variable environments on terms in Theorems \ref{EffectSubstitutionOnTerms}, \ref{EffectWeakeningOnTerms}. This then allows us to find denotations for the weakening of term substitutions and type-environment weakening, which set us up to prove the typical weakening and substitution theorems upon term variables and type environments in Theorems \ref{TermSubstitutionOnTerms}, \ref{TermWeakeningOnTerms}. 

Separately, we prove that all derivable denotations for a typing relation instance, $\gpetyperelation{v}{A}$ have the same denotation (Section \ref{UniqueDenotations}). This is important, since subtyping allows us to find multiple distinct typing derivations for terms, which initially look like they may have distinct denotations. Using a reduction function to transform typing derivations into a unique form, I prove that all typing derivations yield equal denotations.

This collection of theorems finally allows us to complete all cases of the equational-equivalence soundness theorem.

\begin{figure}[H]
    \begin{center}
        \scalebox{0.8}{
        \begin{tikzpicture}[->,>=stealth',shorten >=1pt,auto,node distance=3cm,
            thick,main node/.style={rectangle,fill=blue!20,draw,
            font=\sffamily\small\bfseries,minimum size=15mm}]
        
            \node[main node,text width=20mm, fill=red] (IndexCategory) {Indexed Category and Adjunction (\ref{PECRequirements})};
            \node[main node,text width=20mm, fill=red] (MulNaturality) [below of=IndexCategory] {Naturality of Mul Operator (Def. \ref{MulNaturality})};
        
            \node[main node,text width=20mm, fill=red] (BeckChevalley) [left of=MulNaturality]{Beck-Chevalley Condition (\ref{PECRequirements})};
            \node[main node,text width=20mm, fill=red](SClosure) [right of=MulNaturality]{Re-indexing Functors are S-Preserving (Def. \ref{SPreservingDefinition})};
        
            \node[main node, text width=20mm](EffectWeakeningEffects) [below left of=MulNaturality]{Effect Weakening on Effects (Theorem \ref{EffectWeakeningOnEffects})};
        
            \node[main node, text width=20mm](EffectSubEffects) [below right of=MulNaturality]{Effect Substitution on Effects (Theorem \ref{EffectSubstitutionOnEffects})};
        
            \node[main node, text width=20mm](EffectWeakeningTypes)[below of=EffectWeakeningEffects]{Effect Weakening on Types (Theorem \ref{EffectWeakeningOnTypes})};
        
            \node[main node, text width=20mm](EffectSubTypes)[below of=EffectSubEffects]{Effect Substitution on Types (Theorem \ref{EffectSubstitutionOnTypes})};
        
            \node[main node, text width=20mm](EffectWeakeningTerms)[below of=EffectWeakeningTypes]{Effect Weakening on Terms (Theorem \ref{EffectWeakeningOnTerms})};
        
            \node[main node, text width=20mm](EffectSubTerms)[below of=EffectSubTypes]{Effect Substitution on Terms (Theorem \ref{EffectSubstitutionOnTerms})};
        
            \node[main node, text width=20mm](EffectWeakeningSubTyping)[left of=EffectWeakeningTerms]{Effect Weakening on Subtyping (Theorem \ref{EffectWeakeningOnSubtyping})};
        
            \node[main node, text width=20mm](EffectSubSubTyping)[right of=EffectSubTerms]{Effect Substitution on Subtyping (Theorem \ref{EffectSubstitutionOnSubtyping})};
        
            \node[main node, text width=20mm](EffectWeakeningWeakening)[below of=EffectWeakeningTerms]{Effect Weakening on Term Weakening (Theorem \ref{EffectWeakeningOnTermWeakening})};
        
            \node[main node, text width=20mm](EffectWeakeningSubstitution)[below of=EffectSubTerms]{Effect Weakening on Term Substitution (Theorem \ref{EffectWeakeningOnTermSubstitution})};
        
            \node[main node, text width=20mm](TermWeakening)[below of=EffectWeakeningWeakening]{Term Weakening Theorem (Theorem \ref{TermWeakeningOnTerms})};
        
            \node[main node, text width=20mm](TermSubstitution)[below of=EffectWeakeningSubstitution]{Term Substitution Theorem (Theorem \ref{TermSubstitutionOnTerms})};
        
            \node[main node, text width=20mm](UniqueDenotations)[below left of=TermSubstitution]{Unique Denotations (corollary \ref{DenotationsAreUnique})};
        
            \node[main node, text width=20mm](Soundness)[below of=UniqueDenotations]{Soundness (Theorem \ref{SOundness})};
        
            \draw [->] (IndexCategory) edge (BeckChevalley) (IndexCategory) edge (MulNaturality) (IndexCategory) edge (SClosure);
            \draw [->] (MulNaturality) edge (EffectWeakeningEffects) (MulNaturality) edge (EffectSubEffects);
        
            
            \draw [->] (EffectWeakeningEffects) edge (EffectWeakeningTypes) (EffectSubEffects) edge (EffectSubTypes);
        
            \draw [->] (EffectWeakeningTypes) edge (EffectWeakeningSubTyping) (EffectSubTypes) edge (EffectSubSubTyping);
        
            
            \draw [->] (EffectWeakeningSubTyping) edge (EffectWeakeningTerms) (EffectSubSubTyping) edge (EffectSubTerms);
        
        
            \draw [->] (EffectWeakeningTypes) edge (EffectWeakeningTerms) (EffectSubTypes) edge (EffectSubTerms);
        
            \draw [->] (EffectWeakeningTerms) edge (EffectWeakeningSubstitution) (EffectWeakeningSubTyping) edge (EffectWeakeningWeakening);
        
            \draw [->] (EffectWeakeningWeakening) edge (TermWeakening) (EffectWeakeningSubstitution) edge (TermSubstitution);
        
            \draw [->] (EffectSubTerms) edge [bend left=40] (TermSubstitution);
            \draw [->] 
                (BeckChevalley) edge [bend right=30] (EffectWeakeningTypes)
                (SClosure) edge [bend right=20] (EffectWeakeningTypes)
                (BeckChevalley) edge [bend left=20]  (EffectSubTypes)
                (SClosure) edge [bend left=30] (EffectSubTypes);
            
        
            \draw [->] (TermWeakening) edge (UniqueDenotations) (TermSubstitution) edge (UniqueDenotations) (TermWeakening) edge [bend right=20] (Soundness)
            (TermSubstitution) edge [bend left=20] (Soundness)
            (UniqueDenotations) edge (Soundness);
          \end{tikzpicture}
    }
    \end{center}
\caption{A road map of the proof dependencies. Assumptions are in red, theorems in blue.}
\label{RoadMap}
\end{figure}


\section{Denotations}\label{PECDenotations}
We are now equipped to define the denotations of structures in PEC. Firstly, we define the denotation of effect-variable environments. In the base category, with the kind of effects indicated by $E$, the denotation of an effect-variable environment $\P$ is given by a finite product.

\[
\deno{\nil} = \1 \quad\quad \deno{\P, \a} = \deno{\P}\times E    
\]

Hence, the denotation of an  effect-variable environment $\P$ is given by a finite product on $E$ of width $n$ where $n$ is the length of $\P$. As stated in Section \ref{PECRequirements}, the denotation of a well-formed effect is a function $\deno{\typerelation{\P}{\e}{\effect}}: E^n \rightarrow E$ in $\Eff$. The denotations for effects are shown in Figure \ref{EffectDenotations}.
\begin{figure}[H]
    \centering
    \begin{framed}
        \[
    \deno{\typerelation{\P}{e}{\effect}} = \deno{e} \after \term{E^n}: E^n\rightarrow E
    \quad\quad
    \deno{\typerelation{\P,\a}{\a}{\effect}} = \pp: E^n\times E \rightarrow E
\]\[
    \deno{\typerelation{\P, \b}{\a}{\effect}} = \deno{\typerelation{\P}{\a}{\effect}}\after\p: E^n\times E\rightarrow E
\]\[
    \deno{\typerelation{\P}{\e_1\dot \e_2}{\effect}} = \Mul_{E^n}(\deno{\typerelation{\P}{\e_2}{\effect}},\deno{\typerelation{\P}{\e_1}{\effect}}): E^n \rightarrow E
\]
    \end{framed}
    \caption{The denotations of effects in the base category.}
    \label{EffectDenotations}
\end{figure}


Using these denotations, we are now equipped to define the denotations of types. As stated above, types that are well formed in $\P$ are denoted by objects in the fibre category $\C(E^n)$ given by the denotation of $\P$. These type denotations are given in Figure \ref{TypeDenotations}. Since the fibre category $\C(E^n)$ is an S-category, it has objects for all ground types, a terminal object, graded monad $\T{}{}$, exponentials, products, and co-product, $\1+\1$.

\begin{figure}[H]
    \centering
    \begin{framed}
\[
    \deno{\typerelation{\P}{\U}{\type}} = \1
    \quad\quad
    \deno{\typerelation{\P}{\B}{\type}} = \1+\1
    \quad\quad
    \deno{\typerelation{\P}{\g}{\type}} = \deno{\g}
\] 

\[
    \deno{\typerelation{\P}{\ab}{\type}} = (\deno{\typerelation{\P}{B}{\type}})^{(\deno{\typerelation{\P}{A}{\type}})}
\]

\[
    \deno{\typerelation{\P}{\mea}{\type}} =\T{\deno{\typerelation{\P}{\e}{\effect}}}{\deno{\typerelation{\P}{A}{\type}}}
    \quad\quad
    \deno{\typerelation{\P}{\all{\a}{A}}{\type}} =\allEn(\deno{\typerelation{\P,\a}{A}{\type}})
\]


    \end{framed}
    \caption{The denotations of type expressions within the appropriate fibre.}
    \label{TypeDenotations}
\end{figure}

By using the terminal objects and products present in each fibre, we can now derive denotations of type environments. $\deno{\oke{\P}{\G}}$ should be an object in the fibre $\C(E^n)$ induced by $\P$. Specific denotations are given in Figure \ref{TypeEnvDenotations}.

\begin{figure}[H]
    \centering
    \begin{framed}
        \[
            \deno{\wellformedok{\P}{\nil}} = \1
            \quad\quad
            \deno{\wellformedok{\P}{\gax}} = (\deno{\wellformedok{\P}{\G}} \times \deno{\typerelation{\P}{A}{\type}})
        \]  
    \end{framed}

    \caption{Denotations of type environments within the appropriate fibre.}
    \label{TypeEnvDenotations}
\end{figure}


Another important construction is the denotation of subtyping. For each instance of the subtyping relation in $\P$, $A\subtypep B$, there exists a denotation in the fibre induced by $\P$. $\deno{A\subtypep B} \in\C(E^n)(A, B)$. Since the fibres are S-categories, the ground instances of the subtyping relation exist in each fibre anyway. In addition, for each instance of the subeffect relation $\e_1\subeffectp \e_2$ each fibre contains the natural transformation $\dsep{\e_1}{\e_2}: \T{\e_1}{}\rightarrow \T{\e_2}{}$. Specific subtyping denotations are given in Figure \ref{SubtypingDenotations}.

\begin{figure}[H]
    \centering
    \begin{framed}
        \begin{align*}
            \deno{\g_1\subtypep \g_2} &= \deno{\g_1\subtypeg \g_2}
            \\
            \deno{\ab \subtypep \fntype{A'}{B'}} &= \deno{B\subtypep B'}^{A'}\after B^{\deno{A'\subtypep A}}
            \\
            \deno{\M{\e_1}{A}\subtypep\M{\e_2}{B}} &= \deno{\e_1\subeffectp\e_2}_B\after\T{\e_1}{\deno{A\subtypep B}}
            \\
            \deno{\all{\a}{A}\subtypep\all{\a}{B}} &= \allEn{\deno{A\subtypepa B}}
        \end{align*}
    \end{framed}
    
    \caption{The denotations of subtyping relations.}
    \label{SubtypingDenotations}
\end{figure}


This finally gives us the ability to express the denotations of well-typed terms in an effect-variable environment, $\P$, as morphisms in the fibre induced by $\P$, $\C(E^n)$.  Term denotations are in fact defined inductively with respect to the derivation of the typing relation. Writing $\G_{E^n}$ and $A_{E^n}$ for $\deno{\wellformedok{\P}{\G}}$ and $\deno{\typerelation{\P}{A}{\type}}$, we can derive $\deno{\gpetyperelation{v}{A}}$ as a morphism in $\C(E^n)(\G_{E^n}, A_{E^n})$. Since each fibre is an S-category, for each constant, $\const{A}$, there exists $\deno{\const{A}}: \1 \rightarrow A_{E^n}$ in $\C(E^n)$. Where is is clear from the context, I shall now drop the $E^n$ subscript. These denotations can be seen in Figure \ref{TermDenotations}. We need to be  careful when dealing with denotations of terms; due to subtyping, typing relations on terms may have multiple derivations. As denotations are defined inductively on these derivations, each denotation is implicitly associated with a derivation. In Section \ref{UniqueDenotations}, I prove that all derivations for a given type relation yield equal denotations. However, for the intervening theorems which deal with term denotations, we need to take more care to refer to the specific type derivation. 

\begin{figure}
    \footnotesize
    \begin{framed}
        \[
            \ntreeruleI{\vunit}{\wellformedok{\P}{\G}}{\deno{\etyperelation{\P}{\G}{\u}{\U}} = \term{\G} : \G \rightarrow \1}
            \quad
            \ntreeruleI{\vconst}{\wellformedok{\P}{\G}}{\deno{\etyperelation{\P}{\G}{\const{A}}{A}} = \deno{\const{A}} \after \term{\G} : \G \rightarrow A}
        \]
        \\
        \\        
        \[
            \ntreeruleI{\vtrue}{\wellformedok{\P}{\G}}{\deno{\etyperelation{\P}{\G}{\t}{\B}} = \inl \after \term{\G} : \G \rightarrow \deno{\B} = \1+\1}
        \]
        \\
        \\
        \[
            \ntreeruleI{\vfalse}{\wellformedok{\P}{\G}}{\deno{\etyperelation{\P}{\G}{\f}{\B}} = \inr \after \term{\G} : \G \rightarrow \deno{\B} = \1+\1}
        \]
        \\
        \\        
        \[
            \ntreeruleI{\vvar}{\wellformedok{\P}{\G}}{\deno{\etyperelation{\P}{\gax}{x}{A}} = \pp: \G \times A \rightarrow A}
            \quad    
            \ntreeruleI{\vweaken}{f = \deno{\gpetyperelation{x}{A}}: \G \rightarrow A}{\deno{\etyperelation{\P}{\gby}{x}{A}} = f \after \p: \G \times B \rightarrow A}
        \]
        \\
        \\        
        \[
            \ntreeruleI{\vfun}{f = \deno{\etyperelation{\P}{\gax}{v}{B}} : \G \times A \rightarrow B}{\deno{\etyperelation{\P}{\G}{\lam{x}{A}{v}}{\ab}} = \cur{f} : \G \rightarrow (B)^A}
        \]
        \\
        \\        
        \[
            \ntreeruleII{\vsubtype}{f = \deno{\etyperelation{\P}{\G}{v}{A}} : \G \rightarrow A}{g = \deno{A \subtypep B}}{\deno{\etyperelation{\P}{\G}{v}{B}} = g \after f : \G \rightarrow B}
            \quad 
            \ntreeruleI{\vreturn}{f = \deno{\etyperelation{\P}{\G}{v}{A}}}{\deno{\etyperelation{\P}{\G}{\return{v}}{\moa}} = \point{A} \after f}   
        \]
        \\
        \\        
        \[
            \ntreeruleIII{\vif}{f = \deno{\etyperelation{\P}{\G}{v}{\B}}: \G\rightarrow\1+\1}{g = \deno{\etyperelation{\P}{\G}{v_1}{\mea}}}{ h = \deno{\etyperelation{\P}{\G}{v_2}{\mea}}}{\deno{{\etyperelation{\P}{\G}{\ifthenelse{\e}{A}{v}{v_1}{v_2}}{\mea}}} = \app\after((\fld{\cur{g\after\pp}}{\cur{h\after\pp}}\after f)\times \idg)\after \diag{\G} : \G \rightarrow \tea}    
        \]
        \\
        \\
        \[
            \ntreeruleII{\vbind}{f = \deno{\etyperelation{\P}{\G}{v_1}{\M{\e_1}{A}} : \G \rightarrow \T{\e_1}{A}}}{{ g = \deno{\etyperelation{\P}{\gax}{v_2}{\M{\e_2}{B}}}}: \G \times A \rightarrow \T{\e_2}{B}}{\deno{\etyperelation{\P}{\G}{\doin{x}{v_1}{v_2}}{\M{\e_1 \dot \e_2}{B}}} = \bind{\e_1}{\e_2}{B} \after \T{\e_1}{g} \after \tstrength{\G}{A}{\e_1} \after \pr{\idg}{f}: \G \rightarrow \T{\e_1 \dot \e_2}{B}}  
        \]
        \\
        \\        
        \[
            \ntreeruleII{\vapply}{f = \deno{\gpetyperelation{v_1}{\ab}}: \G \rightarrow (B)^{A}}{g=\deno{\gpetyperelation{v_2}{A}}: \G \rightarrow A}{\deno{\gpetyperelation{\apply{v_1}{v_2}}{B}}= \app\after\pr{f}{g}: \G \rightarrow B }
        \]
        \\
        \\        
        \[
            \ntreeruleI{\vgen}{f = \deno{\etyperelation{\P,\a}{\G}{v}{A}}: \C(E^n\times E)(\G, A)}{\deno{\gpetyperelation{\elam{\a}{A}}{\all{\e}{A}}} = \bar{f}: \C(E^n)(\G, \allEn(A))}    
        \] 
        \\
        \\
        \begin{equation}\label{EffectApplication}
            \ntreeruleII{\vspec}{g=\deno{\gpetyperelation{v}{\all{\a}{A}}}: \C(E^n)(\G, \allEn(A))}{ h = \deno{\typerelation{\P}{\e}{\effect}}: E^n \rightarrow E}{\deno{\gpetyperelation{\eapp{v}{\e}}{A\ssub{\a}{\e}}} = \pr{\Id{E^n}}{h}\star(\counit{\deno{\typerelation{\P,\b}{A\ssub{\a}{\b}}{\type}}})\after g: \C(E^n)(\G, A\ssub{\a}{\e})}
        \end{equation}                
    \end{framed}
    \caption{The denotations of terms in PEC.}
    \label{TermDenotations}
\end{figure}

The least intuitive of these denotations is that of effect-application (\textit{\vspec}) in Equation \ref{EffectApplication}. As explained later in Section \ref{SubsAndWeakening}, the re-indexing functor represents application of the substitution $\ssub{\b}{\e}$. By doing type analysis on the co-unit morphism $ \counit{\deno{\typerelation{\P,\b}{A\ssub{\a}{\b}}{\type}}}$, as in Figure \ref{CoUnitType}, we can see that before the application of the substitution, the co-unit natural transformation takes a quantified type to a substituted type under an extended effect-variable environment. By applying the substitution re-indexing functor in Figure \ref{ReIndexedCoUnitType}, we substitute $\e$ for the extra effect variable, $\b$. In the case of the quantified type, as $\b$ is not used, the substitution has no effect, but in the case of the substituted type, the substitutions compose to yield the substitution $\ssub{\a}{\e}$. This composes with the expression denotation $g$ to give the applied denotation in Figure \ref{EffectSpecComp}.


\begin{figure}
    \begin{framed}
        \begin{align*}
            \counit{\deno{\typerelation{\P,\b}{A\ssub{\a}{\b}}{\type}}} & = \widehat{\Id{\allEn(\deno{\typerelation{\P,\b}       {A\ssub{\a}{\b}}{\type}})}} \\ 
            & : \quad \pstar \allEn(\deno{\typerelation{\P,\b}{A\ssub{\a}{\b}}{\type}}) \rightarrow \deno{\typerelation{\P,\b}       {A\ssub{\a}{\b}}{\type}} \\
            & : \quad\pstar \deno{\typerelation{\P}{\all{\b}{A\ssub{\a}{\b}}}{\type}} \rightarrow \deno{\typerelation{\P,\b}        {A\ssub{\a}{\b}}{\type}} \\
            & : \quad \pstar \deno{\typerelation{\P}{\all{\a}{A}}{\type}} \rightarrow \deno{\typerelation{\P,\b}{A\ssub     {\a}{\b}}{\type}}\\
            & : \quad \deno{\typerelation{\P, \b}{\all{\a}{A}}{\type}} \rightarrow \deno{\typerelation{\P,\b}{A\ssub{\a}{\b}}{\type}}
        \end{align*}
    \end{framed}
    \caption{Analysis of the type of the co-unit natural transformation.}
    \label{CoUnitType}
\end{figure}

\begin{figure}
    \begin{framed}
        \begin{align*}
            \pr{\Id{E^n}}{h}\star(\counit{\deno{\typerelation{\P,\b}{A\ssub{\a}{\b}}{\type}}}) & : \quad \deno{\typerelation{\P}{\all{\a}{A}}{\type}} \rightarrow \deno{\typerelation{\P}{A\ssub{\a}{\b}\ssub{\b}{\e}}{\type}}\\
            & : \quad \deno{\typerelation{\P}{\all{\a}{A}}{\type}} \rightarrow \deno{\typerelation{\P}{A\ssub{\a}{\e}}{\type}}
        \end{align*}
    \end{framed}
    \caption{Analysis of the type of the re-indexed co-unit.}
    \label{ReIndexedCoUnitType}
\end{figure}


 
\begin{figure}
    \begin{framed}
        \centering
        \begin{tikzcd}
            &  \deno{\typerelation{\P, \b}{\all{\a}{A}}{\type}}  \arrow{rrrr}{\counit{\deno{\typerelation{\P,\b}{A\ssub{\a}{\b}}{\type}}}}&&&&  \deno{\typerelation{\P,\b}{A\ssub{\a}{\b}}{\type}} &
            \\
            \G \arrow{rr}{g}&&
            \allEn (A) \arrow{rrrr}{ \pr{\Id{E^n}}{h}\star(\counit{\deno{\typerelation{\P,\b}{A\ssub{\a}{\b}}{\type}}})}
            &&&&
            A\ssub{\a}{\e}
        \end{tikzcd}
    \end{framed}
    \caption{Composition of the effect-specialisation denotation.}
    \label{EffectSpecComp}
\end{figure}



\section{Effect Substitution and Weakening Theorems}\label{SubsAndWeakening}


In this section, I introduce and prove a series of utility theorems which will help us prove cases in future theorems. These weakening and substitution theorems are concerned with a change in environment of typing derivations and their associated denotations. If $\gpetyperelation{v}{A}$, then it should be the case that $\etyperelation{\P}{\gax}{v}{A}$ if $x$ is fresh in $\G$. We also want to know what happens to the denotation when we change the type environment. In this section, I introduce the tools for manipulating the type and effect-variable environments in this fashion.

Substitutions and weakenings are two distinct ways of manipulating effect or type environments. Weakening acts as a kind of subtyping of the environment. If we insert fresh variables into an environment, then any expression that was typeable under the previous environment should also be typeable under the the new environment. This change of environment should also have a predictable effect on the denotations of any expressions to which it is applied.

Substitution considers what happens when we simultaneously replace all variables in one expression, that is typeable under an environment, with expressions that are well formed under another environment. The resulting substituted expression should be typeable under the new environment, and the denotation of the new expression should be composed from the denotation of the old relation and the denotations of the expressions that replace the variables. 

As we go on, I define the denotations of specific substitutions and weakenings, upon both the effect-variable and the type environments.

In this dissertation, substitutions and weakenings each come in two flavours: weakening and substitution of the effect-variable environment and of the type environment. For each of these, there is a family of theorems defining the effects of the applying a substitution and weakening to the various language structures and their denotations, such as well-formedness and typing relations.

\begin{framed}
    \begin{aside}[Weakening Proofs]
        The structure of weakening theorems and their proofs are often rather similar to their respective substitution theorem. In these cases, I have placed the example cases of the weakening proof in Appendix \ref{WeakeningProofs}.
    \end{aside}
\end{framed}

\subsection{Substitution and Weakening on Effects}\label{SectionEffectSubstitution}

The first family of theorems is that of weakening and substitution of the effect-variable environment. Weakenings are relations between effect-variable environments, written $\wrel{\w}{\P'}{\P}$, that are defined in Figure \ref{EffectWeakeningDefinition}. We can define the denotation of an effect-environment weakening as a morphism in the base category: $\deno{\wrelw{\P'}{\P}}: E^m \rightarrow E^n$. The denotations are defined in Figure \ref{EffectWeakeningDenotations}.

\begin{figure}[H]
    \centering
    \begin{framed}
        \[
    \ntreeruleI{\eid}{\ok{\P}}{\wrel{\i}{\P}{\P}}
    \quad
    \condtreeruleI{\eproject}{\wrel{\w}{\P'}{\P}}{\wrel{\w\pi}{(\P', \a)}{\P}}{\a\notin\P'}
    \quad
    \condtreeruleI{\eextend}{\wrel{\w}{\P'}{\P}}{\wrel{\w\x}{(\P', \a)}{(\P, \a)}}{\a\notin\P'}
\]
    \end{framed}
    \caption{Effect weakening definitions.}
    \label{EffectWeakeningDefinition}
\end{figure}


\begin{figure}[H]
    \centering
    \begin{framed}
        \begin{align*}
            \deno{\wrel{\i}{\P}{\P}} & = \Id{E^n}: E^n \rightarrow E^n
            \\
            \deno{\wrel{\w\pi}{\P',\a}{\P}} & = \deno{\wrelw{\P'}{\P}}\after \p: E^m\times E\rightarrow E^n
            \\
            \deno{\wrel{\w\x}{\P',\a}{\P,\a}} & = (\deno{\wrelw{\P'}{\P}}\times \Id{E}): E^m\times E\rightarrow E^n\times E 
        \end{align*}
    \end{framed}
    \caption{Effect weakening denotations.}
    \label{EffectWeakeningDenotations}
\end{figure}

Substitutions are also inductively defined relations between effect-variable environments. Finite substitutions may be represented as a snoc-list of variable-effect pairs. The substitution relation matches  a substitution to the environments it operates between. The relation instance $\typerelation{\P'}{\si}{\P}$ means that $\si$ is a substitution from $\P$ to $\P'$. It is defined inductively using rules in Figure \ref{EffectSubstitutionDefinition}.

\begin{figure}[H]
    \centering
    \begin{framed}
        \[
    \si \gens \nil \mid \si, \a \setto \e    
\]



\[
    \ntreeruleI{\esubnil}{\ok{\P'}}{\typerelation{\P'}{\nil}{\nil}}
    \quad\quad
    \condtreeruleII{\esubextend}{\typerelation{\P'}{\si}{\P}}{\wellformedeffect{\P'}{\e}}{\typerelation{\P'}{(\si, \a \setto\e)}{(\P, \a)}}{\a\notin\P}
\]
    \end{framed}
    
    \caption{Effect substitution definition.}
    \label{EffectSubstitutionDefinition}
\end{figure}


We can define the action of substitutions on effects, as in Figure \ref{EffectSubstitutionActionEffects}. Furthermore, Figure \ref{EffectSubstitutionActionEffects} also gives the denotation of substitutions. The denotation of an effect-environment substitutions, $\deno{\typerelation{\P'}{\si}{\P}}$, is a morphism $E^m \rightarrow E^n$ in the base category.



\begin{figure}[H]
    \centering
    \begin{minipage}{0.47\linewidth}
      \begin{framed}
        \centering
        \textbf{Action}
  
        \scalebox{.8}{\parbox{\linewidth}{%
        \begin{align*}
            \si(e) & = e \\
            \si(\e_1\dot\e_2) & = (\si(\e_1))\dot(\si(\e_2))\\
            \nil(\a) & = \a\\
            (\si, \b\setto \e)(\a) & = \si(\a) \\
            (\si, \a\setto \e)(\a) & =\e
        \end{align*}
        
        }}
      \end{framed}
    \end{minipage}
    \quad
    \begin{minipage}{0.47\linewidth}
      \begin{framed}
        \centering
        \textbf{Denotations}
  
        \scalebox{.8}{\parbox{\linewidth}{%
        \begin{align*}
            \\
            \deno{\typerelation{\P'}{\nil}{\nil}} & = \term{E^n}: \C(E^m, \1)
            \\
            \deno{\typerelation{\P'}{(\si, \a\setto\e)}{\P,\a}} & = \pr{\deno{\typerelation{\P'}{\si}{\P}}}{\deno{\typerelation{\P}{\e}{\effect}}}\\&\quad: \C(E^m, E^n\times E)
            \\
        \end{align*}
        }}
      \end{framed}
    \end{minipage}
    \caption{The action on effects and denotations of an effect substitution.}
    \label{EffectSubstitutionActionEffects}
\end{figure}

The general construction for effect-variable substitution allows us to define, in particular, the identity substitution, $\typerelation{\P}{\Id{\P}}{\P}$ and the singleton substitution, $\typerelation{\P}{\ssub{\a}{\e}}{\P,\a}$ in Figure \ref{EffectSpecialSubs}. By inspection, we can also obtain the denotations of these special-case substitutions. Note that in particular, the denotation of the single substitution completes our intuition for the denotation of the (\textit{\vspec}) rule in Section \ref{PECDenotations}. These substitutions will form a useful shorthand later. Finally, it will be useful to examine how the substitutions generated by extending the effect-variable environment, such as in the case of polymorphic types, relate to the original substitution (Lemma \ref{ExtensionLemmaOnEffectSubstitutions}).

\begin{figure}[H]
    \centering
    \begin{minipage}{.45\linewidth}
        \begin{framed}
            \begin{align*}
                \Id{\nil} & = \nil\\
                \Id{\P,\a} & = (\Id{\P}, \a\setto\a)\\
                \ssub{\a}{\e} & = (\Id{\P}, \a\setto \e)
            \end{align*}
        \end{framed}
    \end{minipage}
    \quad
    \begin{minipage}{.45\linewidth}
        \begin{framed}
            \begin{align*}
                \deno{\typerelation{\P}{\Id{\P}}{\P}} & = \Id{E^n}\\
                \deno{\typerelation{\P}{\ssub{\a}{\e}}{(\P,\a)}} & = \pr{\Id{E^n}}{\deno{\typerelation{\P}{\e}{\effect}}}
                \\
            \end{align*}
        \end{framed}
    \end{minipage}
    \caption{Special case substitutions and their denotations.}
    \label{EffectSpecialSubs}
\end{figure}

\begin{framed}
    \begin{definition}[Freshness]\label{EffectFreshness}
      We define $\a \# \si$ to mean that $\a$ does not occur in the domain or any of the substitution expression of $\si$.
    \end{definition}
\end{framed}

\begin{framed}
    \begin{theorem}[Effect Substitution on Effects]\label{EffectSubstitutionOnEffects} 
        If $\wellformedeffect{\P}{\e}$ and $\typerelation{\P'}{\si}{\P}$ then: 
        \begin{enumerate}[label=\roman*.]
            \item $\wellformedeffect{\P'}{\si(\e)}$
            \item Writing $\si$ for $\deno{\typerelation{\P'}{\si}{\P}}$,  $\deno{\wellformedeffect{\P'}{\si(\e)}} = \deno{\wellformedeffect{\P}{\e}}\after\si$
        \end{enumerate}
          
    \end{theorem}
    
    \begin{proof}
        This proof depends on the naturality of $\Mul_{E^n}$ and inversion to narrow down case splitting on the structure of the effect-variable environments. It proceeds by induction on the definition of denotations of effects given in Figure \ref{EffectDenotations}.
    
    \case{\ecompose}
    
    We make use of the naturality of $\Mul_{E^n}$ and induction upon the subterms.
    \begin{align*}
        \deno{\typerelation{\P}{\e_1\dot\e_2}{\effect}} \after\si &=
        \Mul_{E^n}(\deno{\typerelation{\P}{\e_1}{\effect}}, \deno{\typerelation{\P}{\e_2}{\effect}})\after \si \\
        & = \Mul_{E^n}(\deno{\typerelation{\P}{\e_1}{\effect}}\after \si, \deno{\typerelation{\P}{\e_2}{\effect}}\after \si)\qt{By Naturality}\\
        & = \Mul_{E^m}(\deno{\typerelation{\P'}{\si(\e_1)}{\effect}}, \deno{\typerelation{\P}{\si(\e_2)}{\effect}})\\
        & = \deno{\typerelation{\P'}{\si(\e_1)\dot\si(\e_2)}{\effect}}\\
        & = \deno{\typerelation{\P'}{\si(\e_1\dot\e_2)}{\effect}}&\square
    \end{align*}   
    \end{proof}
\end{framed}

The weakening theorem proceeds similarly.

\begin{framed}
    \begin{theorem}[Effect Weakening on Effects]\label{EffectWeakeningOnEffects}
       If $\wellformedeffect{\P}{\e}$ and $\wrelw{\P'}{\P}$ then: 
       \begin{enumerate}[label=\roman*.]
           \item $\wellformedeffect{\P'}{\e}$
           \item Writing $\w$ for $\deno{\wrelw{\P'}{\P}}$,  $\deno{\wellformedeffect{\P'}{\e}} = \deno{\wellformedeffect{\P}{\e}}\after\w$.
       \end{enumerate}    
    \end{theorem}
    
    \begin{proof}
        This proof also depends on the naturality of $\Mul_{E^n}$ and case splitting on the structure of $\w$.  Some cases can be found in Appendix \ref{AppendixEffectWeakeningOnEffects}.
    \end{proof}
\end{framed}


\begin{framed}
    \begin{lemma}[Extension Lemma on Effect Substitutions]\label{ExtensionLemmaOnEffectSubstitutions}
        If $\typerelation{\P'}{\si}{\P}$, and $\a\#\si$, then $\typerelation{(\P', \a)}{(\si, \a\setto\a)}{(\P,\a)}$ with denotation $\deno{\typerelation{(\P', \a)}{(\si, \a\setto\a)}{(\P,\a)}} = (\deno{\typerelation{\P'}{\si}{\P}}\times\Id{E})$
    \end{lemma}
    \begin{proof}
       This holds by the weakening-theorem on effects, Theorem \ref{EffectWeakeningOnEffects}. $\square$
    \end{proof}
\end{framed}


\subsection{Substitution and Weakening on Typing}
We can now move on to state and prove the weakening and substitution theorems on types, subtyping, and type environments. The general structure of these theorems, as well as the term-based theorems later, is that when we want to quantify the action of a morphism $\theta: E^m \rightarrow E^n$ between objects in the base category on structure in the fibres $\C(E^n)$, we should simply apply the associated re-indexing functor (as defined in Section \ref{PECRequirements}) $\theta\star: \C(E^n) \rightarrow \C(E^m)$ to the structure. The proof of the soundness of this operation is driven by the fact that the re-indexing functor is S-preserving. Specifically, effect substitutions ($\si$) have the actions given in Figure \ref{SubstitutionActionTypesTypeEnvs} on types and type-environments.

\begin{figure}[H]
    
    \begin{framed}
        \centering
        \textbf{Action of Effect Substitution}

        \begin{minipage}{.47\linewidth}
            
\begin{align*}
    \g\sub{\si} &= \g \\
    (\fntype{A}{B})\sub{\si} &= \fntype{(A\sub{\si})}{(B\sub{\si})} \\
    (\M{\e}{A})\ssi &= \M{\si(\e)}{(A\ssi)}\\
    (\all{\a}{A})\ssi &= \all{\a}{(A\ssi)}\qt{If $\a\#\si$}
\end{align*}
        \end{minipage}
        \quad
        \begin{minipage}{.47\linewidth}
            \begin{align*}
                \nil\ssi & = \nil \\
                (\gax)\ssi &= (\G\ssi, x:(A\ssi))\\
            \end{align*}            
        \end{minipage}
    \end{framed}
    
    \caption{The action of effect substitutions on types and type environments.}
    \label{SubstitutionActionTypesTypeEnvs}
\end{figure}




\begin{framed}
    \begin{theorem}[Effect Substitution on Types]\label{EffectSubstitutionOnTypes}
       If $\wellformedtype{\P}{A}$ and $\typerelation{\P'}{\si}{\P}$, then:
       \begin{enumerate}[label=\roman*.]
           \item $\wellformedtype{\P'}{A\ssi}$
           \item $\deno{\wellformedtype{\P'}{A\ssi}} = \si\star\deno{\wellformedtype{\P}{A}}$
       \end{enumerate}
    \end{theorem}
    
    \begin{proof}
       By induction on the derivation of $\deno{\wellformedtype{\P}{A}}$ (Figure \ref{TypeDenotations}) and the fact that $\si\star$ is S-preserving. This specific case shows the dependency on the Beck-Chevalley condition and the extension lemma (Lemma \ref{ExtensionLemmaOnEffectSubstitutions}).
            
        \case{\tquant}
        This case makes use of the Beck-Chevalley condition and the fact that $\deno{\typerelation{(\P', \a)}{(\si, \a\setto\a)}{(\P, \a)}} = \si\times\Id{E}$, which we can induct upon.
            \begin{align*}
                \si\star\deno{\typerelation{\P}{\all{\a}A}{\type}} & = \si\star(\allEn(\deno{\typerelation{\P,\a}{A}{\type}}))\\
                & = \allEn((\si\times\Id{E})\star\deno{\typerelation{\P,\a}{A}{\type}})\qt{By Beck-Chevalley}\\
                & = \allEn(\deno{\typerelation{\P',\a}{A\sub{\si, \a\setto\a}}{\type}})\qt{By induction}\\
                & = \deno{\typerelation{\P'}{\all{\a}{(A\sub{\si, \a\setto\a})}}{\type}}\\
                & = \deno{\typerelation{\P'}{(\all{\a}{A})\ssi}{\type}} & \square
            \end{align*}
    \end{proof}
\end{framed}

Similarly, we can extend the effect-substitution theorem to type environments.

\begin{framed}
    \begin{theorem}[Effect Substitution on Type Environments]\label{EffectSubstitutionOnTypeEnvs}
        If $\wellformedok{\P}{\G}$, then:
        \begin{enumerate}[label=\roman*.]
            \item $\wellformedok{\P'}{\G\ssi}$
            \item $\deno{\wellformedok{\P'}{\G\ssi}} = \si\star\deno{\wellformedok{\P}{\G}}$.
        \end{enumerate}
         
    \end{theorem}
    
    \begin{proof}
        By induction on the derivation of $\deno{\wellformedok{\P}{\G}}$ (Figure \ref{TypeEnvDenotations}) whilst making use of the S-preserving property of the re-indexing functor.
            
        \case{\envextend}
        The S-preserving property means that $\si\star(A \times B) = (\si\star A)\times (\si\star B)$.
        \begin{align*}
           \si\star\deno{\wellformedok{\P}{\gax}} &= \si\star(\deno{\wellformedok{\P}{\G}} \times \deno{\typerelation{\P}{A}{\type}}) \\
           & = (\si\star\deno{\wellformedok{\P}{\G}} \times \si\star\deno{\typerelation{\P}{A}{\type}})\\
            & = (\deno{\wellformedok{\P'}{\G\ssi}} \times \deno{\typerelation{\P'}{A\ssi}{\type}})\\
            & = \deno{\wellformedok{\P'}{\G\ssi, x: A\ssi}}\\
            & = \deno{\wellformedok{\P'}{(\gax)\ssi}} & \square
        \end{align*}
    \end{proof}
    
\end{framed}

The effect-weakening theorem on types and type environments is formulated analogously.

\begin{framed}
    \begin{theorem}[Effect Weakening on Types and Type Environments]\label{EffectWeakeningOnTypes}  
        If $\wrelw{\P'}{\P}$ then:
        \begin{enumerate}[label=\roman*.]
            \item $\wellformedtype{\P}{A}$ implies $\wellformedtype{\P'}{A}$  and $\deno{\wellformedtype{\P'}{A}} = \w\star\deno{\wellformedtype{\P}{A}}$
            \item $\wellformedok{\P}{\G}$ implies $\wellformedok{\P'}{\G}$ and $\deno{\wellformedok{\P'}{\G}} = \w\star\deno{\wellformedok{\P}{\G}}$
        \end{enumerate}        
    \end{theorem}
    

\begin{proof}
    The proof for types proceeds in a similar fashion the proof of effect substitution on types (Theorem \ref{EffectSubstitutionOnTypes}). That is, by inducting over the derivation of $\deno{\typerelation{\P}{A}{\type}}$, and making use of the Beck-Chevalley condition and the S-preserving property of $\w\star$.
   
    The proof for type environments follows the same steps as the effect-substitution proof (Theorem \ref{EffectSubstitutionOnTypeEnvs}). $\square$
\end{proof}
\end{framed}

The above theorems confirm that to model the action of  an effect weakening or substitution on the denotation of a type or type environment, we simply apply the appropriate re-indexing functor to the denotation of the type.

Next, we consider the action of weakening and substitution on subtyping relations. The denotations of the subtyping relations, given in Figure \ref{SubtypingDenotations}, are morphisms as opposed to objects. However, the action of substitution and weakening can still be achieved by applying the appropriate re-indexing functor.


\begin{framed}
    \begin{theorem}[Effect Substitution on Subtyping]\label{EffectSubstitutionOnSubtyping}
        If $A$ is a subtype of $B$ under environment $\P$ ($A \subtypep B$), and $\si$ is a substitution from $\P'$ to $\P$ ($\typerelation{\P'}{\si}{\P}$), then:
        \begin{enumerate}[label=\roman*.]
            \item $A\ssi$ is also a subtype of $B\ssi$ under $\P'$ ($A\ssi\subtypepp B\ssi$)
            \item $\deno{A\ssi\subtypepp B\ssi} = \si\star\deno{A\subtypep B}$.
        \end{enumerate}
    \end{theorem}


\begin{proof}
    By rule induction over the definition of the subtype relation (Figures \ref{FullSubtypingDefinition}, \ref{SubtypingDenotations}), making use of S-preserving property and the effect-substitution theorem on types.

    \case{\seffect}
    This case holds due to the S-preservation properties of $\si\star$ on the graded-monad endofunctor $\si\star\T{\e}{f} = \T{\si(\e)}{(\si\star f)}$ and on subeffect natural transformations $\dsep{\e_1}{\e_2}$.

    \begin{align*}
        \si\star\deno{\M{\e_1}{A} \subtypep \M{\e_2}{B}} & = \si\star(\deno{\e_1\subeffectp\e_2})\after\si\star(\T{\e_1}(\deno{A\subtypep B})) \\ 
         &= \deno{\si(\e_1)\subeffectpp\si(\e_2)} \after \T{\si(\e_1)}{\deno{A\ssi\subtypepp B\ssi}}\qt{By S-Preservation}\\
         &= \deno{\M{\si(\e_1)}{A\ssi}\subtypepp\M{\si(\e_2)}{B\ssi}}\\
         &= \deno{(\M{\e_1}{A})\ssi\subtypepp\M{\e_2}{B}\ssi} & \square
    \end{align*}
\end{proof}
\end{framed}

Similarly we can form the symmetrical weakening theorem.

\begin{framed}
    \begin{theorem}[Effect Weakening on Subtyping]\label{EffectWeakeningOnSubtyping}
        If $A \subtypep B$ and $\wrelw{\P'}{\P}$, then $A \subtypepp B$ and $\deno{A\subtypepp B} = \w\star\deno{A\subtypep B}$.
    \end{theorem}
    
    
    \begin{proof}
        The cases hold the same as in the corresponding substitution theorem. $\square$
    \end{proof}
\end{framed}

These subtyping theorems complete our understanding of how the type system is affected by the application of substitutions and weakenings.

\subsection{Effect Substitution and Weakening on Terms}
Now that we have defined the action of effect substitutions on all other components of the PEC type system, we are now at a point to define and prove the effect weakening and substitution theorems on terms. Following the intuition above that changes of index object should be modelled by applying the re-indexing functor to the morphisms denoting the terms, we can construct the theorems. Firstly, we must define the operation of effect substitutions on terms, as in Figure \ref{EffectSubstitutionActionTerms}.

\begin{figure}[H]
    \centering
    \begin{framed}
        
\begin{align*}
    x\ssi & = x \\
    \const{A}\ssi & = \const{(A\ssi)} \\
    (\lam{x}{A}{v})\ssi &= \lam{x}{(A\ssi)}{(v\ssi)}\\
    (\pifthenelse{A}{v}{v_1}{v_2})\ssi &= \pifthenelse{(A\ssi)}{v\ssi}{v_1\ssi}{v_2\ssi}\\
    (\apply{v_1}{v_2})\ssi&= \apply{(v_1\ssi)}{v_2\ssi}\\
    (\return{v})\ssi &= \return{(v\ssi)}\\
    (\doin{x}{v_1}{v_2})\ssi&= \doin{x}{(v_1\ssi)}{(v_2\ssi)}\\
    (\elam{\a}{v})\ssi & = \elam{\a}{(v\ssi)}\qt{If $\a\#\si$}\\
    (\eapp{v}{\e})\ssi & = \eapp{(v\ssi)}{\si(\e)}\\
\end{align*}
    \end{framed}
    
    \caption{Action of effect substitution on terms.}
    \label{EffectSubstitutionActionTerms}
\end{figure}

We can now formulate the substitution theorem on terms. In the theorems in this section, I use the symbol $\D$ to indicate both a typing relation derivation and its denotation.


\begin{framed}
    \begin{theorem}[Effect Substitution on Terms]\label{EffectSubstitutionOnTerms}
        If $\typerelation{\P'}{\si}{\P}$ and typing derivation tree $\D$ derives $\gpetyperelation{v}{A}$ then there exists a typing derivation tree  $\D'$ deriving $\etyperelation{\P'}{\G\ssi}{v\ssi}{A\ssi}$ and $\D' = \si\star(\D)$.
    \end{theorem}
    
    
    \begin{proof}
        This proof makes use of the previous effect-substitution theorems in Section \ref{SubsAndWeakening}, and the adjunction of quantification and the re-indexing functor of projection, $\pstar$. It proceeds by induction on the typing relation (Figure \ref{TypeRules}) and references term denotations (Figure \ref{TermDenotations}).
    
        \case{\vgen}
    
    This case makes use of the naturality condition (Equation \ref{NaturalityCondition}) and simple reductions. 
    If $\D$ derives $\gpetyperelation{\elam{\a}{v}}{\all{\a}{A}}$ then by inversion, there exists $\D_1$ deriving  $\etyperelation{\P,\a}{\G}{v}{A}$ such that $\D = \bar{\D_1}$. By the extension lemma, we can pick $\a\notin\P'\wedge\a\notin\P$ so that we have: $\typerelation{(\P',\a)}{(\si, \a\setto\a)}{(\P,\a)}$. By induction, there exists $\D_1'$ deriving $\etyperelation{(\P', \a)}{\G\sub{\si, \a\setto\a}}{v\sub{\si,\a\setto\a}}{A\sub{\si, \a\setto\a}}$, which we can simplify to $\etyperelation{\P',\a}{\G\ssi}{v\ssi}{A\ssi}$ since $\a$ does not occur in $\P$ or $\P'$. Hence $\etyperelation{\P'}{\G\ssi}{v\ssi}{(\all{\a}{A})\ssi}$ by the tree given in Equation \ref{EffectSubTermsEffectLambda} ($\D'$).
    
    \begin{equation}\label{EffectSubTermsEffectLambda}
        \D' = \ntreeruleI{\vgen}{\treeruleI{\D_1'}{\etyperelation{\P',\a}{\G\ssi}{v\ssi}{A\ssi}}}{\etyperelation{\P'}{\G\ssi}{v\ssi}{(\all{\a}{A})\ssi}}
    \end{equation}
    
    It is also the case that:
    
    \begin{equation}
        \si\times\Id{} = \deno{\typerelation{(\P',\a)}{(\si, \a\setto\e)}{(\P,\a)}}
    \end{equation}
    
    So
    \begin{align*}
        \si\star\D &= \si\star(\bar{\D_1})\\
        & = \bar{(\si\times\Id{E})\star\D_1}\qt{By naturality}\\
        & = \bar{\D_1'}\qt{By induction}\\
        & = \D'
    \end{align*}
    
    \case{\vspec}
    
    This is a more complex case, as it makes use of several naturality properties and the adjunction $\pstar\dashv\allEn$. By inversion, if $\D$ derives $\gpetyperelation{\eapp{v}{\e}}{A\ssub{\a}{\e}}$ then there exists $\D_1$ deriving $\gpetyperelation{v}{\all{\a}{A}}$ and $h = \deno{\typerelation{\P}{\e}{\effect}}$ such that $\D$ is derived from $\D_1$ and $h$ as in Equations \ref{EffectSubTermsEffAppTree} and \ref{EffectSubTermsEffAppDeno}.
    
    
    \begin{equation}\label{EffectSubTermsEffAppTree}
        \D = \ntreeruleII{\vspec}{\treeruleI{\D_1}{\gpetyperelation{v}{\all{\a}{A}}}}{\treeruleI{h}{\typerelation{\P}{\e}{\effect}}}{\gpetyperelation{\eapp{v}{\e}}{A\ssub{\a}{\e}}}
    \end{equation}
    \begin{equation}\label{EffectSubTermsEffAppDeno}
        \D = \pr{\Id{\G}}{h}\star(\e_{\deno{\typerelation{\P,\b}{A\ssub{\a}{\b}}{\type}}})\after\D_1
    \end{equation}
    
    So by induction and effect-substitution preserving well-formedness of effects, we have $\D_1'$ deriving $\etyperelation{\P'}{\G\ssi}{v\ssi}{(\all{\a}{A})\ssi}$ and $\wellformedeffect{\P'}{\si(\e)}$. Using the Effect-Spec type rule, we can construct $\D'$ deriving $\etyperelation{\P'}{\G\ssi}{\eapp{(v\ssi)}{(\si(\e))}}{A\ssi\ssub{\a}{\si(\e)}}$ from $\D_1'$, as in Equation \ref{EffectSubTermsEffAppNewTree}. Since $\a\#\si$, we can commute the applications of substitution. As a result, $\D'$ also derives $\etyperelation{\P'}{\G\ssi}{(\eapp{v}{\e})\ssi}{A\ssub{\a}{\e}\ssi}$.
    
    
    \begin{equation}
        \label{EffectSubTermsEffAppNewTree}
        \D' = \ntreeruleII{\vspec}{
        \treeruleI{\D_1'}{\etyperelation{\P'}{\G\ssi}{v\ssi}{(\all{\a}{A})\ssi}}
        }{
        \wellformedeffect{\P'}{\si(\e)}
        }{\etyperelation{\P'}{\G\ssi}{\eapp{(v\ssi)}{(\si(\e))}}{A\ssi\ssub{\a}{\si(\e)}}}
    \end{equation}
    
    
    
    So, due to the substitution theorem on effects,
    \begin{equation}
        h\after\si = \deno{\typerelation{\P}{\e}{\effect}}\after\si = \deno{\typerelation{\P'}{\si(\e)}{\effect}} = h'
    \end{equation}
    
    Hence, by applying the re-indexing functor to $\D$, we have:
    
    \begin{align}
        \si\star{\D} & = \si\star(\pr{\Id{\G}}{h}\star(\e_{\deno{\typerelation{\P,\b}{A\ssub{\a}{\b}}{\type}}})\after\D_1)\\
        & = (\pr{\Id{\G}}{h}\after\si)\star(\e_{\deno{\typerelation{\P,\b}{A\ssub{\a}{\b}}{\type}}})\after\si\star(\D_1)\\
        & = ((\si\times\Id{E})\after\pr{\Id{\G}}{h\after\si})\star(\e_{\deno{\typerelation{\P,\b}{A\ssub{\a}{\b}}{\type}}})\after\D_1'\\
        & = (\pr{\Id{\G}}{h'})\star((\si\times\Id{E})\star\e_{\deno{\typerelation{\P,\b}{A\ssub{\a}{\b}}{\type}}})\after\D_1' \label{SubsProofLineNumber}
    \end{align}
    
    Looking at the inner part of the functor application:
    Let \begin{align*}
        A & = \deno{\typerelation{\P,\b}{A\ssub{\a}{\b}}{\type}}\\
    \end{align*}
    \begin{align*}
        (\si\times\Id{E})\star\e_{\deno{\typerelation{\P,\b}{A\ssub{\a}{\b}}{\type}}} &= (\si\times\Id{E})\star\counit{A}\\
        & = (\si\times\Id{E})\star(\widehat{\Id{\allEn(A)}})\\
        & = \widehat{\bar{(\si\times\Id{E})\star(\widehat{\Id{\allEn(A)}})}}\qt{By bijection}\\
        & = \widehat{\si\star(\bar{\widehat{\Id{\allEn(A)}}})}\qt{By naturality}\\
        & = \widehat{\si\star(\Id{\allEn(A)})}\qt{By bijection}\\
        & = \widehat{\Id{\allEm((\si\times\Id{E})\star A)}}\qt{By S-preserving property, naturality}\\
        & = \widehat{\Id{\allEm(A\sub{\si,\a\setto\a})}}\qt{By Substitution theorem}\\
        & = \e_{A\ssi}
    \end{align*}
    
    Going back to the expression in Equation \ref{SubsProofLineNumber}
    
    \begin{align*}
        \si\star{\D} & = (\pr{\Id{\G}}{h'})\star(\e_{A\ssi})\after\D_1')\\
        & = \D' & \square
    \end{align*}

    \end{proof}
    
\end{framed}

Similarly, we can derive the weakening theorem on terms.


\begin{framed}
    \begin{theorem}[Effect Weakening on Terms]\label{EffectWeakeningOnTerms}
        If $\wrelw{\P'}{\P}$ and $\D$ derives $\gpetyperelation{v}{A}$ then:
        \begin{enumerate}[label=\roman*.]
            \item there exists $\D'$ deriving $\etyperelation{\P'}{\G}{v}{A}$ 
            \item $\D' = \w\star\D$.
        \end{enumerate}
    \end{theorem}
    
    \begin{proof}
        This theorem is proved in a similar fashion to the effect substitution theorem (Theorem \ref{EffectSubstitutionOnTerms}) and many of its cases are the same. Some cases of this proof can be found in Appendix \ref{AppendixEffectWeakeningOnTerms}.
    \end{proof}
\end{framed}

This concludes our treatment of effect-environment substitution and weakening.

\subsection{Term Substitution and Weakening}

Having analysed the manipulation of the effect-variable environment, we are now at a point to start considering weakenings and substitution of the type environments. Type environment weakenings are inductively defined in Figure \ref{TermWeakeningRelation} with respect to an effect-variable environment.


\begin{figure}[H]
    \centering
    \begin{framed}
        \[
            \ntreeruleI{\tid}{\wellformedok{\P}{\G}}{\pewrel{\i}{\G}{\G}}
            \quad  
            \condtreeruleII{\tproject}{\pewrel{\w}{\G'}{\G}}{\typerelation{\P}{A}{\type}}{\pewrel{\w \pi}{\G, x: A}{\G}}{x \notin \dom{\G'}}
        \]
        
        \[
            \condtreeruleII{\textend}{\pewrel{\w}{\G'}{\G}}{ A \subtype B}{\pewrel{\w \x}{\G', x: A}{\G, x: B}}{x \notin \dom{\G'}}
        \] 
    \end{framed}
    \caption{Definition of the term weakening relation.}
    \label{TermWeakeningRelation}
\end{figure}


The denotation of such a weakening is a morphism within the fibre category derived from the effect-variable environment: $\deno{\pewrel{\w}{\G'}{\G}}: \G' \rightarrow \G \in \C(E^n)$. These denotations are defined in Figure \ref{TermWeakeningDenotations}.

\begin{figure}[H]
    \centering
    \begin{framed}
        \begin{align}
            \deno{\pewrel{\i}{\G}{\G}} &= \idg: \G \rightarrow \G \in \C(E^n)
            \\
            \deno{\pewrel{\w\pi}{\G', x: A}{\G}} &= \deno{\pewrel{\w}{\G'}{\G}}\after\p: \G'\times A \rightarrow \G
            \\
            \deno{\pewrel{\w\x}{\G', x: A}{\G, x: B}} &= \deno{\pewrel{\w}{\G'}{\G}}\times \deno{A\subtypep B}: \G'\times A \rightarrow \G\times B
        \end{align}
        
    \end{framed}\caption{Denotations of Term Weakening.}
    \label{TermWeakeningDenotations}
\end{figure}


Similarly to the definition of effect-environment substitutions (Section \ref{SectionEffectSubstitution}), type-environment substitutions are also snoc-lists derived inductively with respect to an effect-variable environment in Figure \ref{TermSubstitutionDefinition}.

\begin{figure}[H]
    \centering
\begin{framed}
    
    \[
        \si\gens\nil\mid\si,x\setto v    
    \]

    \[
    \ntreeruleI{\tsubnil}{\wellformedok{\P}{\G'}}{\etyperelation{\P}{\G'}{\nil}{\nil}}
    \quad
    \condtreeruleII{\tsubextend}{
        \etyperelation{\P}{\G'}{\si}{\G}
    }{
        \etyperelation{\P}{\G'}{v}{A}
    }{
        \etyperelation{\P}{\G'}{(\si, x \setto v)}{(\gax)}
    }{x \notin \dom{\G'}}
\]
\end{framed}
    \caption{Definition of Term Substitutions.}
    \label{TermSubstitutionDefinition}
\end{figure}




We also need to explain the action of these term substitutions on terms. We define the action of applying a substitution $\si$  on term $v$, written $v\ssi$, in Figure \ref{TermSubstitutionAction}. 


\begin{framed}
    \begin{definition}[Freshness]
        We write $x\#\si$ to indicate that x does not occur in the domain of $\si$ or as a free variable in any of its substituted terms.
    \end{definition}
\end{framed}

\begin{figure}[H]
    \centering
    \begin{framed}
        \begin{align*}
            x\sub{\nil} & = x \\
            x\sub{\si,x\setto v} & = v \\
            x\sub{\si,x'\setto v'} & = x\ssi\qt{If }x \neq x'\\
            \const{A}\ssi & = \const{A} \\
            (\lam{x}{A}{v})\ssi &= \lam{x}{A}{(v\ssi)}\qt{If }x\#\si\\
            (\pifthenelse{A}{v}{v_1}{v_2})\ssi &= \pifthenelse{A}{v\ssi}{v_1\ssi}{v_2\ssi}\\
            (\apply{v_1}{v_2})\ssi&= \apply{(v_1\ssi)}{v_2\ssi}\\
            (\doin{x}{v_1}{v_2})&= \doin{x}{(v_1\ssi)}{(v_2\ssi)}\qt{If } x\#\si\\
            (\return{v})\ssi & = \return{(v\ssi)} \\
            (\elam{\a}{v})\ssi & = \elam{\a}{(v\ssi)}\\
            (\eapp{v}{\e})\ssi & = \eapp{(v\ssi)}{\e}\\
        \end{align*}
            
    \end{framed}
    
    \caption{Action of term substitutions.}
    \label{TermSubstitutionAction}
\end{figure}


Denotations of type-environment substitutions are morphisms in the appropriate fibre category: $\deno{\etyperelation{\P}{\G'}{\si}{\G}}: \G' \rightarrow \G \in \C(E^n)$ and are defined in Figure \ref{TermSubstitutionDenotations}.

\begin{figure}[H]
    \centering
    \begin{framed}
\begin{align*}
    \deno{\etyperelation{\P}{\G'}{\nil}{\nil}} = \term{\G'}:&\quad \G' \rightarrow \1
    \\
    \deno{\etyperelation{\P}{\G'}{(\si, x \setto v)}{\gax}} =&\quad \pr{\deno{\etyperelation{\P}{\G'}{\G}}}{\deno{\etyperelation{\P}{\G'}{v}{A}}}: \G' \rightarrow \G\times A
\end{align*}
    \end{framed}
    \caption{Denotations for term substitutions.}
    \label{TermSubstitutionDenotations}
\end{figure}


In order to prove the quantification case of the type-environment weakening and substitution theorems on terms, as can be seen in the road map in Figure \ref{RoadMap}, it will be necessary to be able to weaken the effect-variable environment on type-environment weakenings and substitutions.

Let us first consider the action of effect weakenings on these morphisms. The weakening theorem on type-environment weakenings is formulated as so:



\begin{framed}
    \begin{theorem}[Effect Weakening on Term Weakening]\label{EffectWeakeningOnTermWeakening}
        If $\wrel{\w_1}{\P'}{\P}$ and $\pewrel{\w}{\G'}{\G}$ then:
        \begin{enumerate}[label=\roman*.]
            \item $\ewrel{\P'}{\w}{\G'}{\G}$
            \item $\deno{\ewrel{\P'}{\w}{\G'}{\G}} = \w_1\star\deno{\pewrel{\w}{\G'}{\G}}$.
        \end{enumerate} 
    \end{theorem}
    
    \begin{proof}
        By induction on the derivation of $\w$. making use of weakening on types, type environments, and subtyping.
    
        \case{\textend}
        Then $\w = \w'\x$
        
        \begin{equation}
            \ntreeruleII{\textend}{\ewrel{\P}{\w'}{\G'}{\G}}{A\subtypep B}{\ewrel{\P}{\w\x}{(\G',x:A)}{(\G, x:B)}}
        \end{equation}
        
        So $\w = \w'\times\deno{A\subtypep B}$
        
        Hence
        \begin{align*}
            \w_1\star(\w) &=(\w_1\star(\w')\times\w_1\star(\deno{A\subtypep B})\\
            & = (\deno{\ewrel{\P'}{\w'}{\G'}{\G}}\times\deno{A\subtypepp B})\\
            & = \deno{\ewrel{\P'}{\w}{(\G',x:A)}{(\G,x:B)}} & \square
        \end{align*}
    \end{proof}
\end{framed}


Secondly, we can state and prove the weakening theorem on type-environment substitutions.



\begin{framed}
    \begin{theorem}[Effect Weakening on Term Substitution]\label{EffectWeakeningOnTermSubstitution}
        If $\etyperelation{\P}{\G'}{\si}{\G}$ and $\wrelw{\P'}{\P}$ then:
        
        \begin{enumerate}[label=\roman*.]
            \item $\etyperelation{\P'}{\G'}{\si}{\G}$
            \item $\deno{\etyperelation{\P'}{\G'}{\si}{\G}} = \w\star\deno{\etyperelation{\P}{\G'}{\si}{\G}}$
        \end{enumerate}
    \end{theorem}
    
    
    \begin{proof}
        By induction on the definition of $\si$, making use of the weakening on terms, types, and type environments.
    
        \case{\tsubnil}
        Then $\si = \term{\G'_{E^n}}$, so $\w\star(\si) = \term{\G'_{E^m}} = \deno{\etyperelation{\P'}{\G'}{\si}{\nil}}$
        
        \case{\tsubextend}
        Then $\si = (\si',x\setto v)$
        
        \begin{align*}
            \w\star\si & = \w\star\pr{\si'}{\deno{\gpetyperelation{v}{A}}}\\
            & = \pr{\w\star\si'}{\w\star\deno{\gpetyperelation{v}{A}}}\\
            &=\pr{\deno{\etyperelation{\P'}{\G'}{\si'}{\G}}}{\deno{\etyperelation{\G'}{\P'}{v}{A}}}\\
            &=\deno{\etyperelation{\P'}{\G'}{\si}{\gax}} & \square
        \end{align*}
    \end{proof}
\end{framed}

As with effect substitutions, it is useful to define a couple of special substitutions and lemmas on term substitutions. Firstly, we define the identity and single substitutions in an equivalent way in Figure \ref{TermSubstitutionIdAndSingle}.

\begin{figure}[H]
    \centering

    \begin{minipage}{.45\linewidth}
        \begin{framed}
            \begin{align*}
                \etyperelation{\P}{\G}{&\Id{\G}}{\G}\\
                \Id{\nil} & = \nil\\
                \Id{\gax} & = (\idg, x\setto x)\\
                \ssub{x}{v} & = (\idg, x\setto v)
            \end{align*}
        \end{framed}
    \end{minipage}
    \quad
    \begin{minipage}{.45\linewidth}
        \begin{framed}
            \begin{align*}
                \deno{\etyperelation{\P}{\G}{\Id{\G}}{\G}}: &\quad \G\rightarrow\G\\
                \deno{\etyperelation{\P}{\G}{\Id{\G}}{\G}} = &\quad  \idg\\
                \deno{\etyperelation{\P}{\G}{\ssub{x}{v}}{\gax}}: &\quad \G\rightarrow \G\times A\\
                \deno{\etyperelation{\P}{\G}{\ssub{x}{v}}{\gax}} =\quad & \pr{\idg}{\deno{\gpetyperelation{v}{A}}}
            \end{align*}
        \end{framed}
    \end{minipage}
    
    \caption{Definition and denotation of identity and single term substitutions.}
    \label{TermSubstitutionIdAndSingle}
\end{figure}



By inspection, these also have simple denotations, similar to those seen in the case of effect substitutions. Furthermore, it will be useful to have an analogue of the extension lemma for term substitutions.


\begin{framed}
    \begin{lemma}[Extension Lemma on Term Substitutions]
        If $\etyperelation{\P}{\G'}{\si}{\G}$ and $x\#\si$\\ then $\etyperelation{\P}{(\G',x: A)}{(\si, x\setto x)}{(\gax)}$ with denotation $$\deno{\etyperelation{\P}{(\G',x: A)}{(\si, x\setto x)}{(\gax)}} = \deno{\etyperelation{\P}{\G'}{\si}{\G}} \times \Id{A}$$
    \end{lemma}
    
    \begin{proof}
         Makes use of the weakening on terms (Theorem \ref{TermWeakeningOnTerms}). If $\etyperelation{\P}{\G'}{\si}{\G}$ then $\etyperelation{\P}{(\G', x:A) }{\si}{\G}$ with denotation $\si\after\p$. $\square$
    \end{proof}
\end{framed}


Finally, we can move on to the term substitution and weakening theorems. These theorems are the final step before we prove that derivations of the same typing-relation instance have the same denotation and then move onto soundness. They demonstrate that we can model the action of applying well-formed type-environment changes by pre-composing the morphisms to be acted on with a morphism modelling the change in environment. The term-substitution theorem is formulated as so: 

\begin{framed}
    \begin{theorem}[Term Substitution]\label{TermSubstitutionOnTerms}
        If $\etyperelation{\P}{\G'}{\si}{\G}$ and $\D$ is a derivation of $\gpetyperelation{v}{A}$, then we can construct $\D'$ deriving $\etyperelation{\P}{\G'}{v\ssi}{A}$ with denotation $\D' = \D \after \si$.
    \end{theorem}
    
    
    \begin{proof}
        By induction on the derivation of $\D$ (Figure \ref{TypeRules}) using the denotations of terms (Figure \ref{TermDenotations}). Making use of the weakening of effect-variable environments on term substitutions in the case of (\textit{\vgen})
    
        \case{\vgen}
        
        By inversion, we have $\D_1$ such that the derivation tree in Equation \ref{TermSubGenOne} holds. By induction on $\D_1$, we derive $\D_1'$ such that the derivation in Equation \ref{TermSubGenTwo} holds. By induction, we can decompose the denotation $\D_1'$ into $\D_1$ and a substitution, which is the weakened denotation of $\si$ (Equation \ref{TermSubGenSub}).

        \begin{minipage}{0.45\linewidth}
            \begin{equation} \label{TermSubGenOne}
                \D = \ntreeruleI{\vgen}{\treeruleI{\D_1}{\etyperelation{\P,\a}{\G}{v}{A}}}{\gpetyperelation{\elam{\a}{v}}{\all{\e}{A}}}
            \end{equation}
        \end{minipage}\quad\begin{minipage}{0.45\linewidth}
            \begin{equation}\label{TermSubGenTwo}
                \D' = \ntreeruleI{\vgen}{\treeruleI{\D_1'}{\etyperelation{\P,\a}{\G'}{v\ssi}{A}}}{\etyperelation{\P}{\G'}{(\elam{\a}{v})\ssi}{\all{\e}{A}}}
            \end{equation}
        \end{minipage}

        \begin{centering}
            \parbox{0.5\linewidth}{%
            \centering
                \begin{align*}\numberthis\label{TermSubGenSub}
                    \D_1' & = \D_1\after\deno{\etyperelation{\P,\a}{\G'}{\si}{\G}}\\
                    & = \D_1\after\deno{\wrel{\i\pi}{\P,a}{\P}}\star(\si)\\
                    & = \D_1\after\pstar(\si)
                \end{align*}    
            }
        \end{centering}


        Finally, by using the adjunction property, we can derive $D'$ in terms of $D$. 
        \begin{align*}
            \D\after\si & = \bar{\D_1}\after\si\\
            & = \bar{\D_1\after\pstar(\si)}\\
            & = \bar{\D_1'}\\
            & = \D'
        \end{align*}
    
   
    \end{proof}
\end{framed}

The term-weakening theorem is formulated similarly. 

\begin{framed}
    \begin{theorem}[Term Weakening]\label{TermWeakeningOnTerms}
        If $\pewrel{\w}{\G'}{\G}$ and $\D$ is a derivation of $\gpetyperelation{v}{A}$ then we can derive $\D'$, a derivation of $\etyperelation{\P}{\G'}{v}{A}$ with denotation $\D' = \D\after\w$.
    \end{theorem}
    
    \begin{proof}
        This proof follows similarly to the proof of the substitution theorem (Theorem \ref{TermSubstitutionOnTerms}). Some cases can be found in Appendix \ref{AppendixTermWeakeningTheorem}
    \end{proof}
    
\end{framed}




\subsection{Summary}
In this section, I have stated and proved a number of theorems to do with the substitution of effect and term variables within an expression. In particular, these theorems give us a way of manipulating the denotations of terms when the effect-variable and type environments are changed. As a result, these theorems provide us with a set of tools to tackle problems such as soundness and the uniqueness of denotations.

\section{Uniqueness of Denotations}\label{UniqueDenotations}

Up until this point, due to the subtyping rule, we have had to be careful about the implicit typing derivation for every term denotation $\deno{\gpetyperelation{v}{A}}$. This is because typing relations have multiple derivations, as seen in Figure \ref{NonUniqueDerivations}. We are now equipped with the tools to prove that all derivations of the same typing relation instance $\gpetyperelation{v}{A}$ induce the same denotation. This allows us to no longer worry about the equality of denotations, which will be helpful in the soundness proof.


\begin{figure}[H]
\begin{framed}
        \centering
            \begin{minipage}{0.47\linewidth}
                \centering
                \begin{equation*}
                    \scalebox{.9}{$                    \ntreeruleII{\vsubtype}{
                        \scalebox{.6}{$
                        \ntreeruleII{\vapply}{\gpetyperelation{v_1}{\ab}}{\gpetyperelation{v_2}{A}}{\gpetyperelation{\apply{v_1}{v_2}}{B}}
                        $}
                    }{
                        \scalebox{.9}{$B\subtypep B'$}
                    }{
                        \gpetyperelation{\apply{v_1}{v_2}}{B'}
                    }
                    $}
                \end{equation*}
            \end{minipage}
            \quad
            \begin{minipage}{0.47\linewidth}
                \centering

                \begin{equation*}
                    \scalebox{.9}{$
                    \ntreeruleII{\vapply}{
                        \scalebox{.6}{$
                        \ntreeruleII{\vsubtype}{\gpetyperelation{v_1}{\ab}}{\ab\subtypep \fntype{A}{B'}}{\gpetyperelation{v}{\fntype{A}{B'}}}
                        $}
                    }{
                        \scalebox{.7}{$\gpetyperelation{v_2}{A}$}
                    }{
                        \gpetyperelation{\apply{v_1}{v_2}}{B'}
                    }$}
                \end{equation*}
            \end{minipage}
\end{framed}
    \caption{Two different derivations of the same typing relation.}
    \label{NonUniqueDerivations}
\end{figure}

To prove that all typing derivations have the same denotation, I first introduce the concept of a \textit{reduced} typing derivation that is unique to each term and type in each effect and type environment. Next, I present a function, $\reduce$, which recursively maps  typing derivations to their reduced equivalent. I also prove that this function preserves the denotation of the derivations. That is $\deno{\gpetyperelation{v}{A}} = \deno{\reduce(\gpetyperelation{v}{A})}$. Hence, we can conclude, since all derivations for a typing relation instance reduce to the same, unique typing derivation, and that the reduction function preserves the denotations, that all derivations of a typing derivation have the same denotation.

The need for reduced typing derivations comes about because of subtyping. The subtyping rule can be inserted into different places in a derivation to derive the same typing relation. Hence, the reduction function focuses on only placing subtyping rule instances in specific places. 

In particular, a reduced typing derivation is one where subtyping rule only occurs in three places. Firstly, the subtyping rule must occur exactly once at the root of the tree in order to coerce the the typing relation to the correct type. Secondly, the subtyping rule must occur at the root of any of subtrees deriving the preconditions of the type-annotated (\textit{\vif}) rule. This is to coerce the condition to a boolean type, and each of the branches to the type $A$ in the if-expression itself. Finally, the subtyping rule must occur at the root of the argument subtree of an (apply) rule. This allows us to coerce the argument to the correct type to match the parameter type in a lambda expression. These subtyping rule instances may be the identity subtyping rule, making use of the reflexive relation $A \subtypep A$. These rules have the effect of only introducing subtyping rule uses when it is necessary maintain syntactic correctness of the derivation. 


\begin{framed}
    \begin{aside}[Syntactic Subtyping]
        If subtyping were a syntactic feature, that is if subtyping were induced by explicit casts, then the rules for the typing relation would be entirely syntax directed. Hence type derivations would be unambiguous and we would not require a proof that denotations are unique.
    \end{aside}
\end{framed}


\begin{framed}
    \begin{theorem}[Uniqueness of reduced Derivations]\label{UniquenessOfReducedDenotations}
        These reduced derivations are unique.    
    \end{theorem}
    
    \begin{proof}
        This proof proceeds by induction on the term structure, making use of the unique derivations of the subterms to show that a reduced derivation of the whole term must also be unique. There are no cases for subtyping, as it is not a syntactic feature. 
    
        \case{\vbind} This case makes use of the weakening theorem on type environment. Let the trees in Equations \ref{UniqueBindOne}, \ref{UniqueBindTwo} be the respective unique reduced type derivations of the subterms. By weakening, $\ewrel{\P}{\i\x}{(\G, x:A)}{(\G, x: A')}$, so if there is a derivation of $\etyperelation{\P}{(\G, x:A')}{v_2}{B}$, there is also one of $\etyperelation{\P}{\gax}{v_2}{B}$ (equation \ref{UniqueBindThree}). 
    
        \begin{equation}\label{UniqueBindOne}
            \edeltacrule{\G}{v_1}{\e_1}{A}{\e_1'}{A'}
        \end{equation}
    
        \begin{equation}\label{UniqueBindTwo}
            \edeltacruleprime{\G, x:A'}{v_2}{\e_2}{B}{\e_2'}{B'}
        \end{equation}
    
        \begin{equation}\label{UniqueBindThree}
            \edeltacruleprimeprime{(\G, x:A)}{v_2}{\e_2}{B}{\e_2'}{B'}
        \end{equation}
    
        Since the effects monoid operation is monotone, if $\e_1\subeffectp\e_1'$ and $\e_2\subeffectp\e_2'$ then $\e_1\dot\e_2 \subeffectp \e_1'\dot\e_2'$. Hence the reduced type derivation of $\gpetyperelation{\doin{x}{v_1}{v_2}}{\M{\e_1'\dot\e_2'}{B'}}$ can be seen in Equation \ref{UniqueBindResult}.
    
        \begin{equation}\label{UniqueBindResult}
            \resizebox{.9\hsize}{!}{
            \ntreeruleII{\vsubtype}{
                \ntreeruleII{\vbind}{
                    \treeruleI{\D}{\gpetyperelation{v_1}{\M{\e_1}{A}}}
                }{
                    \treeruleI{\D''}{\etyperelation{\P}{\G, x:A}{v_2}{\M{\e_2}{B}}}
                } {
                    \gpetyperelation{\doin{x}{v_1}{v_2}}{\M{\e_1\dot\e_2}{B}}
                }
            }{
                \subeffecttreep{\e_1\dot\e_2}{B}{\e_1'\dot\e_2'}{B'}
            } {
                \gpetyperelation{\doin{x}{v_1}{v_2}}{\M{\e_1'\dot\e_2'}{B'}}
            }
        }
        \end{equation}
    
        \case{\vgen}
        If Equation \ref{ApplyUniqueOne} is the unique reduced derivation of $\etyperelation{\P,\a}{\G}{v}{B}$, then the unique reduced derivation of $\gpetyperelation{\elam{\a}{A}}{\all{\a}{B}}$ is shown in Equation \ref{ApplyUniqueResult}. $\square$
        \begin{eqnarray}\label{ApplyUniqueOne}
            \ntreeruleII{\vsubtype}{
                \treeruleI{\D}{\etyperelation{\P,\a}{\G}{v}{A}}
            }{
                A\subtypepa B
            }{
                \etyperelation{\P,\a}{\G}{v}{B}
            }
        \end{eqnarray}

        \begin{equation}\label{ApplyUniqueResult}
            \ntreeruleII{\vsubtype}{
                \ntreeruleI{\vgen}{
                    \treeruleI{\D}{\etyperelation{\P,\a}{\G}{v}{A}}
                }{
                    \gpetyperelation{\elam{\a}{v}}{\all{\a}{A}}
                }
            }{
                \all{\a}{A}\subeffectp\all{\a}{B}
            }{
                \gpetyperelation{\elam{\a}{v}}{\all{\a}{B}}
            }
        \end{equation}
    
    \end{proof}
    
\end{framed}

The $\reduce$ function maps each derivation to its reduced equivalent. It does this by pushing subtyping rules from the leaves of the derivation tree down towards the root of the tree. The function case splits on the root of the tree and works recursively. Some cases of the function are given in Figure \ref{ReduceFunctionCases}. 

\begin{figure}[H]
    \begin{framed}
        \begin{center}
            \textbf{The Reduce Function}   \par
        \end{center}

    \case{\vapply}
    To find the reduction of a tree ending in (\textit{\vapply}) as seen in Equation \ref{ApplyBeforeReduction}, we pattern match on the reductions of the subtrees to find trees $\D_1'$, $\D_2'$, as seen in Equations \ref{ApplyReductionOne} and \ref{ApplyReductionTwo}. These trees allow us to build a reduced derivation tree as in Equation \ref{ApplyAfterReduction}.

    \begin{equation}\label{ApplyBeforeReduction}
        \scalebox{.8}{$
        \reduce(\ntreeruleII{\vapply}{
            \treeruleI{\D_1}{
                \gpetyperelation{v_1}{\ab}
            }
        }{
            \treeruleI{\D_2}{
                \gpetyperelation{v_2}{A}
            }
        }{
            \gpetyperelation{\apply{v_1}{v_2}}{B}
        })
        $}
    \end{equation}

    \begin{equation}\label{ApplyReductionOne}
        \scalebox{.8}{$
            \ntreeruleII{\vsubtype}{
                \treeruleI{\D'_1}{\gpetyperelation{v_1}{\fntype{A'}{B'}}}
            }{
                \fntype{A'}{B'}\subtypep\fntype{A}{B}
            }{
                \gpetyperelation{v_1}{\ab}
            }  = \reduce(\D_1)
            $}
    \end{equation}

    \begin{equation}\label{ApplyReductionTwo}
        \scalebox{.8}{$
        \ntreeruleII{\vsubtype}{
            \treeruleI{\D'_2}{\gpetyperelation{v}{A'}}
        }{
            A'\subtypep A
        }{
            \gpetyperelation{v_1}{A}
        } = \reduce(\D_2)
        $}
    \end{equation}


    \begin{equation}\label{ApplyAfterReduction}
        \scalebox{.8}{$
        \ntreeruleII{\vsubtype}{
            \ntreeruleII{\vapply}{
                \treeruleI{
                    \D'_1
                }{
                    \gpetyperelation{v_1}{\fntype{A'}{B'}}
                }
            }{
                \ntreeruleII{\vsubtype}{
                    \treeruleI{\D'_2}{\gpetyperelation{v_2}{A''}}
                }{
                    A'' \subtypep A \subtypep A'
                } {
                    \gpetyperelation{v_2}{A'}
                }
            }{
                \gpetyperelation{\apply{v_1}{v_2}}{B'}
            }
        }{
            B' \subtypep B
        }{
            \gpetyperelation{\apply{v_1}{v_2}}{B}
        }
        $}
    \end{equation}

    \end{framed}

    \caption{An example case of the $\reduce$ function. Further cases can be seen in Appendix \ref{AppendixReduceFunctionCases}.}
    \label{ReduceFunctionCases}
\end{figure}


\begin{framed}
    \begin{theorem}[Reduction preserves denotations]\label{ReductionPreservesDenotations}
       If the derivation $\D'$ is the result of applying  reduce to $\D$ then the denotations of the derivations are equal. That is $\D' = \reduce(\D) \implies \D' = \D$.
    \end{theorem}
    
    
    \begin{proof}
        We proceed by induction over the structure of $\D$, making use of the substitution and weakening theorems. We make use of the the definition of the $\reduce$ function as defined in Figure \ref{ReduceFunctionCases}. We also use the definitions of $\D_1, \D_2$ from the same figure. Some of the other cases of this proof make use of the term weakening theorem (Theorem \ref{TermWeakeningOnTerms}).

        \case{\vapply}
        This case makes use of the fact that composing subtyping morphisms gives the transitive subtyping morphism. Let us define some short-hands.
            \begin{align*}
                f & = \deno{A\subtypep A'}: A\rightarrow A' \\
                f' & = \deno{A''\subtypep A}: A'' \rightarrow A \\
                g & = \deno{B' \subtypep B}: B' \rightarrow B \\
            \end{align*}
    
            Hence 
            \begin{align*}
                \deno{\fntype{A'}{B'}\subtypep \ab} & = (g)^A \after (B')^f \\
                & = \cur{g\after \app}\after\cur{\app\after(\Id{}\times f)}\\
                & = \cur {g\after\app\after(\Id{}\times f)}
            \end{align*}
    
            Then 
            \begin{align*}
                \D & = \app\after\pr{\D_1}{\D_2}\qt{By definition}\\
                & = \app\after\pr{\cur {g\after\app\after(\Id{}\times f)}\after\D'_1}{f'\after\D'_2}\qt{By reductions of $\D_1, \D_2$}\\
                & = \app\after(\cur {g\after\app\after(\Id{}\times f)}\times\Id{A})\after\pr{\D'_1}{f'\after\D'_2} \qt{Factoring out}\\
                & = g\after\app\after(\Id{}\times f)\after\pr{\D'_1}{f'\after\D'_2}\qt{By the exponential property}\\
                & = g\after\app\after\pr{\D'_1}{f\after f'\after \D'_2}\\
                & = \D'\qt{By defintion}
            \end{align*}
    \end{proof}
\end{framed}

\begin{figure}[h!]
    \begin{framed}
        \begin{minipage}{0.47\linewidth}
            \begin{equation}
                \ntreeruleI{\vfun}{\scalebox{.65}{$
                    \ntreeruleII{\vsubtype}{
                        \etyperelation{\P}{\gax}{v}{B}
                    }{
                        B \subtypep B'
                    }{
                        \etyperelation{\P}{\gax}{v}{B'}
                    }
                $}}{\gpetyperelation{\lam{x}{A}{v}}{\fntype{A}{B'}}}
            \end{equation}
        \end{minipage}
        \quad
        \begin{minipage}{0.47\linewidth}
            \begin{equation}
                \ntreeruleII{\vsubtype}{
                    \scalebox{.65}{$
                    \ntreeruleI{\vfun}{
                        \etyperelation{\P}{\gax}{v}{B} 
                    }{    
                        \gpetyperelation{\lam{x}{A}{v}}{\fntype{A}{B}}
                    }
                    $}
                }{ 
                    \scalebox{.7}{$\ab \subtype \fntype{A}{B'}$}
                }{
                    \gpetyperelation{\lam{x}{A}{v}}{\fntype{A}{B'}}
                }
            \end{equation}
        \end{minipage}
        
    \end{framed}

    \caption{Two derivations of the same type relation. The right derivation is in reduced form.}
\end{figure}

We can sum up the importance of these theorems in the corollary below.

\begin{framed}
    \begin{corollary}[Denotations are Unique]\label{DenotationsAreUnique}
      For any two derivations $\D_1, \D_2$ deriving $\gpetyperelation{v}{A}$, the denotations of $\D_1$ and $\D_2$ are equal.
    \end{corollary}

    \begin{proof}
        Since reduced derivations of the typing relation are unique, the derivations $\reduce(\D_1), \reduce(\D_2)$ are equivalent and so have the same denotation. Since reduction preserves denotations, $\D_1 = \reduce(\D_1) = \reduce(\D_2) = \D_2$.
    \end{proof}
\end{framed}

\section{Soundness}
\label{SoundnessDefinition}
We are now at a stage where we can state and prove the most important theorem for a denotational semantics: soundness with respect to an equational equivalence. The desire for soundness follows from the common-sense requirement that equivalent programming language terms should also have equal denotations. In our case, I introduce a $\beta\eta$-based equational equivalence relation and then prove that equivalent terms have equal denotations.

The equational equivalence relation is a rule-based relation with three main flavours of rules. Firstly, as seen in Figure \ref{BetaEtaReductions}, there are the reductions which formalise how we expect the program to execute given an appropriate implementation. We give $\beta\eta$-reductions for each term transition, such as the application of lambda terms or the execution of an \texttt{if} expression, as well as the monad equivalence laws. Secondly, there are congruences, seen in Figure \ref{BetaEtaCongruence}, which formalise how the reduction of subexpressions affects the rest of the expression in a compositional way. Finally, we extend this relation into an equivalence relation by closing it under transitivity, reflexivity and symmetry as seen in Figure \ref{BetaEtaEquivalence}.

\begin{figure}[H]
    
    \begin{framed}
        \[
            \scalebox{0.8}{$
            \ntreeruleII{\eqbeta}{\etyperelation{\P}{\gax}{v_2}{B}}{\gpetyperelation{v_1}{A}}{\gpeberelation{\apply{(\lam{x}{A}{v_1})}{v_2} }{ v_1\ssub{x}{v_2}}{B}}
            \quad
            \ntreeruleI{\eqeta}{\gpetyperelation{v}{\ab}}{\gpeberelation{\lam{x}{A}{(\apply{v}{x}})}{v}{\ab}}
            $}
        \]
    
        \[\scalebox{0.8}{$
            \ntreeruleII{\eqleftunit}{\gpetyperelation{v_1}{A} }{ \etyperelation{\P}{\gax}{v_2}{\meb}}{\gpeberelation{\doin{x}{\return{v_1}}{v_2}}{v_2\ssub{x}{v_1}}{\meb}}
            \quad
            \ntreeruleI{\eqrightunit}{\gpetyperelation{v}{\mea}}{\gpeberelation{\doin{x}{v}{\return{x}} }{v}{\mea}}
        $}\]
    
        \[\scalebox{0.8}{$
            \ntreeruleIII{\eqassociativity}{\gpetyperelation{v_1}{\M{\e_1}{A}} }{\etyperelation{\P}{\gax}{v_2}{\M{\e_2}{B}}}{ \etyperelation{\P}{\gby}{v_3}{\M{\e_3}{C}}}{
                \gpeberelation{\doin{x}{v_1}{(\doin{y}{v_2}{v_3})}}{\doin{y}{(\doin{x}{v_1}{v_2})}{v_3}}{\M{\e_1 \dot \e_2 \dot \e_3}{C}}
            }
        $}\]
    
        \[\scalebox{0.8}{$
            \ntreeruleI{\equnitequiv}{\gpetyperelation{v}{\U}}{\gpeberelation{v}{\u}{\U}}
        $}\]
    
        \[\scalebox{0.8}{$
            \ntreeruleII{\eqiftrue}{\gpetyperelation{v_1}{A}}{\gpetyperelation{v_2}{A}}{\gpeberelation{\pifthenelse{A}{\t}{v_1}{v_2}}{v_1}{A}}
            \quad
            \ntreeruleII{\eqiffalse}{\gpetyperelation{v_2}{A}}{\gpetyperelation{v_1}{A}}{\gpeberelation{\pifthenelse{A}{\f}{v_1}{v_2}}{v_2}{A}}    
        $}\]
    
        \[\scalebox{0.8}{$
            \ntreeruleII{\eqifeta}{\etyperelation{\P}{\G, x: \B}{v_2}{A}}{\gpetyperelation{v_1}{\B}}{\gpeberelation{\pifthenelse{A}{v_1}{v_2\ssub{x}{\t}}{v_2\ssub{x}{\f}}}{v_2\ssub{x}{v_1}}{A}}
        $}\]
    
        \[\scalebox{0.8}{$
            \ntreeruleII{\eqeffbeta}{\wellformedeffect{\P}{\e}}{\etyperelation{\P, \a}{\G}{v}{A}}{\gpeberelation{\eapp{(\elam{\a}{v}}{\e})}{v\ssub{\a}{\e}}{A\ssub{\a}{\e}}}
            \quad 
            \ntreeruleI{\eqeffeta}{\etyperelation{\P}{\G}{v}{\all{\a}{A}}}{\gpeberelation{\elam{\a}{(\eapp{v}{\a})}}{v}{\all{\a}{A}}}
        $}\]
    \end{framed}
    \caption{The reduction rules for PEC.}
    \label{BetaEtaReductions}
\end{figure}

\begin{figure}[H]
   \begin{framed}
        \[\scalebox{0.8}{$
            \ntreeruleI{\eqgen}{\eberelation{\P, \a}{\G}{v_1}{v_2}{A}}{\gpeberelation{\elam{\a}{v_1}}{\elam{\a}{v_2}}{\all{\a}{A}}}
            \quad
            \ntreeruleII{\eqspec}{\gpeberelation{v_1}{v_2}{\all{\a}{A}}}{\wellformedeffect{\P}{\e}}{\gpeberelation{\eapp{v_1}{\e}}{\eapp{v_2}{\e}}{A\ssub{\a}{\e}}}
        $}\]
    
        \[\scalebox{0.8}{$
            \ntreeruleI{\eqfun}{\eberelation{\P}{\gax}{v_1}{v_2}{B}}{\gpeberelation{\lam{x}{A}{v_1}}{\lam{x}{A}{v_2}}{\ab}}
            \quad
            \ntreeruleI{\eqreturn}{\gpeberelation{v_1}{v_2}{A}}{\gpeberelation{\return{v_1}}{\return{v_2}}{\moa}}
        $}\]
    
        \[\scalebox{0.8}{$
            \ntreeruleII{\eqapply}{\gpeberelation{v_1}{v_1'}{\ab}}{\gpeberelation{v_2}{v_2'}{A}}{\gpeberelation{\apply{v_1}{v_2}}{\apply{v_1'}{v_2'}}{B}}
            \quad   
            \ntreeruleII{\eqbind}{\gpeberelation{v_1}{v_1'}{\M{\e_1}{A}} }{\eberelation{\P}{\gax}{v_2}{v_2'}{\M{\e_2}{B}}}{\gpeberelation{\doin{x}{v_1}{v_2}}{\doin{x}{v_1'}{v_2'}}{\M{\e_1 \dot \e_2}{B}}} 
        $}\]
    
        \[\scalebox{0.8}{$
            \ntreeruleIII{\eqif}{\gpeberelation{v}{v'}{\B} }{ \gpeberelation{v_1}{v_1'}{A}}{\gpeberelation{v_2}{v_2'}{A}}{\gpeberelation{\pifthenelse{A}{v}{v_1}{v_2}}{\pifthenelse{A}{v'}{v_1'}{v_2'}}{A}}
            \quad    
            \ntreeruleII{\eqsubtype}{\gpeberelation{v}{v'}{A}}{A \subtypep B}{\gpeberelation{v}{v'}{B}}
        $}\]
   \end{framed}
    \caption{The congruence rules for PEC.}
    \label{BetaEtaCongruence}
\end{figure}

\begin{figure}[H]
    
    \begin{framed}
        \[
            \ntreeruleI{\eqreflexive}{\gpetyperelation{v}{A}}{\gpeberelation{v}{v}{A}}
            \quad
            \ntreeruleI{\eqsymmetric}{\gpeberelation{v_1}{v_2}{A}}{\gpeberelation{v_2}{v_1}{A}}
        \]
    
        \[
            \ntreeruleII{\eqtransitive}{\gpeberelation{v_1}{v_2}{A}}{\gpeberelation{v_2}{v_3}{A}}{\gpeberelation{v_1}{v_3}{A}}
        \]
    \end{framed}
    \caption{Rules expanding the reduction and congruence relation to an equivalence relation.}
    \label{BetaEtaEquivalence}
\end{figure}


Now we can state the soundness theorem. 


\begin{framed}
    \begin{theorem}[Soundness]\label{SOundness}
        If $\gpeberelation{v_1}{v_2}{A}$, then $\gpetyperelation{v_1}{A}$, $\gpetyperelation{v_2}{A}$, and $\deno{\gpetyperelation{v_1}{A}} = \deno{\gpetyperelation{v_2}{A}}$.
    \end{theorem}
    
    
    \begin{proof}
        The proof proceeds by induction on the definition of the equational equivalence relation.
    
        This proof has a lot of cases and each of the reduction cases makes use of many of the requirements we have placed on the indexed S-category.
    
        I have omitted the congruence cases here as they hold through simple application of the inductive hypothesis on subterms. This occurs because this denotational semantics is compositional, meaning that denotations of terms are defined entirely in terms of the denotations of their subexpression. Similarly, the equivalence relation cases hold simply because equality on morphisms is an equivalence relation by definition. I do, however, give a selection of the reduction cases to demonstrate the necessity of the S-category requirements.
    
    \case{\eqrightunit}
    This case makes use of the right-unit monad law. Let $f = \deno{\gpetyperelation{v}{\mea}}$.
        \begin{equation}
        \begin{split}
            \deno{\gpetyperelation{\doin{x}{v}{\return{x}}}{\mea}}  & = \bind{\e}{\1}{A} \after \T{\e}{(\point{A} \after \pp)} \after \tstrength{\e}{\G}{A}\after \pr{\idg}{f} \\
            & = \T{\e}{\pp} \after \tstrength{\e}{\G}{A} \after \pr {\idg}{f} \\
            & = \pp \after \pr{\idg}{f}\\
            & = f
        \end{split}
    \end{equation}
    
    \case{\eqiftrue}
    This case makes use of the co-product diagram on $\1 + \1$. Let $f = \deno{\gpetyperelation{v_1}{A}}$  and  $g = \deno{\gpetyperelation{v_2}{A}}$. Then we can simplify the denotation of the whole expression.
    
    \begin{equation}
        \begin{split}
            \deno{\gpetyperelation{\pifthenelse{A}{\t}{v_1}{v_2}}{A}} & = \ifMorph{\inl\after\term{\G}}{f}{g} \\
            & = \app\after((\cur{f\after\pp}\after\term{\G})\times\idg)\after\diag{\G}\\
            & = \app\after(\cur{f\after\pp}\times\idg)\after(\term{\G}\times\idg)\after\diag{\G}\\
            & = f\after\pp\after\pr{\term{\G}}{\idg}\\
            & = f \\
            & = \deno{\gpetyperelation{v_1}{A}}\\
        \end{split}
    \end{equation}
    
    \case{\eqeffbeta}
    This case makes use of the adjunction properties of $\allEn, \pstar$. Let  $h = \deno{\typerelation{\P}{\e}{\effect}}$, $f = \deno{\etyperelation{\P,\a}{\G}{v}{A}}$, and $A = \deno{\typerelation{\P,\a}{A\ssub{\a}{\a}}{\type}}$. Then we can find the sub-term denotation $\deno{\gpetyperelation{\elam{\a}{v}}{\all{\a}{A}}} = \bar{f}$, which allows us to express the denotation of the whole expression.
    
    \begin{align*}
        \deno{\gpetyperelation{\eapp{(\elam{\a}{v})}{\e}}{\all{\a}{A}}} & = \pr{\Id{E^n}}{h}\star(\counit{A})\after\bar{f}\\
        & = \pr{\Id{E^n}}{h}\star(\counit{A})\after\pr{\Id{E^n}}{h}\star(\pstar(\bar{f}))\qt{Identity functor}\\
        &= \pr{\Id{E^n}}{h}\star(\counit{A}\after\pstar(\bar{f}))\\
        &= \pr{\Id{E^n}}{h}\star(f)\qt{By adjunction}\\
        &= \deno{\gpetyperelation{v\ssub{\a}{\e}}{A\ssub{\a}{e}}}\qt{By substitution theorem}\\
    \end{align*}
    
    \case{\eqeffeta}
        Let $f  = \deno{\gpetyperelation{v}{\all{\a}{A}}}$, and $A  = \deno{\typerelation{\P,\a}{A}{\type}}$. We can now express the denotation of the entire expression.
    
        \begin{align*}
            \deno{\gpetyperelation{\elam{\a}{(\eapp{v}{\a})}}{\all{\a}{A}}} & = \bar{\deno{\etyperelation{\P,\a}{\G}{\eapp{v}{\a}}{A}}} \\
            & = \bar{\pr{\Id{E^n\times E}}{\pp}\star(\e_{\deno{\typerelation{\P,\a,\b}{A\ssub{\a}{\b}}{\type}}})\after\pstar(f)}
        \end{align*}
    
        Let us look at $\deno{\typerelation{\P,\a,\b}{A\ssub{\a}{\b}}{\type}}$. We have the weakening $\wrel{\i\pi\x}{\P,\a,\b}{\P,\b}$, so by the weakening theorem on type denotations, we can re-arrange the quantification and projection functors.
    
        \begin{align*}
            \allIU(\deno{\typerelation{\P,\a,\b}{A\ssub{\a}{\b}}{\type}}) & = \allEn( (\p\times \Id{E})\star\deno{\typerelation{\P,\b}{A\ssub{\a}{\b}}{\type}})\\ 
            & = \pstar\allEn(\deno{\typerelation{\P,\b}{A\ssub{\a}{\b}}{\type}})
        \end{align*}
    
        Using this and the adjunction and naturality properties (\ref{NaturalityCondition}), we can unwind the definition of the co-unit on $\deno{\typerelation{\P,\a,\b}{A\ssub{\a}{\b}}{\type}}$.
    
    
        \begin{align*}
            \counit{\deno{\typerelation{\P,\a,\b}{A\ssub{\a}{\b}}{\type}}} 
            & = \widehat{\Id{\allIU(\deno{\typerelation{\P,\a,\b}{A\ssub{\a}{\b}}{\type}})}}\\
            & = \widehat{\Id{\pstar\allEn(\deno{\typerelation{\P,\b}{A\ssub{\a}{\b}}{\type}})}}\\
            & =\widehat{\Id{\pstar\allEn A}}\\
            & = \widehat{\pstar(\Id{\allEn A})}\\
            & = \widehat{\pstar(\bar{\counit{A}})}\\
            & = \widehat{\bar{(\p\times\Id{E})\star(\counit{A})}}\\
            & = (\p\times\Id{E})\star(\counit{A}) 
        \end{align*}
    
        This rearrangement can now be used in conjunction with the contravariant composition of re-indexing functors to show the denotation $\deno{\gpetyperelation{\elam{\a}{(\eapp{v}{\a})}}{\all{\a}{A}}}$ is equal to the morphism $f$.
        \begin{align*}
            \deno{\gpetyperelation{\elam{\a}{(\eapp{v}{\a})}}{\all{\a}{A}}} & = \bar{\pr{\Id{E^n\times E}}{\pp}\star(\counit{\deno{\typerelation{\P,\a,\b}{A\ssub{\a}{\b}}{\type}}})\after\pstar(f)}\\
            & = \bar{\pr{\Id{E^n\times E}}{\pp}\star((\p\times\Id{E})\star(\counit{A}))\after\pstar(f)}\\
            & = \bar{\pr{\p}{\pp}\star(\counit{A})\after\pstar(f)}\\
            &= \bar{\Id{E^n\times E}\star(\counit{A})\after\pstar(f)} \\
            & = \bar{\counit{A}\after \pstar(f)}\qt{The identity re-indexing functor is the identity.}\\
            & = f \qt{By adjunction} & \square
        \end{align*}
    \end{proof}
\end{framed}

This completes the proof of soundness for this semantics.

\section{Summary}

In this chapter, I have introduced the notion of an indexed S-category structure, given a denotational semantics for PEC in terms of  this structure, and proved that these semantics soundly model PEC. However, I have not shown whether it is possible to construct an indexed S-category, or how complex such categories might be. These questions will be addressed in the next chapter.
