\documentclass{report}

%% Don't import the header multiple times

\ifdefined\HEADERIMPORTED
\else
\newcommand\HEADERIMPORTED[0]{This file is HEADERIMPORTED}
\usepackage{amssymb}

\usepackage{amsmath}
\usepackage[a4paper,includeheadfoot,margin=2.54cm]{geometry}
\usepackage{breqn}
\usepackage{amssymb}

\usepackage{amsmath}
\usepackage[a4paper,includeheadfoot,margin=2.54cm]{geometry}
\usepackage{breqn}
\usepackage{array}   % for \newcolumntype macro
\usepackage{tikz-cd}
\usepackage{tabstackengine}
\setstackEOL{\cr}
\setstackgap{L}{\normalbaselineskip}

\newcommand\todo[1]{\textbf{TODO: #1}}

\newcommand{\s}{\;}
\newcommand{\doin}[3]{\texttt{do}\s #1 \leftarrow #2 \s\texttt{in}\s #3}
\newcommand\apply[2]{#1\s#2}
\newcommand\ifthenelse[5]{\texttt{if}_{#1, #2}\s#3\s \texttt{then}\s #4 \s\texttt{else} \s#5}
\newcommand\const[1]{\texttt{C}^{#1}}
\newcommand\return[1]{\texttt{return} #1}
\newcommand\lam[3]{\lambda #1 : #2. #3}
\renewcommand\u[0]{\texttt{()}}
\newcommand{\U}[0]{\texttt{Unit}}
\renewcommand\t[0]{\texttt{true}}
\newcommand\f[0]{\texttt{false}}
\newcommand{\B}[0]{\texttt{Bool}}
\newcommand{\G}[0]{\Gamma}
\newcommand\D{\Delta}


% draw type relations
\newcommand{\typerelation}[3]{#1 \vdash #2 \colon #3}
\newcommand{\gtyperelation}[2]{\typerelation{\G}{#1}{#2}}

%draw tree rules
\newcommand{\treerule}[3]{(\text{#1}) \frac{#2}{#3}}
\newcommand{\condtreerule}[4]{\treerule{#1}{#2}{#3}(\text{if } #4)}

\newcommand{\subtype}[0]{\leq:}
\newcommand\subeffect[0]{\leq}

\newcommand{\M}[2]{\texttt{M}_{#1}{#2}}

\newcommand\lamtype[3]{#1 \rightarrow \M{#2}{#3}}
\newcommand{\1}[0]{\texttt{1}}

\newcommand\e[0]{\epsilon}

\newcommand{\db}[1]{{\bf [\![}#1{\bf ]\!]}}
\newcommand{\deno}[1]{\db{#1}_M}
\newcommand\after\circ
\newcommand\term[1]{\left\langle\right\rangle_{#1}}

\newcommand\point[1]{\eta_{#1}}
\newcommand\bind[3]{\mu_{#1, #2, #3}}

\newcommand\T[2]{T_{#1}{#2}}

\newcommand\pr[2]{\left\langle#1, #2\right\rangle}

% tensor strength Nat-tran
\newcommand\tstrength[3]{\texttt{t}_{#1, #2, #3}}

% Id morphism
\newcommand\Id[1]{\texttt{Id}_{#1}}

\newcommand\idg[0]{\Id{\G}}
% beta-eta equivalence
\newcommand\beequiv[0]{=_{\beta\eta}}
% Substitutions
\newcommand\si{\sigma}

\newcommand{\sub}[1]{\left[#1\right]}
\newcommand{\ssub}[2]{\left[#2 / #1\right]}
\newcommand{\ssi}[0]{\sub{\si}}

% beta-eta equivalence relation
\newcommand{\berelation}[4]{\typerelation{#1}{#2 \beequiv #3}{#4}}
\newcommand{\gberelation}[3]{\gtyperelation{#1 \beequiv #2}{#3}}


% Shortcuts for denotational equality
\newcommand{\denoequality}[4]{\deno{\typerelation{#1}{#2}{#4}} = \deno{\typerelation{#1}{#3}{#4}}}
\newcommand{\gdenoequality}[3]{\denoequality{\G}{#1}{#2}{#3}}

% Shorthand for monad types
\newcommand\mea[0]{\M{\e}{A}}
\newcommand\meb[0]{\M{\e}{B}}
\newcommand\mec[0]{\M{\e}{C}}

\newcommand\tea[0]{\T{\e}{A}}
\newcommand\teb[0]{\T{\e}{B}}
\newcommand\tec[0]{\T{\e}{C}}


\newcommand\moa[0]{\M{\1}{A}}
\newcommand\mob[0]{\M{\1}{B}}
\newcommand\moc[0]{\M{\1}{C}}

\newcommand\toa[0]{\T{\1}{A}}
\newcommand\tob[0]{\T{\1}{B}}
\newcommand\toc[0]{\T{\1}{C}}

\newcommand\aeb[0]{\lamtype{A}{\e}{B}}

% Shorthand for Gammas
\newcommand{\gax}[0]{\G, x: A}
\newcommand{\gby}[0]{\G, y: B}

% reduction function
\newcommand{\reduce}[0]{reduce}



% Combinators for building delta-based tree proof terms
\newcommand{\deltavrule}[4]{
    \treerule{Subtype}{\treerule{}{\D}{\typerelation{#1}{#2}{#3}}\s\s #3 \subtype #4}{\typerelation{#1}{#2}{#4}}}

\newcommand{\deltavruleprime}[4]{
    \treerule{Subtype}{\treerule{}{\D'}{\typerelation{#1}{#2}{#3}}\s\s #3 \subtype #4}{\typerelation{#1}{#2}{#4}}}

\newcommand{\deltavruleprimeprime}[4]{
        \treerule{Subtype}{\treerule{}{\D'}{\typerelation{#1}{#2}{#3}}\s\s #3 \subtype #4}{\typerelation{#1}{#2}{#4}}}
    
\newcommand{\deltacrule}[6]{
            \treerule{Subeffect}{\treerule{}{\D}{\typerelation{#1}{#2}{\M{#3}{#4}}}\s\s #4 \subtype #6\s\s #3 \subeffect #5}{\typerelation{#1}{#2}{\M{#5}{#6}}}}
\newcommand{\deltacruleprime}[6]{
    \treerule{Subeffect}{\treerule{}{\D'}{\typerelation{#1}{#2}{\M{#3}{#4}}}\s\s #4 \subtype #6\s\s #3 \subeffect #5}{\typerelation{#1}{#2}{\M{#5}{#6}}}}
\newcommand{\deltacruleprimeprime}[6]{
    \treerule{Subeffect}{\treerule{}{\D''}{\typerelation{#1}{#2}{\M{#3}{#4}}}\s\s #4 \subtype #6\s\s #3 \subeffect #5}{\typerelation{#1}{#2}{\M{#5}{#6}}}}
                            

\newcommand{\p}[0]{\pi_1}
\newcommand{\pp}[0]{\pi_2}

% short-hands for weakening
\newcommand{\wrel}[3]{#1 : #2 \triangleright #3}
\newcommand{\ok}[1]{#1 \texttt{Ok}}
\renewcommand\i[0]{\iota}
\newcommand\w{\omega}
\newcommand\dom[1]{\texttt{dom}(#1)}
\newcommand\x{\times}


% Combinators to build tree proofs
\newcommand{\truleconst}[0]{\treerule{Const}{\ok{\G}}{\gtyperelation{\const{A}}{A}}}
\newcommand{\truleunit}[0]{\treerule{Unit}{\ok{\G}}{\typerelation{\G}{\u}{\U}}}
\newcommand{\truletrue}[0]{\treerule{True}{\ok{\G}}{\typerelation{\G}{\t}{\B}}}
\newcommand{\trulefalse}[0]{\treerule{False}{\ok{\G}}{\typerelation{\G}{\f}{\B}}}


\newcommand{\E}[0]{\mathbb{E}}
\renewcommand{\dot}{\cdot}
\newcommand{\gens}[0]{::=}
\newcommand{\nil}[0]{\diamond}
\newcommand{\ground}[0]{\gamma}

% Terminal object of C
\newcommand{\terminal}[0]{\texttt{\1}}

% The category C
\newcommand{\C}[0]{\mathbb{C}}

% The category of locally-small categories
\newcommand{\Cat}[0]{\texttt{Cat}}
% Sub-effect Nat-trans
\newcommand{\dse}[2]{\db{#1 \subeffect #2}}

\newcommand\app[0]{\texttt{app}}
\newcommand\cur[1]{\texttt{cur}(#1)}
\newcommand{\ifnt}[1]{\texttt{If}_{#1}}


\newcommand{\setto}{:=}
\newcommand{\fv}[1]{\texttt{fv}(#1)}

% shorthand for inserting text to equations
\newcommand\qt[1]{\quad\text{#1}}


\fi
\newcommand{\allI}[0]{\forall_I}
\newcommand{\allII}[0]{\forall_{I'}}
\newcommand\type[0]{\texttt{Type}}
\newcommand\effect[0]{\texttt{Effect}}
\newcommand\ciw[0]{\C(I, W)}
\newcommand\ciu[0]{\C(I, U)}
\newcommand\ciuw[0]{\C(I\times U, W)}
\newcommand\cipw[0]{\C(I', W)}
\newcommand\cipu[0]{\C(I', U)}
\newcommand\cii[0]{\C(I', I)}
\newcommand\Eff[0]{\texttt{Eff}}
\newcommand\Mul[0]{\texttt{Mul}}
\renewcommand\star[0]{^*}
\renewcommand\bar[1]{\overline{#1}}

\newcommand\subtypeg[0]{\subtype_\g}
\newcommand\subtypepa[0]{\subtype_{\P, \a}}
\newcommand\subtypeppa[0]{\subtype_{\P', \a}}

\usepackage{scalerel,stackengine}
\stackMath
\renewcommand\widehat[1]{%
\savestack{\tmpbox}{\stretchto{%
  \scaleto{%
    \scalerel*[\widthof{\ensuremath{#1}}]{\kern.1pt\mathchar"0362\kern.1pt}%
    {\rule{0ex}{\textheight}}%WIDTH-LIMITED CIRCUMFLEX
  }{\textheight}% 
}{2.4ex}}%
\stackon[-6.9pt]{#1}{\tmpbox}%
}
\parskip 1ex

\newcommand\pstar[0]{\p\star}

\begin{document}

\tableofcontents
\chapter{Preliminaries}
\section{Base Category Requirements}
There are 3 distinct objects in the base category, $\C$:

\begin{itemize}
    \item $U$ - The kind of \effect
    \item $W$ - The kind of \type
    \item $\1$ - A terminal object
\end{itemize}

And we have finite products on $U$.

\begin{itemize}
    \item $U^0 = \1$
    \item $U^{n+1} = U^n \times U$
\end{itemize}

We also have the following natural operations on morphisms in $\C$.

Let $I = U^n$.

\begin{itemize}
    \item $\diamond: \ciw \times \ciw \rightarrow \ciw$ - Generates exponential types.
    \item $\square: \ciw\times\ciw\rightarrow\ciw$ - Generates products of types.
    \item $\allI: \ciuw\rightarrow\ciw$ - generates quantified types.
    \item $\Eff:\ciu\times\ciw\rightarrow\ciw$ - generates monad types.
    \item $\Mul:\ciu\times\ciu\rightarrow\ciu$ - Generates multiplication of effects.
\end{itemize}

With naturality conditions which mean, for $\theta: \U^m \rightarrow\U^n (I' \rightarrow I)$,
\begin{itemize}
    \item $\diamond(\phi,\psi)\after\theta = \diamond(\phi\after\theta,\psi\after\theta)$
    \item $\square(\phi,\psi)\after\theta = \square(\phi\after\theta,\psi\after\theta)$
    \item $\allI(\phi)\after\theta = \allII(\phi\after(\theta\times\Id{U}))$
    \item $\Eff(\phi,\psi)\after\theta = \Eff(\phi\after\theta,\psi\after\theta)$
    \item $\Mul(\phi,\psi)\after\theta = \Mul(\phi\after\theta,\psi\after\theta)$
\end{itemize}
\section{Well-Formed-ness}

Each instance of the well-formed-ness relation on effects, $\wellformed{\P}{\e}$ has a denotation in $\C$: \begin{equation}
    \deno{\typerelation{\P}{\e}{\effect}}: I \rightarrow U
\end{equation}


Each instance of the well-formed-ness relation on types, $\wellformed{\P}{A}$ has a denotation in $\C$:

\begin{equation}
    \deno{\typerelation{P}{A}{\type}}: I \rightarrow W
\end{equation}

It should also be the case that \begin{equation}
    \Mul(\deno{\typerelation{\P}{\e_1}{\effect}}, \deno{\typerelation{\P}{\e_2}{\effect}}) = \deno{\typerelation{\P}{\e_1\dot\e_2}{\effect}} \in \ciu
\end{equation}

That is, $\Mul$ should be have identity $\deno{\typerelation{\P}{\1}{\effect}}$ and be associative.
\section{Substitution and Weakening of the Effect Environment}

For each instance of the well-formed-ness relation on substitution of effects $\typerelation{\P'}{\si}{\P}$, there exists a denotation in $\C$:

\begin{equation}
    \deno{\typerelation{\P'}{\si}{\P}}: I'\rightarrow I
\end{equation}

For each instance of the well-formed weakening relation on effect-environments, $\wrelw{\P'}{\P}$
 there exists a denotation in $\C$:

 \begin{equation}
     \deno{\wrelw{\P'}{\P}}: I'\rightarrow I
 \end{equation}.

\section{Fibre Categories}
Each set of morphisms $\ciw$ forms the objects of a semantic-closed (S-closed) category. That is, a category satisfying all the properties needed for the non-polymorphic language:

\begin{itemize}
    \item Cartesian Closed
    \item Co-product of the terminal object with itself ($\1 + \1$)
    \item Ground morphisms for each ground constant ($\const{A}: \1\rightarrow A$)
    \item Partial order morphisms on ground types ($\deno{A\subtypeg} B$)
    \item A strong, monad, graded by the po-monoid $(E_\P, \dot_\P, \subeffectp, \1)$.
\end{itemize}

\section{Re-indexing Functors}

For each morphism $f: I' \rightarrow I$ in $\C$, there should be a co-variant, re-indexing functor  $f\star: \ciw \rightarrow \cipw$, which is S-closed. That is, it preserves the S-closed properties of $\ciw$. (See below).

$(-)\star$ should be a contra-variant functor in its $\C$ argument and co-variant in its right argument.

\begin{itemize}
    \item $(g\after f)\star(a) = f\star(\g\star(a))$
    \item $\Id{I}\star(a) = a$
    \item $f\star(\Id{A}) = \Id{f\star(A)}$
    \item $f\star(a\after b) = f\star(a)\after f\star(b)$
\end{itemize}

\subsection{$f\star$ Preserves Products}
If $\pr{g}{h}:\ciw(Z, X\times Y)$
Then 
\begin{align}
    f\star(X\times Y) & = f\star(X)\times f\star(Y)\\
    f\star(\pr{g}{h}) & = \pr{f\star(g)}{f\star{h}}&:\cipw(f\star Z, f\star(X)\times f\star(Y))\\
    f\star(\p) & = \p&:\cipw(f\star(X)\times f\star(Y), f\star(X)) \\
    f\star(\pp) &= \pp&:\cipw(f\star(X)\times f\star(Y), f\star(Y))
\end{align}

\subsection{$f\star$ Preserves Terminal Object}
If $\Id{A}:\ciw(A, \1)$
Then 
\begin{align}
    f\star(\1) & = \1 \\
    f\star(\term{A}) & = \term{f\star(A)}&:\cipw(f\star A, \1)\\
\end{align}

\subsection{$f\star$ Preserves Exponentials}
\begin{align}
    f\star(Z^X) & = (f\star(Z))^{(f\star(X))}\\
     f\star(\app) &= \app&:\cipw(f\star(Z^X)\times f\star(X), f\star(Z))\\
     f\star(\cur{g}) &= \cur{f\star(g)}&:\cipw(f\star(Y)\times f\star(X), f\star(Z)^{f\star(X)})
\end{align}

\subsection{$f\star$ Preserves Co-product on Terminal}

\begin{align}
    f\star(\1+\1) &= \1+\1\\
    f\star(\inl)  &= \inl&:\cipw(\1, \1+\1) \\
    f\star(\inr) &= \inr&:\cipw(\1, \1+\1) \\
    f\star([g, h]) &= [f\star(g), f\star(h)]&:\cipw(\1+\1, f\star(Z))
\end{align}

\subsection{$f\star$ Preserves Graded Monad}
\begin{align}
    f\star(\tea) &= \T{f\star(\e)}{f\star(A)}&:\cipw\\
    f\star(\1) &= \1 \qt{The unit effect}\\
    f\star(\point{A}) &= \point{f\star(A)}&:\cipw(f\star(A), f\star(\toa))\\
    f\star(\bind{\e_1}{\e_2}{A}) &= \bind{f\star(\e_1)}{f\star(\e_2)}{f\star(A)}&:\cipw(f\star(\T{\e_1}{\T{\e_2}{A}}), f\star(\T{f\star(\e_1)\dot f\star(\e_2)}{f\star(A)}))\\
    f\star(\e_1\dot\e_2) &= f\star(\e_1)\dot f\star(\e_2)\\
\end{align}

\subsection{$f\star$ Preserves Tensor Strength}
\begin{align}
    f\star(\tstrength{\e}{A}{B}) &= \tstrength{f\star(\e)}{f\star(A)}{f\star(B)} &: \cipw(f\star(A\times\teb), f\star(\T{\e}{(A\times B)}))
\end{align}
\subsection{$f\star$ Preserves Ground Constants}
For each ground constant $\deno{\const{A}}$ in $\ciw$,

\begin{align}
    f\star(\deno{\const{A}}) = \const{f\star(A)} : \cipw(\1, f\star(A))
\end{align}
\subsection{$f\star$ Preserves Ground Sub-effecting}
For ground effects $e_1, e_2$ such that $e_1\subeffect e_2$



\begin{align}
    f\star(e) & = e: \cipu\\
    f\star\db{\e_1\subeffect e_2}_A = \db{e_1 \subeffect e_2}_{f\star(A)} &:\cipw{f\star(\T{e_1}{A}), f\star(\T{e_2}{A})} \\
\end{align}
\subsection{$f\star$ Preserves Ground Sub-typing}
For ground types $\g_1, \g_2$ such that $\g_1\subtypeg\g_2$

\begin{align}
    f\star{\g} = \g: \cipw(\1, \g)\\
    f\star(\deno{\g_1 \subtypeg \g_2}) & = \deno{\g_1 \subtypeg \g_2} &: \cipw(\g_1, \g_2)\\
\end{align}

\subsection{Action on Objects}

It follows that the action of $f\star$ on an object $A$ in $\ciw$ (i.e. a morphism $I \rightarrow U$ in $\C$) is:

\begin{equation}
    f\star(A) = A\after f: I'\rightarrow I\rightarrow W
\end{equation}

\section{Naturality Properties}

\section{The $\allI$ functor}
We expand $\allI: \ciuw \rightarrow \ciw$ to be a functor which is right adjoint to the re-indexing functor $\pstar$.

\begin{equation}
    \bar{(\_)}: \ciuw(\pstar A, B) \leftrightarrow \ciw(A, \allI B) : \widehat{(\_)}
\end{equation}

For $A: \ciw$, $B: \ciuw$.

Hence the action of $\allI$ on a morphism $l : A\rightarrow A'$ is as follows:
\begin{eqnarray}
    \allI(l) = \bar{l\after\e_A}
\end{eqnarray}
Where $\e_A: \ciuw(\pstar\allI A \rightarrow A)$ is the co-unit of the adjunction.

\section{Naturality Corollaries}
Here are some simple corollaries of the adjunction between $\pstar$ and $\allI$.

\subsection{Naturality}
By the definition of an adjunction:

\begin{equation}
    \bar{f\after\pstar(n)} = \bar{f}\after n
\end{equation}

\subsection{$\bar{(-)}$ and Re-indexing Functors}

\todo{Why does this occur? it comes from page 222 of Crole?}
\begin{align}
    \theta\star(\bar{f}) & = (\p\after(\theta\times\Id{U}))\star(\bar{f})\\\
     &= (\theta\times\Id{U})\star(\p\star(\bar{f}))\\
     \\
     \\
     &= \bar{(\theta\times\Id{U})\star f}\\
     \\
\end{align}

\subsection{$\hat{(-)}$ and Re-Indexing Functors}
\begin{align}
    \theta\star(\pr{\Id{I}}{\rho}\star(\widehat{m})) &= (\pr{\Id{I}}{\rho}\after\theta)\star(\widehat{m})\\
    & = ((\theta\times\Id{U})\after\pr{\Id{I}}{\rho})\star(\widehat{m})\\
    & = \pr{\Id{I}}{\rho\after\theta}\star(\theta\times\Id{U})\star(\widehat{m}) \\
    & = \pr{\Id{I}}{\theta\star\rho}\star(\theta\star(\widehat{m}))
\end{align}

\subsection{Pushing Morphisms into $f\star$}

\begin{align}
    \pr{\Id{I}}{\rho}\star(\widehat{m})\after n &= \pr{\Id{I}}{\rho}\star(\widehat{m})\after\pr{\Id{I}}{\rho}\star\p\star(n)\\
    & = \pr{\Id{I}}{\rho}\star(\widehat{m}\after\pstar(n))\\
    &= \pr{\Id{I}}{\rho}\star(\widehat{m\after n})
\end{align}

\chapter{Denotations}
\section{Effects}
For each instance of the well-formed-ness relation on effects, we define a morphism $\deno{\typerelation{\P}{\e}{\effect}}: \ciu$

\begin{itemize}
    \item $\deno{\typerelation{\P}{e}{\effect}} = \deno{\e} \after \term{I}: \rightarrow U$
    \item $\deno{\typerelation{\P,\a}{\a}{\effect}} = \pp: I\times U \rightarrow U$
    
    \item $\deno{\typerelation{\P, \b}{\a}{\effect}} = \deno{\typerelation{\P}{\a}{\effect}}\after\p: I\times U\rightarrow U$
    
    \item $\deno{\typerelation{\P}{\e_1\dot \e_2}{\effect}} = \Mul(\deno{\typerelation{\P}{\e_2}{\effect}},\deno{\typerelation{\P}{\e_1}{\effect}}): I \rightarrow U$
\end{itemize}
\section{Types}
For each instance of the well-formed-ness relation on types, we define a morphism $\deno{\typerelation{\P}{A}{\type}}: \ciw$.

$\deno{\U}$ is the morphism generating the terminal object of $\ciw$. $\B$ is the morphism generating the co-product of this terminal object, $\1 + \1$.
\begin{itemize}
    \item $\deno{\typerelation{\P}{\U}{\type}} = \deno{\U}\after\term{I}: I \rightarrow W$
    
    \item $\deno{\typerelation{\P}{\B}{\type}} = \deno{\B}\after\term{I}: I \rightarrow W$
    
    \item $\deno{\typerelation{\P}{\g}{\type}} = \deno{\g}\after\term{I} : I \rightarrow W$
    
    \item $\deno{\typerelation{\P}{\ab}{\type}} = \diamond(\deno{\typerelation{\P}{A}{\type}},\deno{\typerelation{\P}{B}{\type}}): I \rightarrow W$
    
    \item $\deno{\typerelation{\P}{\mea}{\type}} =\Eff(\deno{\typerelation{\P}{\e}{\effect}},\deno{\typerelation{\P}{A}{\type}}): I \rightarrow W$
    \item $\deno{\typerelation{\P}{\all{\a}{A}}{\type}} =\allI(\deno{\typerelation{\P,\a}{A}{\type}}): I \rightarrow W$
\end{itemize}

\section{Effect Substitution}

For each effect-substitution well-formed-ness-relation, define a denotation morphism, $\deno{\typerelation{\P'}{\si}{\P}}: \cii$

\begin{itemize}
    \item $\deno{\typerelation{\P'}{\nil}{\nil}} = \term{I}: \C(I', \1)$
    \item $\deno{\typerelation{\P'}{(\si, \a\setto\e)}{\P,\a}} = \pr{\deno{\typerelation{\P'}{\si}{\P}}}{\deno{\typerelation{\P}{\e}{\effect}}}: \C(I', I\times U)$
\end{itemize}
\section{Effect Weakening}

For each instance of the effect-environment weakening relation, define a denotation morphism: $\deno{\wrelw{\P'}{P}}: \cii$

\begin{itemize}
    \item $\deno{\wrel{\i}{\P}{\P}} = \Id{I}: I \rightarrow I$
    \item $\deno{\wrel{w\pi}{\P',\a}{\P}} = \deno{\wrelw{\P'}{\P}}\after \p: I'\times U\rightarrow I$
    \item $\deno{\wrel{w\x}{\P',\a}{\P,\a}} = (\deno{\wrelw{\P'}{\P}}\times \Id{U}): I'\times U\rightarrow I\times U$
\end{itemize}
\section{Sub-Typing}
For each instance of the sub-typing relation with respect  to an effect environment, there exists a denotation, $\deno{A\subtypep B}: \ciw(A, B)$.

\begin{itemize}
    \item $\deno{\g_1\subtypep \g_2} = \deno{\g_1\subtypeg \g_2} : \ciw(\g_1, \g_2)$
    \item $\deno{\ab \subtypep \fntype{A'}{B'}} = \deno{B\subtypep B'}^{A'}\after B^{\deno{A'\subtypep A}}$
    \item $\deno{\M{\e_1}{A}\subtypep\M{\e_2}{B}} = \deno{\e_1\subeffectp\e_2}\after\T{\e_1}{\deno{A\subtypep B}}$
    \item $\deno{\all{\a}{A}\subtypep\all{\a}{B}} = \allI{\deno{A\subtypepa B}}$
\end{itemize}
\section{Type-Environments}

For each instance of the well-formed relation on type environments, define an object in $\deno{\wellformedok{I}{W}}\in\ciw$.

\begin{itemize}
    \item $\deno{\wellformedok{\P}{\nil}} = \1: \ciw$
    \item $\deno{\wellformedok{\P}{\gax}} = \square(\deno{\wellformedok{\P}{\G}}, \deno{\typerelation{\P}{A}{\type}})$
\end{itemize}

\section{Terms}
For each instance of the typing relation, define a denotation morphism: $\deno{\gpetyperelation{v}{A}}: \ciw(\G_I, A_I)$. Writing $\G_I$ and $A_I$ for $\deno{\wellformedok{\P}{\G}}$ and $\deno{\typerelation{\P}{A}{\type}}$.

For each ground constant, $\const{A}$, there exists $c: \1 \rightarrow A_I$ in $\ciw$.

\begin{itemize}
    \item $\treerule{Unit}{\wellformedok{\P}{\G}}{\deno{\etyperelation{\P}{\G}{\u}{\U}} = \term{\G} : \G_I \rightarrow \1}$
        
    \item $\treerule{Const}{\wellformedok{\P}{\G}}{\deno{\etyperelation{\P}{\G}{\const{A}}{A}} = \deno{\const{A}} \after \term{\G} : \G \rightarrow \deno{A}}$
         
    \item $\treerule{True}{\wellformedok{\P}{\G}}{\deno{\etyperelation{\P}{\G}{\t}{\B}} = \inl \after \term{\G} : \G \rightarrow \deno{\B} = \1+\1}$
        
    \item $\treerule{False}{\wellformedok{\P}{\G}}{\deno{\etyperelation{\P}{\G}{\f}{\B}} = \inr \after \term{\G} : \G \rightarrow \deno{\B} = \1+\1}$
        
    \item $\treerule{Var}{\wellformedok{\P}{\G}}{\deno{\etyperelation{\P}{\gax}{x}{A}} = \pp: \G \times A \rightarrow A}$
    \item $\treerule{Weaken}{f = \deno{\gpetyperelation{x}{A}}: \G \rightarrow A}{\deno{\etyperelation{\P}{\gby}{x}{A}} = f \after \p: \G \times B \rightarrow A}$
    \item $\treerule{Lambda}{f = \deno{\etyperelation{\P}{\gax}{C}{\meb}} : \G \times A \rightarrow \teb}{\deno{\etyperelation{\P}{\G}{\lam{x}{A}{C}}{A \rightarrow \meb}} = \cur{f} : \G \rightarrow (\teb)^A}$
    
    \item $\treerule{Subtype}{f = \deno{\etyperelation{\P}{\G}{v}{A}} : \G \rightarrow A\s\s g = \deno{A \subtype B}}{\deno{\etyperelation{\P}{\G}{v}{B}} = g \after f : \G \rightarrow B}$
  
    \item $\treerule{Return}{f = \deno{\etyperelation{\P}{\G}{v}{A}}}{\deno{\etyperelation{\P}{\G}{\return{v}}{\moa}} = \point{A} \after f}$
        
 
    \item $\treerule{If}{f = \deno{\etyperelation{\P}{\G}{v}{\B}}: \G\rightarrow\1+\1 \s\s g = \deno{\etyperelation{\P}{\G}{C_1}{\mea}}\s\s h = \deno{\etyperelation{\P}{\G}{C_2}{\mea}}}{\deno{{\etyperelation{\P}{\G}{\ifthenelse{\e}{A}{v}{C_1}{C_2}}{\mea}}} = \app\after((\fld{\cur{g\after\pp}}{\cur{h\after\pp}}\after f)\times \idg)\after \diag{\G} : \G \rightarrow \tea}$
        
    \item $\treerule{Bind}{f = \deno{\etyperelation{\P}{\G}{C_1}{\M{\e_1}{A}} : \G \rightarrow \T{\e_1}{A}\s\s g = \deno{\etyperelation{\P}{\gax}{C_2}{\M{\e_2}{B}}}}: \G \times A \rightarrow \T{\e_2}{B}}{\deno{\etyperelation{\P}{\G}{\doin{x}{C_1}{C_2}}{\M{\e_1 \dot \e_2}}} = \bind{\e_1}{\e_2}{B} \after \T{\e_1}{g} \after \tstrength{\G}{A}{\e_1} \after \pr{\idg}{f}: \G \rightarrow \T{\e_1 \dot \e_2}{B}}$ 
   
    \item $\treerule{Apply}{f = \deno{\gpetyperelation{v_1}{\lamtype{A}{\e}B}}: \G \rightarrow (\teb)^{A} \s\s g=\deno{\gpetyperelation{v_2}{A}}: \G \rightarrow A}{\deno{\gpetyperelation{\apply{v_1}{v_2}}{\meb}}= \app\after\pr{f}{g}: \G \rightarrow \teb }$
    \item $\treerule{Effect-Lambda}{f = \deno{\etyperelation{\P,\a}{\G}{v}{A}}: \ciuw(\G, A)}{\deno{\gpetyperelation{\elam{\a}{A}}{\all{\e}{A}}} = \bar{f}: \ciw(\G, \allI(A))}$
    \item $\treerule{Effect-App}{g=\deno{\gpetyperelation{v}{\all{\a}{A}}}: \ciw(\G, \allI(A))\s\s h = \deno{\typerelation{\P}{\e}{\effect}}: \ciu}{\deno{\gpetyperelation{\eapp{v}{\e}}{A\ssub{\a}{\e}}} = \pr{\Id{I}}{h}\star(\e_{\deno{\typerelation{\P,\b}{A\ssub{\a}{\b}}{\type}}})\after g: \ciw(\G, A\ssub{\a}{\e})}$
\end{itemize}

\chapter{Effect Substitution Theorem}
In this section, we state and prove a theorem that the action of a simultaneous effect-variable substitution upon a structure in the language has a consistent effect upon the denotation of the language. More formally, for the denotation morphism $\D$ of some relation, the denotation of the substituted relation, $\D' = \si\star(\D)$.
\section{Effects}
If $\si = \deno{\typerelation{\P'}{\si}{\P}}$ then $\deno{\typerelation{\P'}{\si(\e)}{\effect}} = \si\star\deno{\typerelation{\P}{\e}{\effect}} = \deno{\typerelation{\P}{\e}{\effect}}\after\si$.

\proof
By induction on the derivation on $\deno{\typerelation{\P}{\e}{\effect}}$

\case{Ground}
\begin{align}
    \deno{\typerelation{\P}{e}{\effect}}\after\si & = \deno{e}\after\term{I}\after\si \\
    & = \deno{e}\after\term{I'} \\
    & = \deno{\typerelation{\P'}{e}{\type}}\\
\end{align}

\case{Var}
\begin{align}
    \deno{\typerelation{\P,\a}{\a}{\effect}}\after\si' &= \pp\after\pr{\si}{\deno{\typerelation{\P'}{\e}{\effect}}}\qt{By inversion $\si' = (\si, \a\setto\e)$}\\
    & =\deno{\typerelation{\P'}{\e}{\effect}} \\
    &= \deno{\typerelation{\P'}{\si'(\a)}{\effect}}\\
\end{align}

\case{Weaken}
\begin{align}
    \deno{\typerelation{\P, \b}{\a}{\type}}\after\si' &= \deno{\typerelation{\P}{\a}{\type}} \after \p\after\pr{\si}{\deno{\typerelation{\P'}{\e}{\effect}}}\qt{By inversion, $\si' = (\si, \b\setto\e)$}\\
    & = \deno{\typerelation{\P}{\a}{\type}}\after\si\\
    & = \deno{\typerelation{\P'}{\si(\a)}{\type}}\\
    & = \deno{\typerelation{\P'}{\si'(\a)}{\type}}\\
\end{align}

\case{Multiply}
\begin{align}
    \deno{\typerelation{\P}{\e_1\dot\e_2}{\type}} \after\si &=
    \Mul(\deno{\typerelation{\P}{\e_1}{\effect}}, \deno{\typerelation{\P}{\e_2}{\effect}})\after \si \\
    & = \Mul(\deno{\typerelation{\P}{\e_1}{\effect}}\after \si, \deno{\typerelation{\P}{\e_2}{\effect}}\after \si)\qt{By Naturality}\\
    & = \Mul(\deno{\typerelation{\P'}{\si(\e_1)}{\effect}}, \deno{\typerelation{\P}{\si(\e_2)}{\effect}})\\
\end{align}

\section{Types}
If $\si = \deno{\typerelation{\P'}{\si}{\P}}$ then $\deno{\typerelation{\P'}{A\ssi}{\type}} = \si\star\deno{\typerelation{\P}{A}{\type}} = \deno{\typerelation{\P}{A}{\type}}\after\si$.

\proof
By induction on the derivation on $\deno{\typerelation{\P}{A}{\type}}$. Making use of naturality properties of the type constructors.

\case{Ground}
\begin{align}
    \deno{\typerelation{\P}{\g}{\type}}\after\si &= \deno{\g}\after\term{I}\after\si\\
    & =  \deno{\g}\after\term{I'}\\
    & = \deno{\typerelation{\P'}{\g}{\type}}\\
    & = \deno{\typerelation{\P'}{\g\ssi}{\type}}
\end{align}

\case{Monad}
\begin{align}
    \deno{\typerelation{\P}{\mea}{\type}}\after\si & =  \Eff(\deno{\typerelation{\P}{\e}{\effect}}, \deno{\typerelation{\P}{A}{\type}})\after\si \\
    & = \Eff(\deno{\typerelation{\P}{\e}{\effect}}\after\si, \deno{\typerelation{\P}{A}{\type}}\after\si) \qt{By naturality}\\
    & = \Eff(\deno{\typerelation{\P'}{\si(\e)}{\effect}}, \deno{\typerelation{\P'}{A\ssi}{\type}})\\
    & = \deno{\typerelation{\P'}{\M{\si(\e)}{A\ssi}}{\type}}\\
    & = \deno{\typerelation{\P'}{(\mea)\ssi}{\type}}
\end{align}
\case{Quantification}
    \begin{align}
        \deno{\typerelation{\P}{\all{\a}A}{\type}}\after\si & = \allI(\deno{\typerelation{\P,\a}{A}{\type}})\after\si\\
        & = \allI(\deno{\typerelation{\P,\a}{A}{\type}}\after(\si\times\Id{U}))\\
        & = \allI(\deno{\typerelation{\P',\a}{A\sub{\si, \a\setto\e}}{\type}})\\
        & = \allI(\deno{\typerelation{\P',\a}{A\ssi}{\type}})\\
        & = \deno{\typerelation{\P'}{\all{\a}{A\ssi}}{\type}}\\
        & = \deno{\typerelation{\P'}{(\all{\a}{A})\ssi}{\type}}\\
    \end{align}

\case{Function}
\begin{align}
    \deno{\typerelation{\P}{\ab}{\type}}\after\si &= \diamond(\deno{\typerelation{\P}{A}{\type}},\deno{\typerelation{\P}{B}{\type}})\after\si\\
    &= \diamond(\deno{\typerelation{\P}{A}{\type}}\after\si,\deno{\typerelation{\P}{B}{\type}}\after\si)\qt{By Naturality}\\
    & = \diamond(\deno{\typerelation{\P'}{A\ssi}{\type}},\deno{\typerelation{\P'}{B\ssi}{\type}})\\
    & = \deno{\typerelation{\P'}{\fntype{(A\ssi)}{(B\ssi)}}{\type}}\\
    & = \deno{\typerelation{\P'}{(\ab)\ssi}{\type}}\\
\end{align}

\section{Sub-typing}
If $\si = \deno{\typerelation{\P'}{\si}{\P}}$ then $\deno{A\ssi\subtypepp B\ssi} = \si\star\deno{A\subtypep B}: \cipw(A, B)$.

\proof
By induction on the derivation on $\deno{A\subtypep B}$. Using S-closure of $\si\star$ 

\case{Ground}
\begin{align}
    \si\star(\g_1\subtypeg\g_2) &= (\g_1\subtypeg\g_2)
\end{align}

Since $\si\star$ is s-closed.

\case{Monad}
\begin{align}
    \si\star\deno{\M{\e_1}{A} \subtypep \M{\e_2}{B}} & = \si\star(\deno{\e_1\subeffectp\e_2})\after\si\star(\T{\e_1}(\deno{A\subtypep B})) \\ 
     &= \deno{\si(\e_1)\subeffectpp\si(\e_2)} \after \T{\si(\e_1)}{\deno{A\ssi\subtypepp B\ssi}}\qt{By S-Closure}\\
     &= \deno{\M{\si(\e_1)}{A\ssi}\subtypepp\M{\si(\e_2)}{B\ssi}}\\
     &= \deno{(\M{\e_1}{A})\ssi\subtypepp\M{\e_2}{B}\ssi}\\
\end{align}

\case{For All}
    \begin{align}
        \si\star\deno{\all{\a}{A}\subtypep\all{\a}{B}} &= \si\star(\allI(\deno{A\subtypepa B})) \\
        &=\allII((\si\times\Id{U})\star(\deno{A\subtypepa B}))\\
        &=\allII(\deno{A\sub{\si,\a\setto\a}\subtypeppa B\sub{\si,\a\setto\a}})\\
        &= \deno{(\all{\a}{A})\ssi \subtypepp(\all{\a}{B})\ssi}\\
    \end{align}

\case{Fn}
\begin{align}
    \si\star\deno{(\ab)\subtypep\fntype{A'}{B'}} &= \si\star(\deno{B\subtypep B'}^{A'}\after B^{\deno{A'\subtypep A}})\\
    &= \si\star(\cur{\deno{B\subtypep B'}\after\app})\after\si\star(\cur{\app\after(\Id{B}\times\deno{A'\subtypep A})})\\
    & = \cur{\si\star(\deno{B\subtypep B'})\after\app}\after\cur{\app\after(\Id{B}\times\si\star(\deno{A'\subtypep A}))}\\
    & = \cur{\deno{B\ssi\subtypepp B'\ssi}\after\app}\after\cur{\app\after(\Id{B\ssi}\times\deno{A'\ssi\subtypepp A\ssi})}\\
    &= \deno{\fntype{(A\ssi)}{(B\ssi)}\subtypepp\fntype{(A'\ssi)}{(B'\ssi)}}\\
    &= \deno{(\ab)\ssi\subtypepp(\fntype{A'}{B'})\ssi}
\end{align}
\section{Type Environments}
If $\si = \deno{\typerelation{\P'}{\si}{\P}}$ then $\deno{\wellformedok{\P'}{\G\ssi}} = \si\star\deno{\wellformedok{\P}{\G}} = \deno{\wellformedok{\P}{\G}}: \cipw$.

\proof
By induction on the derivation on $\deno{\wellformedok{\P}{\G}}$. Using Naturality.

\case{Nil}
\begin{align}
    \si\star\deno{\wellformedok{\P}{\nil}} &= \term{I}\after\si\\
    &= \term{I'}\\
    &= \deno{\wellformedok{\P'}{\nil}}\\
    \deno{\wellformedok{\P'}{\nil\ssi}}\\
\end{align}

\case{Var}
\begin{align}
   \si\star\deno{\wellformedok{\P}{\gax}} &= \si\star(\square(\deno{\wellformedok{\P}{\G}},\deno{\typerelation{\P}{A}{\type}})) \\
    & = \square(\deno{\wellformedok{\P}{\G}},\deno{\typerelation{\P}{A}{\type}})\after\si\\
   & = \square(\deno{\wellformedok{\P}{\G}}\after\si, \deno{\typerelation{\P}{A}{\type}}\after\si)\\
    & = \square(\deno{\wellformedok{\P'}{\G\ssi}},\deno{\typerelation{\P'}{A\ssi}{\type}})\\
    & = \deno{\wellformedok{\P'}{\G\ssi, x: A\ssi}}\\
    & = \deno{\wellformedok{\P'}{(\gax)\ssi}}\\
\end{align}

\section{Terms}
If 
\begin{align}
    \si &= \deno{\typerelation{\P'}{\si}{\P}}\\
    \D &= \deno{\gpetyperelation{v}{A}}\\
    \D' &= \deno{\etyperelation{\P'}{\G\ssi}{v\ssi}{A\ssi}}\\
\end{align}

Then \begin{eqnarray}
    \D' = \si\star(\D)
\end{eqnarray}

\proof
By induction over the derivation of $\D$. Using the S-Closure of $\si\star$. We use $\G_I$ to indicate $\deno{\wellformedok{\P}{\G}}$, an $A_I$ to indicate $\deno{\typerelation{\P}{A}{\effect}}$

\case{Unit}

\begin{equation}
    \D = \term{\G_I}
\end{equation}

So

\begin{equation}
    \si\star(\D) = \term{\G_I\ssi} = \D'
\end{equation}

\case{True, False}
Giving the case for true as false is the same but using $\inr$
\begin{equation}
    \D = \inl\after\term{\G_I}
\end{equation}

So

\begin{equation}
    \si\star(\D) = \inl\after \term{\G_I\ssi} = \D'
\end{equation}

Since $\si\star$ is S-closed.

\case{Constant}


\begin{equation}
    \D = \deno{\const{A}}\after\term{\G_I}
\end{equation}

So

\begin{equation}
    \si\star(\D) = \si\star\deno{\const{A}}\after \term{\G_I\ssi}=\deno{\const{A\ssi}}\after \term{\G_I\ssi}  = \D'
\end{equation}

Since $\si\star$ is S-closed.

\case{Subtype}

Let \begin{equation}
    \D_1 = \deno{\gpetyperelation{v}{A}}
\end{equation}

Then

\begin{equation}
    \D = \deno{A\subtypep B}\after \D_1\\
\end{equation}

So 
\begin{align}
    \si\star(\D) & = \si\star{\deno{A\subtypep B}}\after\si\star\D_1 \\
    & = \deno{A\ssi\subtypepp B\ssi}\after\D_1'\qt{By induction}\\
    & = D'
\end{align}

\case{Lambda}
Let \begin{equation}
    \D_1 = \deno{\etyperelation{\P}{\gax}{v}{B}}
\end{equation}

Then

\begin{equation}
    \D = \cur(\D_1)\\
\end{equation}

So
\begin{align}
    \si\star(\D) & = \si\star(\cur{\D_1})\\
    & = \cur{\si\star(\D_1)}\qt{By S-closure}\\
    & = \cur{\D_1'}\qt{By induction}\\
    & = \D'
\end{align}

\case{Application}
Let \begin{align}
    \D_1 &= \deno{\gpetyperelation{v_1}{\ab}}\\
    \D_2 &= \deno{\gpetyperelation{v_2}{A}}
\end{align}

Then

\begin{equation}
    \D = \app\after\pr{\D_1}{\D_2}\\
\end{equation}

So

\begin{align}
    \si\star\D & = \si\star(\app\after\pr{\D_1}{\D_2})\\
    & = \app\after\pr{\si\star(\D_1)}{\si\star(\D_2)}\qt{By S-closure}\\
    & = \app\after\pr{\D_1'}{\D_2'}\qt{By Induction}\\
    & = \D'
\end{align}

\case{Return}
Let \begin{equation}
    \D_1 = \deno{\gpetyperelation{v}{A}}
\end{equation}

Then

\begin{equation}
    \D = \point{A_I}\after \D_1\\
\end{equation}

So

\begin{align}
    \si\star(\D) &= \si\star(\point{A_I}\after \D_1)\\
            & = \point{A_{I'}} \after\si\star(\D_1)\qt{By S-closure}\\
            & = \point{A_{I'}} \after\D_1'\\
            & = \D'
\end{align}

\case{Bind}
Let \begin{align}
    \D_1 &= \deno{\gpetyperelation{v_1}{\M{\e_1}{A}}}\\
    \D_2 &= \deno{\etyperelation{\P}{\gax}{v_2}{\M{\e_2}{B}}}
\end{align}

Then

\begin{equation}
    \D = \M{\e_1}{\e_2}{A_I}\after\T{\e_1}{\D_2}\after\tstrength{\e_1}{\G_I}{A_I}\after\pr{\Id{\G_{I}}}{\D_1}\\
\end{equation}

So

\begin{align}
    \si\star(\D) &= \si\star(\bind{\e_1}{\e_2}{A}\after\T{\e_1}{\D_2}\after\tstrength{\e_1}{\G}{A}\after\pr{\Id{\G}}{\D_1})\\
    & = \si\star(\bind{\e_1}{\e_2}{A})\after\si\star(\T{\e_1}{\D_2})\after\si\star(\tstrength{\e_1}{\G}{A})\after\pr{\si\star(\Id{\G_{I})}}{\si\star(\D_1)}\qt{By S-Closure}\\
    &= \bind{\si(\e_1)}{\si(\e_2)}{A\ssi'}\after\T{\si(\e_1)}{\si\star(\D_2)}\after\tstrength{\si(\e_1)}{\G\ssi}{A\ssi}\after\pr{\si\star(\Id{\G_{I})}}{\si\star(\D_1)}\qt{By S-Closure}\\
    &= \bind{\si(\e_1)}{\si(\e_2)}{A\ssi'}\after\T{\si(\e_1)}{\D_2'}\after\tstrength{\si(\e_1)}{\G\ssi}{A\ssi}\after\pr{\si\star(\Id{\G_{I})}}{\D_1'}\qt{By Induction}\\
    &= \D'\\
\end{align}

\case{If}

Let \begin{align}
    \D_1 &= \deno{\gpetyperelation{v}{\B}}\\
    \D_2 &= \deno{\gpetyperelation{v_1}{A}}\\
    \D_3 &= \deno{\gpetyperelation{v_2}{A}}\\
\end{align}

Then

\begin{equation}
    \D = \app\after(([\cur{\D_2\after\pp}, \cur{\D_3\after\pp}]\after\D_1)\times\Id{\G})\after\diag{\G}\\
\end{equation}

So

\begin{align}
    \si\star(\D) &= \si\star(\app\after(([\cur{\D_2\after\pp}, \cur{\D_3\after\pp}]\after\D_1)\times\Id{\G})\after\diag{\G})\\
    &= \app\after(([\cur{\si\star(\D_2)\after\pp}, \cur{\si\star(\D_3)\after\pp}]\after\si\star(\D_1))\times\Id{\G\ssi})\after\diag{\G\ssi}\qt{By S-Closure}\\
    &= \app\after(([\cur{\D_2'\after\pp}, \cur{\D_3'\after\pp}]\after\D_1')\times\Id{\G\ssi})\after\diag{\G\ssi}\qt{By Induction}\\
    & = \D'\\
\end{align}


\case{Effect-Lambda}

Let \begin{equation}
    \D_1 = \deno{\etyperelation{\P,\a}{\G}{v}{A}}
\end{equation}

Then

\begin{equation}
    \D = \hat{\D_1}\\
\end{equation}

And also

\begin{equation}
    \si\times\Id{} = \deno{\typerelation{(\P',\a)}{(\si, \a\setto\e)}{(\P,\a)}}
\end{equation}

So
\begin{align}
    \si\star\D &= \si\star(\hat{\D_1})\\
    & = \hat{(\si\times\Id{U})\star\D_1}\qt{By naturality}\\
    & = \hat{\D_1'}\qt{By induction}\\
    & = \D'
\end{align}

\case{Effect-Application}

Let \begin{align}
    \D_1 &= \deno{\gpetyperelation{v}{\all{\a}{A}}}\\
    h &= \deno{\typerelation{\P}{\e}{\effect}}\\
\end{align}

Then

\begin{equation}
    \D = \pr{\Id{\G}}{h}\star(\e_{\deno{\typerelation{\P,\b}{A\ssub{\a}{\b}}{\type}}})\after\D_1\\
\end{equation}

So
Due to the substitution theorem on effects
\begin{equation}
    h\after\si = \deno{\typerelation{\P}{\e}{\effect}}\after\si = \deno{\typerelation{\P'}{\si(\e)}{\effect}} = h'
\end{equation}

\begin{align}
    \si\star{\D} & = \si\star(\pr{\Id{\G}}{h}\star(\e_{\deno{\typerelation{\P,\b}{A\ssub{\a}{\b}}{\type}}})\after\D_1)\\
    & = (\pr{\Id{\G}}{h}\after\si)\star(\e_{\deno{\typerelation{\P,\b}{A\ssub{\a}{\b}}{\type}}})\after\si\star(\D_1)\\
    & = ((\si\times\Id{U})\after\pr{\Id{\G}}{h\after\si})\star(\e_{\deno{\typerelation{\P,\b}{A\ssub{\a}{\b}}{\type}}})\after\D_1)'\\
    & = (\pr{\Id{\G}}{h'})\star((\si\times\Id{U})\star\e_{\deno{\typerelation{\P,\b}{A\ssub{\a}{\b}}{\type}}})\after\D_1)'\\
\end{align}

Looking at the inner part of the functor application:
Let \begin{align}
    A & = \deno{\typerelation{\P,\b}{A\ssub{\a}{\b}}{\type}}\\
\end{align}
\begin{align}
    (\si\times\Id{U})\star\e_{\deno{\typerelation{\P,\b}{A\ssub{\a}{\b}}{\type}}} &= (\si\times\Id{U})\star\e_{A}\\
    & = (\si\times\Id{U})\star(\widehat{\Id{\allI(A)}})\\
    & = \widehat{\bar{(\si\times\Id{U})\star(\widehat{\Id{\allI(A)}})}}\qt{By bijection}\\
    & = \widehat{\si\star(\bar{\widehat{\Id{\allI(A)}}})}\qt{By naturality}\\
    & = \widehat{\si\star(\Id{\allI(A)})}\qt{By bijection}\\
    & = \widehat{\Id{\allI(A\after(\si\times\Id{U}))}}\qt{By S-Closure, naturality}\\
    & = \widehat{\Id{\allI(A\sub{\si,\a\setto\a})}}\qt{By Substitution theorem}\\
    & = \e_{A\ssi}
\end{align}

Going back to the original expression:

\begin{align}
    \si\star{\D} & = (\pr{\Id{\G}}{h'})\star((\si\times\Id{U})\star\e_{A\ssi})\after\D_1)'\\
    & = \D'\\
\end{align}

\chapter{Effect Weakening Theorem}
\section{Effects}
\section{Types}
\section{Type Environments}
\section{Sub-typing}
\section{Terms}

\chapter{Value Substitution Theorem}

\chapter{Type-Environment Weakening Theorem}

\chapter{Unique Denotation Theorem}

\chapter{Beta-Eta-Equivalence Theorem (Soundness)}


    
\end{document}
