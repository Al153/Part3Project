%Won't compile on its own
% Appendix document proving the S-category properties of the functor categories [E^n, \Set]


    \chapter{Functorial S-Categories}
    This appendix provides a full, although terse, exposition of S-category structure of functor categories $[E^n, \set]$ and the S-preserving nature of the re-indexing functors between them.
    
    \section{Base Category is Monoidal}
    
    We construct the base category, $\Eff$, as follows:
    
    \begin{itemize}
        \item $U = E$, the set of ground effects in the non-polymorphic language.
        \item $\1$ is a singleton set.
        \item $U^n = E^n$, set of $n$-wide tuples of effects, $\vec{\e}$
    \end{itemize}
    
    Hence when we treat effects that are well formed in $\P$ as morphisms, $E^n \rightarrow E$ in $\Eff$, we should treat them as functions $f: E^n \rightarrow E$. Ground effects become point functions: $e: \1 \rightarrow E$, so the denotation of a ground effect is the constant value function: $\deno{\typerelation{\P}{e}{\effect}} = \ev \mapsto e$. We extend the multiplication of ground effects to multiplication on effect functions, giving us our $\Mul_n$ operation $\Mul_n(f, g)(\ev) = (f\ev)\dot(g\ev)$. This $\Mul_n$ satisfies the naturality and monoidal requirements, as seen in figure \ref{MulMonoidAndNatural}. It is also trivially the case that $\Mul_n(\deno{\typerelation{\P}{e_1}{\effect}}, \deno{\typerelation{\P}{e_2}{\effect}}) = \deno{\typerelation{\P}{e_1\dot e_2}{\effect}}$.
    
    \begin{figure}
        \begin{framed}
            \begin{framed}
                \centering\textbf{Naturality} 
                \begin{equation*}
                    (\Mul_m(f, g)\after \theta)\ev = (f(\theta\ev))\dot(g(\theta\ev)) = (\Mul_n(f\after\theta), (g\after\theta))\ev
                \end{equation*}
            \end{framed} 
            
            \begin{framed}
                \centering\textbf{Left and Right Identity}
    
    
                \begin{minipage}{.45\textwidth}
                    \begin{align*}
                        \Mul_n(\deno{\typerelation{\P}{\1}{\effect}}, f)(\ev) & = (\1\ev)\dot(f\ev)\\
                        & = (\1)\dot(f\ev)\\
                        & = f\ev\\
                    \end{align*}
                \end{minipage}
                \quad
                \begin{minipage}{.45\textwidth}
                    \begin{align*}
                        \Mul_n(f, \deno{\typerelation{\P}{\1}{\effect}})(\ev) & = (f\ev)\dot(\1\ev)\\
                        & = (f\ev)\dot(\1)\\
                        & = f\ev\\
                    \end{align*}
                \end{minipage}
            \end{framed}
    
            \begin{framed}
                \centering\textbf{Monoid Associativity}
    
                \begin{align*}
                    \Mul_n(f, \Mul_n(g, h))(\ev) & = (f\ev)\dot((g\ev)\dot(h\ev))\\
                    & = ((f\ev)\dot(g\ev))\dot(h\ev)\\
                    & = \Mul_n(\Mul_n(f,g), h)(\ev)
                \end{align*}
            \end{framed}
        \end{framed}
        
        
        \caption{Our pick for $\Mul$ is natural and is a monoid at each $n$.}
        \label{MulMonoidAndNatural}
    \end{figure}
    
    \section{S-Categories}
    The semantic category, $[E^0, \set]$ of the effect-environment $\nil$ is isomorphic to $\set$. Since each effect-environment is alpha equivalent to a natural number, the semantic category for $\P$ shall be represented as $\C(\P) = \C(n) = [E^n, \set]$, the category of functions $E^n\rightarrow \set$. Objects in $[E^n, \set]$ are functions and we describe them by their actions on a generic vector of ground effects, $\ev$. Morphisms in $[E^n, \set]$ are natural transformations between the functions. So:
    
    \begin{align*}
        m:\quad&A \rightarrow B \qt{ In $[E^n, \set]$}\\
        m\ev:\quad& A\ev \rightarrow B\ev \qt{In $\set$}\\
        (f\after g)\ev =\quad& (f\ev)\after(g\ev)\\
        \Id{A}(\ev) =\quad& \Id{A\ev}
    \end{align*}
    
    So morphisms are dependently typed functions from a vector of ground effects to morphisms in $\set$.
    \subsection{Each S-Category is a CCC}
    Since $\set$ is complete and a CCC, and $E^n$ is small, since $E$ is small, $[E^n, \set]$ is a CCC.
    
    \begin{align*}
        (A\times B)\ev & = (A\ev)\times(B\ev)\\
        \1\ev & = \1\\
        (B^A)\ev &= (B\ev)^{(A\ev)}\\
        \p\ev & = \p\\
        \pp\ev & = \pp\\
        \app\ev & = \app\\
        \cur{f}\ev & = \cur{f\ev}\\
        \pr{f}{g}\ev & = \pr{f\ev}{g\ev}\\
    \end{align*}
    
    \subsection{The Terminal Co-Product}
    
    We can define the co-product point-wise.
    
    \begin{align*}
        (\1 + \1)\ev & = (\1\ev + \1\ev)\\
        & = (\1 + \1)\\
        \inl\ev &= \inl\\
        \inr\ev &= \inr\\
        [f, g]\ev &= [f\ev, g\ev]\\
    \end{align*}
    
    This preserves the co-product diagram.
    
    \begin{align*}
        ([f, g]\after\inl)\ev & = [f\ev, g\ev]\after \inl\\
        & = f\ev\\
        & \square\\
        ([f, g]\after\inr)\ev& = [f\ev, g\ev]\after \inr\\
        & = f\ev\\
        & \square\\
    \end{align*}
    
    $[f, g]$ is also unique in $[E^n, \set]$. Suppose $l \after \inl = f$ and $l\after \inr = g$ in $[E^n, \set]$. Then $l\ev \after \inl = f\ev$ and $l\ev \after\inr = g\ev$. Hence by the co-product in $\set$, $l = [f\ev, g\ev]$ so $l = [f, g]$.
    
    
    \subsection{Ground Types and Terms}
    Each ground type in the non-polymorphic calculus has a fixed denotation $\deno{\g}\in\obj\set$. The ground type in the polymorphic calculus hence has a denotation represented by the constant function.
    
    \begin{align*}
        \deno{\g}:&\quad E^n\rightarrow \obj\set\\
        \ev \mapsto&\quad  \deno{\g}\\
    \end{align*}
    
    Each constant term $\const{A}$ in the non-polymorphic calculus has a fixed denotation $\deno{\const{A}}\in\set(\1, A)$. So the morphism $\deno{\const{A}}$ in $[E^n, \set]$ is the corresponding constant dependently typed morphism returning the $\deno{\const{A}}$ function in $\set$.
    
    \begin{align*}
        \deno{\const{A}}:&\quad[E^n, \set](\1, A)\\
        \ev\mapsto&\quad\deno{\const{A}}
    \end{align*}
    
    \subsection{Graded Monad}\label{NaturalityAndMonadLawsProof}
    Given the strong graded monad $(\Tz{}{}, \pointz{}{}{}, \bindmu^0, \strengtht^0)$ on $\set$, we can construct an appropriate graded monad $(\Tn{}{}, \pointn{}{}{}, \bindmu^n,\strengtht^n)$ on $[E^n, \set]$. Through some mechanical proof and the naturality of the $\set$ strong graded monad, these morphisms are natural in their type parameters and form a strong graded monad in $[E^n, \set]$.
    
    \begin{align*}
        \Tn{}{}:&\quad(E^n, \dot, \subeffect_n, \1_n) \rightarrow [[E^n, \set], [E^n, \set]]\\
        (\Tn{f}{A})\ev=&\quad \Tz{(f\ev)}{A\ev}\\
        (\pointn{A})\ev=&\quad\pointz{A\ev}\\
        (\bindn{f}{g}{A})\ev=&\quad\bindz{(f\ev)}{(g\ev)}{(A\ev)}\\
        (\tstrengthn{f}{A}{B})\ev=&\quad\tstrengthz{(f\ev)}{(A\ev)}{(B\ev)}
    \end{align*}
    
    The required naturality, monad, and tensor-strength laws hold, as proved in in figures \ref{NaturalitySquares}-\ref{TensorStrengthLawProofs}
    
    
    
    \begin{figure}
        \begin{framed}
            \centering\textbf{Naturality of the Graded Monad Transformations}
    
    
            \begin{minipage}{.45\textwidth}
                \centering
                \begin{tikzcd}[column sep=huge]
                    A\ev 
                    \arrow{r}{\pointz{(A\ev)}}
                    \arrow{d}{f\ev}
                    &
                    \Tz{\1}{(A\ev)}
                    \arrow{d}{\Tz{\1}{(f\ev)}}
                    \\
                    B\ev
                    \arrow{r}{\pointz{(B\ev)}}
                    &
                    \Tz{\1}{(B\ev)}
                \end{tikzcd}
            \end{minipage}
            \quad\begin{minipage}{.45\textwidth}
                \centering
                \begin{tikzcd}[column sep=huge]
                    \Tz{(f\ev)}{\Tz{(g\ev)}{(A\ev)}}
                    \arrow{r}{\bindz{f\ev}{g\ev}{(B\ev)}}
                    \arrow{d}{\Tz{f\ev}{\Tz{g\ev}{m\ev}}}
                    &
                    \Tz{(f\ev)\dot(g\ev)}{(A\ev)}
                    \arrow{d}{\Tz{(f\ev)\dot(g\ev)}{m\ev}}
                    \\
                    \Tz{(f\ev)}{\Tz{(g\ev)}{(B\ev)}}
                    \arrow{r}{\bindz{f\ev}{g\ev}{(B\ev)}}
                    &
                    \Tz{(f\ev)\dot(g\ev)}{(A\ev)}
                \end{tikzcd}
            \end{minipage}
            \\
    
    
            
           \begin{minipage}{.45\textwidth}
            \centering
                \begin{tikzcd}[column sep=huge]
                    A\ev \times \Tz{f\ev}{(B\ev)}
                    \arrow{r}{\tstrengthz{f\ev}{(A\ev)}{(B\ev)}}
                    \arrow{d}{(m\ev\times\Id{\Tz{f\ev}{B}})}
                    &
                    \Tz{f\ev}{(A\ev\times B\ev)}
                    \arrow{d}{\Tz{(f\ev)}{(m\ev\times \Id{B\ev})}}
                    \\
                    A'\ev \times \Tz{f\ev}{(B\ev)}
                    \arrow{r}{\tstrengthz{f\ev}{(A'\ev)}{(B\ev)}}
                    &
                    \Tz{f\ev}{(A'\ev\times B\ev)}
                \end{tikzcd}
            \end{minipage}
            \quad\begin{minipage}{.45\textwidth}
                \centering
                \begin{tikzcd}[column sep=huge]
                    A\ev \times \Tz{f\ev}{(B\ev)}
                    \arrow{r}{\tstrengthz{f\ev}{(A\ev)}{(B\ev)}}
                    \arrow{d}{(\Id{A\ev}\times \Tz{f\ev}{(m\ev)})}
                    &
                    \Tz{f\ev}{(A\ev\times B\ev)}
                    \arrow{d}{\Tz{(f\ev)}{(\Id{A\ev} \times m\ev)}}
                    \\
                    A\ev \times \Tz{f\ev}{(B'\ev)}
                    \arrow{r}{\tstrengthz{f\ev}{(A\ev)}{(B'\ev)}}
                    &
                    \Tz{f\ev}{(A\ev\times B'\ev)}
                \end{tikzcd}
            \end{minipage}
        \end{framed}
        \caption{Naturality Squares for the graded monad natural transformations. The naturality of each transformation $\a$ in $[E^n, \set]$ derives from the naturality of each of its component transformations $\a\ev$ in $\set$}
        \label{NaturalitySquares}
    \end{figure}
    
    \begin{figure}
    \begin{framed}
            \centering\textbf{Monad Laws}
        
            \begin{minipage}{.45\textwidth}
                \centering\textbf{Left Unit}
        
        
                \begin{align*}
                    (\bindn{\1}{g}{A}\after\pointn{\Tn{f}{A}})\ev &= \bindz{\1}{(f\ev)}{(A\ev)} \after (\pointz{\Tz{f\ev}{A\ev}})\\
                    &= \Id{\Tz{f\ev}{A\ev}}\\
                    &=(\Id{\Tn{f}{A}})\ev
                \end{align*}
            \end{minipage}\quad
            \begin{minipage}{.45\textwidth}
                \centering\textbf{Right Unit}
        
        
                \begin{align*}
                    (\bindn{f}{\1}{A}\after\Tn{f}{\pointn{A}})\ev &= \bindz{(f\ev)}{\1}{(A\ev)} \after \Tz{f\ev}{(\pointz{A\ev})}\\
                    &= \Id{\Tz{f\ev}{A\ev}}\\
                    &=(\Id{\Tn{f}{A}})\ev
                \end{align*}
            \end{minipage}
            \\
        
            \begin{minipage}{\textwidth}
                \centering\textbf{Monad Associativity} 
        
        
                \begin{align*}
                    ((\bindn{f}{(g\dot h)}{A})\after\Tn{f}{(\bindn{g}{h}{A})})\ev &= \bindz{(f\ev)}{((g\ev)\dot(h\ev))}{(A\ev)}\after\Tz{f\ev}{\bindz{(h\ev)}{(g\ev)}{A\ev}}\\
                    &= \bindz{((f\ev)\dot(g\ev))}{(h\ev)}{(A\ev)}\after\bindz{(f\ev)}{(g\ev)}{(\Tz{h\ev}{(A\ev)})}\\
                    & = (\bindn{f\dot g}{h}{A}\after\bindn{f}{g}{\Tz{h}{A}})\ev
                \end{align*}
            \end{minipage}
    \end{framed}
        
        \caption{The monad laws for the graded monad $(\Tn{}{}, \pointn{}{}{}, \bindmu^n,\strengtht^n)$ can be proved component-wise from the graded monad $(\Tz{}{})$ on $\set$.}
        \label{MonadLaws}
    \end{figure}
    
    \begin{figure}
    \begin{framed}
    \centering\textbf{Tensor Strength Laws}
    
            \begin{framed}
                \centering
                \textbf{Bind Law}
        
                \begin{minipage}{.45\textwidth}
                    \centering
                    \begin{tikzpicture}[baseline= (a).base]
                        \node[scale=.7] (a) at (0,0){
                    \begin{tikzcd}
                        A \times \Tn{f}{\Tn{g}{B}} 
                        \arrow [r, "\tstrength{f}{A}{\Tn{g}{B}}"]
                        \arrow [dr, "\Id{A} \times \bindn{f}{g}{B}"]
                        & 
                        \Tn{f}{(A \times \Tn{g}{B})} 
                        \arrow [r, "\Tn{f}{\tstrength{g}{A}{B}}"]
                        & 
                        \Tn{f}{\Tn{g}{(A \times B)}} 
                        \arrow [d, "\bindn{f}{g}{A \times B}"]
                        \\
                        &
                        A \times \Tn{f \dot g}{B}  
                        \arrow [r, "\tstrength{f \dot g}{A}{B}"] 
                        &
                        \Tn{f \dot g}({A \times B)}
                    \end{tikzcd}};
                \end{tikzpicture}
                \end{minipage}
                \quad
                \begin{minipage}{.45\textwidth}
                    \centering
                    \scalebox{.75}{\parbox{\linewidth}{%
                        \begin{align*}
                            & (\tstrengthn{(f\dot g)}{A}{B}\after (\Id{A}\times \bindn{f}{g}{B}))\ev \\
                            & = (\tstrengthz{((f\ev)\dot (g\ev))}{(A\ev)}{(B\ev)}\after (\Id{A\ev}\times \bindn{(f\ev)}{(g\ev)}{(B\ev)}))\\
                            & = \bindz{(f\ev)}{(g\ev)}{(A\times B)\ev}\after\Tz{f\ev}{(\tstrengthz{(g\ev)}{(A\ev)}{(B\ev)})}\after\tstrengthz{(f\ev)}{(A\ev)}{\Tz{g\ev}{(B\ev)}}\\
                            & = (\bindn{f}{g}{(A\times B)}\after\Tn{f}{(\tstrengthn{g}{A}{B})}\after\tstrengthn{f}{A}{\Tn{g}{(B)}})\ev
                        \end{align*}
                    }}
                \end{minipage}
            \end{framed}
        
            \begin{framed}
                \centering
                \textbf{Commutes with Unit}
        
                \begin{minipage}{.45\textwidth}
                    \centering
                    \begin{tikzpicture}[baseline= (a).base]
                        \node[scale=.7] (a) at (0,0){
                    
                        \begin{tikzcd}
                            A \times B
                            \arrow [r, "\Id{A} \times \point{B}"]
                            \arrow [rd, "\point{A \times B}"]
                            &
                            A \times \tob 
                            \arrow [d, "\tstrength{\1}{A}{B}"]
                            \\
                            &
                            \Tn{\1}{(A \times B)}
                        \end{tikzcd}
                    };
                \end{tikzpicture}
                \end{minipage}
                \quad
                \begin{minipage}{.45\textwidth}
                    \centering
                    \scalebox{.75}{\parbox{\linewidth}{%
                    \begin{align*}
                        (\tstrengthn{\1}{A}{B}\after(\Id{A}\times \pointn{A}))\ev & = \tstrengthz{\1}{(A\ev)}{(B\ev)}\after(\Id{A\ev}\times \pointz{A\ev})\\
                        & = \pointz{A\ev\times B\ev}\\
                        & = (\pointn{A\times B})\ev
                    \end{align*}
                    }}
                \end{minipage}
            \end{framed}
        
            \begin{framed}
                \centering
                \textbf{Commutes with $\alpha$}
    
                Let $\alpha_{A, B, C} = \pr{\p\after\p}{\pr{\pp\after\p}{\pp}}: ((A \times B) \times C) \rightarrow (A \times (B \times C))$
        
        
                \begin{minipage}{.45\textwidth}
                    \centering
                    \begin{tikzpicture}[baseline= (a).base]
                        \node[scale=.7] (a) at (0,0){
                        \begin{tikzcd}
                            (A\times B)\times \Tn{\e}{C} 
                            \arrow [rr, "\tstrength{\e}{(A\times B)}{C}"]
                            \arrow [d, "\alpha_{A, B, \Tn{\e}{C}}"]
                            & & \Tn{\e}{((A \times B)\times C)}
                            \arrow [d, "\Tn{\e}{\alpha_{A, B, C}}"]
                            \\
                            A \times (B \times \Tn{\e}{C}) 
                            \arrow [r, "\Id{A}\times\tstrength{\e}{B}{C}"]
                            &
                            A\times\Tn{\e}{(B \times C)} 
                            \arrow [r, "\tstrength{\e}{A}{(B \times C)}"]
                            & \Tn{\e}{(A \times (B \times C))}
                            \\
                        \end{tikzcd}
                    };
                \end{tikzpicture}
                \end{minipage}
                \quad
                \begin{minipage}{.45\textwidth}
                    \centering
                    \scalebox{.75}{\parbox{\linewidth}{%
                    \begin{align*}
                        & (\Tn{f}{\alpha_{A, B, C}}\after\tstrengthn{f}{A\times B}{C})\ev \\
                        & = \Tz{f\ev}{\alpha_{A\ev, B\ev, C\ev}} \after\tstrengthz{(f\ev)}{(A\times B)\ev}{(C\ev)}\\
                        & = \tstrengthz{(f\ev)}{(A\ev)}{(B\ev\times C\ev)}\after(\Id{A\ev}\times \tstrengthz{(f\ev)}{(B\ev)}{(C\ev)})\after\alpha_{A\ev, B\ev, C\ev}\\
                        & = (\tstrengthn{f}{A}{(B\times C)}\after(\Id{A}\times \tstrengthn{f}{B}{C})\after\alpha_{A, B, C})\ev\\
                    \end{align*}            
                    }}
                \end{minipage}
            \end{framed}
    \end{framed}
    
        \caption{The tensor strength laws can be proved component wise from the strength of the monad $\Tz{}{}$}
        \label{TensorStrengthLawProofs}
    \end{figure}
    
    
    
    \subsection{Subeffecting}
    Given a collection of subeffecting natural transformation in $\set$, $\deno{\e_1\subeffectz \e_2}: \quad \Tz{\e_1}{} \rightarrow \Tz{\e_2}{}$ we can form subeffect natural transformations in $[E^n, \set]$ as seen in figure \ref{SubEffecting}. This natural transformation has all the required properties. These are proved in figure \ref{SubeffectProperties}.
    
    \begin{figure}
        \begin{framed}
            \centering\textbf{Subeffect Natural Transformation}
        
            \begin{align*}
                \deno{f\subeffectn g}:&\quad\Tn{f}{}\rightarrow \Tn{g}{}\\
                \deno{f\subeffectn g} A \ev:&\quad\Tn{f\ev}{(A\ev)}\rightarrow \Tn{g\ev}{(B\ev)}\\
                =&\quad \deno{f\ev\subeffectz g\ev}A\ev
            \end{align*}
        \end{framed}
        \caption{Definition of subeffecting natural transformations}
        \label{SubEffecting}
    \end{figure}
    
    \begin{figure}
        \begin{framed}
            \centering\textbf{Subeffect Natural Transformation Properties}
    
            \begin{framed}
                \centering
                \textbf{Naturality}
                
                \begin{tikzcd}[column sep=large]
                    \Tz{f\ev}{A\ev}
                    \arrow{r}{\deno{f\ev\subeffectz g\ev}A\ev}
                    \arrow{d}{\Tz{f\ev}{m\ev}}
                    &
                    \Tz{g\ev}{A\ev}
                    \arrow{d}{\Tz{g\ev}{m\ev}}
                    \\
                    \Tz{f\ev}{B\ev}
                    \arrow{r}{\deno{f\ev\subeffectz g\ev}B\ev}
                    &
                    \Tz{g\ev}{B\ev}
                \end{tikzcd}
            \end{framed}
    
            \begin{framed}
                \centering
                \textbf{Commutes With Tensor Strength}
    
    
    
    
        
                \begin{minipage}{.45\textwidth}
                    \centering
                    \begin{tikzpicture}[baseline= (a).base]
                        \node[scale=.7] (a) at (0,0){\begin{tikzcd}
                            A \times \Tn{f}{B} \arrow [r, "\Id{A} \times \dsen{f}{g}_B"] \arrow [d, "\tstrengthn{f}{A}{B}"] &
                            A \times \Tn{g}{B} \arrow [d, "\tstrengthn{g}{A}{B}"] \\
                            \Tn{f}{(A \times B)} \arrow [r, "\dsen{f}{g}_{ A \times B}"] &
                            \Tn{g}{(A \times B)} 
                        \end{tikzcd}
                        };
                \end{tikzpicture}
                \end{minipage}
                \quad
                \begin{minipage}{.45\textwidth}
                    \centering
                    \scalebox{.75}{\parbox{\linewidth}{%
                    \begin{align*}
                        & (\tstrengthn{g}{A}{B}\after(\Id{A}\times \db{f\subeffectn g}_B))\ev \\
                        &= \tstrengthz{(g\ev)}{(A\ev)}{(B\ev)}\after(\Id{A\ev}\times \db{f\ev \subeffectz g\ev}_{B\ev})\\
                        & = \db{f\ev \subeffectz g\ev}_{(A\times B)\ev} \after\tstrengthz{(f\ev)}{(A\ev)}{(B\ev)}\\
                        & = (\db{f \subeffectn g}_{(A\times B)} \after\tstrengthn{f}{A}{B})\ev\\
                    \end{align*}
                    }}
                \end{minipage}
            \end{framed}
    
            \begin{framed}
                \centering
                \textbf{Commutes with Join}
    
                \begin{minipage}{.45\textwidth}
                    \centering
                    \begin{tikzpicture}[baseline= (a).base]
                        \node[scale=.7] (a) at (0,0){
                        \begin{tikzcd}
                            \Tn{f}{\Tn{g}{}} 
                            \arrow [rr, "\Tn{f}{\deno{g\subeffectn g'}
                            }"]
                            \arrow [d, "\bindn{f}{g}{}"]
                            &  &
                            \Tn{f}{\Tn{g'}{}}
                            \arrow [rr, "\db{f \subeffectn f'}_{M, \Tn{g'}{}}"]
                            & &
                             \Tn{f'}{\Tn{g'}{}} 
                             \arrow [d, "\bindn{f'}{g'}{}"]
                             \\
                            \Tn{f\dot g}{}
                            \arrow [rrrr, "\deno{f\dot g\subeffectn f'\dot g'}"]
                            & &
                             & &
                            \Tn{f'\dot g'}{}
                        \end{tikzcd}                    
                        };
                \end{tikzpicture}
                \end{minipage}
                \quad
                \begin{minipage}{.45\textwidth}
                    \centering
                    \scalebox{.75}{\parbox{\linewidth}{%
                    \begin{align*}
                        & (\dsen{f\dot g}{f'\dot g'}_A\after\bindn{f}{g}{A})\ev \\ & = \dsez{(f\ev)\dot(g\ev)}{(f'\ev)\dot (g\ev)}_{A\ev}\after\bindz{(f\ev)}{(g\ev)}{(A\ev)}\\
                        & = \bindz{(f\ev)}{(g\ev)}{(A\ev)}\after\dsez{f\ev}{f'\ev}_{\Tz{g'\ev}{(A\ev)}}\after \Tz{f\ev}{\dsez{g\ev}{g'\ev}}_{(A\ev)}\\
                        &= \bindn{f}{g}{A}\after\dsen{f}{f'}_{\Tn{g'}{A}}\after\Tn{f}{\dsen{g}{g'}}_A
                    \end{align*}
                    }}
                \end{minipage}
            \end{framed}
            
        \end{framed}
        \caption{The required properties of the subeffect natural transformations can be proved component-wise from the appropriate of the subeffect natural transformation on $\set$.}
        \label{SubeffectProperties}
    \end{figure}
    
    \section{Re-Indexing Functors}
    \label{ReindexingFunctorProperties}
    For a function $\theta: E^m \rightarrow E^n$ , the re-indexing functor $\theta\star$ is defined as follows:
    
    \begin{align*}
        \theta\star:\quad& [E^n, \set] \rightarrow [E^m, \set]\\
        \theta\star(A)\emv=&\quad A(\theta (\emv))\\
        f:\quad& A \rightarrow B \in[E^n, \set]\\
        \theta\star(f)\emv =&\quad f(\theta(\emv)): A(\theta(\emv)\rightarrow B(\theta(\emv)))
    \end{align*}
    
    This functor preserves all the S-category properties.
    These can be seen in figures \ref{PreservesCCC}-\ref{PreservesSubtypingSubeffecting}
    
    
    \begin{figure}
        \centering
            \begin{framed}
                \centering\textbf{$\theta\star$ is Cartesian Closed}
    
                \begin{minipage}{.45\linewidth}
                    \begin{align*}
                        (\theta\star(A\times B))\ev & = (A\times B)(\theta\ev)\\
                        & = (A(\theta \ev)\times B(\theta\ev))\\
                        & = (\theta\star A\times \theta\star B)\ev\\
                    \end{align*}              
                \end{minipage}
                \quad
                \begin{minipage}{.45\linewidth}
                    \begin{align*}
                        (\theta\star \p)\ev & = \p(\theta\ev)\\
                        & = \p \qt{Constant function}\\
                        & = \p\ev
                    \end{align*}                
                \end{minipage}
    
    
                \begin{minipage}{.45\linewidth}
                    \begin{align*}
                        (\theta\star \pp)\ev & = \pp(\theta\ev)\\
                        & = \pp \qt{Constant function}\\
                        & = \pp\ev
                    \end{align*}                
                \end{minipage}
                \quad
                \begin{minipage}{.45\linewidth}
                    \begin{align*}
                        (\theta\star\pr{f}{g})\ev & = (\pr{f}{g})(\theta\ev)\\
                        & = \pr{f(\theta\ev)}{g(\theta\ev)}\\
                        &= \pr{\theta\star f}{\theta\star g}\ev
                    \end{align*}
                \end{minipage}
    
                \begin{minipage}{.45\linewidth}
                    \begin{align*}
                        (\theta\star(A^B))\ev & = (A^B)(\theta\ev)\\
                         & = (A (\theta\ev))^{(B (\theta\ev))}\\
                         & = (\theta\star A)^{(\theta\star B)}\ev\\
                    \end{align*}                
                \end{minipage}
                \quad
                \begin{minipage}{.45\linewidth}
                    \begin{align*}
                    (\theta\star \app)\ev & = \app(\theta\ev)\\
                    & = \app \qt{Constant fn}\\
                    & = \app\ev
                \end{align*}
                \end{minipage}
    
            \begin{minipage}{.45\linewidth}
                \begin{align*}
                    (\theta\star\cur{f})\ev & = \cur{f}(\theta\ev)\\
                    & = \cur{f(\theta\ev)}\\
                    & = \cur{\theta\star f}
                \end{align*}             
            \end{minipage}
            \quad
            \begin{minipage}{.45\linewidth}
                \begin{align*}
                    (\theta\star \1)\ev & = \1(\theta\ev)\\
                    & = \1 \\
                    & = \1 \ev\\
                \end{align*}
            \end{minipage}      
            
            \begin{minipage}{\linewidth}
                \begin{align*}
                    (\theta\star\term{A})\ev & = \term{A}(\theta\ev)\\
                    & = \term{A(\theta\ev)}\\
                    & = \term{\theta\star A}\ev
                \end{align*}
            \end{minipage}        
        \end{framed}
        
        \caption{Proof of the CCC-preserving property of re-indexing functors.}
        \label{PreservesCCC}
    \end{figure}
    
    \begin{figure}
        \centering
        \begin{framed}
            \centering\textbf{$\theta\star$ Preserves Co-products}
    
            \begin{minipage}{.45\textwidth}
                \begin{align*}
                    (\theta\star(\1 + \1))\ev & = (\1 + \1)(\theta\ev)\\
                    & = (\1 + \1)\qt{Constant function}\\
                    & = (\1 + \1)\ev
                \end{align*}
            \end{minipage}
            \quad
            \begin{minipage}{.45\textwidth}
                \begin{align*}
                    (\theta\star\inl )\ev & = \inl(\theta\ev)\\
                    & = \inl \qt{Constant Fn}\\
                    & = \inl \ev
                \end{align*}
            \end{minipage}
    
            \begin{minipage}{.45\textwidth}
                \begin{align*}
                    (\theta\star\inr )\ev & = \inr(\theta\ev)\\
                    & = \inr \qt{Constant Fn}\\
                    & = \inr \ev
                \end{align*}
            \end{minipage}
            \quad
            \begin{minipage}{.45\textwidth}
                \begin{align*}
                    (\theta\star [f, g])\ev & = [f, g](\theta\ev)\\
                    & = [f (\theta\ev), g(\theta\ev)]\\
                    & = [\theta\star f, \theta\star g]\ev\\
                \end{align*}
            \end{minipage}        
        \end{framed}
        
        \caption{Proof that re-indexing functors preserve the $\1 + \1$ co-product.}
        \label{PreservesCoProduct}
    \end{figure}
    
    
    
    \begin{figure}
        \centering
        \begin{framed}
            \centering\textbf{$\theta\star$ Preserves the Graded Monad}
    
            \begin{minipage}{.45\textwidth}
                \begin{align*}
                    (\theta\star\Tn{f}{A})\ev & = \Tn{f}{A}(\theta\ev) \\
                    & = \Tz{(f(\theta\ev))}{(A(\theta\ev))}\\
                    & = (\Tm{(f\after\theta)}{\theta\star A})\ev\\
                \end{align*}
            \end{minipage}
            \quad
            \begin{minipage}{.45\textwidth}
                \begin{align*}
                    (\theta\star\pointn{A})\ev & = \pointn{A}(\theta\ev) \\
                    & = \pointz{A(\theta\ev)}\\
                    & = \pointm{\theta\star A}\ev\\
                \end{align*}
            \end{minipage}
    
            \begin{minipage}{.45\textwidth}
                \begin{align*}
                    (\theta\star \bindn{f}{g}{A})\ev & = \bindn{f}{g}{A}(\theta\ev)\\
                    & = \bindz{f(\theta\ev)}{g(\theta\ev)}{A(\theta\ev)}\\
                    & = \bindm{f\after\theta}{g\after\theta}{\theta\star(A)}(\ev)
                \end{align*}
            \end{minipage}
            \quad
            \begin{minipage}{.45\textwidth}
                \begin{align*}
                    (\theta\star \tstrengthn{f}{A}{B})\ev & = \tstrengthn{f}{A}{B}(\theta \ev)\\
                    & = \tstrengthz{(f(\theta\ev))}{(A(\theta\ev))}{(B(\theta\ev))}\\
                    & = \tstrengthm{f\after\theta}{\theta\star A}{\theta\star B}\ev\\
                \end{align*}
            \end{minipage}        
        \end{framed}
        
        \caption{Re-indexing functors preserve the graded monad structure}
        \label{GradedMonadPreserved}
    \end{figure}
    
    \begin{figure}
        \begin{framed}
            \centering\textbf{$\theta\star$ preserves Ground Subtyping and Subeffecting}
    
            \begin{minipage}{.45\textwidth}
                \begin{align*}
                    \theta\star(\deno{A\subtypeg B})\ev  & = \deno{A\subtypeg B}(\theta\ev)\\
                    & = \deno{A\subtype B} \qt{Constant Function}\\
                    & = \deno{A\subtype B} \ev\\
                \end{align*}
            \end{minipage}
            \quad
            \begin{minipage}{.45\textwidth}
                \begin{align*}
                    (\theta\star(\dsen{f}{g} A))\ev & = (\dsen{f}{g} A)(\theta \ev)\\
                    & = (\dsen{f(\theta\ev)}{g(\theta\ev)}(A(\theta\ev)))\\
                    & = (\dsem{\theta\star f}{\theta\star g} (\theta\star A)) \ev\\
                \end{align*}
            \end{minipage}
        \end{framed}
        \caption{Re-indexing functors preserve the ground subtyping and subeffecting morphisms.}
        \label{PreservesSubtypingSubeffecting}
    \end{figure}
