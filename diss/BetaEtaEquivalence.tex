\documentclass{report}


%% Don't import the header multiple times

\ifdefined\HEADERIMPORTED
\else
\newcommand\HEADERIMPORTED[0]{This file is HEADERIMPORTED}
\usepackage{amssymb}

\usepackage{amsmath}
\usepackage[a4paper,includeheadfoot,margin=2.54cm]{geometry}
\usepackage{breqn}
\usepackage{amssymb}

\usepackage{amsmath}
\usepackage[a4paper,includeheadfoot,margin=2.54cm]{geometry}
\usepackage{breqn}
\usepackage{array}   % for \newcolumntype macro
\usepackage{tikz-cd}
\usepackage{tabstackengine}
\setstackEOL{\cr}
\setstackgap{L}{\normalbaselineskip}

\newcommand\todo[1]{\textbf{TODO: #1}}

\newcommand{\s}{\;}
\newcommand{\doin}[3]{\texttt{do}\s #1 \leftarrow #2 \s\texttt{in}\s #3}
\newcommand\apply[2]{#1\s#2}
\newcommand\ifthenelse[5]{\texttt{if}_{#1, #2}\s#3\s \texttt{then}\s #4 \s\texttt{else} \s#5}
\newcommand\const[1]{\texttt{C}^{#1}}
\newcommand\return[1]{\texttt{return} #1}
\newcommand\lam[3]{\lambda #1 : #2. #3}
\renewcommand\u[0]{\texttt{()}}
\newcommand{\U}[0]{\texttt{Unit}}
\renewcommand\t[0]{\texttt{true}}
\newcommand\f[0]{\texttt{false}}
\newcommand{\B}[0]{\texttt{Bool}}
\newcommand{\G}[0]{\Gamma}
\newcommand\D{\Delta}


% draw type relations
\newcommand{\typerelation}[3]{#1 \vdash #2 \colon #3}
\newcommand{\gtyperelation}[2]{\typerelation{\G}{#1}{#2}}

%draw tree rules
\newcommand{\treerule}[3]{(\text{#1}) \frac{#2}{#3}}
\newcommand{\condtreerule}[4]{\treerule{#1}{#2}{#3}(\text{if } #4)}

\newcommand{\subtype}[0]{\leq:}
\newcommand\subeffect[0]{\leq}

\newcommand{\M}[2]{\texttt{M}_{#1}{#2}}

\newcommand\lamtype[3]{#1 \rightarrow \M{#2}{#3}}
\newcommand{\1}[0]{\texttt{1}}

\newcommand\e[0]{\epsilon}

\newcommand{\db}[1]{{\bf [\![}#1{\bf ]\!]}}
\newcommand{\deno}[1]{\db{#1}_M}
\newcommand\after\circ
\newcommand\term[1]{\left\langle\right\rangle_{#1}}

\newcommand\point[1]{\eta_{#1}}
\newcommand\bind[3]{\mu_{#1, #2, #3}}

\newcommand\T[2]{T_{#1}{#2}}

\newcommand\pr[2]{\left\langle#1, #2\right\rangle}

% tensor strength Nat-tran
\newcommand\tstrength[3]{\texttt{t}_{#1, #2, #3}}

% Id morphism
\newcommand\Id[1]{\texttt{Id}_{#1}}

\newcommand\idg[0]{\Id{\G}}
% beta-eta equivalence
\newcommand\beequiv[0]{=_{\beta\eta}}
% Substitutions
\newcommand\si{\sigma}

\newcommand{\sub}[1]{\left[#1\right]}
\newcommand{\ssub}[2]{\left[#2 / #1\right]}
\newcommand{\ssi}[0]{\sub{\si}}

% beta-eta equivalence relation
\newcommand{\berelation}[4]{\typerelation{#1}{#2 \beequiv #3}{#4}}
\newcommand{\gberelation}[3]{\gtyperelation{#1 \beequiv #2}{#3}}


% Shortcuts for denotational equality
\newcommand{\denoequality}[4]{\deno{\typerelation{#1}{#2}{#4}} = \deno{\typerelation{#1}{#3}{#4}}}
\newcommand{\gdenoequality}[3]{\denoequality{\G}{#1}{#2}{#3}}

% Shorthand for monad types
\newcommand\mea[0]{\M{\e}{A}}
\newcommand\meb[0]{\M{\e}{B}}
\newcommand\mec[0]{\M{\e}{C}}

\newcommand\tea[0]{\T{\e}{A}}
\newcommand\teb[0]{\T{\e}{B}}
\newcommand\tec[0]{\T{\e}{C}}


\newcommand\moa[0]{\M{\1}{A}}
\newcommand\mob[0]{\M{\1}{B}}
\newcommand\moc[0]{\M{\1}{C}}

\newcommand\toa[0]{\T{\1}{A}}
\newcommand\tob[0]{\T{\1}{B}}
\newcommand\toc[0]{\T{\1}{C}}

\newcommand\aeb[0]{\lamtype{A}{\e}{B}}

% Shorthand for Gammas
\newcommand{\gax}[0]{\G, x: A}
\newcommand{\gby}[0]{\G, y: B}

% reduction function
\newcommand{\reduce}[0]{reduce}



% Combinators for building delta-based tree proof terms
\newcommand{\deltavrule}[4]{
    \treerule{Subtype}{\treerule{}{\D}{\typerelation{#1}{#2}{#3}}\s\s #3 \subtype #4}{\typerelation{#1}{#2}{#4}}}

\newcommand{\deltavruleprime}[4]{
    \treerule{Subtype}{\treerule{}{\D'}{\typerelation{#1}{#2}{#3}}\s\s #3 \subtype #4}{\typerelation{#1}{#2}{#4}}}

\newcommand{\deltavruleprimeprime}[4]{
        \treerule{Subtype}{\treerule{}{\D'}{\typerelation{#1}{#2}{#3}}\s\s #3 \subtype #4}{\typerelation{#1}{#2}{#4}}}
    
\newcommand{\deltacrule}[6]{
            \treerule{Subeffect}{\treerule{}{\D}{\typerelation{#1}{#2}{\M{#3}{#4}}}\s\s #4 \subtype #6\s\s #3 \subeffect #5}{\typerelation{#1}{#2}{\M{#5}{#6}}}}
\newcommand{\deltacruleprime}[6]{
    \treerule{Subeffect}{\treerule{}{\D'}{\typerelation{#1}{#2}{\M{#3}{#4}}}\s\s #4 \subtype #6\s\s #3 \subeffect #5}{\typerelation{#1}{#2}{\M{#5}{#6}}}}
\newcommand{\deltacruleprimeprime}[6]{
    \treerule{Subeffect}{\treerule{}{\D''}{\typerelation{#1}{#2}{\M{#3}{#4}}}\s\s #4 \subtype #6\s\s #3 \subeffect #5}{\typerelation{#1}{#2}{\M{#5}{#6}}}}
                            

\newcommand{\p}[0]{\pi_1}
\newcommand{\pp}[0]{\pi_2}

% short-hands for weakening
\newcommand{\wrel}[3]{#1 : #2 \triangleright #3}
\newcommand{\ok}[1]{#1 \texttt{Ok}}
\renewcommand\i[0]{\iota}
\newcommand\w{\omega}
\newcommand\dom[1]{\texttt{dom}(#1)}
\newcommand\x{\times}


% Combinators to build tree proofs
\newcommand{\truleconst}[0]{\treerule{Const}{\ok{\G}}{\gtyperelation{\const{A}}{A}}}
\newcommand{\truleunit}[0]{\treerule{Unit}{\ok{\G}}{\typerelation{\G}{\u}{\U}}}
\newcommand{\truletrue}[0]{\treerule{True}{\ok{\G}}{\typerelation{\G}{\t}{\B}}}
\newcommand{\trulefalse}[0]{\treerule{False}{\ok{\G}}{\typerelation{\G}{\f}{\B}}}


\newcommand{\E}[0]{\mathbb{E}}
\renewcommand{\dot}{\cdot}
\newcommand{\gens}[0]{::=}
\newcommand{\nil}[0]{\diamond}
\newcommand{\ground}[0]{\gamma}

% Terminal object of C
\newcommand{\terminal}[0]{\texttt{\1}}

% The category C
\newcommand{\C}[0]{\mathbb{C}}

% The category of locally-small categories
\newcommand{\Cat}[0]{\texttt{Cat}}
% Sub-effect Nat-trans
\newcommand{\dse}[2]{\db{#1 \subeffect #2}}

\newcommand\app[0]{\texttt{app}}
\newcommand\cur[1]{\texttt{cur}(#1)}
\newcommand{\ifnt}[1]{\texttt{If}_{#1}}


\newcommand{\setto}{:=}
\newcommand{\fv}[1]{\texttt{fv}(#1)}

% shorthand for inserting text to equations
\newcommand\qt[1]{\quad\text{#1}}


\fi

\begin{document}
    \section{Beta and Eta Equivalence}
    \subsection{Beta conversions}
    \begin{itemize}
        \item $\treerule{Lambda}{\typerelation{\gax}{C}{\meb}\s\s \gtyperelation{v}{A}}{\gberelation{\apply{(\lam{x}{A}{C})}{v} }{ C\ssub{v}{x}}{\meb}}$
        
        \item $\treerule{Left Unit}{\gtyperelation{v}{A} \s\s \typerelation{\gax}{C}{\meb}}{\gberelation{\doin{x}{\return{v}}{C}}{C\ssub{x}{V}}{\meb}}$
        
        \item $\treerule{Right Unit}{\gtyperelation{C}{\mea}}{\gberelation{\doin{x}{C}{\return{x}} }{C}{\mea}}$
        \item $\treerule{Associativity}{\gtyperelation{C_1}{\M{\e_1}{A}} \s\s \typerelation{\gax}{C_2}{\M{\e_2}{B}}\s\s \typerelation{\gby}{C_3}{\M{\e_3}{C}}}{
            \gberelation{\doin{x}{C_1}{(\doin{y}{C_2}{C_3})}}{\doin{y}{(\doin{x}{C_1}{C_2})}{C_3}}{\M{\e_1 \cdot \e_2 \cdot \e_3}{C}}
        }$
    \end{itemize}
    \subsection{Equivalence Relation}
    \begin{itemize}
        \item $\treerule{Reflexive}{\gtyperelation{t}{\tau}}{\gberelation{t}{t}{\tau}}$
        \item $\treerule{Symmetric}{\gberelation{t_1}{t_2}{\tau}}{\gberelation{t_2}{t_1}{\tau}}$
        \item $\treerule{Transitive}{\gberelation{t_1}{t_2}{\tau}\s\s\gberelation{t_2}{t_3}{\tau}}{\gberelation{t_1}{t_3}{\tau}}$
    \end{itemize}
    \subsection{Congruences}
    \begin{itemize}
        \item $\treerule{Lambda}{\berelation{\gax}{C_1}{C_2}{\meb}}{\gberelation{\lam{x}{A}{C_1}}{\lam{x}{A}{C_2}}{\lamtype{A}{\e}{B}}}$
        
        \item $\treerule{Return}{\gberelation{v_1}{v_2}{A}}{\gberelation{\return{v_1}}{\return{v_2}}{\M{\1}{A}}}$
        
        \item $\treerule{Apply}{\gberelation{v_1}{v_1'}{\lamtype{A}{\e}{B}}\s\s\gberelation{v_2}{v_2'}{A}}{\gberelation{\apply{v_1}{v_2}}{\apply{v_1'}{v_2'}}{\meb}}$
        
        \item $\treerule{Bind}{\gberelation{C_1}{C_1'}{\M{\e_1}{A}} \s\s \berelation{\gax}{C_2}{C_2'}{\M{\e_2}{B}}}{\gberelation{\doin{x}{C_1}{C_2}}{\doin{c}{C_1'}{C_2'}}{\M{\e_1 \cdot \e_2}{B}}}$
        
        \item $\treerule{If}{\gberelation{v}{v'}{\B} \s\s \gberelation{C_1}{C_1'}{\mea}\s\s\gberelation{C_2}{C_2'}{\mea}}{\gberelation{\ifthenelse{\e}{A}{v}{C_1}{C_2}}{\ifthenelse{\e}{A}{v}{C_1'}{C_2'}}{\mea}}$
        \item $\treerule{Subtype}{\gberelation{v}{v'}{A} \s\s A \subtype B}{\gberelation{v}{v'}{B}}$
        \item $\treerule{Subeffect}{\gberelation{C}{C'}{\M{\e_1}{A}}\s\s A \subtype B \s\s \e_1 \subeffect \e_2}{\gberelation{C}{C'}{\M{\e_2}{B}}}$
    \end{itemize}
    \section{Beta-Eta equivalence implies both have same type}
    Each derivation of $\gberelation{t}{t'}{\tau}$ can be converted to a derivation of $\gtyperelation{t}{\tau}$ and $\gtyperelation{t'}{\tau}$ by induction over the beta-eta equivalence relation derivation.

    \subsection{Equivalence Relations}
    \paragraph{Case Reflexive}
    By inversion we have a derivation of $\gtyperelation{t}{\tau}$.

    \paragraph{Case Symmetric}
    By inversion $\gberelation{t'}{t}{\tau}$. Hence by induction, derivations of $\gtyperelation{t'}{\tau}$ and $\gtyperelation{t}{\tau}$ are given.

    \paragraph{Case Transitive}
    By inversion, there exists $t_2$ such that $\gberelation{t_1}{t_2}{\tau}$ and $\gberelation{t_2}{t_3}{\tau}$. Hence by induction, we have derivations of $\gtyperelation{t_1}{\tau}$ and $\gtyperelation{t_3}{\tau}$

    \subsection{Beta conversions}
    
    \paragraph{Case Lambda}
        By inversion, we have $\typerelation{\gax}{C}{\meb}$ and $\gtyperelation{v}{A}$. Hence by the typing rules, we have:
        $$\treerule{Apply}{
            \treerule{Lambda}{\typerelation{\gax }{C}{\meb}}{\gtyperelation{\lam{x}{A}{C}}{\lamtype{A}{\e}{B}}}
            \s\s
            \gtyperelation{v}{A}
        }{\gtyperelation{\apply{(\lam{x}{A}{C})}{v}}{\mea}}$$

        By the substitution rule \todo{which?}, we have 
        $$\treerule{Substitution}{\typerelation{\gax}{C}{\meb}\s\s\gtyperelation{v}{A}}{\gtyperelation{C\ssub{x}{v}}{\meb}}$$
    \paragraph{Case Left Unit}
    \todo{Use left unit commutivity diagram}
    \paragraph{Case Right Unit}
    \todo{Use left unit commutivity diagram}
    \paragraph{Case Associative}
    \todo{Long proof from book. Maybe use a big diagram.}

    \subsection{Congruences}
    Each congruence rule corresponds exactly to a type derivation rule. To convert to a type derivation, convert all preconditions, then use the equivalent type derivation rule.
    \todo{These can be proved simply by using the recursive case and substituting values}
    \paragraph{Case Lambda}
    \paragraph{Case Return}
    \paragraph{Case Apply}
    \paragraph{Case Bind}
    \paragraph{Case If}
    \paragraph{Case Subtype}
    \paragraph{Case subeffect}

    \section{Beta-Eta equivalent terms have equal denotations}
    If $\berelation{t}{t'}{\tau}$ then $\gdenoequality{t}{t'}{\tau}$

    By induction over Beta-eta equivalence relation.
    \subsection{Equivalence Relation}
    The cases over the equivalence relation laws hold by the uniqueness of denotations and the fact that equality over morphisms is an equivalence relation.
    \paragraph{Case Reflexive}
    Equality is reflexive, so if $\gtyperelation{t}{\tau}$ then $\deno{\gtyperelation{t}{\tau}}$ is equal to itself.
    \paragraph{Case Symmetric}
    By inversion, if $\gberelation{t}{t'}{\tau}$ then $\gberelation{t'}{t}{\tau}$, so by induction $\gdenoequality{t'}{t}{\tau}$ and hence $\gdenoequality{t}{t'}{\tau}$
    \paragraph{Case Transitive}
    There must exist $t_2$ such that $\gberelation{t_1}{t_2}{\tau}$ and $\gberelation{t_2}{t_3}{\tau}$, so by induction,
    $\gdenoequality{t_1}{t_2}{\tau}$ and $\gdenoequality{t_2}{t_3}{\tau}$. Hence by transitivity of equality, $\gdenoequality{t_1}{t_3}{\tau}$

    \subsection{Beta Conversions}
    These cases are typically proved using the properties of a cartesian closed category with a strong graded monad.

    \paragraph{Case Lambda}
    \paragraph{Case Left Unit}
    \paragraph{Case Right Unit}
    \paragraph{Case Associative}

    \subsection{Congruences}
    These cases can be proved fairly mechanically by assuming the preconditions, using induction to prove that the matching pairs of subexpressions have equal denotations, then constructing the denotations of the expressions using the equal denotations which gives trivially equal denotations.

    \paragraph{Case Lambda}
    \paragraph{Case Return}
    \paragraph{Case Apply}
    \paragraph{Case Bind}
    \paragraph{Case If}
    \paragraph{Case Subtype}
    \paragraph{Case subeffect}
\end{document}