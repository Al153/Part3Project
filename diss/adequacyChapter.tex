\chapter{Adequacy}
    An important result, though often not proved in presentations of denotational semantics, is that of adequacy. A denotational semantics is adequate when two closed terms of ground type having equal denotations implies that the terms are equationally equivalent. Proving adequacy ensures that a denotational semantics may be used for program analysis. In this chapter, I shall introduce an instantiation of PEC,  $\pecput$ with a suitable denotational semantics, generated using the methods described in Chapter \ref{HowToBuildAModel}. Following this, I shall introduce a set of logical relation between the denotations and closed terms. Using these relations, I shall then prove that closed terms with the same denotations are equivalent under the equational equivalence given in Section \ref{SoundnessDefinition}.

    \section{Instantiation of the Polymorphic Effect Calculus}
    Let us instantiate the polymorphic effect calculus to be a language in which one can write programs which can create an output signal using the statement $\put$. The effect system of EC is then used to count an upper bound on the number of outputs that a program can make. This language shall be called $\pecput$.

    \subsection{Ground Types}
    We simply use the basic ground types: $\g \gens \B \mid \U$.

    \subsection{Graded Monad}
    We index the base graded monad with a partially ordered monoid derived from the natural numbers $E = (\N, 0, +, \leq)$. This base monoid is extended as described in Section \ref{PECTypeSystem} to symbolically include variables $\a, \b, \g$ which range over the natural numbers. The abstract effect of a term therefore computes an upper bound on the number of outputs produced.  This means that the $\doin{x}{v}{v'}$ type rule adds together the upper bounds on the two expressions to give an upper bound on the number of outputs of the sequenced expression. The $\return{v}$ type rule acknowledges that a pure expression does not have any output. Subtyping  allows the type system to compute a looser bound.
    
    

    \subsection{Constants}
    As stated, we extend the set of constant, built in expressions to include a \put\s statement which performs a single output action. $\const{A} \gens \t^\B \mid \f^\B \mid \u^\U \mid \put^{\M{1}{\U}}$

    \subsection{Subtyping}
    The ground subtyping relation is the identity relation, meaning the $\U$ and $\B$ types are subtypes only of themselves. This is extended using the subeffect and function subtyping rules given in \ref{PECTypeSystem} to give a full subtyping relation.

    \section{Instantiation of a Model of the Polymorphic Effect Calculus}

    Let us now instantiate a model of $\pecput$ in the indexed category derived as in Chapter \ref{HowToBuildAModel} from a model of the non-polymorphic version of $\pecput$ in $\set$, the category of sets and functions. To do this, we build an S-category for $\pecput$. Since $\set$ is already a CCC and has co-products, we simply need to define a strong graded monad, a denotation for $\put$, and the subeffecting natural transformations.
    
    \subsection{Non-Polymorphic Model}
    \paragraph{Graded Monad}
    The strong graded monad on $\set$ is given by tagging values of the underlying type with the number of output operations required to compute that value (Figure \ref{AdequacyGradedMonad}).

\begin{figure}
    \centering
    \begin{minipage}[t]{0.45\linewidth}
        \begin{framed}
            \begin{align*}
                \Tz{n}{A} & = \setcomp{(n', a)}{n'\leq n\wedge a\in A}\\
                \bindz{m}{n}{A} & = (m', (n', a))\mapsto (n' + m', a)\\
                \pointz{A} & = a\mapsto (0, a)\\
                \tstrengthz{n}{A}{B} & = (a, (n', b)) \mapsto (n', (a, b))
            \end{align*}
        \end{framed}
        \caption{The graded monad.}
        \label{AdequacyGradedMonad}
    \end{minipage}\quad
    \begin{minipage}[t]{0.45\linewidth}
        \begin{framed}
            \begin{align*}
                \deno{\u} & = \singleton\mapsto \singleton\\
                \deno{\t} & = \singleton\mapsto \top\\
                \deno{\f} & = \singleton\mapsto \bot\\
                \deno{\put} & = \singleton\mapsto (1, \singleton)
            \end{align*}
            \end{framed}
            
                \caption{Denotations for ground terms. \todo{Align}}
                \label{PECPUTGroundDenotations}
    \end{minipage}
\end{figure}


   


\paragraph{Subeffecting Natural Transformations}
These natural transformations are given by inclusion functions, since $n\leq m\wedge (n',a)\in\Tz{n}{A} \implies (n'\leq n \leq m, a\in A)\implies (n',a)\in\Tz{m}{A}$. Other subtyping morphisms are generated using the usual method according to the subtype derivation (Figure \ref{SubtypingDenotations}).

\paragraph{Ground Denotations}
We define denotations for ground terms as in Figure \ref{PECPUTGroundDenotations}, using $\top, \bot$ to represent the elements of $1+1$.






\subsection{Making the Model Polymorphic}
Using the methods explained in Chapter \ref{HowToBuildAModel}, we can lift this model into an indexed S-Category using the functor categories $[E^n, \set]$. Hence the model can be proved sound for $\pecput$.
    

\section{Programming with Put}
We can now introduce some syntactic sugar for $\pecput$. Specifically, we introduce a notion of powers of $\put$ in Definition \ref{PutPowers}. These powers have simple denotations in our indexed category. Due to soundness, all terms in the equivalence class have the same denotation.

\begin{framed}
    \begin{definition}[Powers of Put as an Equational Equivalence Class]
        \label{PutPowers}
    Define $\put^n$ as follows:
    
    \begin{align*}
        \label{PutnDefinition}
        \gpeberelation{\put^0}{&{\s}\return{\u}}{\M{0}{\U}}\\ 
        \gpeberelation{\put^{m+1}}{&{\s}\doin{\_}{\put^n}{\put}}{\M{m+1}{\U}}
    \end{align*}
    \end{definition}
\end{framed}


\begin{framed}
    \begin{lemma}[Denotations of Powers of Put]
    $\deno{\gpetyperelation{\put^m}{\M{m}{\U}}} = \ev\mapsto\rho\mapsto(m, \singleton)$    
    \end{lemma}
    
    \begin{proof}
        By induction on $m$.
    
        \case{0}
        
        \begin{equation}
            \deno{\gpetyperelation{\put^0}{\M{0}{\U}}}\fundRho = \pointn{}(\singleton) = (0, \singleton)
        \end{equation}
        
        \case{m+1}
        \begin{align*}
            \deno{\gpetyperelation{\put^{m+1}}{\M{0}{\U}}}\fundRho & = (\bindmu^n\after\Tn{m}{(\deno{\typerelation{\nil}{\put}{\M{1}{\U}}}\after\p)}\after \strengtht^n)
            \\&\quad(\singleton, \deno{\gpetyperelation{\put^m}{\M{n}{\U}}}\fundRho)\\
            & = (m+1, \singleton)
        \end{align*}
    \end{proof}
\end{framed}



\section{Logical Relations}
It is difficult to directly prove adequacy. This is because we cannot directly perform induction over morphisms. So instead we use a system of logical relations which relate elements of denotations of types to closed terms of the same type (Definition \ref{LogicalRelationDefinition}). Specifically we define a relation for a given type, effect environment, and tuple of ground effects. Since they are defined structurally with respect to the type, we can induct over the type structure to prove properties. Specifically, using such induction, we show that these logical relations are preserved by subtyping (Theorem \ref{LogRelSubtype}), they respect equational equivalence (Theorem \ref{LogRelBeequiv}), and that they behave well with respect to single effect-variable substitutions (Theorem \ref{EnvironmentLemma}). These properties allow us to prove a fundamental property of these logical relations (Theorem \ref{FundPropTheorem}) which asserts that the denotation of a term relates to the term itself. In Section \ref{AdequacySection}, we finally use the logical relations to prove adequacy.

\begin{framed}
    \begin{definition}[Logical Relation]\label{LogicalRelationDefinition}
        \begin{equation}
            \label{LogicalRelationType}
            \erelates{\P}{A}(\ev: E^n) \in \powset{\deno{\wellformedt{\P}{A}}\ev\times \pecput^{A\ssub{\a}{\e}}}
        \end{equation}
   
        \begin{align*}
            \inLogRelPE{d}{v}{\all{\a}{A}} 
            & \Leftrightarrow \forall\e. (\pe(d), \eapp{v}{\e})\in\erelates{\P}{A\ssub{\a}{\e}}(\ev)\\
                    \inLogRelPE{d}{v}{\U} & \Leftrightarrow (d=\singleton \wedge \zberelation{v}{()}{\U})
                    \\
                    \inLogRelPE{d}{v}{\B} & \Leftrightarrow ((d=\top \wedge \zberelation{v}{\t}{\B}) 
                    \\& \quad\vee (d=\bot \wedge \zberelation{v}{\f}{\B}))
                    \\
                    \inLogRelPE{d}{v}{\ab}  &\Leftrightarrow \forall e, u. (e, u)\in\erelates{\P}{A}(\ev) \implies (d(e), \apply{v}{u})\in \erelates{\P}{B}(\ev)
                    \\
                    \inLogRelPE{(n, d)}{v}{\M{f}{A}} &\Leftrightarrow \exists v'. (\inLogRelPE{d}{v'}{A}
                    \\ & \quad\wedge  \zberelation{v}{\doin{\_}{\put^n}{\return{v'}}}{(\M{f}{A})\ssub{\av}{\ev}}))
        \end{align*}
    \end{definition}
        
\end{framed}

\begin{framed}
    \begin{theorem}[Logical Relation and Subtyping]\label{LogRelSubtype}
        If $A\subtypep B$ and $\inLogRelPE{d}{v}{A}$ then $\inLogRelPE{d}{v}{B}$
    \end{theorem}
    
    \begin{proof}
        By induction on the derivation of $A\subtypep B$. Cases are given in Appendix \ref{AdequacySubtypingProof}.
    \end{proof}
\end{framed}

\begin{framed}
    \begin{theorem}[Logical Relation and Equational Equivalence]\label{LogRelBeequiv}
        For all $v_1, v_2, A, \ev.$
        \begin{equation}
            \zberelation{v_1}{v_2}{A\ssub{\av}{\ev}}\implies (\inLogRelPE{d}{v_1}{A} \Leftrightarrow \inLogRelPE{d}{v_2}{A})
        \end{equation}
        
    \end{theorem}
    
    \begin{proof}
        By a simple induction on the type $A$. Cases are given in Appendix \ref{AdequacyEqEquivProof}.
    \end{proof}
\end{framed}

\begin{framed}
    \begin{lemma}[Environment Lemma]\label{EnvironmentLemma}
        $$\erelates{\P,\a}{A}(\ev, \e) = \erelates{\P}{A\ssub{\e}{\a}}(\ev)$$
    \end{lemma}
    
    \begin{proof}
        By induction over the type $A$, proving that $\inLogRel{d}{v}{\P,\a}{A}{\ev, \e} \Leftrightarrow \inLogRelPE{d}{v}{A\ssub{\a}{\e}}$. Cases are given in Appendix \ref{AdequacyEnvironmentsProof}.
    \end{proof}
\end{framed}




\subsection{Fundamental Property}\label{FundProp}
At this point, we know how the logical relations interact with subtyping and the change of environment. We can now use these properties to prove the fundamental property relating terms to their denotations.

\begin{framed}
    \begin{theorem}[Fundamental Property]\label{FundPropTheorem}
        If $\forall x. \inLogRelPE{\rho(x)}{\si(x)}{\G(x)}$ then $\inLogRelPE{\deno{\gpetyperelation{v}{A}}\fundRho}{v\fundSub}{A}$ up to equational equivalence.    
    \end{theorem}
    
    \begin{proof}
        By induction over the derivation of $\gpetyperelation{v}{A}$. The full set of  cases is given in Appendix \ref{AdequacyFundamentalPropertyProof}.
    \end{proof}
\end{framed}


\section{Adequacy}
\label{AdequacySection}

The fundamental property gives us the ability to prove the main theorem of this chapter: adequacy.

\begin{framed}
    
    \begin{theorem}[Adequacy]
        For types $G$ defined as: $ G\gens \B \mid \U \mid \M{n}{G}$, equality of denotations implies equational equality.
        
        \begin{equation}
            \deno{\ztyperelation{v}{G}} = \deno{\ztyperelation{u}{G}} \implies \zberelation{v}{u}{G}
        \end{equation}
        
    \end{theorem}
        
        \begin{proof}
            By induction on the structure of $G$, making use of the fundamental Property \ref{FundProp}.
        
            \case{Boolean}

            Let $d=\deno{\ztyperelation{v}{\B}}=\deno{\ztyperelation{v}{\B}}\in \{ \top, \bot\}$. By the fundamental property, $\inLogRelPE{d}{v}{\B}$  and $\inLogRelPE{d}{v}{\B}$.
        
            \subcase{$d=\top$}
            Then $\zberelation{v\beequiv\t}{u}{\B}$
        
            
            \subcase{$d=\bot$}
            Then $\zberelation{v\beequiv\f}{u}{\B}$
        
            \case{Unit}

            Let $\singleton=\deno{\ztyperelation{v}{\U}}=\deno{\ztyperelation{v}{\U}}\in \{\singleton\}$. By the fundamental property, $\inLogRelPE{d}{v}{\U}$  and $\inLogRelPE{d}{v}{\U}$. Hence $\zberelation{v\beequiv\u}{u}{\U}$.
        
            \case{\teffect}
        
            Let $( n',  d) = \deno{\ztyperelation{v}{\M{n}{G}}} = \deno{\ztyperelation{u}{\M{n}{G}}}$. By the fundamental property, $\inLogRelPE{((n', d))}{v}{\M{n}{G}}$ and $\inLogRelPE{((n', d))}{u}{\M{n}{G}}$. So there exists $u', v'$ such that $\inLogRelPE{d'}{u'}{G}$ and $\inLogRelPE{d'}{u'}{G}$ and:
        
            \begin{align*}
                \zberelation{v&}{\doin{\_}{\put^{n'}}{\return{v'}}}{\M{n}{G}}\\
                &\beequiv \doin{\_}{\put^{n'}}{\return{u'}}\\
                & \beequiv u & \square
            \end{align*}
        \end{proof}
\end{framed}

Adequacy is important as it tells us that the model is indeed useful for program analysis. Adequacy is often omitted in the presentation of denotational semantics, as it is often difficult to prove. However, this $\set$ based construction shows that logical relations required to prove adequacy are fairly simple. Indeed, simply by changing the relation for $\wellformedt{\P}{\mea}$, adequacy can be proved for polymorphic languages with different effect systems.