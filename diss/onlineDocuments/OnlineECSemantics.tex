\documentclass{report}

%% Don't import the header multiple times

\ifdefined\HEADERIMPORTED
\else
\newcommand\HEADERIMPORTED[0]{This file is HEADERIMPORTED}
\usepackage{amssymb}

\usepackage{amsmath}


% For typesetting tree rules
\usepackage{mathpartir}

% For colouring code
\usepackage{xcolor}


\usepackage{array}   % for \newcolumntype macro
\usepackage{tikz-cd}
\usepackage{tabstackengine}
\usepackage{breqn}
\usepackage{stmaryrd}

\usepackage{float} % extra options for figure placement

% For drawing boxed
\usepackage{framed}

% for code fragments + highlighting
\usepackage{listings}

% For roman numerals
\usepackage{enumitem}


\usepackage{amsthm}
%Theorems
\usepackage[utf8]{inputenc}
\usepackage[english]{babel}

\ifdefined\PRESENTATIONMODE
\else
\usepackage[a4paper,includeheadfoot,margin=2.54cm]{geometry}
\newtheorem{theorem}{Theorem}[section]
\newtheorem{corollary}{Corollary}[theorem]
\newtheorem{lemma}[theorem]{Lemma}
\newtheorem{definition}{Definition}[section]

\newtheorem{aside}{Aside}[section]
\newtheorem{property}[theorem]{Property}
\theoremstyle{definition}
\fi



\usepackage{tikz}

\definecolor{grey}{rgb}{0.75, 0.75, 0.75}
\definecolor{DarkGreen}{rgb}{0.1, 0.6, 0.1}

\usetikzlibrary{shapes.geometric,fit}
\usetikzlibrary{arrows,automata,positioning}
\usetikzlibrary{decorations.pathreplacing,calc}



\setstackEOL{\cr}
\setstackgap{L}{\normalbaselineskip}

\newcommand\todo[1]{\textbf{TODO: #1}}
\newcommand\needsRef[1]{\textbf{Reference Needed: (#1)}}
\newcommand\fixLayout[1]{\textbf{Fix Layout: #1}}


%% Rule Names
% Prefixes
\newcommand{\tprefix}[0]{T-}
\newcommand{\eprefix}[0]{E-}
\newcommand{\sprefix}[0]{S-}
\newcommand\equationalprefix[0]{Eq-}
\newcommand\envprefix[0]{Env-}
\newcommand\pprefix[0]{\eprefix\envprefix}

\newcommand\subprefix[0]{Sb-}
\newcommand\weakenprefix[0]{Wk-}

% Base  rule names
\newcommand\basenil[0]{Nil}
\newcommand\baseextend[0]{Extend}

\newcommand{\baseground}[0]{Ground}
\newcommand{\baseweaken}[0]{Weaken}
\newcommand{\basevar}[0]{Var}
\newcommand\basefn[0]{Fn}
\newcommand\baseeffect[0]{Effect}
\newcommand\basequant[0]{Quantification}


\newcommand\baseunit[0]{Unit}
\newcommand\basetrue[0]{True}
\newcommand\basefalse[0]{False}
\newcommand\baseconst[0]{Const}
\newcommand\basesubtype[0]{Subtype}
\newcommand\basegen[0]{Effect-Gen}
\newcommand\basespec[0]{Effect-Spec}
\newcommand\basereturn[0]{Return}
\newcommand\baseapply[0]{Apply}
\newcommand\baseif[0]{If}
\newcommand\basebind[0]{Bind}

\newcommand\basetransitive[0]{Transitive}
\newcommand\basereflexive[0]{Reflexive}

\newcommand{\baseid}[0]{Id}
\newcommand\baseproject[0]{Project}

% Effect Weakening Rule Names
\newcommand{\eid}[0]{\eprefix\baseid}
\newcommand{\eproject}[0]{\eprefix\baseproject}
\newcommand{\eextend}[0]{\eprefix\baseextend}

% Term Weakening Rule Names
\newcommand{\tid}[0]{\tprefix\baseid}
\newcommand{\tproject}[0]{\tprefix\baseproject}
\newcommand{\textend}[0]{\tprefix\baseextend}

% Effect Substitution Rule Names
\newcommand\esubnil[0]{\eprefix\basenil}
\newcommand\esubextend[0]{\eprefix\baseextend}

% Term Substitution Rule Names

\newcommand\tsubnil[0]{\tprefix\basenil}
\newcommand\tsubextend[0]{\tprefix\baseextend}

% Type environment Rule Names
\newcommand\envnil[0]{\envprefix\basenil}
\newcommand\envextend[0]{\envprefix\baseextend}
% Effect Environment rule names
\newcommand\pnil[0]{\pprefix\basenil}
\newcommand\pextend[0]{\pprefix\baseextend}
% Equational equality rule names
\newcommand{\eqbeta}[0]{\equationalprefix Lambda-Beta}
\newcommand{\eqeta}[0]{\equationalprefix Lambda-Eta}
\newcommand{\eqeffbeta}[0]{\equationalprefix Effect-Beta}
\newcommand{\eqeffeta}[0]{\equationalprefix Effect-Eta}
\newcommand\eqleftunit[0]{\equationalprefix Left-Unit}
\newcommand\eqrightunit[0]{\equationalprefix Right-Unit}
\newcommand\equnitequiv[0]{\equationalprefix Unit}
\newcommand\eqiftrue[0]{\equationalprefix If-True}
\newcommand\eqiffalse[0]{\equationalprefix If-False}
\newcommand\eqifeta[0]{\equationalprefix If-Eta}
\newcommand\eqassociativity[0]{\equationalprefix Associativity}

\newcommand{\eqreflexive}[0]{\equationalprefix\basereflexive}
\newcommand\eqtransitive[0]{\equationalprefix\basetransitive}
\newcommand\eqsymmetric[0]{\equationalprefix Symmetric}

\newcommand\equnit[0]{\equationalprefix\baseunit}
\newcommand\eqtrue[0]{\equationalprefix\basetrue}
\newcommand\eqfalse[0]{\equationalprefix\basefalse}
\newcommand\eqconst[0]{\equationalprefix\baseconst}
\newcommand{\eqvar}[0]{\equationalprefix\basevar}
\newcommand\eqweaken[0]{\equationalprefix\baseweaken}
\newcommand\eqfun[0]{\equationalprefix\basefn}
\newcommand\eqsubtype[0]{\equationalprefix\basesubtype}
\newcommand\eqgen[0]{\equationalprefix\basegen}
\newcommand\eqspec[0]{\equationalprefix\basespec}
\newcommand\eqreturn[0]{\equationalprefix\basereturn}
\newcommand\eqapply[0]{\equationalprefix\baseapply}
\newcommand\eqif[0]{\equationalprefix\baseif}
\newcommand\eqbind[0]{\equationalprefix\basebind}

% Term rule names
\newcommand\vunit[0]{\baseunit}
\newcommand\vtrue[0]{\basetrue}
\newcommand\vfalse[0]{\basefalse}
\newcommand\vconst[0]{\baseconst}
\newcommand{\vvar}[0]{\basevar}
\newcommand\vweaken[0]{\baseweaken}
\newcommand\vfun[0]{\basefn}
\newcommand\vsubtype[0]{\basesubtype}
\newcommand\vgen[0]{\basegen}
\newcommand\vspec[0]{\basespec}
\newcommand\vreturn[0]{\basereturn}
\newcommand\vapply[0]{\baseapply}
\newcommand\vif[0]{\baseif}
\newcommand\vbind[0]{\basebind}

%Effect rule names
\newcommand\eground[0]{\eprefix\baseground}
\newcommand\evar[0]{\eprefix\basevar}
\newcommand\eweaken[0]{\eprefix\baseweaken}
\newcommand\ecompose[0]{\eprefix Compose}

% Type rule names
\newcommand{\tground}[0]{\tprefix\baseground}
\newcommand{\tfun}[0]{\tprefix\basefn}
\newcommand{\teffect}[0]{\tprefix\baseeffect}
\newcommand{\tquant}[0]{\tprefix\basequant}

% Subtyping rule names
\newcommand{\stransitive}[0]{\sprefix\basetransitive}
\newcommand{\sreflexive}[0]{\sprefix\basereflexive}
\newcommand{\sground}[0]{\sprefix\baseground}
\newcommand{\sfun}[0]{\sprefix\basefn}
\newcommand{\seffect}[0]{\sprefix\baseeffect}
\newcommand{\squant}[0]{\sprefix\basequant}


\newcommand{\s}{\;}
\newcommand{\doin}[3]{\texttt{do}\s #1 \leftarrow #2 \s\texttt{in}\s #3\s}
\newcommand\apply[2]{#1\s#2}
\newcommand{\pifthenelse}[4]{\texttt{if}_{\textcolor{purple}{#1}}\s#2\s \texttt{then}\s #3 \s\texttt{else} \s#4\s}
\newcommand\ifthenelse[5]{\pifthenelse{#1, #2}{#3}{#4}{#5}}
\newcommand\const[1]{\texttt{k}^{\color{purple} #1}}
\newcommand\return[1]{\texttt{return} \s#1\s}


\newcommand\lam[3]{\lambda #1 \colon {\color{purple}#2}. #3\s}
\renewcommand\u[0]{\texttt{()}}
\newcommand{\U}[0]{\texttt{Unit}}
\renewcommand\t[0]{\texttt{true}}
\newcommand\f[0]{\texttt{false}}
\newcommand{\B}[0]{\texttt{Bool}}
\newcommand{\G}[0]{\Gamma}
\newcommand\D{\Delta}


% draw type relations
\newcommand{\typerelation}[3]{{\color{DarkGreen}#1} \vdash #2 \colon {\color{blue}#3}}
\newcommand\wellformed[2]{{\color{DarkGreen}#1}\vdash {\color{blue}#2}}
\newcommand\wellformedok[2]{\ok{{\color{DarkGreen}#1}\vdash {\color{blue} #2}}}

\newcommand{\wellformedtype}[2]{\typerelation{#1}{#2}{\type}}
\newcommand{\wellformedeffect}[2]{\typerelation{#1}{#2}{\effect}}
\newcommand{\wellformedF}[2]{\typerelation{#1}{#2}{F}}



\newcommand{\gtyperelation}[2]{\typerelation{\G}{#1}{#2}}
 

\newcommand\treerulez[1]{\inferrule{ }{#1}}
\newcommand\treeruleI[2]{\inferrule{#1}{#2}}
\newcommand\treeruleII[3]{\inferrule{#1 \\ #2}{#3}}
\newcommand\treeruleIII[4]{\inferrule{#1 \\ #2 \\ #3}{#4}}
\newcommand\treeruleIV[5]{\inferrule{#1 \\ #2 \\ #3 \\ #4}{#5}}
\newcommand\treeruleV[6]{\inferrule{#1 \\ #2 \\ #3 \\ #4 \\ #5}{#6}}

\newcommand\ntreerulez[2]{(\text{#1})\inferrule{ }{#2}}
\newcommand\ntreeruleI[3]{(\text{#1})\inferrule{#2}{#3}}
\newcommand\ntreeruleII[4]{(\text{#1})\inferrule{#2 \\ #3}{#4}}
\newcommand\ntreeruleIII[5]{(\text{#1})\inferrule{#2 \\ #3 \\ #4}{#5}}
\newcommand\ntreeruleIV[6]{(\text{#1})\inferrule{#2 \\ #3 \\ #4 \\ #5}{#6}}
\newcommand\ntreeruleV[7]{(\text{#1})\inferrule{#2 \\ #3 \\ #4 \\ #5 \\ #6}{#7}}

\newcommand\condtreerulez[3]{(\text{#1})\inferrule{ }{#2}(\text{if } #3)}
\newcommand\condtreeruleI[4]{(\text{#1})\inferrule{#2}{#3}(\text{if } #4)}
\newcommand\condtreeruleII[5]{(\text{#1})\inferrule{#2 \\ #3}{#4}(\text{if } #5)}
\newcommand\condtreeruleIII[6]{(\text{#1})\inferrule{#2 \\ #3 \\ #4}{#5}(\text{if } #6)}
\newcommand\condtreeruleIV[7]{(\text{#1})\inferrule{#2 \\ #3 \\ #4 \\ #5}{#6}(\text{if } #7)}
\newcommand\condtreeruleV[8]{(\text{#1})\inferrule{ #2 \\ #3 \\ #4 \\ #5 \\ #6 }{#7}(\text{if } #8)}



\newcommand{\subtype}[0]{\leq\colon}
\newcommand\subeffect[0]{\leq}

\newcommand{\M}[2]{\texttt{M}_{#1}{#2}}

\newcommand\lamtype[3]{#1 \rightarrow \M{#2}{#3}}
\newcommand{\1}[0]{\texttt{1}}

\newcommand\e[0]{\epsilon}

\newcommand{\db}[1]{{\bf [\![}#1{\bf ]\!]}}
\newcommand{\deno}[1]{\db{#1}}
\newcommand\after\circ
\newcommand\term[1]{\langle\rangle_{#1}}

\newcommand\bindmu[0]{\mu}
\newcommand\point[1]{\eta_{#1}}
\newcommand\bind[3]{\bindmu_{#1, #2, #3}}

\newcommand\T[2]{T_{#1}{#2}}

\newcommand\pr[2]{\langle#1, #2\rangle}
\newcommand\finpr[2]{\langle #1\rangle_{#2}}

\newcommand\strengtht[0]{\texttt{t}}
% tensor strength Nat-tran
\newcommand\tstrength[3]{\strengtht_{#1, #2, #3}}

% Id morphism
\newcommand\Id[1]{\texttt{Id}_{#1}}

\newcommand\idg[0]{\Id{\G}}
% beta-eta equivalence
\newcommand\beequiv[0]{\approx}
% Substitutions
\newcommand\si{\sigma}

\newcommand{\sub}[1]{[#1]}
\newcommand{\ssub}[2]{[#2 / #1]}
\newcommand{\ssi}[0]{\sub{\si}}

% beta-eta equivalence relation
\newcommand{\berelation}[4]{\typerelation{#1}{#2 \beequiv #3}{#4}}
\newcommand{\gberelation}[3]{\gtyperelation{#1 \beequiv #2}{#3}}


% Shortcuts for denotational equality
\newcommand{\denoequality}[4]{\deno{\typerelation{#1}{#2}{#4}} = \deno{\typerelation{#1}{#3}{#4}}}
\newcommand{\gdenoequality}[3]{\denoequality{\G}{#1}{#2}{#3}}

% Shorthand for monad types
\newcommand\mea[0]{\M{\e}{A}}
\newcommand\meb[0]{\M{\e}{B}}
\newcommand\mec[0]{\M{\e}{C}}

\newcommand\tea[0]{\T{\e}{A}}
\newcommand\teb[0]{\T{\e}{B}}
\newcommand\tec[0]{\T{\e}{C}}


\newcommand\moa[0]{\M{\1}{A}}
\newcommand\mob[0]{\M{\1}{B}}
\newcommand\moc[0]{\M{\1}{C}}

\newcommand\toa[0]{\T{\1}{A}}
\newcommand\tob[0]{\T{\1}{B}}
\newcommand\toc[0]{\T{\1}{C}}

\newcommand\aeb[0]{\lamtype{A}{\e}{B}}

% Shorthand for Gammas
\newcommand{\gax}[0]{\G, x\colon A}
\newcommand{\gby}[0]{\G, y\colon B}

% reduction function
\newcommand{\reduce}[0]{reduce}



% Combinators for building delta-based tree proof terms
\newcommand{\deltavrule}[4]{
    \ntreeruleII{\vsubtype}{\treeruleI{\D}{\typerelation{#1}{#2}{#3}}}{#3 \subtype #4}{\typerelation{#1}{#2}{#4}}}

\newcommand{\deltavruleprime}[4]{
    \ntreeruleII{\vsubtype}{\treeruleI{\D'}{\typerelation{#1}{#2}{#3}}}{#3 \subtype #4}{\typerelation{#1}{#2}{#4}}}

\newcommand{\deltavruleprimeprime}[4]{
        \ntreeruleII{\vsubtype}{\treeruleI{\D'}{\typerelation{#1}{#2}{#3}}}{#3 \subtype #4}{\typerelation{#1}{#2}{#4}}}
    
\newcommand{\deltacrule}[6]{
            \ntreeruleII{Subeffect}{\treeruleI{\D}{\typerelation{#1}{#2}{\M{#3}{#4}}}}{\subeffecttree{#3}{#4}{#5}{#6}}{\typerelation{#1}{#2}{\M{#5}{#6}}}}
\newcommand{\deltacruleprime}[6]{
    \ntreeruleII{Subeffect}{\treeruleI{\D'}{\typerelation{#1}{#2}{\M{#3}{#4}}}}{
    \subeffecttree{#3}{#4}{#5}{#6}}{\typerelation{#1}{#2}{\M{#5}{#6}}}}
\newcommand{\deltacruleprimeprime}[6]{
    \ntreeruleII{\vsubtype}{\treeruleI{\D''}{\typerelation{#1}{#2}{\M{#3}{#4}}}}{
        \subeffecttree{#3}{#4}{#5}{#6}}{\typerelation{#1}{#2}{\M{#5}{#6}}}}
                            

\newcommand{\p}[0]{\pi_1}
\newcommand{\pp}[0]{\pi_2}

% short-hands for weakening
\newcommand{\wrel}[3]{#1 \colon {\color{blue}#2} \triangleright {\color {blue} #3}}
\newcommand{\ok}[1]{{\color{blue} #1} \texttt{ Ok}}
\newcommand\okt[0]{\texttt{Ok}}
\renewcommand\i[0]{\iota}
\newcommand\w{\omega}
\newcommand\dom[1]{\texttt{dom}(#1)}
\newcommand\x{\times}


\newcommand\fev[1]{fev(#1)}
\newcommand\union[0]{\cup}


% Combinators to build tree proofs
\newcommand{\truleconst}[0]{\ntreeruleI{\vconst}{\ok{\G}}{\gtyperelation{\const{A}}{A}}}
\newcommand{\truleunit}[0]{\ntreeruleI{\vunit}{\ok{\G}}{\typerelation{\G}{\u}{\U}}}
\newcommand{\truletrue}[0]{\ntreeruleI{\vtrue}{\ok{\G}}{\typerelation{\G}{\t}{\B}}}
\newcommand{\trulefalse}[0]{\ntreeruleI{\vfalse}{\ok{\G}}{\typerelation{\G}{\f}{\B}}}


\newcommand{\E}[0]{\mathbb{E}}
\renewcommand{\dot}{\cdot}
\newcommand{\gens}[0]{\colon\colon=}
\newcommand{\nil}[0]{\diamond}
\newcommand{\ground}[0]{\gamma}

% Terminal object of C
\newcommand{\terminal}[0]{\texttt{\1}}

% The category C
\newcommand{\C}[0]{\mathbb{C}}
\newcommand{\Cz}[0]{\C_0}
\newcommand\DC[0]{\mathbb{D}}

% The category of locally-small categories
\newcommand{\Cat}[0]{\texttt{Cat}}
% Sub-effect Nat-trans
\newcommand{\dse}[2]{\db{#1 \subeffect #2}}

\newcommand\app[0]{\texttt{app}}
\newcommand\cur[1]{\texttt{cur}(#1)}
\newcommand{\ifnt}[1]{\texttt{If}_{#1}}


\newcommand{\setto}{\colon=}
\newcommand{\fv}[1]{\texttt{fv}(#1)}

% shorthand for inserting text to equations
\newcommand\qt[1]{\quad\text{#1}}

% Co-product short-hands
\newcommand\inr[0]{\texttt{inr}}
\newcommand\inl[0]{\texttt{inl}}
    
\newcommand\fld[2]{[#1,#2]}
\newcommand{\diag}[1]{\delta_{#1}}
\newcommand{\twist}[2]{\tau_{#1, #2}}

\newcommand\ifMorph[3]{\app\after((\fld{\cur{#2\after\pp}}{\cur{#3\after\pp}}\after #1)\times \idg)\after \diag{\G}}


% Polymorphic short-hands
\newcommand\elam[2]{\Lambda #1. #2}
\newcommand{\eapp}[2]{#1\s#2}
\renewcommand{\a}[0]{\alpha}
\newcommand{\all}[2]{\forall #1. #2}
\renewcommand{\P}[0]{\Phi}

\renewcommand{\b}[0]{\beta}
\newcommand{\g}[0]{\gamma}
\renewcommand\d[0]{\delta}
\newcommand\oke[2]{\wellformedok{#1}{#2}}
\newcommand\etyperelation[4]{\typerelation{#1\mid#2}{#3}{#4}}
\newcommand{\gpetyperelation}[2]{\etyperelation{\P}{\G}{#1}{#2}}
\newcommand{\gppetyperelation}[2]{\etyperelation{\P'}{\G}{#1}{#2}}


\newcommand{\eberelation}[5]{\berelation{#1\mid#2}{#3}{#4}{#5}}
\newcommand{\gpeberelation}[3]{\berelation{\P\mid\G}{#1}{#2}{#3}}
\newcommand{\gppeberelation}[3]{\berelation{\P'\mid\G}{#1}{#2}{#3}}

\newcommand{\dotp}[0]{\dot_\P}
\newcommand{\fntype}[2]{#1\rightarrow #2}
\newcommand{\ab}[0]{\fntype{A}{B}}

\newcommand\wrelw[2]{\wrel{\w}{#1}{#2}}
\renewcommand\proof[0]{\paragraph{Proof:}}
\newcommand{\case}[1]{\paragraph{Case #1:}}
\newcommand{\subcase}[1]{\subparagraph{Case: #1}}
\newcommand\bi[0]{By inversion}

%pre-filled effect-weakening relations
\newcommand\ewrel[4]{\wellformed{#1}{\color{black}\wrel{#2}{#3}{#4}}}
\newcommand\pewrel[3]{\ewrel{\P}{#1}{#2}#3}
\newcommand\ppewrel[3]{\ewrel{\P'}{#1}{#2}#3}

\newcommand\subtypep[0]{\subtype_\P}
\newcommand\subtypepp[0]{\subtype_{\P'}}
\newcommand\subeffectp[0]{\subeffect_{\P}}
\newcommand\subeffectpp[0]{\subeffect_{\P'}}
\newcommand\subeffectn[0]{\subeffect_{n}}
\newcommand\subeffectz[0]{\subeffect_{0}}

\newcommand{\allI}[0]{\forall_I}
\newcommand{\allII}[0]{\forall_{I'}}
\newcommand\allIU[0]{\forall_{I\times U}}
\newcommand\type[0]{\texttt{Type}}
\newcommand\effect[0]{\texttt{Effect}}
\newcommand\ciw[0]{\C(I, W)}
\newcommand\ciu[0]{\C(I, U)}
\newcommand\ciuw[0]{\C(I\times U, W)}
\newcommand\cipw[0]{\C(I', W)}
\newcommand\cipu[0]{\C(I', U)}
\newcommand\ciuu[0]{\C(I\times U, U)}
\newcommand\cii[0]{\C(I', I)}
\newcommand\Eff[0]{\texttt{Eff}}
\newcommand\Mul[0]{\texttt{Mul}}
\newcommand\singleton[0]{\ast}
\renewcommand\star[0]{^*}
\renewcommand\bar[1]{\overline{#1}}

\newcommand\subtypeg[0]{\subtype_\g}
\newcommand\subtypepa[0]{\subtype_{\P, \a}}
\newcommand\subtypeppa[0]{\subtype_{\P', \a}}

\newcommand\subtypez[0]{\subtype_{0}}
\newcommand\subtypen[0]{\subtype_{n}}

\usepackage{scalerel,stackengine}
\stackMath
\renewcommand\widehat[1]{%
\savestack{\tmpbox}{\stretchto{%
  \scaleto{%
    \scalerel*[\widthof{\ensuremath{#1}}]{\kern.1pt\mathchar"0362\kern.1pt}%
    {\rule{0ex}{\textheight}}%WIDTH-LIMITED CIRCUMFLEX
  }{\textheight}% 
}{2.4ex}}%
\stackon[-6.9pt]{#1}{\tmpbox}%
}
\parskip 1ex

\newcommand\pstar[0]{\p\star}

\newcommand\edeltavrule[5]{\deltavrule{#1 \mid #2}{#3}{#4}{#5}}

\newcommand\subeffecttreep[4]{\ntreeruleII{\teffect}{
    #1\subeffectp #3}{#2 \subtypep #4
}{\M{#1}{
    #2
}\subtypep\M{#3}{#4}}}
\newcommand\subeffecttree[4]{\ntreeruleII{\teffect}{
    #1\subeffect #3}{#2 \subtype #4
}{\M{#1}{
    #2
}\subtype\M{#3}{#4}}}


\newcommand{\edeltavruleprime}[5]{
        \deltavruleprime{#1\mid #2}{#3}{#4}{#5}}
    
\newcommand{\edeltavruleprimeprime}[5]{
        \deltavruleprimeprime{#1\mid #2}{#3}{#4}{#5}}
    
\newcommand{\edeltacrule}[6]{
            \ntreeruleII{\vsubtype}{
                \treeruleI{
                    \D
                }{
                    \typerelation{\P\mid#1}{#2}{\M{#3}{#4}}
                }
            }{
                \ntreeruleII{\teffect}{
                    #4 \subtypep #6
                    }{
                         #3 \subeffectp #5
                }{
                    \M{#3}{#4}\subtypep{\M{#5}{#6}}
                }
            }{
                \typerelation{\P\mid #1}{#2}{\M{#5}{#6}}
            }
        }
        

        \newcommand{\edeltacruleprime}[6]{
            \ntreeruleII{\vsubtype}{
                \treeruleI{
                    \D'
                }{
                    \typerelation{\P\mid #1}{#2}{\M{#3}{#4}}
                }
            }{
                \ntreeruleII{\teffect}{
                    #4 \subtypep #6
                }{#3 \subeffectp #5
                }{
                    \M{#3}{#4}\subtypep{\M{#5}{#6}}
                }
            }{
                \typerelation{\P\mid #1}{#2}{\M{#5}{#6}}
            }
        }
                   

        \newcommand{\edeltacruleprimeprime}[6]{
            \ntreeruleII{\vsubtype}{
                \treeruleI{
                    \D''
                }{
                    \typerelation{\P\mid #1}{#2}{\M{#3}{#4}}
                }
            }{
                \ntreeruleII{\teffect}{
                    #4 \subtypep #6
                    }{ #3 \subeffectp #5
                }{
                    \M{#3}{#4}\subtypep{\M{#5}{#6}}
                }
            }{
                \typerelation{\P\mid #1}{#2}{\M{#5}{#6}}
            }
        }

        \newcommand\obj[0]{\texttt{obj }}


        \newcommand{\Tz}[2]{\texttt{T}^0_{#1}#2}
        \newcommand{\Tn}[2]{\texttt{T}^n_{#1}#2}
        \newcommand{\Tm}[2]{\texttt{T}^m_{#1}#2}
        
        \newcommand{\pointz}[1]{\point{#1}^0}
        \newcommand{\pointn}[1]{\point{#1}^n}
        \newcommand{\pointm}[1]{\point{#1}^m}
        
        \newcommand{\bindz}[3]{\bind{#1}{#2}{#3}^0}
        \newcommand{\bindn}[3]{\bind{#1}{#2}{#3}^n}
        \newcommand{\bindm}[3]{\bind{#1}{#2}{#3}^m}
        
        \newcommand\tstrengthz[3]{\tstrength{#1}{#2}{#3}^0}
        \newcommand\tstrengthn[3]{\tstrength{#1}{#2}{#3}^n}
        \newcommand\tstrengthm[3]{\tstrength{#1}{#2}{#3}^m}
        
        \newcommand\set[0]{\texttt{Set}}
        \newcommand\cccat[0]{\textit{CCCat}}

        \newcommand\ev[0]{\vec{\e}}
        \newcommand\emv[0]{\vec{\e_m}}
        \newcommand\env[0]{\vec{\e_n}}
        
        \newcommand\subeffectm[0]{\subeffect_m}
        
        \newcommand\dsem[2]{\db{#1 \subeffectm #2}}
        \newcommand\dsen[2]{\db{#1 \subeffectn #2}}
        \newcommand\dsez[2]{\db{#1 \subeffectz #2}}
        \newcommand\dsep[2]{\db{#1 \subeffectp #2}}
        \newcommand\dsepp[2]{\db{#1 \subeffectpp #2}}
        
        \newcommand\allEn[0]{\forall_{E^n}}
        \newcommand\allEm[0]{\forall_{E^m}}
        
        \newcommand\counit[1]{\boldsymbol{\epsilon}_{#1}}
        \newcommand\unit[1]{\boldsymbol{\eta}_{#1}}


        
%% Adequacy shorthands
\newcommand{\relates}[0]{\lhd}
\newcommand{\logRel}[3]{#1 \relates_{#2} #3}
\newcommand{\plogRel}[4]{#1 \relates_{\wellformed{#2}{#3}} #4}

\newcommand{\zberelation}[3]{\berelation{}{#1}{#2}{#3}}
\newcommand\ztyperelation[2]{\typerelation{}{#1}{#2}}

\newcommand{\N}[0]{\mathbb{N}}
\renewcommand\put[0]{\texttt{put}}
\newcommand\ecput[0]{\texttt{EC}_\put}
\newcommand\ecputA[0]{\texttt{EC}_\put^A}
\newcommand\ecputG[0]{\texttt{EC}_\put^G}

\newcommand\mna[0]{\M{n}{A}}
\newcommand\mmb[0]{\M{m}{B}}
\newcommand\mnb[0]{\M{n}{B}}
\newcommand\mma[0]{\M{m}{A}}

\newcommand{\setcomp}[2]{\{#1 \mid #2 \}}
\fi

\begin{document}

\abstract
This document contains a terse explanation of the semantics of the Effect Calculus in an S-Category.

\tableofcontents
\chapter{Language Definition}

\section{Terms}

\begin{equation}
    \begin{split}
        v \gens& \mid  x \\
        & \mid \lam{x}{A}{v} \\
        & \mid \const{A} \\
        & \mid \u \\
        & \mid \t \mid \f \\
        & \mid \apply{v_1}{v_2} \\
        & \mid \pifthenelse{A}{v}{v_1}{v_2} \\
        & \mid \doin{x}{v_1}{v_2} \\
        & \mid \return{v}
        \end{split}
    \end{equation}

\section{Type System}
\subsection{Effects}
The effects should form a monotonous, partially-ordered monoid $(E, \dot, \1, \subeffect)$ with elements $\e$
\subsection{Types}
    \paragraph{Ground Types}
        There exists a set $\ground$ of ground types, including \U, \B
    \paragraph{Types}
    $$ A, B, C \gens \ground \mid \mea \mid \ab $$

\subsection{Subtyping}
    There exists a subtyping partial-order relation $\subtype_{\ground}$ over ground types that is:
    \begin{itemize}
        \item $\ntreerulez{\sreflexive}{A \subtype_{\ground} A}$
        \item $\ntreeruleII{\stransitive}{A \subtype_{\ground} B }{ B \subtype_{\ground} C}{A \subtype_{\ground} C}$
    \end{itemize}

    We extend this relation with the function subtyping rule to yield the full subtyping relation $\subtype$

    \begin{itemize}
        \item $\ntreeruleI{\sground}{A \subtype_{\ground} B}{A \subtype B}$
        \item $\ntreeruleII{\sfun}{A \subtype A' }{ B' \subtype B }{\fntype{A'}{B'} \subtype \ab}$
        \item $\ntreeruleII{\seffect}{A\subtype A'}{\e\subeffect\e'}{\mea\subtype\M{\e'}{A'}}$
    \end{itemize}
\subsection{Type Environments}
An environment, $G \gens \nil \mid \gax$ 
\paragraph{Domain Function}
\begin{itemize}
    \item $\dom{\nil} = \emptyset$
    \item $\dom{\gax} =  \dom{\G}  \cup \left\{x \right\}$
\end{itemize}
\paragraph{$\okt$ Predicate}
\begin{itemize}
    \item $\ntreerulez{\envnil}{\ok{\nil}}$
    \item $\ntreeruleII{\envextend}{\ok{\G} }{ x\notin \dom{\G}}{\ok{\gax}}$
\end{itemize}



\subsection{Type Rules}
\paragraph{Typing Rules}
\begin{itemize}
    \item $\truleconst$
    \item $\truleunit$
    \item $\truletrue$
    \item $\trulefalse$
    \item $\ntreeruleI{\vvar}{\ok{\gax}}{\typerelation{\gax}{x}{A}}$
    \item $\condtreeruleI{\vweaken}{\typerelation{\G}{x}{A}}{\typerelation{\gby}{x}{A}}{x \neq y}$
    \item $\ntreeruleI{\vfun}{\typerelation{\gax}{v}{B}}{\typerelation{\G}{\lam{x}{A}{v}}{\ab}}$
    \item $\ntreeruleII{\vsubtype}{\typerelation{\G}{v}{A}}{ A \subtype B}{\typerelation{\G}{v}{B}}$
    \item $\ntreeruleI{\vreturn}{\typerelation{\G}{v}{A}}{\typerelation{\G}{\return{v}}{\moa}}$
    \item $\ntreeruleII{\vapply}{\typerelation{\G}{v_1}{\ab}}{\typerelation{\G}{v_2}{A}}{\typerelation{\G}{\apply{v_1}{v_2}}{B}}$
    \item $\ntreeruleIII{\vif}{\typerelation{\G}{v}{\B} }{ \typerelation{\G}{v_1}{ A}}{\typerelation{\G}{v_2}{ A}}{\typerelation{\G}{\pifthenelse{A}{V}{v_1}{v_2}}{ A}}$
    \item $\ntreeruleII{\vbind}{\typerelation{\G}{v_1}{\M{\e_1}{A}} }{ \typerelation{\gax}{v_2}{\M{\e_2}{B}}}{\typerelation{\G}{\doin{x}{v_1}{v_2}}{\M{\e_1 \dot \e_2}{B}}}$
\end{itemize}

\subsection{Ok Lemma}

\begin{lemma}[Ok Lemma]\label{OkLemma}
    If $\gtyperelation{v}{A}$ then $\ok{\G}$.
\end{lemma}


\begin{framed}
    \begin{proof}
        If $\ok{\gax}$ then by inversion $\ok{\G}$
        Only the type rule \textit{\vweaken} adds terms to the environment from its preconditions to its post-condition and it does so in an $\okt$ preserving way. Any type derivation tree has at least one leaf. All leaves are axioms which require $\ok{\G}$. And all non-axiom derivations preserve the $\okt$ property.
    \end{proof}
\end{framed}

\chapter{Category Requirements}
$\C$ should be an S-Category instantiated with the relevant subtyping and subeffecting morphisms and natural transformations.

\chapter{Denotations}
\section{Helper Morphisms}
\subsection{Diagonal and Twist Morphisms}

In the definition and proofs (Especially of the the If cases), I make use of the morphisms twist and diagonal.

\begin{align*}
    \twist{A}{B}: (A\times B) \rightarrow (B\times A) = \pr{\pp}{\p}\\
    \diag{A}: A \rightarrow (A\times A)  = \pr{\Id{A}}{\Id{A}}
\end{align*}

\section{Denotations of Types}
\subsection{Denotation of Ground Types}
The denotations of the default ground types, $\U, \B$ should be as follows:
\begin{equation}
    \deno{\U} = \1
\end{equation}
\begin{equation}
    \deno{\B} = \1 + \1
\end{equation}

The mapping $\deno{\_}$ should then map each other ground type $\g$ to an object $\deno{\g}$ in $\C$.

\subsection{Denotation of Computation Types}
Given a function $\deno{\_}$ mapping value types to objects in the category $\C$, we write the denotation of Computation types $\mea$ as so:

$$\deno{\mea} = \T{\e}{\deno{A}}$$

Since we can infer the denotation function, we can include it implicitly an drop the denotation sign.

$$\deno{\mea} = \tea$$

\subsection{Denotation of Function Types}
Given a function $\deno{\_}$ mapping types to objects in the category $\C$, we write the denotation of a function type $\ab$ as so:
$$\deno{\ab} = \deno{B}^{\deno{A}}$$

Again, since we can infer the denotation function, Let us drop the denotation syntax.

$$\deno{\ab} = (B)^A$$

\subsection{Denotation of Type Environments}
Given a function $\deno{\_}$ mapping types to objects in the category $\C$, we can define the denotation of an $\okt$ type environment $\G$.
$$\deno{\nil} = \1$$
$$\deno{\gax} = (\deno{\G}\times\deno{A}) $$

For ease of notation, and since we normally only talk about one denotation function at a time, I shall typically drop the denotation notation when talking about the denotation of value types and type environments. Hence, $$\deno{\gax} = \G \times A$$


\section{Denotation of Terms}
Given the denotation of types and typing environments, we can now define denotations of well typed terms.

$$\deno{\gtyperelation{v}{A}}: \G \rightarrow A$$

Denotations are defined recursively over the typing derivation of a term. Hence, they implicitly depend on the exact derivation used. Since, as proven in the chapter on the uniqueness of derivations, the denotations of all type derivations yielding the same type relation $\gtyperelation{v}{A}$ are equal, we need not refer to the derivation that yielded each denotation.

\subsection{Denotation of Terms}
\begin{itemize}
    \item $\ntreeruleI{\vunit}{\ok{\G}}{\deno{\typerelation{\G}{\u}{\U}} = \term{\G} : \G \rightarrow \1}$
        
    \item $\ntreeruleI{\vconst}{\ok{\G}}{\deno{\typerelation{\G}{\const{A}}{A}} = \deno{\const{A}} \after \term{\G} : \G \rightarrow A}$
         
    \item $\ntreeruleI{\vtrue}{\ok{\G}}{\deno{\typerelation{\G}{\t}{\B}} = \inl \after \term{\G} : \G \rightarrow \deno{\B} = \1+\1}$
        
    \item $\ntreeruleI{\vfalse}{\ok{\G}}{\deno{\typerelation{\G}{\f}{\B}} = \inr \after \term{\G} : \G \rightarrow \deno{\B} = \1+\1}$
        
    \item $\ntreeruleI{\vvar}{\ok{\G}}{\deno{\typerelation{\gax}{x}{A}} = \pp: \G \times A \rightarrow A}$
    \item $\ntreeruleI{\vweaken}{f = \deno{\gtyperelation{x}{A}}: \G \rightarrow A}{\deno{\typerelation{\gby}{x}{A}} = f \after \p: \G \times B \rightarrow A}$
    \item $\ntreeruleI{\vfun}{f = \deno{\typerelation{\gax}{v}{B}} : \G \times A \rightarrow B}{\deno{\typerelation{\G}{\lam{x}{A}{v}}{A \rightarrow B}} = \cur{f} : \G \rightarrow (B)^A}$
    
    \item $\ntreeruleII{\vsubtype}{f = \deno{\typerelation{\G}{v}{A}} : \G \rightarrow A}{ g = \deno{A \subtype B}}{\deno{\typerelation{\G}{v}{B}} = g \after f : \G \rightarrow B}$
  
    \item $\ntreeruleI{\vreturn}{f = \deno{\typerelation{\G}{v}{A}}}{\deno{\typerelation{\G}{\return{v}}{\moa}} = \point{A} \after f}$
        
 
    \item $\ntreeruleIII{\vif}{f = \deno{\typerelation{\G}{v}{\B}}: \G\rightarrow\1+\1 }{ g = \deno{\typerelation{\G}{v_1}{ A}}}{ h = \deno{\typerelation{\G}{v_2}{ A}}}{\deno{{\typerelation{\G}{\pifthenelse{A}{v}{v_1}{v_2}}{ A}}} = \app\after((\fld{\cur{g\after\pp}}{\cur{h\after\pp}}\after f)\times \idg)\after \diag{\G} : \G \rightarrow A}$
        
    \item $\ntreeruleII{\vbind}{f = \deno{\typerelation{\G}{v_1}{\M{\e_1}{A}} : \G \rightarrow \T{\e_1}{A}}}{{ g = \deno{\typerelation{\gax}{v_2}{\M{\e_2}{B}}}}: \G \times A \rightarrow \T{\e_2}{B}}{\deno{\typerelation{\G}{\doin{x}{v_1}{v_2}}{\M{\e_1 \dot \e_2}{B}}} = \bind{\e_1}{\e_2}{B} \after \T{\e_1}{g} \after \tstrength{\G}{A}{\e_1} \after \pr{\idg}{f}: \G \rightarrow \T{\e_1 \dot \e_2}{B}}$ 
    
    \item $\ntreeruleII{\vapply}{f = \deno{\gtyperelation{v_1}{\ab}}: \G \rightarrow (B)^{A} }{ g=\deno{\gtyperelation{v_2}{A}}: \G \rightarrow A}{\deno{\gtyperelation{\apply{v_1}{v_2}}{B}}= \app\after\pr{f}{g}: \G \rightarrow B }$
\end{itemize}  

\chapter{Unique Denotations}
 
\section{Reduced Type Derivation}
A reduced type derivation is one where instances of the subtype rule must, and may only, occur at the root or directly above an \textit{\vif}, or \textit{\vapply} rule.

In this section, I shall prove that there is at most one reduced derivation of $\gtyperelation{v}{A}$. Secondly, I shall present a function for generating reduced derivations from arbitrary typing derivations, in a way that does not change the denotations. These imply that all typing derivations of a type-relation have the same denotation.

\section{Reduced Type Derivations are Unique}

\begin{theorem}[Reduced Type Derivations are Unique]
    For each instance of the relation $\gtyperelation{v}{A}$,there exists at most one reduced derivation of  $\gtyperelation{v}{A}$.  
\end{theorem}



\begin{framed}
    \begin{proof}
        This is proved by induction over the typing rules on the bottom rule used in each derivation.
        \subsection{Variables}
        To find the unique derivation of $\gtyperelation{x}{A}$, we case split on the type-environment, $\G$.
        
        \case{$\G = \G', x: A'$}
        Then the unique reduced derivation of $\gtyperelation{x}{A}$ is, if $A' \subtype A$, as below:
        
        \begin{equation}
            \ntreeruleII{\vsubtype}{\ntreeruleI{\vvar}{\ok{\G', x: A'}}{\typerelation{\G,x: A'}{x}{A'}}}{ A' \subtype A}{\typerelation{\G', x: A'}{x}{A}}
        \end{equation}
        
        \case{$\G = \G', y: B$} with $y \neq x$.
        
        Hence, if $\gtyperelation{x}{A}$ holds, then so must $\typerelation{\G'}{x}{A}$.
        
        Let 
        \begin{equation}
            \deltavrule{\G'}{x}{A'}{A}
        \end{equation}
        Be the  unique reduced derivation of $\typerelation{\G'}{x}{A}$.
        
        Then the unique reduced derivation of $\gtyperelation{x}{A}$ is:
        
        
        \begin{equation}
            \ntreeruleII{\vsubtype}{
                \ntreeruleI{\vweaken}{
                    \treeruleI{\D}{\typerelation{\G, x: A'}{x}{A'}}
                }{\gtyperelation{x}{A'}}   
            }{ A' \subtype A
            }{\gtyperelation{x}{A}}
        \end{equation}
        
        \subsection{Constants}
        For each of the constants, ($\const{A}$, $\t$, $\f$, $\u$), there is exactly one possible derivation for $\typerelation{\G}{c}{A}$ for a given A. I shall give examples using the case $\const{A}$
        
        
        $$
            \ntreeruleII{\vsubtype}{\truleconst}{ A \subtype B}{\gtyperelation{\const{A}}{B}}
        $$
        
        If $A = B$, then the subtype relation is the identity subtype ($A \subtype A$).
        
        \case{\vfun}
        The reduced derivation of $\gtyperelation{\lam{x}{A}{v}}{\fntype{A'}{B'}}$ is:
        
        
        $$
        \ntreeruleII{\vsubtype}
        {\ntreeruleI{\vfun}{\treeruleI{\D}{\typerelation{\gax}{v}{B}}}
        {
            \gtyperelation{\lam{x}{A}{B}}{\ab}}
            }{
            \ab \subtype \fntype{A'}{B'}
        }{
           \gtyperelation{\lam{x}{A}{v}}{\fntype{A'}{B'}} 
        }
        $$
        
        Where 
        \begin{equation}
            \ntreeruleII{\vsubtype}{\treeruleI{\D}{\typerelation{\gax}{v}{B}}}{ B\subtype B'}{\typerelation{\gax}{v}{B'}}
        \end{equation}
        is the reduced derivation of $\typerelation{\gax}{v}{\M{\e'}{B}}$ if it exists.
        
        \case{\vreturn}
        The reduced denotation of $\gtyperelation{\return{v}}{\meb}$ is 
        $$
            \ntreeruleII{\vsubtype}
            {
                \ntreeruleI{\vreturn}
                {\treeruleI{\D}{\gtyperelation{v}{A}}}
                {\gtyperelation{\return{v}}{\moa}}
                }{
                    \subeffecttree{\1}{A}{\e}{B}
            }
            {\gtyperelation{\return{v}}{\meb}}
        $$
        
        Where $$\deltavrule{\G}{v}{A}{B}$$ is the reduced derivation of $\gtyperelation{v}{B}$
        
        \case{\vapply}
        If 
        $$
            \deltavrule{\G}{v_1}{\ab}{\fntype{A'}{B'}}
        $$ and $$
            \deltavruleprime{\G}{v_2}{A''}{A'}
        $$
        
        Are the reduced type derivations of $\gtyperelation{v_1}{\fntype{A'}{B'}}$ and $\gtyperelation{v_2}{A'}$
        
        
        
        Then we can construct the reduced derivation of $\gtyperelation{\apply{v_1}{v_2}}{B'}$ as
        
        $$
            \ntreeruleII{\vsubtype}{
                \ntreeruleII{\vapply}{
                    \treeruleI{\D}{
                        \gtyperelation{v_1}{\ab}
                    }
                    }{
                    \ntreeruleII{\vsubtype}{
                        \treeruleI{\D'}{
                            \gtyperelation{v}{A''}
                        } }{ A'' \subtype A
                    }
                    {\gtyperelation{v}{A}}
                }{
                    \gtyperelation{\apply{v_1}{v_2}}{B}
                }
                }{
                B \subtype B'
                }{
                \gtyperelation{\apply{v_1}{v_2}}{B'}
            }
        $$
        \case{\vif}
        Let
        
        \begin{equation}
            \deltavrule{\G}{v}{B}{\B}
        \end{equation}
        
        \begin{equation}
            \deltacruleprime{\G}{v_1}{\e'}{A'}{\e}{A}
        \end{equation}
        
        \begin{equation}
            \deltacruleprimeprime{\G}{v_2}{\e''}{A''}{\e}{A}
        \end{equation}
        
        Be the unique reduced reduced derivations of $\gtyperelation{v}{\B}$, $\gtyperelation{v_1}{ A}$, $\gtyperelation{v_2}{ A}$.
        
        Then the only reduced derivation of $\gtyperelation{\pifthenelse{A}{v}{v_1}{v_2}}{ B}$ is:
                
        \begin{equation}
        \scalebox{.8}{$
            \ntreeruleII{\vsubtype}{
                \ntreeruleIII{\vif}{
                    \scalebox{.5}{$\deltavrule{\G}{v}{B}{\B}$}
                }{
                    \scalebox{.5}{$\deltacruleprime{\G}{v_1}{\e'}{A'}{\e}{A}$}
                }{
                    \scalebox{.5}{$\deltacruleprimeprime{\G}{v_2}{\e''}{A''}{\e}{A}$}
                }{
                    \gtyperelation{\pifthenelse{A}{v}{v_1}{v_2}}{ A} 
                } 
            }{
                A\subtype B
            }{
                \gtyperelation{\pifthenelse{A}{v}{v_1}{v_2}}{B}
                }
                $}
        \end{equation}
        
        \case{\vbind}
        
        Let 
        
        \begin{equation}
            \deltacrule{\G}{v_1}{\e_1}{A}{\e_1'}{A'}
        \end{equation}
        
        \begin{equation}
            \deltacruleprime{\G, x:A}{v_2}{\e_2}{B}{\e_2'}{B'}
        \end{equation}
        
        Be the respective unique reduced type derivations of the subterms]
        
        By weakening, $\wrel{\i\x}{\G, x:A}{\G, x: A'}$ so if there is a derivation of $\typerelation{\G, x:A'}{v_2}{\M{\e_2}{B}}$, there is also one of   $\typerelation{\gax}{v_2}{\M{\e_2}{B}}$.
        
        \begin{equation}
            \deltacruleprimeprime{\G, x:A'}{v_2}{\e_2}{B}{\e_2'}{B'}
        \end{equation}
        
        Since the effects monoid operation is monotone, if $\e_1\subeffect\e_1'$ and $\e_2\subeffect\e_2'$ then $\e_1\dot\e_2 \subeffect \e_1'\dot\e_2'$
        
        Hence the reduced type derivation of $\gtyperelation{\doin{x}{v_1}{v_2}}{\M{\e_1'\dot\e_2'}{B'}}$ is the following:
        
        \begin{equation}
            \ntreeruleII{\vsubtype} {\scalebox{0.5}{$
                \ntreeruleII{\vbind}{
                    \scalebox{0.8}{$
                    \deltacrule{\G}{v_1}{\e_1}{A}{\e_1'}{A'}
                    $}}{\scalebox{0.8}{$
                    \deltacruleprimeprime{\G, x:A'}{v_2}{\e_2}{B}{\e_2'}{B'}
                    $}
                } {
                    \gtyperelation{\doin{x}{v_1}{v_2}}{\M{\e_1\dot\e_2}{B}}
                }
                $}
            }{\scalebox{0.5}{$
                    \subeffecttree{\e_1\dot\e_2}{B}{\e_1'\dot\e_2'}{B'}     
            $}}{
                \gtyperelation{\doin{x}{v_1}{v_2}}{\M{\e_1'\dot\e_2'}{B'}}
            }
        \end{equation}
        
    \end{proof}
    
\end{framed}

\section{Each Type Derivation has a Reduced Equivalent with the Same Denotation}

\begin{theorem}[Each Type Derivation has a Reduced Equivalent with the Same Denotation]
    We introduce a function, $\reduce$ that maps each valid type derivation of $\gtyperelation{v}{A}$ to a reduced equivalent with the same denotation. To do this, we do case analysis over the root type rule of a derivation and prove that the denotation is not changed.
\end{theorem}


\begin{framed}
\begin{proof}
    
        
            \subsection{Constants}
                For the constants $\t, \f, \const{A}$, etc, $\reduce$ simply returns the derivation, as it is already reduced. This trivially preserves the denotation.
        
                $\reduce(\truleconst) = \truleconst$
        
                \case{\vvar}
                \begin{equation}
                    \reduce(\ntreeruleI{\vvar}{\ok{\G}}{\typerelation{\gax}{x}{A}}) =  \ntreeruleI{\vvar}{\ok{\G}}{\typerelation{\gax}{x}{A}}
                \end{equation}
        
                Preserves denotation trivially.
        
                \case{\vweaken}
                \subparagraph{$\reduce$ definition}
                To find:
                \begin{equation}
                    \reduce(\ntreeruleI{\vweaken}{\treeruleI{\D}{\gtyperelation{x}{A}}}{\typerelation{\gby}{x}{A}})
                \end{equation}
        
                Let 
                \begin{equation}\label{WeakenDeltaReduction}
                    \ntreeruleII{\vsubtype}{\treeruleI{\D'}{\gtyperelation{x}{A}}}{ A'\subtype A}{\gtyperelation{x}{A}} = \reduce(\D)
                \end{equation}
        
                In 
                \begin{equation}
                    \ntreeruleII{\vsubtype}{
                    \ntreeruleI{\vweaken}{
                        \treeruleI{\D'}{\gtyperelation{x}{A'}}
                    } {
                        \typerelation{\gby}{x}{A'}
                    }
                    }{
                    A' \subtype A
                    }{\typerelation{\gby}{x}{A}}
                \end{equation}
        
                \subparagraph{Preserves Denotation}
                Using the construction of denotations, we can find the denotation of the original derivation to be:
                \begin{equation}
                    \deno{\ntreeruleI{\vweaken}{\treeruleI{\D}{\gtyperelation{x}{A}}}{\typerelation{\gby}{x}{A}}} = \D \after \p
                \end{equation}
        
                Similarly, the denotation of the reduced denotation is:
                \begin{equation}
                    \deno{\ntreeruleII{\vsubtype}{
                        \ntreeruleI{\vweaken}{
                            \treeruleI{\D'}{\gtyperelation{x}{A'}}
                        } {
                            \typerelation{\gby}{x}{A'}
                        }
                        }{
                        A' \subtype A
                        }{\typerelation{\gby}{x}{A}}} = \deno{A'\subtype A}\after \D' \after \p
                \end{equation}
        
        
                By induction on $\reduce$ preserving denotations and the reduction of $\D$ (\ref{WeakenDeltaReduction}), we have:
        
                \begin{equation}
                    \D = \deno{A' \subtype A}\after\D'
                \end{equation}
        
                So the denotations of the un-reduced and reduced derivations are equal.
        
                \case{\vfun}
                \subparagraph{$\reduce$ definition}
                    To find:
                
                    \begin{equation}
                        \reduce(\ntreeruleI{\vfun}{
                            \treeruleI{\D}{\typerelation{\gax}{v}{\M{\e_2}{B}}}
                        }{\gtyperelation{\lam{x}{A}{v}}{\ab}})
                    \end{equation}
        
                    Let 
        
                    \begin{equation}
                        \ntreeruleII{\vsubtype}{
                            \treeruleI{\D'}{\typerelation{\gax}{v}{B'}}
                            }{B\subtype B'}{
                            \typerelation{\gax}{v}{B}
                        } = \reduce(\D)
                    \end{equation}
        
                    In
        
                    \begin{equation}
                        \ntreeruleII{\vsubtype}{
                            \ntreeruleI{\vfun}{
                                \D'
                            }{
                                \typerelation{\gax}{v}{\M{\e_1}{B'}}
                            }
                            }{
                            \fntype{A}{B'}\subtype\ab
                        } {
                            \gtyperelation{\lam{x}{A}{v}}{\ab}
                        }
                    \end{equation}
        
                \subparagraph{Preserves Denotation}
                    Let
                    \begin{align*}
                        f &= \deno{\M{\e_1}{B'}\subtype\M{\e_2}{B}} = \deno{\e_1\subeffect\e_2}_B\after\T{\e_1}{(\deno{B'\subtype B})} \\
                        \deno{\fntype{A}{B'}\subtype\ab} & = f^A = \cur{f\after \app}
                    \end{align*}
        
                    Then
        
                    \begin{align*}
                        before & = \cur{\D} \qt{By definition}\\
                                & = \cur{f\after\D'} \qt{By reduction of $\D$} \\
                                & = f^A\after\cur{\D'}\qt{By the property of $f^X\after\cur{g} = \cur{f\after g}$} \\
                                & = after \qt{By definition}\\
                    \end{align*}
        
                \case{\vsubtype}
                \subparagraph{$\reduce$ definition}
                To find:
                \begin{equation}
                    \reduce(\ntreeruleII{\vsubtype}{\treeruleI{\D}{\gtyperelation{v}{A}}}{ A \subtype B}{\gtyperelation{v}{B}})
                \end{equation}
        
                Let 
                \begin{equation}\label{SubtypeDeltaReduction}
                    \ntreeruleII{\vsubtype}{\treeruleI{\D'}{\gtyperelation{x}{A}}}{ A'\subtype A}{\gtyperelation{x}{A}} = \reduce(\D)
                \end{equation}
        
                In 
                \begin{equation}
                    \ntreeruleII{\vsubtype}{
                        \treeruleI{\D'}{\gtyperelation{v}{A'}}
                    }{
                    A' \subtype A \subtype B
                    }{\gtyperelation{v}{B}}
                \end{equation}
        
                \subparagraph{Preserves Denotation}
                    \begin{align*}
                        before & = \deno{A \subtype B} \after \D \\
                        & = \deno{A \subtype B} \after (\deno{A' \subtype A} \after \D') \qt{ byDenotation of reduction of $\D$.}\\
                        & = \deno{A' \subtype B} \after \D'\qt{Subtyping relations are unique} \\
                        & = after \\
                    \end{align*}
            
                \case{\vreturn}
                \subparagraph{$\reduce$ definition}
                To find:
                
                    \begin{equation}
                        \reduce(\ntreeruleI{\vreturn}{
                            \treeruleI{\D}{\gtyperelation{v}{A}}
                        }{\gtyperelation{\return{v}}{\moa}})
                    \end{equation}
        
                    Let 
        
                    \begin{equation}
                        \ntreeruleII{\vsubtype}{
                            \treeruleI{\D'}{\gtyperelation{v}{A'}}
                            }{
                            A' \subtype A
                        }{
                            \gtyperelation{v}{A}
                        } = \reduce(\D)
                    \end{equation}
        
                    In
        
                    \begin{equation}
                        \ntreeruleII{\vsubtype}{
                            \ntreeruleI{\vreturn}{
                                \D'
                            }{
                                \gtyperelation{v}{A'}
                            }
                            }{\subeffecttree{\1}{A'}{\1}{A}
                        } {
                            \gtyperelation{\return{v}}{\moa}
                        }
                    \end{equation}
        
        
                    Then
                    \begin{align*}
                        before &= \point{A}\after\D\qt{By definition}\qt{By defintion}\\
                        &{}= \point{A}\after\deno{A'\subtype A}\after \D'\qt{BY reduction of $\D$}\\
                        &{}=\T{\1}{\deno{A'\subtype A}}\after \point{A'}\after\D'\qt{By naturality of $\point{}$}\\
                        &{}=\deno{\1\subeffect\1}_{A} \after \T{\1}{\deno{A'\subtype A}}\after\point{A'}\after\D'\qt{Since $\deno{\1\subeffect\1}$ is the identity Nat-Trans }\\
                        &{}=after\qt{By definition}\\
                    \end{align*}
                \case{\vapply}
                \subparagraph{$\reduce$ definition}
                To find:
                \begin{equation}
                    \reduce(\ntreeruleII{\vapply}{
                        \treeruleI{\D_1}{
                            \gtyperelation{v_1}{\ab}
                        }
                        }{
                        \treeruleI{\D_2}{
                            \gtyperelation{v_2}{A}
                        }
                    }{
                        \gtyperelation{\apply{v_1}{v_2}}{B}
                    })
                \end{equation}
        
                Let
                \begin{align*}
                    \ntreeruleII{\vsubtype}{
                        \treeruleI{\D'_1}{\gtyperelation{v_1}{\fntype{A'}{B'}}}
                        }{
                        \fntype{A'}{B'}\subtype\ab
                    }{
                        \gtyperelation{v_1}{\ab}
                    } & = \reduce(\D_1)\\
                    \ntreeruleII{\vsubtype}{
                        \treeruleI{\D'_2}{\gtyperelation{v}{A'}}
                        }{
                        A'\subtype A
                    } {
                        \gtyperelation{v_1}{A}
                    } & = \reduce(\D_2)
                \end{align*}
        
                In
                \begin{equation}
                    \ntreeruleII{\vsubtype}{
                        \ntreeruleII{\vapply}{
                            \treeruleI{
                                \D'_1
                            }{
                                \gtyperelation{v_1}{\fntype{A'}{B'}}
                            }
                        }{
                            \ntreeruleII{\vsubtype}{
                                \treeruleI{\D'_2}{\gtyperelation{v_2}{A''}}
                                }{
                                A'' \subtype A \subtype A'
                            } {
                                \gtyperelation{v_2}{A'}
                            }
                        }{
                            \gtyperelation{\apply{v_1}{v_2}}{B'}
                        }
                        }{
                        B' \subtype B
                    }{
                        \gtyperelation{\apply{v_1}{v_2}}{B}
                    }
                \end{equation}
                \subparagraph{Preserves Denotation}
                    Let
                    \begin{align*}
                        f & = \deno{A\subtype A'}: A\rightarrow A' \\
                        f' & = \deno{A''\subtype A}: A'' \rightarrow A \\
                        g & = \deno{B' \subtype B}: B' \rightarrow B \\
                        h & = \deno{\e' \subeffect \e}: \T{\e'}{} \rightarrow \T{\e}{}
                    \end{align*}
        
                    Hence 
                    \begin{align*}
                        \deno{\fntype{A'}{B'}\subtype \ab} & = (h_B\after\T{\e'}{g})^A \after (\T{\e'}{B'})^f \\
                        & = \cur{h_B\after\T{\e'}{g}\after \app}\after\cur{\app\after(\Id{}\times f)}\\
                        & = \cur {h_B\after\T{\e'}{g}\after\app\after(\Id{}\times f)}
                    \end{align*}
        
                    Then 
                    \begin{align*}
                        before & = \app\after\pr{\D_1}{\D_2}\qt{By definition}\\
                        & = \app\after\pr{\cur {h_B\after\T{\e'}{g}\after\app\after(\Id{}\times f)}\after\D'_1}{f'\after\D'_2}\qt{By reductions of $\D_1, \D_2$}\\
                        & = \app\after(\cur {h_B\after\T{\e'}{g}\after\app\after(\Id{}\times f)}\times\Id{A})\after\pr{\D'_1}{f'\after\D'_2} \qt{Factoring out}\\
                        & = h_B\after\T{\e'}{g}\after\app\after(\Id{}\times f)\after\pr{\D'_1}{f'\after\D'_2}\qt{By the exponential property}\\
                        & = h_B\after\T{\e'}{g}\after\app\after\pr{\D'_1}{f\after f'\after \D'_2}\\
                        & = after\qt{By defintion}
                    \end{align*}
                \case{\vif}
               
                \subparagraph{$\reduce$ definition}
                \begin{equation}
                    \scalebox{.95}{%
                        \reduce(\ntreeruleIII{\vif}{
                            \treeruleI{\D_1}{\gtyperelation{v}{\B}}
                            }{
                            \treeruleI{\D_2}{\gtyperelation{v_1}{ A}}
                            }{
                            \treeruleI{\D_3}{\gtyperelation{v_2}{ A}}
                        }{
                            \gtyperelation{\pifthenelse{A}{v}{v_1}{v_2}}{ A}
                        }) = \ntreeruleIII{\vif}{
                            \treeruleI{\reduce(\D_1)}{\gtyperelation{v}{\B}}
                            }{
                            \treeruleI{\reduce(\D_2)}{\gtyperelation{v_1}{ A}}
                            }{
                            \treeruleI{\reduce(\D_3)}{\gtyperelation{v_2}{ A}}
                        }{
                            \gtyperelation{\pifthenelse{A}{v}{v_1}{v_2}}{ A}
                        }
                        }
                    \end{equation}
                
        
                \subparagraph{Preserves Denotation}
                    Since calling $\reduce{}$ on the sub-derivations preserves their denotations, this definition trivially preserves the denotation of the derivation.
        
        
                    \case{\vbind}
                    \subparagraph{$\reduce$ definition}
        
                    To find
                    \begin{equation}
                        \reduce(
                            \ntreeruleII{\vbind}{
                                \treeruleI{
                                    \D_1
                                }{
                                    \gtyperelation{v_1}{\M{\e_1}{A}}
                                }
                                }{
                                \treeruleI{
                                    \D_2
                                }{
                                    \typerelation{\gax}{v_2}{\M{\e_2}{B}}
                                }
                            } {
                                \gtyperelation{\doin{x}{v_1}{v_2}}{\M{\e_1\dot\e_2}{B}}
                            }
                        )
                    \end{equation}
        
                    Let \begin{equation}
                        \ntreeruleII{\vsubtype}{
                            \treeruleI{\D'_1}{\gtyperelation{v_1}{\M{\e'_1}{A'}}}
                        }{
                            \subeffecttree{\e_1'}{A'}{\e_1}{A}
                        }{
                            \gtyperelation{v_1}{\M{\e_1}{A}}
                        } = \reduce(\D_1)
                    \end{equation}
        
                    Since $\wrel{i,\x}{\G, x: A'}{\gax}$ if $A' \subtype A$, and by $\D_2$, $\typerelation{(\gax)}{v_2}{\M{\e_2}{B}}$, there also exists a derivation $\D_3$ of $\typerelation{(\G, x: A')}{v_2}{\M{\e_2}{B}}$. $\D_3$ is derived from $\D_2$ simply by inserting a (Subtype) rule below all instances of the (Var) rule.
        
                    Let \begin{equation}
                        \ntreeruleII{\vsubtype}{
                            \treeruleI{\D'_3}{\typerelation{\G, x: A'}{v_2}{\M{\e'_2}{B'}}}
                        }{
                            \subeffecttree{\e_2'}{B'}{\e_2}{B}
                        }{
                            \typerelation{\G, x: A'}{v_2}{\M{\e_2}{B}}
                        } = \reduce(\D_3)
                    \end{equation}
                    
        
                    Since the effects monoid operation is monotone, if $\e_1\subeffect\e_1'$ and $\e_2\subeffect\e_2'$ then $\e_1\dot\e_2 \subeffect \e_1'\dot\e_2'$
                    
        
        
                    Then the result of reduction of the whole bind expression is:
        
        
                    \begin{equation}
                        \ntreeruleII{\vsubtype}{
                            \ntreeruleII{\vbind}{
                                \treeruleI{
                                    \D'_1
                                }{
                                    \gtyperelation{v_1}{\M{\e_1'}{A'}}
                                }
                                }{
                                \treeruleI{
                                    \D'_3
                                }{
                                    \typerelation{\G, x: A'}{v_2}{\M{\e_2'}{B'}}
                                }
                            }{
                            \gtyperelation{\doin{x}{v_1}{v_2}}{\M{\e_1'\dot\e_2'}{B}}
                            }
                            }{
                                \subeffecttree{\e_1'\dot\e_2'}{B'}{\e_1\dot\e_3}{B}
                        }{
                            \gtyperelation{\doin{x}{v_1}{v_2}}{\M{\e_1\dot\e_2}{B}}
                        }
                    \end{equation}
                    \subparagraph{Preserves Denotation}
        
                    Let \begin{align*}
                        f & = \deno{A' \subtype A}: A' \rightarrow A\\
                        g & = \deno{B' \subtype B}: B' \rightarrow B\\
                        h_1 & = \deno{\e_1' \subeffect \e_1} : \T{\e_1'}{} \rightarrow \T{\e_1}{} \\
                        h_2 & = \deno{\e_2'\subeffect \e_2}:\T{\e_2'}{} \rightarrow \T{\e_2}{}\\
                        h & = \deno{\e_1'\dot\e_2'\subeffect\e_1\dot\e_2}: \T{\e_1'\dot\e_2'}{}\rightarrow \T{\e_1\dot\e_2}{}
                    \end{align*}
        
                    Due to the denotation of the weakening used to derive $\D_3$ from $\D_2$, we have 
                    \begin{equation}
                        \D_3 = \D_2\after(\idg\times f)
                    \end{equation}
        
                    And due to the reduction of $\D_3$,
                    we have 
                    \begin{equation}
                        \D_3 = h_{2, B} \after \T{\e_2'}{g}\after \D_3'
                    \end{equation}
        
                    So:
        
                    \begin{align*}
                        before &= \bind{\e_1}{\e_2}{B}\after \T{\e_1}{\D_2}\after\tstrength{\e_1}{\G}{A}\after\pr{\idg}{\D_1}\qt{By definition.}\\
                        &= \bind{\e_1}{\e_2}{B}\after \T{\e_1}{\D_2}\after\tstrength{\e_1}{\G}{A}\after\pr{\idg}{h_{1, A}\after\T{\e_1'}{f}\after\D_1'}\qt{By reduction of $\D_1$.}\\
                        &= \bind{\e_1}{\e_2}{B}\after \T{\e_1}{\D_2}\after\tstrength{\e_1}{\G}{A}\after(\idg\times h_{1, A})\after\pr{\idg}{\T{\e_1'}{f}\after\D_1'}\qt{Factor out $h_1$}\\
                        &= \bind{\e_1}{\e_2}{B}\after \T{\e_1}{\D_2}\after
                        h_{1, (\G\times A)}\after
                        \tstrength{\e_1'}{\G}{A}\after\pr{\idg}{\T{\e_1'}{f}\after\D_1'}\qt{Tensor strength and subeffecting $h_1$}\\
                        &= \bind{\e_1}{\e_2}{B}\after 
                        h_{1, B}\after\T{\e_1'}{\D_2}\after
                        \tstrength{\e_1'}{\G}{A}\after\pr{\idg}{\T{\e_1'}{f}\after\D_1'}\qt{Naturality of $h_1$}\\
                        &= \bind{\e_1}{\e_2}{B}\after 
                        h_{1, B}\after\T{\e_1'}{\D_2}\after
                        \tstrength{\e_1'}{\G}{A}\after(\idg\times \T{\e_1'}{f})\after\pr{\idg}{\D_1'}\qt{Factor out pairing again}\\
                        &= \bind{\e_1}{\e_2}{B}\after 
                        h_{1, B}\after\T{\e_1'}{(\D_2\after(\idg\times f))}\after
                        \tstrength{\e_1'}{\G}{A'}\after\pr{\idg}{\D_1'}\qt{Tensorstrength}\\
                        &= \bind{\e_1}{\e_2}{B}\after 
                        h_{1, B}\after\T{\e_1'}{(\D_3)}\after
                        \tstrength{\e_1'}{\G}{A'}\after\pr{\idg}{\D_1'}\qt{By the definition of $\D_3$}\\
                        &= \bind{\e_1}{\e_2}{B}\after 
                        h_{1, B}\after\T{\e_1'}{(h_{2, B}\after\T{\e_2'}{g}\after \D_3')}\after
                        \tstrength{\e_1'}{\G}{A'}\after\pr{\idg}{\D_1'}\qt{By the reduction of $\D_3$}\\
                        &= \bind{\e_1}{\e_2}{B}\after 
                        h_{1, B}\after\T{\e_1'}{h_{2, B}}\after\T{\e_1'}{\T{\e_2'}{g}}\after \T{\e_1'}{\D_3'}\after
                        \tstrength{\e_1'}{\G}{A'}\after\pr{\idg}{\D_1'}\qt{Factor out the functor}\\
                        &= h_B\after\bind{\e_1'}{\e_2'}{B}\after\T{\e_1'}{\T{\e_2'}{g}}\after \T{\e_1'}{\D_3'}\after
                        \tstrength{\e_1'}{\G}{A'}\after\pr{\idg}{\D_1'}\qt{By the $\mu$ and Subtype rule }\\
                        & = h_B\after\T{\e_1'\dot\e_2'}{g}\after\bind{\e_1'}{\e_2'}{B'}\after \T{\e_1'}{\D_3'}\after
                        \tstrength{\e_1'}{\G}{A'}\after\pr{\idg}{\D_1'}\qt{By naturality of $\bind{}{}{}$ }\\
                        & = after \qt{By definition}
                    \end{align*}
    
\end{proof}
\end{framed}

\section{Denotations are Equivalent}
For each type relation instance $\gtyperelation{v}{A}$ there exists a unique reduced derivation of the relation instance. For all derivations $\D$, $\D'$ of the type relation instance, $\deno{\D} = \deno{\reduce{\D}} = \deno{\reduce{\D'}} = \deno{\D'} $, hence the denotation $\deno{\gtyperelation{v}{A}}$ is unique.


\chapter{Weakening}
\section{Weakening Definition}
\subsection{Relation}
We define the ternary weakening relation $\wrel{w}{\G'}{\G}$ using the following rules.


\begin{itemize}
    \item $\ntreeruleI{\tid}{\ok{\G}}{\wrel{\i}{\G}{\G}}$
    \item $\ntreeruleII{\tproject}{\wrel{\w}{\G'}{\G}}{ x \notin \dom{\G'}}{\wrel{\w \pi}{\G, x: A}{\G}}$
    \item $\ntreeruleIII{\textend}{\wrel{\w}{\G'}{\G}}{ x \notin \dom{\G'}}{ A \subtype B}{\wrel{w \x}{\G', x: A}{\G, x: B}}$
\end{itemize}

\subsection{Weakening Denotations}
The denotation of a weakening relation is defined as follows:

\begin{equation}
    \deno{\wrel{\w}{\G'}{\G}}: \G' \rightarrow \G
\end{equation}

\begin{itemize}
    \item $\deno{\wrel{\i}{\G}{\G}} = \idg: \G \rightarrow \G$
    
    \item $\ntreeruleI{\tproject}{f = \deno{\wrel{\w}{\G'}{\G}}: \G' \rightarrow \G}{\deno{\wrel{\w \pi}{\G, x: A}{\G}} = f\after \p: \G'\times A \rightarrow \G}$
    
    \item $\ntreeruleII{\textend}{f = \deno{\wrel{\w}{\G'}{\G}}: \G' \rightarrow \G }{ g = \deno{A \subtype B}: A \rightarrow B}{\deno{\wrel{w \x}{\G', x: A}{\G, x: B}} = (f\times g): (\G\times A)\rightarrow(\G\times B)}$
\end{itemize}

\section{Weakening Theorems}

\begin{lemma}[Domain Lemma]
    If $\wrel{\w}{\G'}{\G}$, then $\dom{\G}\subseteq \dom{\G'}$.
\end{lemma}



\begin{framed}
    \begin{proof}
        \case{\tid}
        Then $\G' = \G$ and so $\dom{\G'} = \dom{\G}$.
        \case{\tproject}
        By inversion and induction, $\dom{\G}\subseteq\dom{\G'}\subseteq\dom{\G'\cup\left\{x\right\}}$
        \case{\textend}
        By inversion and induction, $\dom{\G}\subseteq\dom{\G'}$ so 
        $$\dom{\gax} = \dom{\G}\cup\left\{x\right\} \subseteq\dom{\G'}\cup\left\{x\right\} = \dom{\G',x:A}$$
    \end{proof}
\end{framed}


\begin{theorem}[Ok Preservation]\label{WeakeningPropertyOne}
    If $\wrel{\w}{\G'}{\G}$ and $\ok{\G}$ then $\ok{\G'}$  
\end{theorem}



\begin{framed}
    \begin{proof}
        
        \case{\tid}
        $$\ntreeruleI{\tid}{\ok{\G}}{\wrel{\i}{\G}{\G}}$$
        
        By inversion, $\ok{\G}$.
        
        \case{\tproject}
        $$\ntreeruleII{\tproject}{\wrel{\w}{\G'}{\G}}{ x \notin \dom{\G'}}{\wrel{\w \pi}{\G, x: A}{\G}}$$
        
        By inversion, $\wrel{\w}{\G'}{\G}$ and $x\notin \dom{\G'}$.
        
        Hence by induction $\ok{\G'}$, $\ok{\G}$. Since $x\notin\dom{\G'}$, we have $\ok{\G', x:A}$.
        
        \case{\textend}
        $\ntreeruleIII{\textend}{\wrel{\w}{\G'}{\G}}{ x \notin \dom{\G'}}{ A \subtype B}{\wrel{w \x}{\G', x: A}{\G, x: B}}$, 
        
        By inversion, we have 
        
        $\wrel{\w}{\G'}{\G}$, $x\notin\dom{\G'}$.
        
        Hence we have $\ok{\G}$, $\ok{\G'}$, and by the domain Lemma, $\dom{\G}\subseteq\dom{\G'}$, hence $x\notin\dom{\G}$. Hence, we have $\ok{\gax}$ and $\ok{\G',x:A}$
        
        
    \end{proof}
\end{framed}



\begin{theorem}[Weakening Preserves Type Relation]
    \label{WeakeningTyping}
    If $\gtyperelation{v}{A}$ and $\wrel{\w}{\G'}{\G}$ then there is a derivation of $\typerelation{\G'}{v}{A}$
\end{theorem}

\begin{proof}
    Proved in parallel with theorem \ref{WeakeningDenotations} below
\end{proof}



\begin{theorem}[Weakening and Denotations]\label{WeakeningDenotations}
    If $\wrel{\w}{\G'}{\G}$ and $\D = \deno{\gtyperelation{v}{A}}$ and $\D' = \deno{\typerelation{\G'}{v}{A}}$, derived using theorem \ref{WeakeningTyping}, then 
    
    $$
        \D \after \deno{\w} = \D' : \G' \rightarrow A
    $$
\end{theorem}

\begin{framed}
    
    \begin{proof}
        
        We induct over the structure of typing derivations of $\gtyperelation{v}{A}$, assuming $\wrel{\w}{\G'}{\G}$ holds. In each case, we construct the new derivation $\D'$ from the derivation $\D$ giving $\gtyperelation{v}{A}$ and show that $\D\after\deno{\wrel{\w}{\G'}{\G}} = \D'$
        
        \subsection{Variable Terms}
        \case{\vvar, \vweaken}
            We case split on the weakening $\w$.
            \subparagraph{If $\w = \i$}
            Then $\G' = \G$, and so $\typerelation{\G'}{x}{A}$ holds and the derivation $\D'$ is the same as $\D$
        
            \begin{equation}
                \D' = \D = \D\after\idg = \D\after\deno{\wrel{\i}{\G}{\G}} 
            \end{equation}
            \subparagraph{If $\w = \w'\pi$}
            Then  $\G' = (\G'',x': A')$ and $\wrel{\w'}{\G''}{\G}$. So by induction, there is a tree, $\D_1$ deriving $\typerelation{\G''}{x}{A}$,  such that 
            \begin{equation}
                \D_1 = \D\after\deno{\wrel{\w'}{\G''}{\G}} \qt{By Induction}
            \end{equation}
            
            , and hence by the weaken rule, we have 
            \begin{equation}
                \ntreeruleI{\vweaken}{\typerelation{\G''}{x}{A}}{\typerelation{\G'', x':A' }{x}{A}}
            \end{equation}
        
            This preserves denotations:
            \begin{align*}
                \D' & = \D_1\after\p\qt{By Definition} \\
                & = \D\after\deno{\wrel{\w'}{\G''}{\G}}\after\p\qt{By induction}\\
                & = \D\after\deno{\wrel{\w'\p}{\G'}{\G}}\qt{By denotation of weakening}
            \end{align*}
        
            \subparagraph{If $\w = \w'\x$} 
            Then 
            \begin{align*}
                \G' & = \G''', x': B\\
                \G &= \G'', x': A'\\
                B & \subtype A
            \end{align*}
        
            \subparagraph{If $x = x'$}
        
            Then $A = A'$.
        
            Then we derive the new derivation, $\D'$ as so:
        
            \begin{equation}
                \ntreeruleII{\vsubtype}{
                    \ntreeruleI{\vvar}{ }{\typerelation{\G''', x: B}{x}{B}}
                    }{
                    B \subtype A
                }{
                    \typerelation{\G'}{x}{A}
                }
            \end{equation}
        
            This preserves denotations:
        
            \begin{align*}
                \D' & = \deno{B\subtype A}\after\pp\qt{By Definition} \\
                 & = \pp\after (\deno{\wrel{\w'}{\G'''}{\G''}}\times \deno{B\subtype A}) \qt{By the properties of binary products}\\
                 & = \D\after\deno{\wrel{\w}{\G'}{\G}}\qt{By Definition}
            \end{align*}
        
            \case{$x \neq x'$}
            Then 
            \begin{equation}
                \D = \ntreeruleI{\vweaken}{\treeruleI{\D_1}{\typerelation{\G''}{x}{A}}}{\gtyperelation{x}{A}}
            \end{equation}
        
            By induction with $\wrel{\w}{\G'''}{\G''}$,
             we have a derivation $\D_1$ of $\typerelation{\G'''}{x}{A}$
        
            We have the weakened derivation:
        
            \begin{equation}
                \D' = \ntreeruleI{\vweaken}{\treeruleI{\D_1'}{\typerelation{\G'''}{x}{A}}}{\typerelation{\G'}{x}{A}}
            \end{equation}
        
            This preserves denotations:
        
            By induction, we have
            \begin{equation}
                \D_1' = \D_1 \after \deno{\wrel{\w}{\G'''}{\G''}}
            \end{equation}
            So we have:
            \begin{align*}
                \D' &= \D_1' \after \p\qt{By denotation definition}\\
                & = \D_1\after\deno{\wrel{\w'}{\G'''}{\G''}}\qt{By induction}\after\p \\
                & = \D_1\after\p\after(\deno{\wrel{\w'}{\G'''}{\G''}}\times\deno{A' \subtype B})\qt{By product properties}\\
                & = \D\after\deno{\wrel{\w}{\G'}{\G}}\qt{By definition}
            \end{align*}
        
        From this point onwards, since we no-longer case split over the weakening relations, we write the denotation $\deno{\wrel{\w}{\G'}{\G'}}$, simply as $\w$.
        
        
        \case{\vconst}
        The constant typing rules, $\u$, $\t$, $\f$, $\const{A}$, all proceed by the same logic. Hence I shall only prove the theorems for the case $\const{A}$.
        
        \begin{equation}
            \truleconst
        \end{equation}
        
        By inversion, we have $\ok{\G}$, so we have $\ok{\G'}$.
        
        Hence
        
        \begin{equation}
            \ntreeruleI{\vconst}{\ok{\G'}}{\typerelation{\G'}{\const{A}}{A}}
        \end{equation}
        Holds.
        
        This preserves denotations:
        
        
        \begin{align*}
            \D' & = \deno{\const{A}} \after \term{\G'}\qt{By definition}\\
            & = \deno{\const{A}} \after \term{\G}\after \w \qt{By the terminal property}\\
            & = \D\qt{By Definition}\\
        \end{align*}
        
        
        \case{\vfun}
        By inversion, we have a derivation $\D_1$ giving
        
        \begin{equation}
            \D = \ntreeruleI{\vfun}{
                \treeruleI{\D_1}{\typerelation{\gax}{v}{B}}
            }{\gtyperelation{\lam{x}{A}{v}}{\ab}}
        \end{equation}
        
        Since $\wrel{\w}{\G'}{\G}$, we have:
        
        \begin{equation}
            \wrel{\w\x}{(\G,x:  A)}{(\gax)}
        \end{equation}
        
        Hence, by induction, using $\wrel{\w\x}{(\G,x:  A)}{(\gax)}$, we derive $\D_1'$:
        
        \begin{equation}
            \D' = \ntreeruleI{\vfun}{
                \treeruleI{\D_1'}{\typerelation{\G',x: A}{v}{B}}
            }{\typerelation{\G',x: A}{\lam{x}{A}{v}}{\ab}}
        \end{equation}
        
        This preserves denotations:
        
        
        \begin{align*}
        \D' & = \cur{\D_1'} \qt{By Definition}\\
        & = \cur{\D_1\after(\w\times \idg)}\qt{By the denotation of $\w\x$} \\
        &= \cur{\D_1}\after\w\qt{By the exponential property}\\
        &= \D\after \w \qt{By Definition}
        \end{align*}
        
        
        \case{\vsubtype}
        
        \begin{equation}
            \ntreeruleII{\vsubtype}{\gtyperelation{v}{A}}{ A\subtype B}{\gtyperelation{v}{B}}
        \end{equation}
        
        by inversion, we have a derivation $\D_1$
        \begin{equation}
            \treeruleI{\D_1}{\gtyperelation{v}{A}}
        \end{equation}
        
        So by induction, we have a derivation $\D_1'$ such that:
        \begin{equation}
            \ntreeruleII{\vsubtype}{\treeruleI{\D_1'}{\typerelation{\G'}{v}{a}}}{ A \subtype B}{\typerelation{\G'}{v}{B}}
        \end{equation}
        
        This preserves denotations:
        
        \begin{align*}
            \D' & = \deno{A\subtype B}\after \D_1' \qt{By Definition} \\
            & = \deno{A\subtype B}\after \D_1\after\w \qt{By induction}\\
            & = \D\after\w \qt{By Definition}\\
        \end{align*}
        
        \case{\vreturn}
        We have the sub-derivation $\D_1$ such that
        \begin{equation}
            \D = \ntreeruleI{\vreturn}{\treeruleI{\D_1}{\gtyperelation{v}{A}}}{\gtyperelation{\return{v}}{\moa}}
        \end{equation}
        
        Hence, by induction, with $\wrel{\w}{\G'}{\G}$, we find the derivation $\D_1'$ such that:
        \begin{equation}
            \D' = \ntreeruleI{\vreturn}{\treeruleI{\D_1'}{\typerelation{\G'}{v}{A}}}{\typerelation{\G'}{\return{v}}{\moa}}
        \end{equation}
        
        This preserves denotations:
        
        \begin{align*}
            \D' & = \point{A}\after\D_1' \qt{By definition}\\
                & = \point{A}\after\D_1\after\w\qt{By induction of $\D_1, \D_1'$}\\
                & = \D\after\w\qt{By Definition}
        \end{align*}
        
        \case{\vapply}
            By inversion, we have derivations $\D_1$, $\D_2$ such that
        
            \begin{equation}
                \D = 
                \ntreeruleII{\vapply}{
                    \treeruleI{\D_1}{\gtyperelation{v_1}{\ab}}
                    }{
                    \treeruleI{\D_2}{\gtyperelation{v_2}{A}}
                } {
                    \gtyperelation{\apply{v_1}{v_2}}{B}
                }
            \end{equation}
        
            By induction, this gives us the respective derivations: $\D_1',\D_2'$ such that
        
            
            \begin{equation}
                \D' = 
                \ntreeruleII{\vapply}{
                    \treeruleI{\D_1'}{\typerelation{\G'}{v_1}{\ab}}
                    }{
                    \treeruleI{\D_2'}{\typerelation{\G'}{v_2}{A}}
                } {
                    \typerelation{\G'}{\apply{v_1}{v_2}}{B}
                }
            \end{equation}
        
            This preserves denotations:
        
            \begin{align*}
                \D' &= \app\after\pr{\D_1'}{\D_2'} \qt{By Definition}\\
                &= \app\after\pr{\D_1\after\w}{\D_2\after\w} \qt{By induction on $\D_1, \D_2$}\\
                &= \app\after\pr{\D_1}{\D_2}\after\w\\
                &= \D\after\w\qt{By Definition}
            \end{align*}
        \case{\vif}
        By inversion, we have the sub-derivations $\D_1,\D_2,\D_3$, such that:
        
        
        \begin{equation}
            \D = \ntreeruleIII{\vif}{
                \treeruleI{\D_1}{\typerelation{\G}{v}{\B}}
                }{
                \treeruleI{\D_2}{\typerelation{\G}{v_1}{ A}}
                }{
                \treeruleI{\D_3}{\typerelation{\G}{v_2}{ A}}
            }{
                \typerelation{\G}{\pifthenelse{A}{v}{v_1}{v_2}}{ A}
            }
        \end{equation}
        
        By induction, this gives us the sub-derivations $\D_1', \D_2', \D_3'$ such that
        
        \begin{equation}
            \D' = \ntreeruleIII{\vif}{
                \treeruleI{\D_1'}{\typerelation{\G'}{v}{\B}}
                }{
                \treeruleI{\D_2'}{\typerelation{\G'}{v_1}{ A}}
                }{
                \treeruleI{\D_3'}{\typerelation{\G'}{v_2}{ A}}
            }{
                \typerelation{\G'}{\pifthenelse{A}{v}{v_1}{v_2}}{ A}
            }
        \end{equation}
        
        And 
        \begin{align*}
            \D_1' & =\D_1 \after \w\\
            \D_3' & =\D_2 \after \w\\
            \D_3' & =\D_3 \after \w 
        \end{align*}
        
        
        This preserves denotations.
        Since $\w: \G' \rightarrow \G$, \\
        Let $(A)^{\w}: A^{\G}\rightarrow A^{\G'}$ be as defined in ExSh 3 (\footnote{https://www.cl.cam.ac.uk/teaching/1819/L108/exercises/L108-exercise-sheet-3.pdf})
        That is:
        
        \begin{align*}
            (A)^{\w} & = \cur{\app\after(\Id{A}\times w)}
        \end{align*}.
        And hence, we have:
        
        \begin{align*}
            \cur{f\after(\Id{}\times \w)} & = (A)^{\w} \after\cur{f}
        \end{align*}
        
        \scalebox{0.8}{\parbox{1.2\linewidth}{%
        \begin{align*}
            \D' & =\app\after((\fld{\cur{\D_2'\after\pp}}{\cur{\D_3'\after\pp}}\after\D_1')\times \Id{\G'})\after\diag{\G'}\qt{By Definition}\\
            & =\app\after((\fld{\cur{\D_2\after\w\after\pp}}{\cur{\D_3\after\w\after\pp}}\after\D_1')\times \Id{\G'})\after\diag{\G'}\qt{By Induction}\\
            & = \app\after((\fld{\cur{\D_2\after\pp\after(\Id{\1}\times \w)}}{\cur{\D_3\after\pp\after(\Id{\1}\times \w)}}\after\D_1\after\w)\times \Id{\G'})\after\diag{\G'}\qt{By product property}\\
            & = \app\after((\fld{(A)^{\w}\after\cur{\D_2\after\pp}}{(A)^{\w}\after\cur{\D_3\after\pp}}\after\D_1\after\w)\times \Id{\G'})\after\diag{\G'}\qt{By $(A)^{\w}$ property}\\
            & = \app\after(((A)^{\w}\after\fld{\cur{\D_2\after\pp}}{\cur{\D_3\after\pp}}\after\D_1\after\w)\times \Id{\G'})\after\diag{\G'}\qt{Factor out transformation}\\
            & = \app\after((A)^{\w}\times\Id{\G'})\after((\fld{\cur{\D_2\after\pp}}{\cur{\D_3\after\pp}}\after\D_1)\times \Id{\G'})\after(\w \times \Id{\G'})\after\diag{\G'}\qt{Factor out Identity pairs}\\
            & = \app\after(\Id{(A)}\times\w)\after((\fld{\cur{\D_2\after\pp}}{\cur{\D_3\after\pp}}\after\D_1) \times \Id{\G'})\after(\w \times \Id{\G'})\after\diag{\G'}\qt{By defintion of $\app, (A)^{\w}$}\\
            & = \app\after((\fld{\cur{\D_2\after\pp}}{\cur{\D_3\after\pp}}\after\D_1)\times \idg)\after(\w \times \w)\after\diag{\G'}\qt{Push through pairs}\\
            & = \app\after((\fld{\cur{\D_2\after\pp}}{\cur{\D_3\after\pp}}\after\D_1)\times \idg)\after\diag{\G}\after\w\qt{By Definition of the diagonal morphism.}\\
            & = \D\after\w
        \end{align*}
        }}
        
        
        \case{\vbind}
        By inversion, we have derivations $\D_1, \D_2$ such that:
        
        
        \begin{equation}
            \D = \ntreeruleII{\vbind}{
                \treeruleI{\D_1}{\typerelation{\G}{v_1}{\M{\E_1}{A}}}
                }{
                \treeruleI{\D_2}{\typerelation{\G,x: A}{v_2}{\M{\e_2}{B}}}
            }{
                \typerelation{\G}{\doin{x}{v_1}{v_2}}{\M{\e_1\dot\e_2}{B}}
            }
        \end{equation}
        
        If $\wrel{\w}{\G'}{\G}$ then $\wrel{\w\x}{\G',x:A}{\gax}$, so by induction, we can derive $\D_1'$, $\D_2'$ such that:
        
        \begin{equation}
            \D' = \ntreeruleII{\vbind}{
                \treeruleI{\D_1'}{\typerelation{\G'}{v_1}{\M{\E_1}{A}}}
                }{
                \treeruleI{\D_2'}{\typerelation{\G',x: A}{v_2}{\M{\e_2}{B}}}
            }{
                \typerelation{\G'}{\doin{x}{v_1}{v_2}}{\M{\e_1\dot\e_2}{B}}
            }
        \end{equation}
        
        This preserves denotations:
        
        \begin{align*}
            \D' & = \bind{\e_1}{\e_2}{B}\after\T{\e_1}{\D_2'}\after\tstrength{\e_1}{\G'}{A}\after\pr{\Id{G'}}{\D_1'}\qt{By definition}\\
            & = \bind{\e_1}{\e_2}{B}\after\T{\e_1}{(\D_2\after(\w\times\Id{A}))}\after\tstrength{\e_1}{\G'}{A}\after\pr{\Id{G'}}{\D_1\after\w}\qt{By induction on $\D_1', \D_2'$}\\
            & = \bind{\e_1}{\e_2}{B}\after\T{\e_1}{\D_2}\after\tstrength{\e_1}{\G}{A}\after\pr{\w}{\D_1\after\w}\qt{By tensor strength}\\
            & = \bind{\e_1}{\e_2}{B}\after\T{\e_1}{\D_2}\after\tstrength{\e_1}{\G}{A}\after\pr{\idg}{\D_1}\after\w\qt{By product property}\\
            & = \D \qt{By definition}
        \end{align*}
        
    \end{proof}
\end{framed}


\chapter{Substitution}
\section{Introduce Substitutions}
\subsection{Substitutions as SNOC lists}

\begin{equation}
   \si \gens \nil \mid \si, x \setto v
\end{equation}

\subsection{Trivial Properties of substitutions}
\paragraph{$\fv{\si}$}
\begin{align*}
    \fv{\nil} &= \emptyset\\
    \fv{\si, x\setto v} &= \fv{\si}\cup\fv{v}
\end{align*}
\paragraph{$\dom{\si}$}
\begin{align*}
    \dom{\nil} & = \emptyset\\
    \dom{\si, x\setto v} & = \dom{\si}\cup\left\{x\right\}
\end{align*}

\paragraph{$x \# \si$}
\begin{equation}
    x \# \si \Leftrightarrow x\notin(\fv{\si}\cup\dom{\si`})
\end{equation}

\subsection{Action of Substitutions}
    We define the action of applying a substitution $\si$ as 
    $$v\ssi$$

    
    \begin{align*}
        x\sub{\nil} & = x \\
        x\sub{\si,x\setto v} & = v \\
        x\sub{\si,x'\setto v'} & = x\ssi\qt{If }x \neq x'\\
        \const{A}\ssi & = \const{A} \\
        (\lam{x}{A}{v})\ssi &= \lam{x}{A}{(v\ssi)}\qt{If }x\#\si\\
        (\pifthenelse{A}{v}{v_1}{v_2})\ssi &= \pifthenelse{A}{v\ssi}{v_1\ssi}{v_2\ssi}\\
        (\apply{v_1}{v_2})\ssi&= \apply{(v_1\ssi)}{v_2\ssi}\\
        (\doin{x}{v_1}{v_2})&= \doin{x}{(v_1\ssi)}{(v_2\ssi)}\qt{If } x\#\si\\
    \end{align*}
\subsection{Wellformedness}
Define the relation $$\typerelation{\G'}{\si}{\G}$$ by:

\begin{itemize}
    \item $\ntreeruleI{Nil}{\ok{\G'}}{\typerelation{\G'}{\nil}{\nil}}$
    \item $\ntreeruleIII{\textend}{
        \typerelation{\G'}{\si}{\G}
        }{
        x\notin\dom{\G}
        }{
        \typerelation{\G'}{v}{A}
    }{
        \typerelation{\G'}{(\si, x \setto v)}{(\gax)}
    }$
\end{itemize}
\subsection{Simple Properties Of Substitution}

\begin{property}[Ok Relation]\label{SubPropOne}
    If $\typerelation{\G'}{\si}{\G}$ then  $\ok{\G}$ and $\ok{\G'}$. Since $\ok{\G'}$ holds by the Nil-axiom. $\ok{\G}$ holds by induction on the wellformedness relation.
\end{property}

\begin{property}[Weakening]\label{SubPropTwo}
    If $\typerelation{\G'}{\si}{\G}$ then $\wrel{\w}{\G''}{\G'}$ implies $\typerelation{\G''}{\si}{\G}$.
    This holds by induction over wellformedness relation. For each $x\setto v$ in $\si$, $\typerelation{\G''}{v}{A}$ holds if $\typerelation{\G'}{v}{A}$ holds.
\end{property}

\begin{property}[Extension]\label{SubPropThree}
    If $\typerelation{\G'}{\si}{\G}$ then $x \notin (\dom{\G} \cup \dom{\G''})$ implies $\typerelation{(\G', x: A)}{(\si, x \setto x)}{(\gax)}$.
    This occurs since $\wrel{\i\pi}{\G',x: A}{\G'}$, so by property \ref{SubPropTwo}, 
    $\typerelation{\G', x: A}{\si}{\G}$.
    In addition, $\typerelation{\G', x: A}{x}{A}$ trivially, so by the rule \textit{\textend}, wellformedness holds for
    \begin{equation}
        \typerelation{(\G', x:A)}{(\si, x\setto x)}{(\gax)}
    \end{equation}
\end{property}

     
    
   
    

\section{Substitution Preserves Typing}
We have the following non-trivial property of substitution:


\begin{theorem}[Substitution Preserves Typing Relation]\label{SubsTyping}

    If $\typerelation{\G}{g}{A}$ and $\typerelation{\G'}{\si}{\G}$ then $\typerelation{\G'}{v\ssi}{A}$.
\end{theorem}



\begin{framed}
    \begin{proof}
        
        Assuming $\typerelation{\G'}{\si}{\G}$, we induct over the typing relation, proving $\gtyperelation{v}{A}\rightarrow\typerelation{\G'}{v}{A}$
        
            \case{\vvar}
                By inversion $\G = (\G'', x:A)$
                So \begin{equation}
                    \typerelation{\G'',x:A}{x}{A}
                \end{equation}
        
                So by inversion, since $\typerelation{\G'}{\si}{\G'', x:A}$, 
        
                \begin{equation}
                    \si = \si', x\setto v \wedge \typerelation{\G'}{v}{A}
                \end{equation}
        
                By the definition of the action of substitutions, $x\ssi = v$, So
                \begin{equation}
                    \typerelation{\G'}{x\ssi}{A}
                \end{equation}
        
                holds.
                
            \case{\vweaken}
                By inversion, $\G = \G'', y: B, x \neq y$, and there exists $\D$ such that
        
                \begin{equation}
                    \ntreeruleI{\vweaken}{\treeruleI{\D}{\typerelation{\G''}{x}{A}}}{\typerelation{\G'', y: B}{x}{A}}
                \end{equation}
        
            
                By inversion, $\si = \si', y\setto v$
                and:
        
                \begin{equation}
                    \typerelation{\G'}{\si'}{\G''}
                \end{equation}
        
                So by induction,
        
                \begin{equation}
                    \typerelation{\G'}{x\sub{\si'}}{A}
                \end{equation}
        
                And so by definition of the action of $\si$, $x\ssi = x\sub{\si'}$
        
                \begin{equation}
                    \typerelation{\G'}{x\sub{\si}}{A}
                \end{equation}
        \case{\vfun}
            By inversion, there exists $\D$ such that:
        
            \begin{equation}
                \ntreeruleI{\vfun}{
                    \treeruleI{\D}{\typerelation{\G, x:A }{v}{B}}
                }{\typerelation{\G}{\lam{x}{A}{v}}{\ab}}
            \end{equation}
            Using alpha equivalence, we pick $x\notin(\dom{\G}\cup\dom{\G'})$
            Hence, by property \ref{SubPropThree}, we have
        
            \begin{equation}
                \typerelation{(\G', x:A )}{(\si, x\setto x)}{\gax}
            \end{equation}
        
            So by induction using $\si, x\setto x$, we have $\D'$ such that:
        
            \begin{equation}
                \ntreeruleI{\vfun}{
                    \treeruleI{\D'}{\typerelation{\G', x:A }{v\sub{\si, x\setto v}}{B}}
                }{\typerelation{\G}{\lam{x}{A}{v\sub{\si, x\setto{x}}}}{\ab}}
            \end{equation} 
            Since $\lam{x}{A}{(v\sub{\si, x\setto{x}})} = {\lam{x}{A}({v\ssi})} = (\lam{x}{A}{v})\ssi$, we have a typing derivation for $\typerelation{\G'}{(\lam{x}{A}{v})\ssi}{\ab}$.
        
        \case{\vconst}
            We use the same logic for all constants, $\u,\t,\f,\const{A}$:
        
            $\typerelation{\G}{\si}{\G} \Rightarrow \ok{\G'}$ and:
        
            \begin{equation}
                \const{A}\ssi =\const{A}
            \end{equation}
        
            So \begin{equation}
                \ntreeruleI{\vconst}{\ok{\G'}}{\typerelation{\G'}{\const{A}}{A}}
            \end{equation}
        
        \case{\vreturn}
            By inversion, we have $\D_1$ such that:
        
            \begin{equation}
                \ntreeruleI{\vreturn}{\treeruleI{\D_1}{\typerelation{\G}{v}{A}}}{\typerelation{\G}{\return{v}}{\moa}}
            \end{equation}
        
            By induction, we have $\D_1'$ such that
        
            \begin{equation}
                \ntreeruleI{\vreturn}{\treeruleI{\D'_1}{\typerelation{\G'}{v\ssi}{A}}}{\typerelation{\G'}{\return{(v\ssi)}}{\moa}}
            \end{equation}
        
            Since $(\return{v})\ssi = \return{(v\ssi)}$, the type derivation above holds for $\typerelation{\G'}{(\return{v})\ssi}{\moa}$.
        
        \case{\vapply}
            By inversion, we have $\D_1$, $\D_2$ such that:
        
            \begin{equation}
                \ntreeruleII{\vapply}{
                    \treeruleI{\D_1}{\typerelation{\G}{v_1}{\ab}}
                    }{
                    \treeruleI{\D_2}{\typerelation{\G}{v_2}{A}}
                }{\typerelation{\G}{\apply{v_1}{v_2}}{B}}
            \end{equation}
        
            By induction on $\D_1, \D_2$, we have $\D_1',\D_2'$ such that
        
            \begin{equation}
                \ntreeruleII{\vapply}{
                    \treeruleI{\D'_1}{\typerelation{\G'}{v_1\ssi}{\ab}}
                    }{
                    \treeruleI{\D_2'}{\typerelation{\G'}{v_2\ssi}{A}}
                }{\typerelation{\G'}{\apply{(v_1\ssi)}{(v_2\ssi)}}{B}}
            \end{equation}
        
            Since $(\apply{v_1}{v_2})\ssi = \apply{(v_1\ssi)}{(v_2\ssi)}$, we the above derivation holds for $\typerelation{\G'}{(\apply{v_1}{v_2})\ssi}{B}$
        \case{\vif}
            By inversion, we have $\D_1,\D_2,\D_3$ such that:
        
            \begin{equation}
                \ntreeruleIII{\vif}{
                    \treeruleI{\D_1}{\typerelation{\G}{v}{\B}}
                    }{
                    \treeruleI{\D_2}{\typerelation{\G}{v_1}{ A}}
                    }{
                    \treeruleI{\D_3}{\typerelation{\G}{v_2}{ A}}
                }{
                    \typerelation{\G}{\pifthenelse{A}{v}{v_1}{v_2}}{ A}
                }
            \end{equation}
        
            By induction on $\D_1,\D_2, \D_3$, we derive 
            $\D_1', \D_2',\D_3'$ such that:
            
            \begin{equation}
                \ntreeruleIII{\vif}{
                    \treeruleI{\D_1'}{\typerelation{\G'}{v\ssi}{\B}}
                    }{
                    \treeruleI{\D_2'}{\typerelation{\G'}{v_1\ssi}{ A}}
                    }{
                    \treeruleI{\D_3'}{\typerelation{\G'}{v_2\ssi}{ A}}
                }{
                    \typerelation{\G'}{\pifthenelse{A}{(v\ssi)}{(v_1\ssi)}{(v_2\ssi)}}{ A}
                }
            \end{equation}
        
            Since $(\pifthenelse{A}{v}{v_1}{v_2})\ssi = \pifthenelse{A}{(v\ssi)}{(v_1\ssi)}{(v_2\ssi)}$ The derivation above holds for $\typerelation{\G'}{(\pifthenelse{A}{v}{v_1}{v_2})\ssi}{ A}$
        
        \case{\vbind}
            By inversion, there exist $\D_1, \D_2$ such that:
            \begin{equation}
                \ntreeruleII{\vbind}{
                    \treeruleI{\D_1}{\typerelation{\G}{v_1}{\M{\e_1}{A}}}
                    }{
                    \treeruleI{\D_2}{\typerelation{\G, x:A}{v_2}{\M{\e_2}{B}}}
                }{\typerelation{\G}{\doin{x}{v_1}{v_2}}{\M{\e_1\dot\e_2}{B}}}
            \end{equation}
        
            Using alpha-equivalence, we pick $x\notin(\dom{\G}\cup\dom{\G'})$. Hence by property \ref{SubPropThree}, $$\typerelation{(\G, x: A)}{(\si, x\setto x)}{(\gax)}$$
            By induction on $\D_1, \D_2$, we have $\D_1', \D_2'$ such that:
            \begin{equation}
                \ntreeruleII{\vbind}{
                    \treeruleI{\D_1'}{\typerelation{\G'}{v_1\ssi}{\M{\e_1}{A}}}
                    }{
                    \treeruleI{\D_2}{\typerelation{\G', x:A}{v_2\sub{\si, x\setto x}}{\M{\e_2}{B}}}
                }{\typerelation{\G'}{\doin{x}{(v_1\ssi)}{(v_2\sub{\si, x\setto x})}}{\M{\e_1\dot\e_2}{B}}}
            \end{equation}
        
            Since $(\doin{x}{v_1}{v_2})\ssi = \doin{x}{(v_1\ssi)}{(v_2\ssi)} = \doin{x}{(v_1\ssi)}{(v_2\sub{\si, x\setto x})}$, the above derivation holds for $\typerelation{\G'}{(\doin{x}{v_1}{v_2})\ssi}{\M{\e_1\dot\e_2}{B}}$
        \case{\vsubtype}
            By inversion, there exists $\D$ such that
            \begin{equation}
                \ntreeruleII{\vsubtype}{\treeruleI{\D}{\typerelation{\G}{v}{A}}}{ A\subtype B}{\typerelation{\G}{v}{B}}
            \end{equation}
        
            By induction on $\D$ we derive $\D'$ such that:
        
            \begin{equation}
                \ntreeruleII{\vsubtype}{\treeruleI{\D'}{\typerelation{\G'}{v\ssi}{A}}}{ A\subtype B}{\typerelation{\G}{v\ssi}{B}}
            \end{equation}
        
    \end{proof}
    
\end{framed}

\subsection{Extension Lemma}

\begin{lemma}[Extension Denotation]
    If $\typerelation{\G'}{\si}{\G}$ and $x\notin(\dom{\G'}\cup\dom{\G})$ then the substitution in property \ref{SubPropThree} has denotation:
    \begin{equation}
        \deno{\typerelation{(\G', x:A)}{(\si, x\setto x)}{(\gax})} = (\deno{\typerelation{\G'}{\si}{\G}}\times \Id{A})
    \end{equation}
\end{lemma}






\begin{framed}
    \begin{proof}
        This holds since 
        \begin{equation}
            \deno{\typerelation{\G', x:A}{x}{A}} = \pp
        \end{equation}
        
        And $\wrel{\i\pi}{(\G',x:A)}{\G'}$
        \begin{equation}
            \deno{\wrel{\i\pi}{(\G',x:A)}{\G'}} = \p
        \end{equation}
        
        So for each denotation $\deno{\typerelation{\G'}{v}{B}}$ of each $y\setto v$ in $\si$, we can pre-pend the denotation with the weakening denotation to yield: 
        \begin{equation}
            \deno{\typerelation{\G',x:A}{v}{B}} = \deno{\typerelation{\G'}{v}{B}}\after\p
        \end{equation}
        Since $\p$ appears in every branch of $\deno{\typerelation{\G', x:A}{\si}{\G}}$, it can be factored out to yield:
        
        \begin{equation}
            \deno{\typerelation{\G', x:A}{\si}{\G}} = \deno{\typerelation{\G'}{\si}{\G}}\after\p
        \end{equation}
        
        Hence,
        
        \begin{equation}
            \deno{\typerelation{(\G', x:A)}{(\si, x\setto x)}{\gax}} = \pr{\deno{\typerelation{\G'}{\si}{\G}}\after\p}{\pp} = (\deno{\typerelation{\G'}{\si}{\G}}\times \Id{A})
        \end{equation}
    \end{proof}
\end{framed}


\subsection{Substitution Theorem}

\begin{theorem}[Substitutions and Denotations]\label{SubsDenotations}
    If $\D$ derives $\gtyperelation{v}{A}$ and $\typerelation{\G'}{\si}{\G}$ then the derivation $\D'$ deriving $\typerelation{\G'}{v\ssi}{A}$ satisfies:


\begin{framed}
    \begin{equation}
        \D' = \D \after\deno{\typerelation{\G'}{\si}{\G}}
    \end{equation}
    \centering
    \begin{tikzcd}[column sep=huge]
        \G' 
            \arrow{rr}{\deno{\typerelation{\G'}{\si}{\G}}} 
            \arrow{rrd}{\deno{\typerelation{\G'}{v\ssi}{A}}} &&
        \G 
            \arrow{d}{\deno{\gtyperelation{v}{A}}}
        \\
        & &\deno{T}
    \end{tikzcd}
\end{framed}
\end{theorem}




\begin{framed}
    \begin{proof}
        This is proved by induction over the derivation of $\typerelation{\G}{v}{A}$.
        We shall use $\si$ to denote $\deno{\typerelation{\G'}{\si}{\G}}$ where it is clear from the context.
        
        \case{\vvar}
        By inversion $\G = \G'', x:A$
        \begin{equation}
            \ntreeruleI{\vvar}{\ok{\G}}{\typerelation{\G'', x:A}{x}{A}}
        \end{equation}
        By inversion, $\si = \si', x\setto v$ and $\typerelation{\G'}{v}{A}$.
        
        Let 
        \begin{align*}
            \si &=\deno{\typerelation{\G'}{\si}{\G}} = \pr{\si'}{\D'}\\
            \D &=\deno{\typerelation{\G'', x:A}{x}{A}} = \pp\\
        \end{align*}
        
        \begin{align*}
            \D\after\si &= \pp\after\pr{\si'}{\D'}\qt{By definition}\\
            &= \D'\qt{By product property}
        \end{align*}
        \case{\vweaken}
        By inversion, $\G = \G', y:B$ and $\si = \si', y\setto v$
        and we have $\D_1$ deriving:
        
        \begin{equation}
            \ntreeruleI{\vweaken}{\treeruleI{\D_1}{\typerelation{\G''}{x}{A}}}{\typerelation{\G'',y: B}{x}{A}}
        \end{equation}
        
        Also by inversion of the wellformedness of $\typerelation{\G'}{\si}{\G}$, we have $\typerelation{\G'}{\si'}{\G''}$ and 
        
        \begin{equation}
            \deno{\typerelation{\G'}{\si}{\G}} = \pr{\deno{\typerelation{\G'}{\si}{\G''}}}{\deno{\typerelation{\G'}{v}{B}}}
        \end{equation}
        
        Hence by induction on $\D_1$ we have $\D_1'$ such that
        
        \begin{equation}
            \treeruleI{\D_1'}{\typerelation{\G'}{x\ssi}{A}}
        \end{equation}
        
        
        
        Hence
        \begin{align*}
            \D' & = \D_1' \qt{By definition}\\
                & = \D_1\after\si'\qt{By induction}\\
                & = \D_1\after\p\after\pr{\si'}{\deno{\typerelation{\G'}{v}{B}}}\qt{By product property}\\
                & = \D_1\after\p\after\si\qt{By defintion of the denotation of $\si$}
                & = \D\after\si\qt{By defintion.}
        \end{align*}
        
        \case{\vconst}
        The logic for all constant terms ($\t,\f,\u\const{A}$) is the same.
        Let
        \begin{equation}
            c = \deno{\const{A}}
        \end{equation}
        \begin{align*}
            \D' & = c\after\term{\G'}\qt{By Definition}\\
                & = c\after\term{G}\after\si\qt{Terminal property}\\
                & = \D\after\si\qt{By definition}
        \end{align*}
        \case{\vfun}
        
        By inversion, we have $\D_1$ such that
        \begin{equation}
            \D = \ntreeruleI{\vfun}{
                \treeruleI{\D_1}{\typerelation{\G, x:A}{v}{B}}
            }{\typerelation{\G}{\lam{x}{A}{v}}{\ab}}
        \end{equation}
        
        By induction of $\D_1$ we have $\D_1'$ such that
        \begin{equation}
            \D' = \ntreeruleI{\vfun}{
                \treeruleI{\D_1'}{\typerelation{\G', x:A}{(v\ssi)}{B}}
            }{\typerelation{\G}{(\lam{x}{A}{v})\ssi}{\ab}}
        \end{equation}
        By induction and the extension lemma, we have:
        \begin{equation}
            \D_1' = \D_1\after(\si\times\Id{A})
        \end{equation}
        
        Hence:
        
        \begin{align*}
            \D' &= \cur{\D_1'}\qt{By definition}\\
                &= \cur{\D_1\after(\si\times\Id{A})}\qt{By induction and extension lemma.}\\
                & = \cur{\D_1}\after\si\qt{By the exponential property (Uniqueness)}\\
                &= \D\after\si\qt{By Definition}\\
        \end{align*}
        \case{\vsubtype}
        By inversion, there exists derivation $\D_1$ such that:
        
        \begin{equation}
            \D = \ntreeruleII{\vsubtype}{\treeruleI{\D_1}{\typerelation{\G}{v}{A}}}{ A\subtype B}{\typerelation{\G}{v}{B}}
        \end{equation}
        
        By induction on $\D_1$, we find $\D_1'$ such that $\D_1' = \D_1\after\si$ and:
        
        \begin{equation}
            \D' = \ntreeruleII{\vsubtype}{\treeruleI{\D_1'}{\typerelation{\G'}{v\ssi}{A}}}{ A\subtype B}{\typerelation{\G'}{v\ssi}{B}}
        \end{equation}
        
        Hence,
        
        \begin{align*}
            \D' &= \deno{A\subtype B}\after\D_1'\qt{By definition}\\
                &= \deno{A\subtype B}\after\D_1\after\si\qt{By induction}\\
                &= \D\after\si\qt{By definition}\\
        \end{align*}
        
        \case{\vreturn}
        
        By inversion, we have $\D_1$ such that:
        \begin{equation}
            \D = \ntreeruleI{\vreturn}{\treeruleI{\D_1}{\typerelation{\G}{v}{A}}}{\typerelation{\G}{\return{v}}{\moa}}
        \end{equation}
        
        By induction on $\D_1$, we find $\D_1'$ such that $\D_1' = \D_1\after\si$ and:
        
        \begin{equation}
            \D' = \ntreeruleI{\vreturn}{\treeruleI{\D_1'}{\typerelation{\G'}{v\ssi}{A}}}{\typerelation{\G'}{(\return{v})\ssi}{\moa}}
        \end{equation}
        
        Hence,
        
        \begin{align*}
            \D' & = \point{A}\after\D_1'\qt{By Definition}\\
            & = \point{A}\after\D_1\after\si\qt{By induction}\\
            & = \D\after\si\qt{By Definition}\\
        \end{align*}
        \case{\vapply}
        By inversion, we find $\D_1, \D_2$ such that
        \begin{equation}
            \D = \ntreeruleII{\vapply}{\treeruleI{\D_1}{\typerelation{\G}{v_1}{\ab}}}{\treeruleI{\D_2}{\typerelation{\G}{v_2}{A}}}{\typerelation{\G}{\apply{v_1}{v_2}}{B}}
        \end{equation}
        
        By induction we find $\D_1', \D_2'$ such that 
        \begin{align*}
            \D_1' &= \D_1\after\si\\
            \D_2' &= \D_2\after\si\\
        \end{align*}
        
        And
        \begin{equation}
            \D' = \ntreeruleII{\vapply}{\treeruleI{\D_1'}{\typerelation{\G'}{v_1\ssi}{\ab}}}{\treeruleI{\D_2'}{\typerelation{\G'}{v_2\ssi}{A}}}{\typerelation{\G'}{(\apply{v_1}{v_2})\ssi}{B}}
        \end{equation}
        
        Hence
        \begin{align*}
            \D' &= \app\after\pr{\D_1'}{\D_2'}\qt{By Definition}\\
                &= \app\after\pr{\D_1\after\si}{\D_2\after\si}\qt{By induction}\\
                &= \app\after\pr{\D_1}{\D_2}\after\si\qt{By Product Property}\\
                & = \D\after\si\qt{By Definition}\\
        \end{align*}
        
        \case{\vif}
        
        
        By inversion, we find $\D_1, \D_2, \D_3$ such that
        \begin{equation}
            \D = \ntreeruleIII{\vif}{\treeruleI{\D_1}{\typerelation{\G}{v}{\B}}}{\treeruleI{\D_2}{\typerelation{\G}{v_1}{ A}}}{\treeruleI{\D_3}{\typerelation{\G}{v_2}{ A}}}{\typerelation{\G}{\pifthenelse{A}{v}{v_1}{v_2}}{ A}}
        \end{equation}
        
        By induction we find $\D_1', \D_2', \D_3'$ such that 
        \begin{align*}
            \D_1' &= \D_1\after\si\\
            \D_2' &= \D_2\after\si\\
            \D_3' &= \D_3\after\si\\
        \end{align*}
        
        And
        \begin{equation}
            \D' = \ntreeruleIII{\vif}{\treeruleI{\D_1'}{\typerelation{\G'}{v\ssi}{\B}}}{\treeruleI{\D_2'}{\typerelation{\G'}{v_1\ssi}{ A}}}{\treeruleI{\D_3'}{\typerelation{\G'}{v_2\ssi}{ A}}}{\typerelation{\G'}{(\pifthenelse{A}{v}{v_1}{v_2})\ssi}{ A}}
        \end{equation}
        
        Since $\si: \G' \rightarrow \G$, \\
        Let $(A)^{\si}: A^{\G}\rightarrow A^{\G'}$ be as defined in ExSh 3 (\footnote{https://www.cl.cam.ac.uk/teaching/1819/L108/exercises/L108-exercise-sheet-3.pdf})
        That is:
        
        \begin{align*}
            ( A)^{\si} & = \cur{\app\after(\Id{ A}\times w)}
        \end{align*}.
        And hence, we have:
        
        \begin{align*}
            \cur{f\after(\Id{}\times \si)} & = ( A)^{\si} \after\cur{f}
        \end{align*}
        
        And so:
        
        \scalebox{0.8}{\parbox{1.2\linewidth}{%
        \begin{align*}
            \D' & =\app\after((\fld{\cur{\D_2'\after\pp}}{\cur{\D_3'\after\pp}}\after\D_1')\times \Id{\G'})\after\diag{\G'}\qt{By Definition}\\
            & =\app\after((\fld{\cur{\D_2\after\si\after\pp}}{\cur{\D_3\after\si\after\pp}}\after\D_1')\times \Id{\G'})\after\diag{\G'}\qt{By Induction}\\
            & = \app\after((\fld{\cur{\D_2\after\pp\after(\Id{\1}\times \si)}}{\cur{\D_3\after\pp\after(\Id{\1}\times \si)}}\after\D_1\after\si)\times \Id{\G'})\after\diag{\G'}\qt{By product property}\\
            & = \app\after((\fld{( A)^{\si}\after\cur{\D_2\after\pp}}{( A)^{\si}\after\cur{\D_3\after\pp}}\after\D_1\after\si)\times \Id{\G'})\after\diag{\G'}\qt{By $( A)^{\si}$ property}\\
            & = \app\after((( A)^{\si}\after\fld{\cur{\D_2\after\pp}}{\cur{\D_3\after\pp}}\after\D_1\after\si)\times \Id{\G'})\after\diag{\G'}\qt{Factor out transformation}\\
            & = \app\after(( A)^{\si}\times\Id{\G'})\after((\fld{\cur{\D_2\after\pp}}{\cur{\D_3\after\pp}}\after\D_1)\times \Id{\G'})\after(\si \times \Id{\G'})\after\diag{\G'}\qt{Factor out Identity pairs}\\
            & = \app\after(\Id{( A)}\times\si)\after((\fld{\cur{\D_2\after\pp}}{\cur{\D_3\after\pp}}\after\D_1) \times \Id{\G'})\after(\si \times \Id{\G'})\after\diag{\G'}\qt{By defintion of $\app, ( A)^{\si}$}\\
            & = \app\after((\fld{\cur{\D_2\after\pp}}{\cur{\D_3\after\pp}}\after\D_1)\times \idg)\after(\si \times \si)\after\diag{\G'}\qt{Push through pairs}\\
            & = \app\after((\fld{\cur{\D_2\after\pp}}{\cur{\D_3\after\pp}}\after\D_1)\times \idg)\after\diag{\G}\after\si\qt{By Definition of the diagonal morphism.}\\
            & = \D\after\si
        \end{align*}
        }}
        
        
        \case{\vbind}
        
        By inversion, we have $\D_1, \D_2$ such that:
        
        \begin{equation}
            \D = \ntreeruleII{\vbind}{
                \treeruleI{\D_1}{\typerelation{\G}{v_1}{\M{\e_1}{A}}}
                }{
                \treeruleI{\D_2}{\typerelation{\G, x:A }{v_2}{\M{\e_2}{B}}}
            }{
                \typerelation{\G}{\doin{x}{v_1}{v_2}}{\M{\e_1\dot\e_2}{B}}
            }
        \end{equation}
        
        By property \ref{SubPropThree},
        
        \begin{equation}
            \typerelation{(\G', x:A)}{(\si, x\setto x}{(\gax)}
        \end{equation}
        
        With denotation (extension lemma)
        
        \begin{equation}
            \deno{\typerelation{(\G', x:A)}{(\si, x\setto x}{(\gax)}} = \si\times\Id{A}
        \end{equation}
        
        By induction, we derive $\D_1', \D_2'$ such that:
        
        \begin{align*}
            \D_1' & = \D_1\after \si\\
            \D_2' & = \D_2\after (\si\times\Id{A})\qt{By Extension Lemma}
        \end{align*}
        
        And:
        
        \begin{equation}
            \D' = \ntreeruleII{\vbind}{
                \treeruleI{\D_1'}{\typerelation{\G'}{v_1\ssi}{\M{\e_1}{A}}}
                }{
                \treeruleI{\D_2'}{\typerelation{\G', x:A }{v_1\ssi}{\M{\e_2}{B}}}
            }{
                \typerelation{\G'}{(\doin{x}{v_1}{v_2})\ssi}{\M{\e_1\dot\e_2}{B}}
            }
        \end{equation}
        
        Hence:
        
        \begin{align*}
            \D' & = \bind{\e_1}{\e_2}{B}\after\T{\e_1}{\D_2'}\after\tstrength{\e_1}{\G'}{A}\after\pr{\Id{\G'}}{\D_1'}\qt{By Definition}\\
            & = \bind{\e_1}{\e_2}{B}\after\T{\e_1}{(\D_2\after(\si\times\Id{A}))}\after\tstrength{\e_1}{\G'}{A}\after\pr{\Id{\G'}}{\D_1\after\si}\qt{By Induction using the extension lemma}\\
            & = \bind{\e_1}{\e_2}{B}\after\T{\e_1}{(\D_2)}\after\tstrength{\e_1}{\G}{A}\after(\si\times\Id{\T{\e_1}{A}})\after\pr{\Id{\G'}}{\D_1\after\si}\qt{By Tensor Strength}\\
            & = \bind{\e_1}{\e_2}{B}\after\T{\e_1}{(\D_2)}\after\tstrength{\e_1}{\G}{A}\after\pr{\si}{\D_1\after\si}\qt{By Product rule}\\
            & = \bind{\e_1}{\e_2}{B}\after\T{\e_1}{(\D_2)}\after\tstrength{\e_1}{\G}{A}\after\pr{\idg}{\D_1}\after\si\qt{By Product rule}\\
            &= \D\after\si\qt{By Defintion}\\
        \end{align*}
        
        
    \end{proof}
\end{framed}


\section{The Identity Substitution}
For each type environment $\G$, define the identity substitution $I_{\G}$ as so:

\begin{itemize}
    \item $I_{\nil} = \nil$
    \item $I_{(\gax} = (I_{\G}, x\setto x)$
\end{itemize}

\subsection{Properties of the Identity Substitution}


\begin{property}[Wellformedness]\label{IdPropOne}
    If $\ok{\G}$ then $\typerelation{\G}{I_{\G}}{\G}$, proved trivially by induction over the wellformedness relation.
\end{property}



\begin{property}[Denotation]\label{IdPropTwo}
    $\deno{\typerelation{\G}{I_{\G}}{\G}} = \idg$, proved trivially by induction over the definition of $I_{\G}$
\end{property}

\section{Single Substitution}

If $\typerelation{\G}{v}{A}$, let the single substitution $\typerelation{\G}{\ssub{x}{v}}{\gax}$, be defined as:
\begin{equation}
    \ssub{x}{v} = (I_{\G}, x\setto v)
\end{equation}

Then by properties \ref{IdPropOne}, \ref{IdPropTwo} of the identity substitution, we have:

\begin{equation}
    \deno{\gtyperelation{\ssub{x}{v}}{\gax}} = \pr{\idg}{\deno{\gtyperelation{v}{A}}}: \G\rightarrow(\G\times A)
\end{equation}

\subsection{The Semantics of Single Substitution}

The following diagram commutes:
\begin{framed}
    
    \begin{align*}
        \deno{\gtyperelation{v_1\ssub{x}{v}}{A}} &= \deno{\typerelation{\gax}{v_1}{A}}\after\pr{\idg}{\deno{\gtyperelation{v}{A}}}
    \end{align*}
    
    \centering
    \begin{tikzcd}[column sep=huge]
        \G \arrow{rr}{\pr{\idg}{\deno{\gtyperelation{v}{A}}}}
        \arrow{rrd}{\deno{\gtyperelation{v_1\ssub{x}{v}}{A}}}
        &&
        \G\times A 
        \arrow{d}{
            \deno{
                \typerelation{\gax}{v_1}{A}
                }
            }
        \\
        &&
        A 
        \\
    \end{tikzcd}
\end{framed}


Since $\deno{\gtyperelation{(I_{\G}, x\setto v)}{(\gax)}} = \pr{\idg}{\deno{\gtyperelation{v}{A}}}$
And $v_1\ssub{x}{v} = v_1\sub{I_{\G}, x\setto{v}}$


\chapter{Soundness}
\section{Equational Equivalence}
\subsection{Reduction Conversions}
\begin{itemize}
    \item $\ntreeruleII{\eqbeta}{\typerelation{\gax}{v_1}{B}}{\gtyperelation{v_2}{A}}{\gberelation{\apply{(\lam{x}{A}{v_1})}{v_2} }{ v_1\ssub{x}{v_2}}{B}}$
    
    \item $\ntreeruleI{\eqeta}{\gtyperelation{v}{\ab}}{\gberelation{\lam{x}{A}{(\apply{v}{x}})}{v}{\ab}}$
 

    \item $\ntreeruleII{\eqleftunit}{\gtyperelation{v_1}{A}}{\typerelation{\gax}{v_2}{\meb}}{\gberelation{\doin{x}{\return{v_1}}{v_2}}{v_2\ssub{x}{v_1}}{\meb}}$
    
    \item $\ntreeruleI{\eqrightunit}{\gtyperelation{v}{\mea}}{\gberelation{\doin{x}{v}{\return{x}} }{v}{\mea}}$
    \item $\ntreeruleIII{\eqassociativity}{
            \gtyperelation{v_1}{\M{\e_1}{A}} 
        }{
            \typerelation{\gax}{v_2}{\M{\e_2}{B}}
        }{
            \typerelation{\gby}{v_3}{\M{\e_3}{C}}
        }{
        \gberelation{\doin{x}{v_1}{(\doin{y}{v_2}{v_3})}}{\doin{y}{(\doin{x}{v_1}{v_2})}{v_3}}{\M{\e_1 \dot \e_2 \dot \e_3}{C}}
    }$

    \item $\ntreeruleI{\equnitequiv}{\gtyperelation{v}{\U}}{\gberelation{v}{\u}{\U}}$

    \item $\ntreeruleII{\eqiftrue}{\gtyperelation{v_1}{ A}}{\gtyperelation{v_2}{ A}}{\gberelation{\pifthenelse{A}{\t}{v_1}{v_2}}{v_1}{ A}}$
    
    \item $\ntreeruleII{\eqiffalse}{\gtyperelation{v_2}{ A}}{\gtyperelation{v_1}{ A}}{\gberelation{\pifthenelse{A}{\f}{v_1}{v_2}}{v_2}{ A}}$
    
    \item $\ntreeruleII{\eqifeta}{\typerelation{\G, x: \B}{v_1}{ A}}{\gtyperelation{v_2}{\B}}{\gberelation{\pifthenelse{A}{v_2}{v_1\ssub{x}{\t}}{v_1\ssub{x}{\f}}}{v_1\ssub{x}{v_2}}{ A}}$
    
\end{itemize}
\subsection{Equivalence Relation}
\begin{itemize}
    \item $\ntreeruleI{\eqreflexive}{\gtyperelation{v}{A}}{\gberelation{v}{v}{A}}$
    \item $\ntreeruleI{\eqsymmetric}{\gberelation{v_1}{v_2}{A}}{\gberelation{v_2}{v_1}{A}}$
    \item $\ntreeruleII{\eqtransitive}{\gberelation{v_1}{v_2}{A}}{\gberelation{v_2}{v_3}{A}}{\gberelation{v_1}{v_3}{A}}$
\end{itemize}
\subsection{Congruences}
\begin{itemize}
    \item $\ntreeruleI{\eqfun}{\berelation{\gax}{v_1}{v_2}{B}}{\gberelation{\lam{x}{A}{v_1}}{\lam{x}{A}{v_2}}{\ab}}$
    
    \item $\ntreeruleI{\eqreturn}{\gberelation{v_1}{v_2}{A}}{\gberelation{\return{v_1}}{\return{v_2}}{\moa}}$
    
    \item $\ntreeruleII{\eqapply}{\gberelation{v_1}{v_1'}{\ab}}{\gberelation{v_2}{v_2'}{A}}{\gberelation{\apply{v_1}{v_2}}{\apply{v_1'}{v_2'}}{B}}$
    
    \item $\ntreeruleII{\eqbind}{\gberelation{v_1}{v_1'}{\M{\e_1}{A}}}{\berelation{\gax}{v_2}{v_2'}{\M{\e_2}{B}}}{\gberelation{\doin{x}{v_1}{v_2}}{\doin{c}{v_1'}{v_2'}}{\M{\e_1 \dot \e_2}{B}}}$
    
    \item $\ntreeruleIII{\eqif}{\gberelation{v}{v'}{\B} }{\gberelation{v_1}{v_1'}{ A}}{\gberelation{v_2}{v_2'}{ A}}{\gberelation{\pifthenelse{A}{v}{v_1}{v_2}}{\pifthenelse{A}{v}{v_1'}{v_2'}}{ A}}$
    \item $\ntreeruleII{\eqsubtype}{\gberelation{v}{v'}{A} }{ A \subtype B}{\gberelation{v}{v'}{B}}$
\end{itemize}
\section{Equational Equivalence Implies Both Sides Have the Same Type}

\begin{theorem}[Equational Equivalence Implies Both Sides Have the Same Type]
    Each derivation of $\gberelation{v}{v'}{A}$ can be converted to a derivation of $\gtyperelation{v}{A}$ and $\gtyperelation{v'}{A}$ by induction over the beta-eta equivalence relation derivation.    
\end{theorem}



\begin{framed}
    \begin{proof}
        
        \subsection{Equivalence Relations}
        \case{\eqreflexive}
        By inversion we have a derivation of $\gtyperelation{v}{A}$.
        
        \case{\eqsymmetric}
        By inversion $\gberelation{v'}{v}{A}$. Hence by induction, derivations of $\gtyperelation{v'}{A}$ and $\gtyperelation{v}{A}$ are given.
        
        \case{\eqtransitive}
        By inversion, there exists $v_2$ such that $\gberelation{v_1}{v_2}{A}$ and $\gberelation{v_2}{v_3}{A}$.
        Hence by induction, we have derivations of $\gtyperelation{v_1}{A}$ and $\gtyperelation{v_3}{A}$
        
        \subsection{Reduction Conversions}
        
        \case{\vfun}
            By inversion, we have $\typerelation{\gax}{v}{B}$ and $\gtyperelation{v}{A}$. Hence by the typing rules, we have:
            $$\ntreeruleII{\vapply}{
                \ntreeruleI{\vfun}{\typerelation{\gax }{v_1}{B}}{\gtyperelation{\lam{x}{A}{v_1}}{\ab}}
            }{
                \gtyperelation{v_2}{A}
            }{\gtyperelation{\apply{(\lam{x}{A}{v_1})}{v_2}}{B}}$$
        
            By the substitution typing theorem \ref{SubsTyping}, we have 
            $$\ntreeruleII{Substitution}{\typerelation{\gax}{v_1}{B}}{\gtyperelation{v_2}{A}}{\gtyperelation{v_1\ssub{x}{v_2}}{B}}$$
        
        
        
            \case{\eqleftunit}
            By inversion, we have $\gtyperelation{v_1}{A}$ and $\typerelation{\gax }{v_2}{B}$
        
            Hence we have:
        
        
            \begin{equation}
                \ntreeruleII{\vbind}{\ntreeruleI{\vreturn}{\gtyperelation{v_1}{A}}{\gtyperelation{\return{v_1}}{\moa}}}{\typerelation{\gax}{v_2}{\meb}}{\gtyperelation{\doin{x}{\return{v_1}}{v_2}}{\M{\1\dot\e}{B} = \meb}}
            \end{equation}
        
            And by the substitution typing theorem \ref{SubsTyping} we have: 
        
            \begin{equation}
                \gtyperelation{v_2\ssub{x}{v_1}}{\meb}
            \end{equation}
            \case{\eqrightunit}
            By inversion, we have $\gtyperelation{v}{\mea}$.
        
            Hence we have:
        
            \begin{equation}
                \ntreeruleII{\vbind}{
                    \gtyperelation{v}{\mea}
                }{
                    \ntreeruleI{\vreturn}{\ntreeruleI{\vvar}{\ok{\G}}{\typerelation{\gax}{x}{A}}}{\typerelation{\gax}{\return{v}}{\moa}}
                }{\gtyperelation{\doin{x}{v}{\return{x}}}{\M{\e\dot\1}{A} = \mea}}
            \end{equation}
            \case{\eqassociativity}
            By inversion, we have $\gtyperelation{v_1}{\M{\e_1}{A}}$, $\typerelation{\gax}{v_2}{\M{\e_2}{B}}$, and $\typerelation{\gby}{v_3}{\M{\e_3}{C}}$.
        
           
            
            $$\wrel{(\i\pi\times)}{(\gax, y: B)}{(\gby)}$$
        
            So by the weakening property \ref{WeakeningTyping}, $\typerelation{\gax, y: B}{v_3}{\M{\e_3}{C}}$
        
            Hence we can construct the type derivations:
            \begin{equation} 
                \ntreeruleII{\vbind}{
                    \gtyperelation{v_1}{\M{\e_1}{A}}
                }{
                    \ntreeruleII{\vbind}{
                        \typerelation{\gax}{v_2}{\M{\e_2}{B}}
                    }{
                        \typerelation{\gax, y: B}{v_3}{\M{\e_3}{C}}
                    }{
                        \typerelation{\gax}{\do{x}{v_2}{v_3}}{\M{\e_2 \dot \e_3}{C}}
                    }
                }{
                    \gtyperelation{
                        \doin{x}{v_1}{
                            (\doin{y}{v_2}{v_3})
                        }
                    }{
                        \M{\e_1\dot\e_2\dot\e_3}{C}
                    }
                }
            \end{equation}
        
            and 
        
            \begin{equation}
                \ntreeruleII{\vbind}{
                    \ntreeruleII{\vbind}{
                        \gtyperelation{v_1}{\M{\e_1}{A}}
                    }{
                        \typerelation{\gax}{v_2}{\M{\e_2}{B}}
                    } {
                        \gtyperelation{\doin{x}{v_1}{v_2}}{\M{\e_1\dot\e_2}{B}}
                    }
                }{
                    \typerelation{\gby}{v_3}{\M{\e_3}{C}}
                }{
                    \gtyperelation{\doin{y}{(\doin{x}{v_1}{v_2})}{v_3}}{\M{\e_1\dot\e_2\dot\e_3}{C}}
                }
            \end{equation}
        
            \case{\eqeta}
            By inversion, we have $\gtyperelation{v}{\ab}$
        
            By weakening, we have $\wrel{\i\pi}{(\gax)}{\G}$
            Hence, we have
        
            \begin{equation}
                \ntreeruleI{\vfun}{
                    \ntreeruleII{\vapply}{
                        \typerelation{(\gax)}{x}{A}
                    }{
                        \ntreeruleII{\vweaken}{
                            \gtyperelation{v}{\ab}
                        }{
                            \wrel{\i\pi}{\gax}{\G}
                        }{
                            \typerelation{\gax}{v}{\ab}
                        }
                    } {
                        \typerelation{\gax}{\apply{v}{x}}{B}
                    }
                }{\gtyperelation{\lam{x}{A}{(\apply{v}{x})}}{\ab}}
            \end{equation}
        
            \case{\eqiftrue}
                By inversion, we have $\gtyperelation{v_1}{ A}$, $\gtyperelation{v_2}{ A}$. Hence by the typing ok lemma \ref{OkLemma}, we have $\ok{\G}$ so $\gtyperelation{\t}{\B}$ by the axiom typing rule.
        
                Hence 
                \begin{equation}
                        \ntreeruleIII{\vif}{
                            \gtyperelation{\t}{\B}
                        }{
                            \gtyperelation{v_1}{ A}
                        }{
                            \gtyperelation{v_2}{ A}
                        }{
                            \gtyperelation{\pifthenelse{A}{\t}{v_1}{v_2}}{ A}
                        }
                \end{equation}
            \case{\eqiffalse}
            As above,
        
            Hence 
            \begin{equation}
                \ntreeruleIII{\vif}{
                    \gtyperelation{\f}{\B}
                }{
                    \gtyperelation{v_1}{ A}
                }{
                    \gtyperelation{v_2}{ A}
                }{
                    \gtyperelation{\pifthenelse{A}{\f}{v_1}{v_2}}{ A}
                }
        \end{equation}
        
        \case{\eqifeta}
            By inversion, we have:
            \begin{equation}
                \gtyperelation{v_2}{\B}
            \end{equation}
            and
            \begin{equation}
                \typerelation{\G, x:\B}{v_1}{ A}
            \end{equation}
            
            Hence we also have $\ok{\G}$. Hence, the following also hold:
        
            $\gtyperelation{\t}{\B}$, and $\gtyperelation{\f}{\B}$.
        
            Hence by the substitution theorem, we have:
        
            \begin{equation}
                \ntreeruleIII{\vif}{
                    \gtyperelation{v_2}{\B}
                }{
                    \gtyperelation{v_1\ssub{x}{\t}}{ A}
                }{
                    \gtyperelation{v_1\ssub{x}{\f}}{ A}
                }{\gtyperelation{\pifthenelse{A}{v_2}{v_1\ssub{x}{\t}}{v_1\ssub{x}{\f}}}{ A}
                }
            \end{equation}
        
            and 
            \begin{equation}
                \gtyperelation{v_1\ssub{x}{v_2}}{ A}
            \end{equation}
        
            
        
            \subsection{Congruences}
        Each congruence rule corresponds exactly to a type derivation rule. To convert to a type derivation, convert all preconditions, then use the equivalent type derivation rule.
        
        \case{\eqfun}
            By inversion, $\berelation{\gax}{v_1}{v_2}{B}$. Hence by induction $\typerelation{\gax}{v_1}{B}$, and $\typerelation{\gax}{v_2}{B}$.
        
            So 
            \begin{equation}
                \gtyperelation{\lam{x}{A}{v_1}}{\ab}
            \end{equation}
        
            and
            \begin{equation}
                \gtyperelation{\lam{x}{A}{v_2}}{\ab}
            \end{equation}
        
            Hold.
        \case{\eqreturn}
            By inversion, $\gberelation{v_1}{v_2}{A}$, so by induction $$\gtyperelation{v_1}{A}$$ and $$\gtyperelation{v_2}{A}$$
        
            Hence we have $$\gtyperelation{\return{v_1}}{\moa}$$
            and
            $$\gtyperelation{\return{v_2}}{\moa}$$
        \case{\eqapply}
        
            By inversion, we have $\gberelation{v_1}{v_1'}{\ab}$ and $\gberelation{v_2}{v_2'}{A}$. Hence we have by induction $\gtyperelation{v_1}{\ab}$, $\gtyperelation{v_2}{A}$, $\gtyperelation{v_1'}{\ab}$, and $\gtyperelation{v_2'}{A}$.
        
            So we have:
        
            \begin{equation}
                \gtyperelation{\apply{v_1}{v_2}}{B}
            \end{equation}
        
            and
        
            
            \begin{equation}
                \gtyperelation{\apply{v_1'}{v_2'}}{B}
            \end{equation}
        
        \case{\eqbind}
            By inversion, we have:
            $\gberelation{v_1}{v_1'}{\M{\e_1}{A}}$ and
            $\berelation{\gax}{v_2}{v_2'}{\M{\e_2}{B}}$.
            Hence by induction, we have 
            $\gtyperelation{v_1}{\M{\e_1}{A}}$,
            $\gtyperelation{v_1'}{\M{\e_1}{A}}$,
            $\typerelation{\gax}{v_2}{\M{\e_2}{B}}$, and 
            $\typerelation{\gax}{v_2'}{\M{\e_2}{B}}$
        
            Hence we have 
            \begin{equation}
                \gtyperelation{\doin{x}{v_1}{v_2}}{\M{\e_1\dot\e_2}{A}}
            \end{equation}
        
            
            \begin{equation}
                \gtyperelation{\doin{x}{v_1'}{v_2'}}{\M{\e_1\dot\e_2}{A}}
            \end{equation}
        \case{\eqif}
        By inversion, we have:
        $\gberelation{v}{v'}{\B}$,
        $\gberelation{v_1}{v_1'}{ A}$, and
        $\gberelation{v_2}{v_2'}{ A}$.
        
        Hence by induction, we have:
        
        $\gtyperelation{v}{\B}$,
        $\gtyperelation{v'}{\B}$,
        
        $\gtyperelation{v_1}{ A}$,
        $\gtyperelation{v_1'}{ A}$,
        
        
        $\gtyperelation{v_2}{ A}$, and
        $\gtyperelation{v_2'}{ A}$.
        
        So 
        \begin{equation}
            \gtyperelation{\pifthenelse{A}{v}{v_1}{v_2}}{ A}
        \end{equation}
        and
        \begin{equation}
            \gtyperelation{\pifthenelse{A}{v}{v_1'}{v_2'}}{ A}
        \end{equation}
        Hold.
        \case{\eqsubtype}
        
        By inversion, we have $A \subtype B$ and $\gberelation{v}{v'}{A}$. By induction, we therefore have $\gtyperelation{v}{A}$ and $\gtyperelation{v'}{A}$.
        
        Hence we have 
        \begin{equation}
            \gtyperelation{v}{B}
        \end{equation}
        
        
        \begin{equation}
            \gtyperelation{v'}{B}
        \end{equation}
        
    \end{proof}
\end{framed}


\section{Equationally Equivalent Terms Have Equal Denotations}

\begin{theorem}[Equationally Equivalent Terms Have Equal Denotations]
    If $\gberelation{v}{v'}{A}$ then $\gdenoequality{v}{v'}{A}$
\end{theorem}


\begin{framed}
    
    \begin{proof}
        By induction over Beta-eta equivalence relation.
        \subsection{Equivalence Relation}
        The cases over the equivalence relation laws hold by the uniqueness of denotations and the fact that equality over morphisms is an equivalence relation.
        \case{\eqreflexive}
        Equality is reflexive, so if $\gtyperelation{v}{A}$ then $\deno{\gtyperelation{v}{A}}$ is equal to itself.
        \case{\eqsymmetric}
        By inversion, if $\gberelation{v}{v'}{A}$ then $\gberelation{v'}{v}{A}$, so by induction $\gdenoequality{v'}{v}{A}$ and hence $\gdenoequality{v}{v'}{A}$
        \case{\eqtransitive}
        There must exist $v_2$ such that $\gberelation{v_1}{v_2}{A}$ and $\gberelation{v_2}{v_3}{A}$, so by induction,
        $\gdenoequality{v_1}{v_2}{A}$ and $\gdenoequality{v_2}{v_3}{A}$. Hence by transitivity of equality, $\gdenoequality{v_1}{v_3}{A}$
        
        \subsection{Reduction Conversions}
        These cases are typically proved using the properties of a cartesian closed category with a strong graded monad.
        
        \case{\eqbeta}
            Let $f = \deno{\typerelation{\gax}{v_1}{B}}: (\G \times A) \rightarrow B$
        
            Let $g = \deno{\gtyperelation{v_2}{A}}: \G \rightarrow A$
        
            By the substitution denotation, $$\deno{\typerelation{\G}{\ssub{x}{v_2}}{\gax}}: \G \rightarrow (\G \times A) = \pr{\idg}{g}$$
        
            We have $$\deno{\gtyperelation{v_1\ssub{x}{v_2}}{B}} = f \after \pr{\idg}{g}$$
        
            and hence
            \begin{equation}
                \begin{split}
                    \deno{\gtyperelation{\apply{(\lam{x}{A}{v_1})}{v_2}}{B}} & = \app\after\pr{\cur{f}}{g} \\
                    & = \app\after(\cur{f}\times\Id{A})\after\pr{\idg}{g}\\
                    & = f \after \pr{\idg}{g} \\
                    & = \deno{\gtyperelation{v_1\ssub{x}{v_2}}{B}}  
                \end{split}
            \end{equation}
           
        
        \case{\eqleftunit}
        Let $f = \deno{\typerelation{\gax}{v_2}{\meb}}$
        
        Let $g = \deno{\gtyperelation{v_1}{A}}: \G \rightarrow A$
        
        By the substitution denotation, $$\deno{\typerelation{\G}{\ssub{x}{v_1}}{\gax}}: \G \rightarrow (\G \times A) = \pr{\idg}{g}$$
        
        We have $$\deno{\gtyperelation{v_2\ssub{x}{v_1}}{\meb}} = f \after \pr{\idg}{g}$$
        
        And hence
        
        \begin{equation}
            \begin{split}
                \deno{\gtyperelation{\doin{x}{\return{v_1}}{v_2}}{\meb}} = &  \bind{\1}{\e}{B} \after \T{\1}{f} \after \tstrength{1}{\G}{A} \after \pr{\idg}{\point{A}\after g} \\
                = & \bind{\1}{\e}{B} \after \T{\1}{f} \after \tstrength{1}{\G}{A} \after (\idg\times \point{A}) \after \pr{\idg}{g} \\
                = & \bind{\1}{\e}{B} \after \T{\1}{f} \after \point{(\G \times A)} \after \pr{\idg}{g} \qt{By Tensor strength + unit}\\
                = & \bind{\1}{\e}{B}\after\point{\teb}\after f\after\pr{\idg}{g} \qt{By Naturality of $\point{}$}\\
                = & f\after\pr{\idg}{g} \qt{By left unit law}\\
                = & \deno{\gtyperelation{v_2\ssub{x}{v_1}}{\meb}}\\
            \end{split}
        \end{equation}
        
        
        
        
        \case{\eqrightunit}
        
        Let $f = \deno{\gtyperelation{v}{\mea}}$ 
            \begin{equation}
            \begin{split}
                \deno{\gtyperelation{\doin{x}{v}{\return{x}}}{\mea}}  & = \bind{\e}{\1}{A} \after \T{\e}{(\point{A} \after \pp)} \after \tstrength{\e}{\G}{A}\after \pr{\idg}{f} \\
                & = \T{\e}{\pp} \after \tstrength{\e}{\G}{A} \after \pr {\idg}{f} \\
                & = \pp \after \pr{\idg}{f}\\
                & = f
            \end{split}
        \end{equation}
        
        
        
        \case{\eqassociativity}
        Let
        \begin{align*}
            f ={}& \deno{\gtyperelation{v_1}{\M{\e_1}{A}}}\\
            g ={}& \deno{\typerelation{\gax}{v_2}{\M{\e_2}{B}}}\\
            h ={}& \deno{\typerelation{\gby}{v_3}{\M{\e_3}{C}}}
        \end{align*}
        
        We also have the weakening:
        \begin{equation}
            \wrel{\i\pi\times}{\gax, y: B}{\gby}
        \end{equation}
        
        With denotation:
        
        \begin{equation}
            \deno{\wrel{\i\pi\times}{\gax, y: B}{\gby}} = (\p \times \Id{B})
        \end{equation}
        
        We need to prove that the following are equal
        
        \begin{align*}
            lhs & =  \deno{\gtyperelation{\doin{x}{v_1}{(\doin{y}{v_2}{v_3})}}{\M{\e_1\dot\e_2\dot\e_3}{C}}} \\
            & = \bind{\e_1}{\e_2\dot\e_3}{C}\after\T{
                \e_1}{(\bind{\e_2}{\e_3}{C}\after\T{\e_2}{h\after(\p \times \Id{B})}\after\tstrength{\e_2}{(\G\times A)}{B}\after\pr{\Id{(\G\times A)}}{g}
                )}\after\tstrength{\e_1}{\G}{A}\after\pr{\idg}{f}\\
                rhs &  = \deno{\gtyperelation{\doin{y}{(\doin{x}{v_1}{v_2})}{v_3}}{\M{\e_1\dot\e_2\dot\e_2}{C}}}  \\
            & = \bind{\e_1\dot\e_2}{\e_3}{C}\after\T{\e_1\dot\e_2}{(h)}\after\tstrength{\e_1\dot\e_2}{\G}{B}\after\pr{\idg}{(\bind{\e_1}{\e_2}{B}\after\T{\e_1}{g}\after\tstrength{\e_1}{\G}{A}\after\pr{\idg}{f})} \\
        \end{align*}
        Let's look at fragment $F$ of $rhs$.
        \begin{equation}
            F = \tstrength{\e_1\dot\e_2}{\G}{B}\after\pr{\idg}{(\bind{\e_1}{\e_2}{B}\after\T{\e_1}{g}\after\tstrength{\e_1}{\G}{A}\after\pr{\idg}{f})}
        \end{equation}
        
        So 
        \begin{equation}
            rhs = \bind{\e_1\dot\e_2}{\e_3}{C}\after\T{\e_1\dot\e_2}{(h)}\after F
        \end{equation}
        
    
        \scalebox{0.6}{\parbox{1.4\linewidth}{%
        \begin{align*}
                F & = \tstrength{\e_1\dot\e_2}{\G}{B}\after(\idg\times\bind{\e_1}{\e_2}{B})\after(\idg\times\T{\e_1}{g})\after\pr{\idg}{\tstrength{\e_1}{\G}{A}\after\pr{\idg}{f}}\\
                &  = \bind{\e_1}{\e_2}{(\G\times B)} \after \T{\e_1}{\tstrength{\e_2}{\G}{B}} \after\tstrength{\e_1}{\G}{(\T{\e_2}{B})} \after (\idg \after\T{\e_1}{g})\after\pr{\idg}{\tstrength{\e_1}{\G}{A}\after\pr{\idg}{f}}\qt{By commutativity of bind and tensor-strength}
                \\ & = \bind{\e_1}{\e_2}{(\G \times B))} \after \T{\e_1}{(\tstrength{\e_2}{\G}{B} \after (\idg \times g))} \after \tstrength{\e_1}{\G}{(\G \times A)} \after \pr{\idg}{\tstrength{\e_1}{\G}{A}\after\pr{\idg}{f}}\qt{By naturality of t-strength}
        \end{align*}
        }}
        
        Since
        $
            rhs = \bind{\e_1\dot\e_2}{\e_3}{C}\after\T{\e_1\dot\e_2}{(h)}\after F
        $, 
        
        \scalebox{0.8}{\parbox{1.2\linewidth}{%
            \begin{align*}
                rhs = &\bind{\e_1\dot\e_2}{\e_3}{C}\after\T{\e_1\dot\e_2}{(h)}\after\bind{\e_1}{\e_2}{(\G \times B))} \after \T{\e_1}{(\tstrength{\e_2}{\G}{B} \after (\idg \times g))} \after \tstrength{\e_1}{\G}{(\G \times A)} \after \pr{\idg}{\tstrength{\e_1}{\G}{A}\after\pr{\idg}{f}}\\
                = &\bind{\e_1\dot\e_2}{\e_3}{C}\after\bind{\e_1}{\e_2}{(\T{\e_3}{C})}\after\T{\e_1}{(\T{\e_2}{(h)} \after \tstrength{\e_2}{\G}{B} \after (\idg \times g))} \after \tstrength{\e_1}{\G}{(\G \times A)} \after \pr{\idg}{\tstrength{\e_1}{\G}{A}\after\pr{\idg}{f}}\qt{Naturality of $\mu$}\\
                = & \bind{\e_1}{\e_2\dot\e_3}{C}\after\T{\e_1}{(\bind{\e_2}{\e_3}{C}\after \T{\e_2}{(h)} \after \tstrength{\e_2}{\G}{B} \after (\idg \times g))} \after \tstrength{\e_1}{\G}{(\G \times A)} \after \pr{\idg}{\tstrength{\e_1}{\G}{A}\after\pr{\idg}{f}}
            \end{align*}    
        }} 
        
        Let's now look at the fragment $G$ of $rhs$
        \begin{equation}
            G = \T{\e_1}{(\idg\times g)}\after\tstrength{\e_1}{\G}{(\G\times A)}\after\pr{\idg}{\tstrength{\e_1}{\G}{A}\after\pr{\idg}{f}}
        \end{equation}
        
        So
        \begin{equation}
            rhs = \bind{\e_1}{\e_2\dot\e_3}{C}\after\T{\e_1}{(\bind{\e_2}{\e_3}{C}\after \T{\e_2}{(h)} \after \tstrength{\e_2}{\G}{B})}\after G
        \end{equation}
        
        
        By folding out the $\pr{...}{...}$, we have
        \begin{equation}
            G = \T{\e_1}{(\idg\times g)}\after\tstrength{\e_1}{\G}{\G \times A}\after(\idg\times\tstrength{\e_1}{\G}{A})\after\pr{\idg}{\pr{\idg}{f}}
        \end{equation}
        
       Using the commutativity of tensor strength with the associativity natural transformation $\a$, we have:
        
        \begin{tikzcd}
            \G 
            \arrow [r, "\pr{\idg}{\pr{\idg}{f}}"]
            &
            \G\times(\G \times \T{\e_1}{A}) 
            \arrow [d, "\idg\times\tstrength{\e_1}{\G}{A}"]
            &
            (\G \times \G) \times\T{\e_1}{A} 
            \arrow [l, "\alpha_{\G, \G, (\T{\e_1}{A})}"]
            \arrow [d, "\tstrength{\e_1}{(\G \times \G)}{A}"]
            \\
            &
            \G \times \T{\e_1}{(\G \times A)}
            \arrow [d, "\tstrength{\e_1}{\G}{\G \times A}"]
            & \T{\e_1}{((\G \times \G)\times A)} 
            \arrow [ld, "\T{\e_1}{\alpha_{\G, \G, A}}"]
            \\
            & \T{\e_1}{(\G \times (\G \times A))}
        \end{tikzcd}
        
        Where $\alpha: ((\_ \times \_) \times \_) \rightarrow (\_ \times (\_ \times \_))$ is a natural isomorphism.
        
        \begin{align*}
            \alpha & = \pr{\p\after\p}{\pr{\pp\after\p}{\pp}}\\
            \alpha^{-1} &= \pr{\pr{\p}{\p\after\pp}}{\pp\after\pp}
        \end{align*}
        
        So:
        
        \scalebox{.8}{\parbox{1.2\linewidth}{%
        \begin{align*}
                G = & \T{\e_1}{((\idg\times g)\after \alpha_{\G, \G, A})} \after \tstrength{\e_1}{(\G\times\G)}{A}\after\alpha^{-1}_{\G, \G, (\T{\e_1}{A})}\after\pr{\idg}{\pr{\idg}{f}} \\
                = & \T{\e_1}{((\idg\times g)\after \alpha_{\G, \G, A})} \after \tstrength{\e_1}{(\G\times\G)}{A}\after(\pr{\idg}{\idg}\times \Id{\T{\e_1}{A}})\after\pr{\idg}{f}\qt{By definition of $\alpha$ and products}\\
                = & \T{\e_1}{((\idg\times g)\after \alpha_{\G, \G, A}\after(\pr{\idg}{\idg}\times \Id{A}))}\after\tstrength{\e_1}{\G}{A}\after\pr{\idg}{f}\qt{By tensor strength's left-naturality}\\
                = & \T{\e_1}{((\p\times\Id{\T{\e_2}{B}})\after\pr{\Id{(\G\times A)}}{g})}\after\tstrength{\e_1}{\G}{A}\after\pr{\idg}{f}
        \end{align*}
        }}
        
        Since 
        \begin{equation}
        rhs = \bind{\e_1}{\e_2\dot\e_3}{C}\after\T{\e_1}{(\bind{\e_2}{\e_3}{C}\after \T{\e_2}{(h)} \after \tstrength{\e_2}{\G}{B})}\after G
        \end{equation}
        
        We have
        
        \scalebox{.8}{\parbox{1.2\linewidth}{%
        \begin{align*}
            rhs ={} & \bind{\e_1}{\e_2\dot\e_3}{C}\after\T{\e_1}{(\bind{\e_2}{\e_3}{C}\after \T{\e_2}{(h)} \after \tstrength{\e_2}{\G}{B}\after(\p\times\Id{\T{\e_2}{B}})\after\pr{\Id{(\G\times A)}}{g})}\after\tstrength{\e_1}{\G}{A}\after\pr{\idg}{f}\\
            ={} & \bind{\e_1}{\e_2\dot\e_3}{C}\after\T{\e_1}{(\bind{\e_2}{\e_3}{C}\after \T{\e_2}{(h \after (\p\times\Id{B}))}\after\tstrength{\e_2}{(\G\times A)}{B}\after\pr{\Id{(\G\times A)}}{g})}\after\tstrength{\e_1}{\G}{A}\after\pr{\idg}{f}\qt{By Left-Tensor Strength}\\
            ={} & lhs \qt{Woohoo!}
        \end{align*}
        }}
        
        %%%%%% END CASE ASSOCIATIVE %%%%%%%
        
        \case{\eqeta}
        Let 
        \begin{equation}
            f = \deno{\gtyperelation{v}{\ab}} : \G \rightarrow (B)^{A}
        \end{equation}
        
        By weakening, we have
        
        \begin{align*}
            \deno{\typerelation{\gax}{v}{\ab}} & = f \after \p : \G \times A \rightarrow (B)^A \\
            \deno{\typerelation{\gax}{\apply{v}{x}}{B}} & = \app\after\pr{f \after \p}{\pp}
        \end{align*}
        
        Hence, we have 
        \begin{align}
                \deno{\gtyperelation{\lam{x}{A}{(\apply{v}{x})}}{\ab}} & = \cur{\app\after\pr{f \after \p}{\pp}}\\
                \app\after(\deno{\gtyperelation{\lam{x}{A}{(\apply{v}{x})}}{\ab}} \times \Id{A}) & = \app \after(\cur{\app\after\pr{f \after \p}{\pp}}\times \Id{A})\\
                & = \app \after\pr{f\after\p}{\pp}\\
                & = \app\after(f\times\Id{A})
        \end{align}
        
        Hence, by the fact that $\cur{f}$ is unique in a cartesian closed category, 
        
        \begin{equation}
            \deno{\gtyperelation{\lam{x}{A}{(\apply{v}{x})}}{\ab}} = f =\deno{\gtyperelation{v}{\ab}}
        \end{equation}
        
        \case{\eqiftrue}
        Let
        \begin{align*}
            f = &\deno{\gtyperelation{v_1}{ A}}\\
            g = &\deno{\gtyperelation{v_2}{ A}}\\
        \end{align*}
        
        Then
        \begin{equation}
            \begin{split}
                \deno{\gtyperelation{\pifthenelse{A}{\t}{v_1}{v_2}}{ A}} & = \ifMorph{\inl\after\term{\G}}{f}{g} \\
                & = \app\after((\cur{f\after\pp}\after\term{\G})\times\idg)\after\diag{\G}\\
                & = \app\after(\cur{f\after\pp}\times\idg)\after(\term{\G}\times\idg)\after\diag{\G}\\
                & = f\after\pp\after\pr{\term{\G}}{\idg}\\
                & = f \\
                & = \deno{\gtyperelation{v_1}{ A}}\\
            \end{split}
        \end{equation}
        
        
        \case{\eqiffalse}
        Let
        \begin{align*}
            f = &\deno{\gtyperelation{v_1}{ A}}\\
            g = &\deno{\gtyperelation{v_2}{ A}}\\
        \end{align*}
        
        Then
        \begin{equation}
            \begin{split}
                \deno{\gtyperelation{\pifthenelse{A}{\f}{v_1}{v_2}}{ A}} & = \ifMorph{\inr\after\term{\G}}{f}{g} \\
                & = \app\after((\cur{g\after\pp}\after\term{\G})\times\idg)\after\diag{\G}\\
                & = \app\after(\cur{g\after\pp}\times\idg)\after(\term{\G}\times\idg)\after\diag{\G}\\
                & = g\after\pp\after\pr{\term{\G}}{\idg}\\
                & = g \\
                & = \deno{\gtyperelation{v_2}{ A}}\\
            \end{split}
        \end{equation}
        
        \subsection{Case If-Eta}
            Let 
            \begin{align*}
                f & = \deno{\typerelation{\G}{v_2}{\B}} \\
                g & = \deno{\typerelation{\G, x: \B}{v_1}{ A}} \\
            \end{align*}
        
        
            Then by the substitution theorem,
            \begin{align*}
                \deno{\typerelation{\G}{v_1\ssub{x}{\t}}{ A}} & = g\after\pr{\idg}{\inl_\1\after{\term{\G}}}\\
                \deno{\typerelation{\G}{v_1\ssub{x}{\f}}{ A}} & = g\after\pr{\idg}{\inr_\1\after{\term{\G}}}\\
                \deno{\gtyperelation{v_1\ssub{x}{v_2}}{ A}} & = g\after\pr{\idg}{f}
            \end{align*}
        
            Hence we have (Using the diagonal and twist morphisms):
         
        
        \scalebox{.6}{\parbox{1.5\linewidth}{%
        \begin{align*}
                \deno{\gtyperelation{&\pifthenelse{A}{v}{v_1\ssub{x}{\t}}{v_1\ssub{x}{\f}}}{ A}}  \\ 
                & = \ifMorph{f}{g\after\pr{\idg}{\inl_\1\after{\term{\G}}}}{g\after\pr{\idg}{\inr_\1\after{\term{\G}}}}\\
                & = \app\after((\fld{
                    \cur{g\after\pr{\pp}{\inl_1\after\term{\G}\after\pp}}
                }{
                    \cur{g\after\pr{\pp}{\inr_1\after\term{\G}\after\pp}}
                }\after f)\times \idg)\after\diag{\G}\qt{Pairing property}\\
                &= \app\after((\fld{
                    \cur{g\after\pr{\pp}{\inl_1\after\term{\G}\after\p}}
                }{
                    \cur{g\after\pr{\pp}{\inr_1\after\term{\G}\after\p}}
                }\after f)\times \idg)\after\diag{\G}\qt{Terminal is unique}\\
                &= \app\after((\fld{
                    \cur{g\after(\idg\times (\inl_\1\after\term{\1}))\after\twist{\1}{\G}}
                }{
                    \cur{g\after(\idg\times (\inr_\1\after\term{\1}))\after\twist{\1}{\G}}
                }\after f)\times \idg)\after\diag{\G}\qt{Definition of the twist morphism}\\
                & = \app\after((\fld{
                    \cur{g\after(\idg\times \inl_\1)\after\twist{\1}{\G}}
                }{
                    \cur{g\after(\idg\times \inr_\1)\after\twist{\1}{\G}}
                }\after f)\times \idg)\after\diag{\G}\qt{Identity = $\Id{\1}$}\\
                & = \app\after((\fld{
                    \cur{g\after\twist{\1+\1}{\G}\after(\inl_\1\times \idg)}
                }{
                    \cur{g\after\twist{\1+\1}{\G}\after(\inr_\1\times \idg)}
                }\after f)\times \idg)\after\diag{\G}\qt{Twist commutivity}\\
                & = \app\after((\fld{
                    \cur{g\after\twist{\1+\1}{\G}}\after\inl_\1
                }{
                    \cur{g\after\twist{\1+\1}{\G}}\after\inr_\1
                }\after f)\times \idg)\after\diag{\G}\qt{Exponential property}\\
                & = \app\after((\cur{g\after\twist{\1+\1}{\G}}\after\fld{
                    \inl_\1
                }{
                    \inr_\1
                }\after f)\times \idg)\after\diag{\G}\qt{Factoring out $\cur{..}$}\\
                & = \app\after((\cur{g\after\twist{\1+\1}{\G}}\after f)\times \idg)\after\diag{\G}\qt{Since $\fld{\inl}{\inr} is the identity$}\\
                & = \app\after(\cur{g\after\twist{\1+\1}{\G}}\times \idg)\after (f\times \idg)\after\diag{\G}\qt{Factoring}\\
               & = g\after\twist{\1+\1}{\G} \after (f\times \idg)\after\diag{\G} \qt{Definition of $\app,\cur{..}$}\\
               & = g\after(\idg\times f)\after \twist{\G}{\G} \after\diag{\G} \qt{Twist commutivity}\\
               & = g\after(\idg\times f) \after\pr{\idg}{\idg} \qt{Twist, diagonal defintions} \\
               & = g\after\pr{\idg}{f} \\
               & = \deno{\gtyperelation{v_1\ssub{x}{v_2}}{ A}}
            \end{align*}
            }}
        
        
        \subsection{Congruences}
        These cases can be proved fairly mechanically by assuming the preconditions, using induction to prove that the matching pairs of subexpressions have equal denotations, then constructing the denotations of the expressions using the equal denotations which gives trivially equal denotations.
        
        \case{\eqfun}
            By inversion, we have $\berelation{\gax}{v_1}{v_2}{B}$
            By induction, we therefore have $\deno{\typerelation{\gax}{v_1}{B}} = \deno{\typerelation{\gax}{v_2}{B}}$
        
            Then let
            \begin{equation}
                f = \deno{\typerelation{\gax}{v_1}{B}} = \deno{\typerelation{\gax}{v_2}{B}}
            \end{equation}
        
            And so
            \begin{equation}
                \deno{\gtyperelation{\lam{x}{A}{v_1}}{\ab}} = \cur{f} = \deno{\gtyperelation{\lam{x}{A}{v_2}}{\ab}}
            \end{equation}
        
        
        \case{\eqreturn}
        By inversion, we have $\gberelation{v_1}{v_2}{A}$
        By induction, we therefore have $\deno{\gtyperelation{v_1}{A}} = \deno{\gtyperelation{v_2}{A}}$
        
        Then let
        \begin{equation}
            f = \deno{\gtyperelation{v_1}{A}} = \deno{\gtyperelation{v_2}{A}}
        \end{equation}
        
        And so
        \begin{equation}
            \deno{\gtyperelation{\return{v_1}}{\M{\1}{A}}} = \point{A}\after f = \deno{\gtyperelation{\return{v_2}}{\M{\1}{A}}}
        \end{equation}
        
        
        \case{\eqapply}
        By inversion, we have $\gberelation{v_1}{v_1'}{\ab}$ and $\gberelation{v_2}{v_2'}{A}$
        By induction, we therefore have $\deno{\gtyperelation{v_1}{\ab}}= \deno{\gtyperelation{v_1'}{\ab}}$ and $\deno{\gtyperelation{v_2}{A}} = \deno{\gtyperelation{v_2'}{A}}$
        
        Then let
        \begin{align*}
            f &{}= \deno{\gtyperelation{v_1}{\ab}} = \deno{\gtyperelation{v_1'}{\ab}}\\
            g &{}= \deno{\gtyperelation{v_2}{A}} = \deno{\gtyperelation{v_2'}{A}}
        \end{align*}
        
        
        
        And so
        \begin{equation}
            \deno{\gtyperelation{\apply{v_1}{v_2}}{\M{\e}{A}}} = \app\after\pr{f}{g} = \deno{\gtyperelation{\apply{v_1'}{v_2'}}{\M{\e}{A}}}
        \end{equation}
        
        
        \case{\eqbind}
        By inversion, we have $\gberelation{v_1}{v_1'}{\M{\e_1}{A}}$ and $\berelation{\gax}{v_2}{v_2'}{\M{\e_2}{B}}$
        By induction, we therefore have $\deno{\gtyperelation{v_1}{\M{\e_1}{A}}}= \deno{\gtyperelation{v_1'}{\M{\e_1}{A}}}$ and $\deno{\typerelation{\gax}{v_2}{\M{\e_1}{A}}} = \deno{\typerelation{\gax}{v_2'}{\M{\e_2}{B}}}$
        
        Then let
        \begin{align*}
            f &{}=\deno{\gtyperelation{v_1}{\M{\e_1}{A}}}= \deno{\gtyperelation{v_1'}{\M{\e_1}{A}}}\\
            g &{}=\deno{\typerelation{\gax}{v_2}{\M{\e_2}{B}}} = \deno{\typerelation{\gax}{v_2'}{\M{\e_2}{B}}}
        \end{align*}
        
        
        
        And so
        \begin{equation}
            \begin{split}
                \deno{\gtyperelation{\doin{x}{v_1}{v_2}}{\M{\e_1\dot\e_2}{A}}} & = \bind{\e_1}{\e_2}{B} \after \T{\e_1}{g} \after \tstrength{\e_1}{\G}{A} \after \pr{\idg}{f}\\
                 & = \deno{\gtyperelation{\doin{x}{v_1}{v_2}}{\M{\e_1\dot\e_2}{A}}}
            \end{split}
        \end{equation}
        \case{\eqif}
        By inversion, we have $\gberelation{v}{v'}{\B}$, $\gberelation{v_1}{v_1'}{\M{\e_1}{A}}$ and $\gberelation{v_2}{v_2'}{\M{\e_2}{B}}$
        By induction, we therefore have $\deno{\gtyperelation{v}{\B}} = \deno{\gtyperelation{v'}{B}}$, $\deno{\gtyperelation{v_1}{ A}}= \deno{\gtyperelation{v_1'}{ A}}$ and $\deno{\typerelation{\gax}{v_2}{\M{\e_2}{B}}} = \deno{\typerelation{\gax}{v_2'}{\M{\e_2}{B}}}$
        
        Then let
        \begin{align*}
            f &{}=\deno{\gtyperelation{v}{\B}} = \deno{\gtyperelation{v'}{B}}\\
            g &{}=\deno{\gtyperelation{v_1}{\M{\e_1}{A}}}= \deno{\gtyperelation{v_1'}{\M{\e_1}{A}}}\\
            h &{}=\deno{\typerelation{\gax}{v_2}{\M{\e_2}{B}}} = \deno{\typerelation{\gax}{v_2'}{\M{\e_2}{B}}}
        \end{align*}
        
        
        
        And so
        \begin{equation}
            \begin{split}
                \deno{\gtyperelation{\pifthenelse{A}{v}{v_1}{v_2}}} & = \ifMorph{f}{g}{h}\\
                 & = \deno{\gtyperelation{\doin{x}{v_1}{v_2}}{\M{\e_1\dot\e_2}{A}}}
            \end{split}
        \end{equation}
        \case{\eqsubtype}
        By inversion, we have $\gberelation{v_1}{v_2}{A}$, and $A \subtype B$ 
        By induction, we therefore have $\deno{\gtyperelation{v_1}{A}} = \deno{\gtyperelation{v_2}{A}}$
        
        Then let
        \begin{align*}
            f &{}=\deno{\gtyperelation{v_1}{A}} = \deno{\gtyperelation{v_2}{A}}\\
            g &{}=\deno{A \subtype B}
        \end{align*}
        
        
        
        And so
        \begin{equation}
                \deno{\gtyperelation{v_1}{B}} = g \after f = \deno{\gtyperelation{v_1}{B}}
        \end{equation}
        
        $$\square$$
    \end{proof}
\end{framed}

\end{document} 
