% Won't compile on its own.

% A list of properties that an S-preserving functor should preserve.


\chapter{S-Category Functors}

This appendix gives a terse introduction to the properties of functors between S-categories. In particular, it covers re-indexing functors $f\star$ and the quantification functor $\allEn$.

\section{Re-Indexing Functors}\label{AppendixReindexingFunctors}
For each morphism $f: E^m \rightarrow E^n$ in $\C$, there should be a co-variant, re-indexing functor  $f\star: \C(E^n) \rightarrow \C(E^m)$, which is S-preserving. That is, it preserves the S-category structure of $\C(E^n)$. (See below). $(-)\star$ should be a contra-variant functor in its $\C$ argument and co-variant in its right argument.

\begin{itemize}
    \item $(g\after f)\star(a) = f\star(\g\star(a))$
    \item $\Id{E^n}\star(a) = a$
    \item $f\star(\Id{A}) = \Id{f\star(A)}$
    \item $f\star(a\after b) = f\star(a)\after f\star(b)$
\end{itemize}
\subsection{Preserves Ground Types}
If $\deno{\g}\in\obj\C(E^n)$ then $f\star\deno{\g} = \deno{\g}\in\obj\C(E^m)$
\subsection{$f\star$ Preserves Products}
If $\pr{g}{h}:\C(E^n)(Z, X\times Y)$
Then 
\begin{align*}
    f\star(X\times Y) & = f\star(X)\times f\star(Y)\\
    f\star(\pr{g}{h}) & = \pr{f\star(g)}{f\star{h}}&:\C(E^m)(f\star Z, f\star(X)\times f\star(Y))\\
    f\star(\p) & = \p&:\C(E^m)(f\star(X)\times f\star(Y), f\star(X)) \\
    f\star(\pp) &= \pp&:\C(E^m)(f\star(X)\times f\star(Y), f\star(Y))
\end{align*}

\subsection{$f\star$ Preserves Terminal Object}
If $\Id{A}:\C(E^n)(A, \1)$
Then 
\begin{align*}
    f\star(\1) & = \1 \\
    f\star(\term{A}) & = \term{f\star(A)}&:\C(E^m)(f\star A, \1)\\
\end{align*}

\subsection{$f\star$ Preserves Exponentials}
\begin{align*}
    f\star(Z^X) & = (f\star(Z))^{(f\star(X))}\\
     f\star(\app) &= \app&:\C(E^m)(f\star(Z^X)\times f\star(X), f\star(Z))\\
     f\star(\cur{g}) &= \cur{f\star(g)}&:\C(E^m)(f\star(Y)\times f\star(X), f\star(Z)^{f\star(X)})
\end{align*}

\subsection{$f\star$ Preserves Co-product on Terminal}

\begin{align*}
    f\star(\1+\1) &= \1+\1\\
    f\star(\inl)  &= \inl&:\C(E^m)(\1, \1+\1) \\
    f\star(\inr) &= \inr&:\C(E^m)(\1, \1+\1) \\
    f\star([g, h]) &= [f\star(g), f\star(h)]&:\C(E^m)(\1+\1, f\star(Z))
\end{align*}

\subsection{$f\star$ Preserves Graded Monad}
\begin{align*}
    f\star(\tea) &= \T{f\star(\e)}{f\star(A)}&:\C(E^m)\\    
    f\star(\point{A}) &= \point{f\star(A)}&:\C(E^m)(f\star(A), f\star(\toa))\\
    f\star(\bind{\e_1}{\e_2}{A}) &= \bind{f\star(\e_1)}{f\star(\e_2)}{f\star(A)}&:\C(E^m)(f\star(\T{\e_1}{\T{\e_2}{A}}), f\star(\T{\e_1\dot\e_2}{A}))\\
\end{align*}

\subsection{$f\star$ and Effects}
\begin{align*}
    f\star(\1) &= \1 \qt{The unit effect}\\
    f\star(\e_1\dot\e_2) &= f\star(\e_1)\dot f\star(\e_2)\qt{Multiplication}\\
\end{align*}

This is done By
\begin{align*}
    f\star\deno{\typerelation{\P}{\e}{\effect}} & = \deno{\typerelation{\P}{\e}{\effect}}\after f
\end{align*}

\subsection{$f\star$ Preserves Tensor Strength}
\begin{align*}
    f\star(\tstrength{\e}{A}{B}) &= \tstrength{f\star(\e)}{f\star(A)}{f\star(B)} &: \C(E^m)(f\star(A\times\teb), f\star(\T{\e}{(A\times B)}))
\end{align*}
\subsection{$f\star$ Preserves Ground Constants}
For each ground constant $\deno{\const{A}}$ in $\C(E^n)$,

\begin{align*}
    f\star(\deno{\const{A}}) = \const{f\star(A)} : \C(E^m)(\1, f\star(A))
\end{align*}
\subsection{$f\star$ Preserves Ground Subeffecting}
For ground effects $e_1, e_2$ such that $e_1\subeffect e_2$



\begin{align*}
    f\star(e) & = e: \C(E^m)\\
    f\star\db{\e_1\subeffect e_2}_A = \db{e_1 \subeffect e_2}_{f\star(A)} &:\C(E^m){f\star(\T{e_1}{A}), f\star(\T{e_2}{A})} \\
\end{align*}
\subsection{$f\star$ Preserves Ground Subtyping}
For ground types $\g_1, \g_2$ such that $\g_1\subtypeg\g_2$

\begin{align*}
    f\star{\g} = \g: \C(E^m)(\1, \g)\\
    f\star(\deno{\g_1 \subtypeg \g_2}) & = \deno{\g_1 \subtypeg \g_2} &: \C(E^m)(\g_1, \g_2)\\
\end{align*}

\section{The $\allEn$ functor}

We expand $\allEn: \C(E^n\times E) \rightarrow \C(E^n)$ to be a functor which is right adjoint to the re-indexing functor $\pstar$.

\begin{equation}
    \bar{(\_)}: \C(E^n\times E)(\pstar A, B) \leftrightarrow \C(E^n)(A, \allEn B) : \widehat{(\_)}
\end{equation}

For $A\in\obj\C(E^n)$, $B\in\obj\C(E^n\times E)$.

Hence the action of $\allEn$ on a morphism $l : A\rightarrow A'$ is as follows:
\begin{eqnarray}
    \allEn(l) = \bar{l\after\counit{A}}
\end{eqnarray}
Where $\e_A: \C(E^n\times E)(\pstar\allEn A \rightarrow A)$ is the co-unit of the adjunction.

\subsection{Beck-Chevalley Condition}
We need to be able to commute the $\allEn$ functor with re-indexing functors. A natural way to do this is:
\begin{align*}
    \theta\star\after\allEn & = \allEm\after(\theta\times\Id{E})\star
\end{align*}

We shall also require that the canonical natural-transformation between these functors is the identity.

That is, $\bar{(\theta\times\Id{E})\star(\counit{})} = \Id{}: \theta\star\after\allEn \rightarrow \allEm\after(\theta\times\Id{E})\star\in \C(E^m)$

This shall be called the Beck-Chevalley condition.


\section{Naturality Corollaries}
Here are some simple corollaries of the adjunction between $\pstar$ and $\allEn$.
    
    \subsection{Naturality}
    By the definition of an adjunction:
    
    \begin{equation}
        \bar{f\after\pstar(n)} = \bar{f}\after n
    \end{equation}
    
    \subsection{$\bar{(-)}$ and Re-indexing Functors}
    By assuming the Beck-Chevalley condition that:
    
    \begin{equation}
        \bar{(\theta\times\Id{E})\star(\counit{})} = \Id{}: \theta\star\after\allEn \rightarrow \allEm\after(\theta\times\Id{E})\star
    \end{equation}
    
    We then have:
    
    \begin{align*}
        \theta\star\unit{A}:\quad&\theta\star A \rightarrow \theta\star\after\allEn\after\pstar A\\
        \theta\star\unit{} =& \bar{(\theta\times\Id{E})\star(\counit{\pstar})}\after\theta\star\unit{}\\
        =& (\allEm\after(\theta\times\Id{E})\star)(\counit{\pstar})\after\unit{(\allEm\after(\theta\times\Id{E})\star)\after\pstar}\after\theta\star\unit{}\\
        = & (\allEm\after(\theta\times\Id{E})\star)(\counit{\pstar})\after\unit{\theta\star\after\allEn\after\pstar}\after\theta\star\unit{}\\
        = & (\allEm\after(\theta\times\Id{E})\star)(\counit{\pstar}) \after(\theta\star\after\allEn\after\pstar)\unit{} \after\unit{(\theta\times\Id{E})\star}\\
        = & (\theta\star\after\allEn)(\counit{\pstar}
        \after\pstar\unit{})\after\unit{(\theta\times\Id{E})\star}\\
        = & (\theta\star\after\allEn)(\Id{})\after\unit{(\theta\times\Id{E})\star}\\
        = & \unit{(\theta\times\Id{E})\star}
    \end{align*}
    \begin{align*}
        \theta\star(\bar{f}) & = \theta\star(\allEn(f)\after\eta_A)\\
        & = \theta\star(\allEn(f))\after\theta\star(\eta_A)\\
        & =  (\allEm\after(\theta\times\Id{E})\star)f \after\unit{(\theta\times\Id{E})\star A}\\
        & = \bar{(\theta\times\Id{E})\star f}\\
    \end{align*}
    
    \subsection{$\hat{(-)}$ and Re-Indexing Functors}
    \begin{align*}
        \theta\star(\pr{\Id{E^n}}{\rho}\star(\widehat{m})) &= (\pr{\Id{E^n}}{\rho}\after\theta)\star(\widehat{m})\\
        & = ((\theta\times\Id{E})\after\pr{\Id{E^n}}{\rho})\star(\widehat{m})\\
        & = \pr{\Id{E^n}}{\rho\after\theta}\star(\theta\times\Id{E})\star(\widehat{m}) \\
        & = \pr{\Id{E^n}}{\theta\star\rho}\star(\theta\star(\widehat{m}))
    \end{align*}

\subsection{Pushing Morphisms into $f\star$}

\begin{align*}
    \pr{\Id{E^n}}{\rho}\star(\widehat{m})\after n &= \pr{\Id{E^n}}{\rho}\star(\widehat{m})\after\pr{\Id{E^n}}{\rho}\star\p\star(n)\\
    & = \pr{\Id{E^n}}{\rho}\star(\widehat{m}\after\pstar(n))\\
    &= \pr{\Id{E^n}}{\rho}\star(\widehat{m\after n})
\end{align*}