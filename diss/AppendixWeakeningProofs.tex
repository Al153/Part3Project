\chapter{Additional Proofs}\label{WeakeningProofs}

In this appendix, I provide introductions to some proofs which are too similar to proofs already in the dissertation to include.

\section{Effect Weakening on Effects}\label{AppendixEffectWeakeningOnEffects}

It is harder to use inversion on the structure of $\w$, since the structure of $\w$ does not depend as strongly on the structure of $\P$. I present here the cases for variables.
    
        \case{\evar}
        We do a case split on $\w$.
        \subcase{$\w = \i$}
        Then $\P' = \P$ and $\w = \Id{I}$. So the theorem holds trivially.
        \subcase{$\w = \w'\x$}
        Then by the definition of its denotation:    
        \begin{align*}
            \deno{\typerelation{\P,\a}{\a}{\effect}}\after\w &= \pp\after(\w'\times \Id{E}) \\
            & = \pp\\
            & = \deno{\typerelation{\P',\a}{\a}{\effect}}
        \end{align*}
        
        \subcase{$\w = \w'\pi$}
        Then \begin{equation}
            \deno{\typerelation{\P,\a}{\a}{\effect}} = \pp\after\w'\after\p
        \end{equation}
        
        Here $\P' = \P,\b$ and $\wrel{\w'}{\P''}{\P}$.
        
        So\begin{align*}
            \pp\after\w' & = \deno{\typerelation{\P''}{\a}{\effect}}
            \\
            \pp\after\w'\after\p & =\deno{\typerelation{\P'',\b}{\a}{\effect}}\\
            &= \deno{\typerelation{\P'}{\a}{\effect}}
        \end{align*}
        
        \case{\eweaken}
        \begin{equation}
            \deno{\typerelation{\P,\b}{\a}{\effect}}\after\w = \deno{\typerelation{\P}{\a}{\effect}}\after\p\after\w
        \end{equation}
        
        Similarly, we perform a case split on structure of $\w$:
        
        \subcase{$\w=\i$}
        Then $\P' = \P,\b$ so $\w=\Id{I}$
        So $\deno{\typerelation{\P,\b}{\a}{\effect}}\after\w = \deno{\typerelation{\P'}{\a}{\effect}}$
        
        \subcase{$\w=\w'\p$}
        Then $\P' = (\P'',\g)$ and $\w=\w'\after\p$
        Here $\wrel{\w'}{\P''}{\P,\b}$.
        So
        \begin{align*}
            \deno{\typerelation{\P,\b}{\a}{\effect}}\after\w & = \deno{\typerelation{\P,\b}{\a}{\effect}}\after\w'\after\p\\
            & = {\typerelation{\P''}{\a}{\effect}}\after\p\\
            & = {\typerelation{\P'',\g}{\a}{\effect}}\\
            & = {\typerelation{\P'}{\a}{\effect}}\\
        \end{align*}
        
        \subcase{$\w=\w'\x$}
        Then $\P'=\P'',\b$ and $\wrel{\w'}{\P''}{\P}$
        
        So \begin{align*}
            \deno{\typerelation{\P,\b}{\a}{\effect}}\after\w &= \deno{\typerelation{\P}{\a}{\effect}}\after\p\after(\w'\times\Id{E})\\
            &=\deno{\typerelation{\P}{\a}{\effect}}\after\w'\after\p\\
            & = \deno{\typerelation{\P''}{\a}{\effect}}\after\p\\
            & = \deno{\typerelation{\P'}{\a}{\effect}}\\
        \end{align*}
    
        $$\square$$


\section{Effect Weakening on Types}\label{AppendixEffectWeakeningOnTypes}

This proof proceeds in a similar fashion the proof of effect substitution on types (Theorem \ref{EffectSubstitutionOnTypes}). That is, by induction over the derivation of $\deno{\typerelation{\P}{A}{\type}}$, and making use of the Beck-Chevalley condition and the S-preserving property of $\w\star$. 

\case{\tquant}
    This case uses the Beck-Chevalley condition and the fact that $\deno{\wrel{\w\x}{(\P', \a)}{(\P,\a)}} = \w\times \Id{E}$. This property is used in conjunction with induction to change the environment from $(\P, \a)$ to $(\P', \a)$.

    \begin{align*}
        \w\star\deno{\typerelation{\P}{\all{\a}A}{\type}} & = \w\star(\allI(\deno{\typerelation{\P,\a}{A}{\type}}))\\
        & = \allI((\w\times\Id{E})\star\deno{\typerelation{\P,\a}{A}{\type}})\qt{By Beck-Chevalley}\\
        & = \allI(\deno{\typerelation{\P',\a}{A}{\type}})\qt{By induction}\\
        & = \allI(\deno{\typerelation{\P',\a}{A}{\type}})\\
        & = \deno{\typerelation{\P'}{\all{\a}{A}}{\type}}\\
    \end{align*}

\case{\tfun}
This case makes use of the S-preserving property that $\w\star$ preserves exponentials. Specifically $\w\star(B^A) = (\w\star B)^{(\w\star A)}$.
\begin{align*}
    \w\star\deno{\typerelation{\P}{\ab}{\type}} &= \w\star(\deno{\typerelation{\P}{B}{\type}}^{\deno{\typerelation{\P}{A}{\type}}})\\
    &=\w\star(\deno{\typerelation{\P}{B}{\type}})^{\w\star(\deno{\typerelation{\P}{A}{\type}})}\\
    & = \deno{\typerelation{\P'}{B}{\type}}^{\deno{\typerelation{\P'}{A}{\type}}}\\
    & = \deno{\typerelation{\P'}{\ab}{\type}}\\
\end{align*}

$$\square$$

\section{Effect Weakening on Terms} \label{AppendixEffectWeakeningOnTerms}
This theorem is proved in a similar fashion to the effect substitution theorem (Theorem \ref{EffectSubstitutionOnTerms}) and many of its cases are the same.

\case{\vsubtype}
If $\D$ derives $\gpetyperelation{v}{B}$ then, by inversion, there exists $\D_1$ deriving $\gpetyperelation{v}{A}$, such that $ \D = \deno{A\subtypep B}\after \D_1$, as shown in Equation \ref{EffectWeakenTermsSubtype}.

\begin{equation}
    \label{EffectWeakenTermsSubtype}
    \D = \ntreeruleII{\vsubtype}{\treeruleI{\D_1}{\gpetyperelation{v}{A}}}{ A \subtypep B}{\gpetyperelation{v}{B}}
\end{equation}

By induction there exists $\D_1'$ deriving $\etyperelation{\P'}{\G}{v}{A}$, so we can construct $\D'$ using the subtyping rule in Equation \ref{EffectWeakenTermsSubtypeNew}.
So, using the weakening of the subtyping morphism, we can derive the denotation of $\D'$.

\begin{equation}
    \label{EffectWeakenTermsSubtypeNew}
    \D' = \ntreeruleII{\vsubtype}{\treeruleI{\D_1'}{\etyperelation{\P'}{\G}{v}{A}}}{A \subtypep B}{\etyperelation{\P'}{\G}{v}{B}}
\end{equation}

\begin{align*}
    \w\star(\D) & = \w\star{\deno{A\subtypep B}}\after\w\star\D_1 \\
    & = \deno{A\subtypepp B}\after\D_1'\qt{By induction}\\
    & = \D'
\end{align*}

\case{\vfun}
This case holds by induction and the S-preserving property of $\w\star$.

If $\D$ derives $\gpetyperelation{\lam{x}{A}{v}}{\ab}$ then by inversion there exists $\D_1$ deriving $\etyperelation{\P}{\gax}{v}{B}$ such that $\D$ and its denotation can be defined as in equations \ref{EffWeakenTermsLambdaTree} and \ref{EffWeakenTermsLambdaDeno}


\begin{equation}
    \label{EffWeakenTermsLambdaTree}
    \D = \ntreeruleI{\vfun}{
        \treeruleI{\D_1}{\etyperelation{\P}{\gax}{v}{B}}
    }{\gpetyperelation{\lam{x}{A}{v}}{\ab}}
\end{equation}


\begin{equation}\label{EffWeakenTermsLambdaDeno}
    \D = \cur{\D_1}
\end{equation}

By induction, there exists $\D_1'$ deriving $\etyperelation{\P'}{\gax}{v}{B}$ with denotation $\D_1' = \w\star(\D_1)$. Using S-preserving property, we can derive $\D'$ and its denotation, seen in equations \ref{EffWeakenTermsLambdaTreeNew} and \ref{EffWeakenTermsLambdaDenoNew}.

\begin{equation}
    \label{EffWeakenTermsLambdaTreeNew}
    \D' = \ntreeruleI{\vfun}{
        \treeruleI{\D_1'}{\etyperelation{\P'}{\gax}{v}{B}}
    }{\etyperelation{\P'}{\G}{\lam{x}{A}{v}}{\ab}}
\end{equation}


\begin{equation}\label{EffWeakenTermsLambdaDenoNew}
    \D' = \cur{\D_1'}\\
\end{equation}

Using S-closure, we can show that $\D' = \w\star(\D)$.

\begin{align*}
    \w\star(\D) & = \w\star(\cur{\D_1})\\
    & = \cur{\w\star(\D_1)}\qt{By S-preserving property}\\
    & = \cur{\D_1'}\qt{By induction}\\
    & = \D'
\end{align*}

$$\square$$

\section{Term Weakening Theorem}\label{AppendixTermWeakeningTheorem}
Proceeds by induction on the derivation of $\D$. Making use of the weakening of effect-variable environments on term weakenings in the case of (\textit{$\vgen$}).

\case{\vgen}
This case makes use of the effect weakening of term weakenings.

If $\D$ derives $\gpetyperelation{\elam{\a}{v}}{\all{\a}{A}}$, then by inversion, we have $\D_1$ such that:

\begin{equation}
    \D = \ntreeruleI{\vgen}{\treeruleI{\D_1}{\etyperelation{\P,\a}{\G}{v}{A}}}{\gpetyperelation{\elam{\a}{v}}{\all{\e}{A}}}
\end{equation}

By induction, we derive $\D_1'$ such that it completes the tree in Equation \ref{TermWeakELamOne}. By induction, its denotation can be related to that of $\D_1$, as in Equation \ref{EffWeakELamDenoOne}. Hence we can show that the denotation of $\D$ is preserved in Equation \ref{EffWeakELamDenoResult}.

\begin{equation}\label{TermWeakELamOne}
    \D' = \ntreeruleI{\vgen}{\treeruleI{\D_1'}{\etyperelation{\P,\a}{\G'}{v}{A}}}{\etyperelation{\P}{\G'}{(\elam{\a}{v})}{\all{\e}{A}}}
\end{equation}

\begin{align*}
    \D_1' & = \D_1\after\deno{\ewrel{\P,\a}{\w}{\G'}{\G}}\\
    & = \D_1\after\deno{\wrel{\i\pi}{\P,a}{\P}}\star(\w) \numberthis\label{EffWeakELamDenoOne}\\
    & = \D_1\after\pstar(\w)
\end{align*}

\begin{align*}
    \D\after\w & = \bar{\D_1}\after\w\\
    & = \bar{\D_1\after\pstar(\w)}\\
    & = \bar{\D_1'}\numberthis\label{EffWeakELamDenoResult}\\
    & = \D'
\end{align*}


\case{\vbind}
This case makes use of the properties of an S-category, specifically the tensor strength on the graded monad. By inversion, we have derivations $\D_1, \D_2$ such that:


\begin{equation}
    \D = \ntreeruleII{\vbind}{
        \treeruleI{\D_1}{\etyperelation{\P}{\G}{v_1}{\M{\E_1}{A}}}
    }{
        \treeruleI{\D_2}{\etyperelation{\P}{\G,x: A}{v_2}{\M{\e_2}{B}}}
    }{
        \etyperelation{\P}{\G}{\doin{x}{v_1}{v_2}}{\M{\e_1\dot\e_2}{B}}
    }
\end{equation}

If $\ewrel{\P}{\w}{\G'}{\G}$ then $\ewrel{\P}{\w\x}{\G',x:A}{\gax}$, so by induction, we can derive $\D_1'$, $\D_2'$ such that:

\begin{equation}
    \D' = \ntreeruleII{\vbind}{
        \treeruleI{\D_1'}{\etyperelation{\P}{\G'}{v_1}{\M{\e_1}{A}}}
    }{
        \treeruleI{\D_2'}{\etyperelation{\P}{\G',x: A}{v_2}{\M{\e_2}{B}}}
    }{
        \etyperelation{\P}{\G'}{\doin{x}{v_1}{v_2}}{\M{\e_1\dot\e_2}{B}}
    }
\end{equation}

This preserves denotations:

\begin{align*}
    \D' & = \bind{\e_1}{\e_2}{B}\after\T{\e_1}{\D_2'}\after\tstrength{\e_1}{\G'}{A}\after\pr{\Id{G'}}{\D_1'}\qt{By definition}\\
    & = \bind{\e_1}{\e_2}{B}\after\T{\e_1}{(\D_2\after(\w\times\Id{A}))}\after\tstrength{\e_1}{\G'}{A}\after\pr{\Id{G'}}{\D_1\after\w}\qt{By induction on $\D_1', \D_2'$}\\
    & = \bind{\e_1}{\e_2}{B}\after\T{\e_1}{\D_2}\after\tstrength{\e_1}{\G}{A}\after\pr{\w}{\D_1\after\w}\qt{By tensor strength}\\
    & = \bind{\e_1}{\e_2}{B}\after\T{\e_1}{\D_2}\after\tstrength{\e_1}{\G}{A}\after\pr{\idg}{\D_1}\after\w\qt{By product property}\\
    & = \D \qt{By definition}
\end{align*}

\case{\vreturn}
    We have the sub-derivation $\D_1$ such that
    \begin{equation}
        \D = \ntreeruleI{\vreturn}{\treeruleI{\D_1}{\gpetyperelation{v}{A}}}{\gpetyperelation{\return{v}}{\moa}}
    \end{equation}

    Hence, by induction, with $\ewrel{\P}{\w}{\G'}{\G}$, we find the derivation $\D_1'$ such that:
    \begin{equation}
        \D' = \ntreeruleI{\vreturn}{\treeruleI{\D_1'}{\etyperelation{\P}{\G'}{v}{A}}}{\etyperelation{\P}{\G'}{\return{v}}{\moa}}
    \end{equation}

    This preserves denotations:

    \begin{align*}
        \D' & = \point{A}\after\D_1' \qt{By definition}\\
            & = \point{A}\after\D_1\after\w\qt{By induction of $\D_1, \D_1'$}\\
            & = \D\after\w\qt{By Definition}
    \end{align*}

    $$\square$$