    
\begin{frame}{Instantiating a Model (1)}

        \begin{tikzpicture}[->,>=stealth', scale=0.5, every node/.style={scale=0.5}]
            % Draw the env objects
            \foreach \y[count=\c,evaluate={\yi=int(\c-1)}] in {3, 4, 5}{
                \node[fill,circle,inner sep=2pt,label=left:{\small $\P_\yi$}] (d\yi) at (0,\y) {};
            }
    
            % draw the ... above the env objects
            \foreach \y[count=\c,evaluate={\yi=int(\c-1)}] in {5.5, 6, 6.5}{
                \node[fill, circle, inner sep=1pt] (dd\yi) at (0,\y){};
            }
            % Draw the index category
            \node[fit=(d0) (d1) (d2) (dd0) (dd1) (dd2),ellipse,draw,minimum width=2cm] {};
    
            %draw the s-category stack
            \foreach \y[count=\c,evaluate={\yi=int(\c-1)}] in {2, 4, 6}{
                \node[circle, draw, inner sep=1pt, fill, label=above:{$\G$}] (g\yi) at (7,\y){};
                \node[circle, draw, inner sep=1pt, fill, label=above:{$A$}] (a\yi) at (9,\y){};
                \draw[->](g\yi) to[bend right=5] node[below]{\tiny $\deno{\etyperelation{\P_\yi}{\G}{v}{A}}$} (a\yi);
                \node[ellipse, draw, minimum width=5cm, minimum height=15mm,label=right:$S_\yi$] (s\yi) at (8,\y){};
            }
    
            % Hidden ellipse to draw functors to
            \node[ellipse, minimum width=5cm, minimum height=15mm] (s3) at (8,8){};
    
            %Draw the ... for the s-category stack
            \foreach \y[count=\c,evaluate={\yi=int(\c-1)}] in {7, 7.5, 8}{
                \node[fill, circle, inner sep=1pt] (p\yi) at (8, \y){};
            }
    
            % Draw index arrows
            \foreach \i in {0, 1, 2}{
                \draw[->, thick] (d\i) to (s\i);
            }
    
            % draw the re-indexing functors
    
            \foreach \source[count=\dest] in {0, 1, 2}{
                \draw[->, thick](s\source.north west) to[bend left=10] node[left]{$\pstar$} (s\dest.south west);
            }
    
            % Draw the quantification functors
            \foreach \dest[count=\source] in {0, 1, 2}{
                \draw[->, thick] 
                (s\source.south east) to[bend left=10] node[right]{$\forall_{\P_\dest}$} (s\dest.north east);
            }
    
            % Draw the internal morphisms in base category
            \foreach \dest[count=\source] in {0, 1}{
                \draw[->]
                (d\source) to[bend right=10] node[right]{$\p$} (d\dest);
            }
    
            %Draw the bracket
    
            \draw [decoration={brace,amplitude=8pt},decorate] ($(s2)+(10em,1ex)$) -- ($(s0)+(10em,-1ex)$);
            \node[text width=20mm] (Label) at (14,5){Fibres for each effect environment};
        \end{tikzpicture}

    \begin{itemize}
        \item Can we actually instantiate a category with the required structure? 
        \item Starting point: a model of EC in $\set$
    \end{itemize}
    
    \script{
        - So far, we've only said what structures are required to model PEC.
        - Haven't shown that there actually exists an indexed category with the required properties
        - so let's do that
        - It's fairly well known that you can instantiate models of languages with the same features as EC in the category of sets and functions
        - We want to extend one of these models into one for PEC
    }
\end{frame}

\begin{frame}{Instantiating a Model (2) - Fibres}
    \begin{itemize}\setlength\itemsep{3em}
        \item The fibre $\C(n)$ is the category of functors $[E^n, \set]$
        \item I.E. objects are functions that take a vector of ground effects and return a set $\deno{\wellformedt{\P}{A}}: E^n \rightarrow \obj(\set)$.
        \item Morphisms are dependent functions that return functions in $f: (\ev: E^n)\rightarrow A\ev\rightarrow B\ev$
        \item These fibres have S-Category features
    \end{itemize}

    \script{
        - Now we can think about the fibres
        - These are functor categories
        - I.E. their objects are functions returning sets and their morphisms are dependently typed functions
        - We can construct all the of the S-Category structure  as pointwise
    }
\end{frame}

\begin{frame}{Instantiating a Model (4) - Functors and Adjunctions}
\begin{itemize}
    \item Re-indexing functors act by pre-composition
    \begin{align*}
        A\in&\quad [E^n, \set]\\
        \theta\star(A) \emv =&\quad  A(\theta(\emv))\\
        \theta\star(f) \emv =&\quad f(\theta(\emv)): \theta\star(A) \rightarrow \theta\star(B)\\
    \end{align*}
    \item The quantification functor takes a product over all ground effects
    \begin{align*}
        \allEn(A)\env =&\Pi_{\e\in E}{A(\env, \e)}
    \end{align*}
\end{itemize}
    


    

    \script{
        - Finally, we need to construct the various functors
        - Reindexing functors are done using precomposition
        - quantification consists of a product over all ground effects
    }
\end{frame}