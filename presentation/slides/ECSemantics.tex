
\begin{frame}{Semantics of EC}
    \begin{itemize}
        \setlength\itemsep{3em}
        \item Can build a model of EC when we have
        \begin{itemize}
            \item CCC
            \item Strong Graded Monad
            \item Co-product and Subtyping (morphisms for if-statements)
        \end{itemize}
        \item We'll call this an S-category
    \end{itemize}
    
    \[
        \scalebox{.8}{$
        \ntreeruleI{Return}{f = \deno{\gtyperelation{v}{A}}}{\deno{\gtyperelation{\return{v}}{\moa}} = \point{A} \after f}
        \quad
        \ntreeruleI{Fn}{f = \deno{\typerelation{\gax}{v}{B}} : \G \times A \rightarrow B}{\deno{\gtyperelation{\lam{x}{A}{v}}{\ab}} = \cur{f} : \G \rightarrow B^A}
        $}
    \]

    \script{
        - As described, all of the language features can be modelled in a cartesian closed category with a graded monad, a coproduct, and subtyping morphisms.

        - Known as an S-Category

        - Here's an example, if we have the denotation of an expression here, we can get the denotation of using it as a pure computation by postcomposing with the unit of the graded monad.

    }
\end{frame}