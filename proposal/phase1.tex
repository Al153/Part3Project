\documentclass[11pt]{article}
\usepackage{a4wide,parskip,times}

\begin{document}

\centerline{\Large A Denotational Semantics for a polymorphic Effects Systems}
\vspace{2em}
\centerline{\Large \emph{A PartIII project proposal}}
\vspace{2em}
\centerline{\large A. J. Taylor (\emph{at736}), St John's College}
\vspace{1em}
\centerline{\large Project Supervisor: Prof Alan Mycroft}
\vspace{1em}

\begin{abstract}
\textsl{
	A category theoretic approach to build a graded monad based denotational semantics for a polymorphic effects system.
} 
\end{abstract}

\section{Introduction, approach and outcomes (500 words)}

\textsl{Provide an introduction to your project or essay. In particular, try to
  motivate the work and explain the relevant context (general
  background, as well as sufficient detail about any related
  work).}

\textsl{What's the basic idea and approach? What are you thinking of 
doing, and how is it going to solve the problem (or need) you've 
identified. What are you going to ``produce''? 
A project will typically produce one (or perhaps more) of the following:
a piece of software, an evaluation of a published result, a proof, or
the design (and perhaps the construction of) a new piece of hardware. An
essay will typically either review and critique a particular area of the
academic literature, or evaluate a published result or proof. Try to 
name the specific things (and describe them) in this part of the 
proposal -- this will allow you to refer to them in the next.} 

\section{Workplan (500 words)}
\textsl{Project students have approximately 26 weeks between the approval of 
the proposal by the Head of Department, and the submission of the dissertation. This section
should account for what you intend to do during that time. You should divide the time into two-week chunks including dates, and 
describe the work to be done (and, as relevant, milestones to be 
achieved) in each chunk. You should leave two 
chunks for writing a project dissertation. You should leave 1 chunk for contingencies.} 

\begin{tabular}{|p{3cm}||p{12cm}|}
\hline
	November & Plan + read\\ 
	December & Construct a simple monadic lambda calculus based language with a type system and operational semantics, prove  simple properties of operational semantics \\
	January &  Construct a framework for constructing a denotational semantics from a kernel.
Explore the effects of different kernels (e.g. Identity kernel for STLC, Optional kernel for PCF, full kernel for a simple imperatiive language) \\
	February & add polymorphism \\
	March & Contingency \\
	April &  Collate results\\
	May & write dissertation \\ 
\hline
\end{tabular}

\newpage
\appendix

\end{document}
