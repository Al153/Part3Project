\documentclass[11pt]{article}
\newcommand\comment[1]{}
\usepackage{a4wide,parskip,times}


\newcommand{\deno}[1]{{\bf [\![}#1{\bf ]\!]}}

\newcommand{\st}{$^{st}$}
\renewcommand{\th}{$^{th}$}
\newcommand{\nd}{$^{nd}$}
\newcommand{\rd}{$^{rd}$}

\begin{document}

\centerline{\Large A Denotational Semantics for a polymorphic Effects Systems}
\vspace{2em}
\centerline{\Large \emph{A PartIII project proposal}}
\vspace{2em}
\centerline{\large A. J. Taylor (\emph{at736}), St John's College}
\vspace{1em}
\centerline{\large Project Supervisor: Prof Alan Mycroft}
\vspace{1em}

\begin{abstract}
\textsl{
	A category theoretic approach to build a graded monad based denotational semantics for a polymorphic effects system.
} 
\end{abstract}

\section{Introduction, approach and outcomes (500 words)}

\textsl{Provide an introduction to your project or essay. In particular, try to
  motivate the work and explain the relevant context (general
  background, as well as sufficient detail about any related
  work).}

\textsl{
Modelling effects of a program is crucial in modern optimising compiler design. It allows statements to be reordered or pruned and simplification of program structure.
}

\textsl{
Denotational semantics allows "full program" analysis
}

\textsl{
There already exist denotational semantics for non-polymorphic effects systems using graded monads, but this may be improved by the use of polymorphism (c.f "Theorems for free" in polymorphic type systems)
}

\textsl{
Although semantics for polymorphic types is hard (russell's paradox) it is likely to be easier for effects as there is a lack of self recursion.
}


\paragraph{Deliverable}{

}

\comment {
What's the basic idea and approach? What are you thinking of  doing, and how is it going to solve the problem (or need) you've  identified. What are you going to ``produce''?  A project will typically produce one (or perhaps more) of the following: a piece of software, an evaluation of a published result, a proof, or the design (and perhaps the construction of) a new piece of hardware. An essay will typically either review and critique a particular area of the academic literature, or evaluate a published result or proof. Try to  name the specific things (and describe them) in this part of the  proposal -- this will allow you to refer to them in the next.} 

\section{Workplan (500 words)}

\begin{tabular}{|p{6cm}||p{10cm}|}
\hline
	2\nd December - 15\th December & Construct a simple graded-monadic lambda calculus based language with a type system and operational semantics. This language shall be designed such that effect polymorphism can be appended onto the core in an easy and intuitive way. I expect that I shall take the route of having an explicit graded monad in the language, and polymorphism shall be added in a similar fashion to the polymorphic lambda calculus with explicit generalisation and specialisation terms.\\\hline
	16\th December - 29\th December & Prove simple properties of operational semantics without effect polymorphism. These shall include Type Preservation, Progress, Type Safety.\\\hline
	30\th December - 12\th January & Characterise an abstract model for the language in category theory using cartesian closed categories. This shall be performed in a similar fashion to Andrew Pitts' example for STLC and the original paper by E. Moggi.  \\\hline
	13\th January - 26\th January &  Add effect polymorphism to the language and extend the proofs of simple operational properties to the new polymorphic language. \\\hline
	27\th January - 9\th February &  Extend denotational model to polymorphic language. I shall attempt to add morphisms between terms and their generalised equivalents ($gen: \deno{T} \rightarrow \deno{\forall \phi. T} $) and between polymorphic terms and their specialised equivaluents 
	($spec: \deno{\forall \phi. T} \rightarrow \deno{T[ \epsilon / \phi ]} $)
	\\\cline{1-1}
	10\th February - 23\rd February &  \\\hline
	24\th February - 9\th March & Continue extension of denotations, aiming to formalise and prove the standard properties of a denotational semantics (Soundness, Adequacy, equal denotations $\Rightarrow$ contextual equivalence \\\hline
	10\th March - 23\rd March & Extensions. \\\hline
	24\th March - 6\th April & Collation of results. \\\hline
	7\th April - 20\th April & Write dissertation. \\\cline{1-1}
	21\st April - 4\th May &\\\hline
	5\th May - 18\th May & Contingency and hand in.\\\cline{1-1}
	19\th May - 31\st May & \\
\hline
\end{tabular}

\newpage
\appendix

\end{document}
