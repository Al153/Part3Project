\documentclass{report}


%% Don't import the header multiple times

\ifdefined\HEADERIMPORTED
\else
\newcommand\HEADERIMPORTED[0]{This file is HEADERIMPORTED}
\usepackage{amssymb}

\usepackage{amsmath}
\usepackage[a4paper,includeheadfoot,margin=2.54cm]{geometry}
\usepackage{breqn}
\usepackage{amssymb}

\usepackage{amsmath}
\usepackage[a4paper,includeheadfoot,margin=2.54cm]{geometry}
\usepackage{breqn}
\usepackage{array}   % for \newcolumntype macro
\usepackage{tikz-cd}
\usepackage{tabstackengine}
\setstackEOL{\cr}
\setstackgap{L}{\normalbaselineskip}

\newcommand\todo[1]{\textbf{TODO: #1}}

\newcommand{\s}{\;}
\newcommand{\doin}[3]{\texttt{do}\s #1 \leftarrow #2 \s\texttt{in}\s #3}
\newcommand\apply[2]{#1\s#2}
\newcommand\ifthenelse[5]{\texttt{if}_{#1, #2}\s#3\s \texttt{then}\s #4 \s\texttt{else} \s#5}
\newcommand\const[1]{\texttt{C}^{#1}}
\newcommand\return[1]{\texttt{return} #1}
\newcommand\lam[3]{\lambda #1 : #2. #3}
\renewcommand\u[0]{\texttt{()}}
\newcommand{\U}[0]{\texttt{Unit}}
\renewcommand\t[0]{\texttt{true}}
\newcommand\f[0]{\texttt{false}}
\newcommand{\B}[0]{\texttt{Bool}}
\newcommand{\G}[0]{\Gamma}
\newcommand\D{\Delta}


% draw type relations
\newcommand{\typerelation}[3]{#1 \vdash #2 \colon #3}
\newcommand{\gtyperelation}[2]{\typerelation{\G}{#1}{#2}}

%draw tree rules
\newcommand{\treerule}[3]{(\text{#1}) \frac{#2}{#3}}
\newcommand{\condtreerule}[4]{\treerule{#1}{#2}{#3}(\text{if } #4)}

\newcommand{\subtype}[0]{\leq:}
\newcommand\subeffect[0]{\leq}

\newcommand{\M}[2]{\texttt{M}_{#1}{#2}}

\newcommand\lamtype[3]{#1 \rightarrow \M{#2}{#3}}
\newcommand{\1}[0]{\texttt{1}}

\newcommand\e[0]{\epsilon}

\newcommand{\db}[1]{{\bf [\![}#1{\bf ]\!]}}
\newcommand{\deno}[1]{\db{#1}_M}
\newcommand\after\circ
\newcommand\term[1]{\left\langle\right\rangle_{#1}}

\newcommand\point[1]{\eta_{#1}}
\newcommand\bind[3]{\mu_{#1, #2, #3}}

\newcommand\T[2]{T_{#1}{#2}}

\newcommand\pr[2]{\left\langle#1, #2\right\rangle}

% tensor strength Nat-tran
\newcommand\tstrength[3]{\texttt{t}_{#1, #2, #3}}

% Id morphism
\newcommand\Id[1]{\texttt{Id}_{#1}}

\newcommand\idg[0]{\Id{\G}}
% beta-eta equivalence
\newcommand\beequiv[0]{=_{\beta\eta}}
% Substitutions
\newcommand\si{\sigma}

\newcommand{\sub}[1]{\left[#1\right]}
\newcommand{\ssub}[2]{\left[#2 / #1\right]}
\newcommand{\ssi}[0]{\sub{\si}}

% beta-eta equivalence relation
\newcommand{\berelation}[4]{\typerelation{#1}{#2 \beequiv #3}{#4}}
\newcommand{\gberelation}[3]{\gtyperelation{#1 \beequiv #2}{#3}}


% Shortcuts for denotational equality
\newcommand{\denoequality}[4]{\deno{\typerelation{#1}{#2}{#4}} = \deno{\typerelation{#1}{#3}{#4}}}
\newcommand{\gdenoequality}[3]{\denoequality{\G}{#1}{#2}{#3}}

% Shorthand for monad types
\newcommand\mea[0]{\M{\e}{A}}
\newcommand\meb[0]{\M{\e}{B}}
\newcommand\mec[0]{\M{\e}{C}}

\newcommand\tea[0]{\T{\e}{A}}
\newcommand\teb[0]{\T{\e}{B}}
\newcommand\tec[0]{\T{\e}{C}}


\newcommand\moa[0]{\M{\1}{A}}
\newcommand\mob[0]{\M{\1}{B}}
\newcommand\moc[0]{\M{\1}{C}}

\newcommand\toa[0]{\T{\1}{A}}
\newcommand\tob[0]{\T{\1}{B}}
\newcommand\toc[0]{\T{\1}{C}}

\newcommand\aeb[0]{\lamtype{A}{\e}{B}}

% Shorthand for Gammas
\newcommand{\gax}[0]{\G, x: A}
\newcommand{\gby}[0]{\G, y: B}

% reduction function
\newcommand{\reduce}[0]{reduce}



% Combinators for building delta-based tree proof terms
\newcommand{\deltavrule}[4]{
    \treerule{Subtype}{\treerule{}{\D}{\typerelation{#1}{#2}{#3}}\s\s #3 \subtype #4}{\typerelation{#1}{#2}{#4}}}

\newcommand{\deltavruleprime}[4]{
    \treerule{Subtype}{\treerule{}{\D'}{\typerelation{#1}{#2}{#3}}\s\s #3 \subtype #4}{\typerelation{#1}{#2}{#4}}}

\newcommand{\deltavruleprimeprime}[4]{
        \treerule{Subtype}{\treerule{}{\D'}{\typerelation{#1}{#2}{#3}}\s\s #3 \subtype #4}{\typerelation{#1}{#2}{#4}}}
    
\newcommand{\deltacrule}[6]{
            \treerule{Subeffect}{\treerule{}{\D}{\typerelation{#1}{#2}{\M{#3}{#4}}}\s\s #4 \subtype #6\s\s #3 \subeffect #5}{\typerelation{#1}{#2}{\M{#5}{#6}}}}
\newcommand{\deltacruleprime}[6]{
    \treerule{Subeffect}{\treerule{}{\D'}{\typerelation{#1}{#2}{\M{#3}{#4}}}\s\s #4 \subtype #6\s\s #3 \subeffect #5}{\typerelation{#1}{#2}{\M{#5}{#6}}}}
\newcommand{\deltacruleprimeprime}[6]{
    \treerule{Subeffect}{\treerule{}{\D''}{\typerelation{#1}{#2}{\M{#3}{#4}}}\s\s #4 \subtype #6\s\s #3 \subeffect #5}{\typerelation{#1}{#2}{\M{#5}{#6}}}}
                            

\newcommand{\p}[0]{\pi_1}
\newcommand{\pp}[0]{\pi_2}

% short-hands for weakening
\newcommand{\wrel}[3]{#1 : #2 \triangleright #3}
\newcommand{\ok}[1]{#1 \texttt{Ok}}
\renewcommand\i[0]{\iota}
\newcommand\w{\omega}
\newcommand\dom[1]{\texttt{dom}(#1)}
\newcommand\x{\times}


% Combinators to build tree proofs
\newcommand{\truleconst}[0]{\treerule{Const}{\ok{\G}}{\gtyperelation{\const{A}}{A}}}
\newcommand{\truleunit}[0]{\treerule{Unit}{\ok{\G}}{\typerelation{\G}{\u}{\U}}}
\newcommand{\truletrue}[0]{\treerule{True}{\ok{\G}}{\typerelation{\G}{\t}{\B}}}
\newcommand{\trulefalse}[0]{\treerule{False}{\ok{\G}}{\typerelation{\G}{\f}{\B}}}


\newcommand{\E}[0]{\mathbb{E}}
\renewcommand{\dot}{\cdot}
\newcommand{\gens}[0]{::=}
\newcommand{\nil}[0]{\diamond}
\newcommand{\ground}[0]{\gamma}

% Terminal object of C
\newcommand{\terminal}[0]{\texttt{\1}}

% The category C
\newcommand{\C}[0]{\mathbb{C}}

% The category of locally-small categories
\newcommand{\Cat}[0]{\texttt{Cat}}
% Sub-effect Nat-trans
\newcommand{\dse}[2]{\db{#1 \subeffect #2}}

\newcommand\app[0]{\texttt{app}}
\newcommand\cur[1]{\texttt{cur}(#1)}
\newcommand{\ifnt}[1]{\texttt{If}_{#1}}


\newcommand{\setto}{:=}
\newcommand{\fv}[1]{\texttt{fv}(#1)}

% shorthand for inserting text to equations
\newcommand\qt[1]{\quad\text{#1}}


\fi

\begin{document}

Given a set based S-Category $\C$ which is a model of the non-polymorphic effect calculus, we generate an indexed category capable of modelling the polymorphic effect calculus.
\section{The Non-Polymorphic Model}
Since $\C$ is a model of the non-polymorphic calculus,
\begin{itemize}
    \item $\C$ is cartesian closed.
    \item $\C$ has a strong graded monad: $\Tz{}{}: (E, \dot, \subeffect_0, \1) \rightarrow [\C, \C]$
    \item $\C$ has a co-product on the terminal object $\1$.
\end{itemize}

In addition, we require that 
\begin{itemize}
    \item $\C$ should be complete (e.g a sub-category of $\set$)
    \item $E$ should be small.
\end{itemize}


\section{Base Category}

We construct the base category, $\Eff$ as follows:

\begin{itemize}
    \item $U = E$, the set of ground effects in the non-polymorphic language.
    \item $\1$ is a singleton set.
    \item $U^n = E^n$, set of $n$-wide tuples of effects, $\vec{\e}$
\end{itemize}

Hence when we treat effects that are well formed in $\P$ as morphisms, $E^n \rightarrow E$ in $\Eff$, we should treat them as functions $f: E^n \rightarrow E$. Ground effects become point functions: $e: \1 \rightarrow E$, so the denotation of a ground effect is the constant value function: $\deno{\typerelation{\P}{e}{\effect}} = \ev \mapsto e$

We extend the multiplication of ground effects to multiplication on effect functions, giving us our $\Mul$ operation

\begin{align}
    \Mul(f, g) & = f\dot g\\
    (f\dot g)(\ev) & = (f\ev)\dot(g\ev)\\
\end{align}

This satisfies naturality of $\Mul$.

\begin{equation}
    ((f\dot g)\after \theta)\ev = (f(\theta\ev))\dot(g(\theta\ev)) = ((f\after\theta)\dot(g\after\theta))\ev
\end{equation}

\section{S-Categories}
The semantic category, $\C_0$ of the effect-environment $\nil$ is $\C$.

Since each effect-environment is alpha equivalent to a natural number, the semantic category for $\P$ shall be represented as $\C_{\P} = \C_n = [E^n, \C]$, the category of functions $E^n\rightarrow \C$.

Objects in $[E^n, \C]$ are functions and we describe them by their actions on a generic vector of ground effects, $\ev$.

Morphisms in $[E^n, \C]$ are natural transformations between the functions. So:

\begin{align}
    m:\quad&A \rightarrow B \qt{ In $[E^n, \C]$}\\
    m\ev:\quad& A\ev \rightarrow B\ev \qt{In $\C$}\\
    (f\after g)\ev =\quad& (f\ev)\after(g\ev)\\
    \1(\ev) =\quad& \1
\end{align}

So morphisms are dependently typed functions from a vector of ground effects to morphisms in $\C$.
\subsection{Each S-Category is a CCC}
Since $\C$ is complete and a CCC, and $E^n$ is small, since $E$ is small, $[E^n, \C]$ is a CCC.

\begin{align}
    (A\times B)\ev & = (A\ev)\times(B\ev)\\
    \1\ev & = \1\\
    (B^A)\ev &= (B\ev)^{(A\ev)}\\
    \p\ev & = \p\\
    \pp\ev & = \pp\\
    \app\ev & = \app\\
    \cur{f}\ev & = \cur{f\ev}\\
    \pr{f}{g}\ev & = \pr{f\ev}{g\ev}\\
\end{align}

\subsection{Ground Types and Terms}
Each ground type in the non-polymorphic calculus has a fixed denotation $\deno{\g}\in\obj\C$. The ground type in the polymorphic calculus hence has a denotation represented by the constant function.

\begin{align}
    \deno{\g}:&\quad E^n\rightarrow \obj\C\\
    \ev \mapsto&\quad  \deno{\g}\\
\end{align}

Each constant term $\const{A}$ in the non-polymorphic calculus has a fixed denotation $\deno{\const{A}}\in\C(\1, A)$.

So the morphism $\deno{\const{A}}$ in $[E^n, \C]$ is the corresponding constant dependently typed morphism.

\begin{align}
    \deno{\const{A}}:&\quad[E^n, \C](\1, A)\\
    \ev\mapsto&\quad\deno{\const{A}}
\end{align}

\subsection{Graded Monad}
Given the strong graded monad $(\Tz{}{}, \pointz{}{}{}, \bindz{}{}{}, \tstrengthz{}{}{})$ on $\C$ we can construct an appropriate graded monad on $[E^n, \C]$.

\begin{align}
    \Tn{}{}:&\quad(E^n, \dot, \subeffect_n, \1_n) \rightarrow [[E^n, \C], [E^n, \C]]\\
    (\Tn{f}{A})\ev=&\quad \Tz{(f\ev)}{A\ev}\\
    (\pointn{A})\ev=&\quad\pointz{A\ev}\\
    (\bindn{f}{g}{A})\ev=&\quad\bindz{(f\ev)}{(g\ev)}{(A\ev)}\\
    (\tstrengthn{f}{A}{B})\ev=&\quad\tstrengthz{(f\ev)}{(A\ev)}{(B\ev)}
\end{align}

Through some mechanical proof and the naturality of the $\C$ strong graded monad, these morphisms are natural in their type parameters and form a strong graded monad in $[E^n, \C]$


\subsubsection{Naturality}

\begin{tikzcd}
    A\ev 
    \arrow{r}{\pointz{(A\ev)}}
    \arrow{d}{f\ev}
    &
    \Tz{\1}{(A\ev)}
    \arrow{d}{\Tz{\1}{(f\ev)}}
    \\
    B\ev
    \arrow{r}{\pointz{(B\ev)}}
    &
    \Tz{\1}{(B\ev)}
\end{tikzcd}

\begin{tikzcd}
    \Tz{(f\ev)}{\Tz{(g\ev)}{(A\ev)}}
    \arrow{r}{\bindz{f\ev}{g\ev}{(B\ev)}}
    \arrow{d}{\Tz{f\ev}{\Tz{g\ev}{m\ev}}}
    &
    \Tz{(f\ev)\dot(g\ev)}{(A\ev)}
    \arrow{d}{\Tz{(f\ev)\dot(g\ev)}{m\ev}}
    \\
    \Tz{(f\ev)}{\Tz{(g\ev)}{(B\ev)}}
    \arrow{r}{\bindz{f\ev}{g\ev}{(B\ev)}}
    &
    \Tz{(f\ev)\dot(g\ev)}{(A\ev)}
\end{tikzcd}


\begin{tikzcd}
    A\ev \times \Tz{f\ev}{(B\ev)}
    \arrow{r}{\tstrengthz{f\ev}{(A\ev)}{(B\ev)}}
    \arrow{d}{(m\ev\times\Id{\Tz{f\ev}{B}})}
    &
    \Tz{f\ev}{(A\ev\times B\ev)}
    \arrow{d}{\Tz{(f\ev)}{(m\ev\times \Id{B\ev})}}
    \\
    A'\ev \times \Tz{f\ev}{(B\ev)}
    \arrow{r}{\tstrengthz{f\ev}{(A'\ev)}{(B\ev)}}
    &
    \Tz{f\ev}{(A'\ev\times B\ev)}
\end{tikzcd}

\begin{tikzcd}
    A\ev \times \Tz{f\ev}{(B\ev)}
    \arrow{r}{\tstrengthz{f\ev}{(A\ev)}{(B\ev)}}
    \arrow{d}{(\Id{A\ev}\times \Tz{f\ev}{(m\ev)})}
    &
    \Tz{f\ev}{(A\ev\times B\ev)}
    \arrow{d}{\Tz{(f\ev)}{(\Id{A\ev} \times m\ev)}}
    \\
    A\ev \times \Tz{f\ev}{(B'\ev)}
    \arrow{r}{\tstrengthz{f\ev}{(A\ev)}{(B'\ev)}}
    &
    \Tz{f\ev}{(A\ev\times B'\ev)}
\end{tikzcd}


\subsubsection{Monad Laws}
\paragraph{Left Unit}

\begin{align}
    (\bindn{f}{\1}{A}\after\Tn{f}{\pointn{A}})\ev &= \bindz{(f\ev)}{\1}{(A\ev)} \after \Tz{f\ev}{(\pointz{A\ev})}\\
    &= \Id{\Tz{f\ev}{A\ev}}\\
    &=(\Id{\Tn{f}{A}})\ev
\end{align}

\paragraph{Right Unit}

\begin{align}
    (\bindn{\1}{g}{A}\after\pointn{\Tn{f}{A}})\ev &= \bindz{\1}{(f\ev)}{(A\ev)} \after (\pointz{\Tz{f\ev}{A\ev}})\\
    &= \Id{\Tz{f\ev}{A\ev}}\\
    &=(\Id{\Tn{f}{A}})\ev
\end{align}


\paragraph{Monad Associativity}

\begin{align}
    ((\bindn{f}{(g\dot h)}{A})\after\Tn{f}{(\bindn{g}{h}{A})})\ev &= \bindz{(f\ev)}{((g\ev)\dot(h\ev))}{(A\ev)}\after\Tz{f\ev}{\bindz{(h\ev)}{(g\ev)}{A\ev}}\\
    &= \bindz{((f\ev)\dot(g\ev))}{(h\ev)}{(A\ev)}\after\bindz{(f\ev)}{(g\ev)}{(\Tz{h\ev}{(A\ev)})}\\
    & = (\bindn{f\dot g}{h}{A}\after\bindn{f}{g}{\Tz{h}{A}})\ev
\end{align}
\subsubsection{Tensorial Strength}

\paragraph{Unitor Law}
\begin{align}
    (\Tn{f}{\pp})\ev&= \Tz{(f\ev)}{(\pp\ev)}\\
    & =  \Tz{(f\ev)}{(\pp)}\\
    & = \pp\\
    &= \pp\ev
\end{align}

\paragraph{Bind Law}

\begin{tikzcd}[ampersand replacement=\&]
    A \times \Tn{f}{\Tn{g}{B}} 
    \arrow [r, "\tstrength{f}{A}{\Tn{g}{B}}"]
    \arrow [dr, "\Id{A} \times \bindn{f}{g}{B}"]
    \& 
    \Tn{f}{(A \times \Tn{g}{B})} 
    \arrow [r, "\Tn{f}{\tstrength{g}{A}{B}}"]
    \& 
    \Tn{f}{\Tn{g}{(A \times B)}} 
    \arrow [d, "\bindn{f}{g}{A \times B}"]
    \\
    \&
    A \times \Tn{f \dot g}{B}  
    \arrow [r, "\tstrength{f \dot g}{A}{B}"] 
    \&
    \Tn{f \dot g}({A \times B)}
\end{tikzcd}

\begin{align}
    (\tstrengthn{(f\dot g)}{A}{B}\after (\Id{A}\times \bindn{f}{g}{B}))\ev & = (\tstrengthz{((f\ev)\dot (g\ev))}{(A\ev)}{(B\ev)}\after (\Id{A\ev}\times \bindn{(f\ev)}{(g\ev)}{(B\ev)}))\\
    & = \bindz{(f\ev)}{(g\ev)}{(A\times B)\ev}\after\Tz{f\ev}{(\tstrengthz{(g\ev)}{(A\ev)}{(B\ev)})}\after\tstrengthz{(f\ev)}{(A\ev)}{\Tz{g\ev}{(B\ev)}}\\
    & = (\bindn{f}{g}{(A\times B)}\after\Tn{f}{(\tstrengthn{g}{A}{B})}\after\tstrengthn{f}{A}{\Tn{g}{(B)}})\ev
\end{align}



\paragraph{Commutativity with Unit}

\begin{tikzcd}[ampersand replacement=\&]
    A \times B
    \arrow [r, "\Id{A} \times \point{B}"]
    \arrow [rd, "\point{A \times B}"]
    \&
    A \times \tob 
    \arrow [d, "\tstrength{\1}{A}{B}"]
    \\
    \&
    \Tn{\1}{(A \times B)}
\end{tikzcd}

\begin{align}
    (\tstrengthn{\1}{A}{B}\after(\Id{A}\times \pointn{A}))\ev & = \tstrengthz{\1}{(A\ev)}{(B\ev)}\after(\Id{A\ev}\times \pointz{A\ev})\\
    & = \pointz{A\ev\times B\ev}\\
    & = (\pointn{A\times B})\ev
\end{align}

\paragraph{Commutativity with $\alpha$}
Let $\alpha_{A, B, C} = \pr{\p\after\p}{\pr{\pp\after\p}{\pp}}: ((A \times B) \times C) \rightarrow (A \times (B \times C))$


\begin{tikzcd}[ampersand replacement=\&]
    (A\times B)\times \Tn{\e}{C} 
    \arrow [rr, "\tstrength{\e}{(A\times B)}{C}"]
    \arrow [d, "\alpha_{A, B, \Tn{\e}{C}}"]
    \& \& \Tn{\e}{((A \times B)\times C)}
    \arrow [d, "\Tn{\e}{\alpha_{A, B, C}}"]
    \\
    A \times (B \times \Tn{\e}{C}) 
    \arrow [r, "\Id{A}\times\tstrength{\e}{B}{C}"]
    \&
    A\times\Tn{\e}{(B \times C)} 
    \arrow [r, "\tstrength{\e}{A}{(B \times C)}"]
    \& \Tn{\e}{(A \times (B \times C))}
    \\
\end{tikzcd}

\begin{align}
    (\Tn{f}{\alpha_{A, B, C}}\after\tstrengthn{f}{A\times B}{C})\ev & = \Tz{f\ev}{\alpha_{A\ev, B\ev, C\ev}} \after\tstrengthz{(f\ev)}{(A\times B)\ev}{(C\ev)}\\
    & = \tstrengthz{(f\ev)}{(A\ev)}{(B\ev\times C\ev)}\after(\Id{A\ev}\times \tstrengthz{(f\ev)}{(B\ev)}{(C\ev)})\after\alpha_{A\ev, B\ev, C\ev}\\
    & = (\tstrengthn{f}{A}{(B\times C)}\after(\Id{A}\times \tstrengthn{f}{B}{C})\after\alpha_{A, B, C})\ev\\
\end{align}


\subsection{Sub-Effecting}
Given a collection of sub-effecting natural transformation in $\C$,

\begin{align}
    \deno{\e_1\subeffectz \e_2}:&\quad \Tz{\e_1}{} \rightarrow \Tz{\e_2}{}
\end{align}

We can form sub-effect natural transformations in $[E^n, \C]$:

\begin{align}
    \deno{f\subeffectn g}:&\quad\Tn{f}{}\rightarrow \Tn{g}{}\\
    \deno{f\subeffectn g} A \ev:&\quad\Tn{f\ev}{(A\ev)}\rightarrow \Tn{g\ev}{(B\ev)}\\
    =&\quad \deno{f\ev\subeffectz g\ev}A\ev
\end{align}

\subsubsection{Naturality}

\begin{tikzcd}
    \Tz{f\ev}{A\ev}
    \arrow{r}{\deno{f\ev\subeffectz g\ev}A\ev}
    \arrow{d}{\Tz{f\ev}{m\ev}}
    &
    \Tz{g\ev}{A\ev}
    \arrow{d}{\Tz{g\ev}{m\ev}}
    \\
    \Tz{f\ev}{B\ev}
    \arrow{r}{\deno{f\ev\subeffectz g\ev}B\ev}
    &
    \Tz{g\ev}{B\ev}
\end{tikzcd}

\subsubsection{Commutes With Tensor Strength}

\begin{tikzcd}[ampersand replacement=\&]
    A \times \Tn{f}{B} \arrow [r, "\Id{A} \times \dsen{f}{g}_B"] \arrow [d, "\tstrengthn{f}{A}{B}"] \&
    A \times \Tn{g}{B} \arrow [d, "\tstrengthn{g}{A}{B}"] \\
    \Tn{f}{(A \times B)} \arrow [r, "\dsen{f}{g}_{ A \times B}"] \&
    \Tn{g}{(A \times B)} 
\end{tikzcd}

\begin{align}
    (\tstrengthn{g}{A}{B}\after(\Id{A}\times \db{f\subeffectn g}_B))\ev & = \tstrengthz{(g\ev)}{(A\ev)}{(B\ev)}\after(\Id{A\ev}\times \db{f\ev \subeffectz g\ev}_{B\ev})\\
    & = \db{f\ev \subeffectz g\ev}_{(A\times B)\ev} \after\tstrengthz{(f\ev)}{(A\ev)}{(B\ev)}\\
    & = (\db{f \subeffectn g}_{(A\times B)} \after\tstrengthn{f}{A}{B})\ev\\
\end{align}

\subsubsection{Commutes with Join}

\begin{tikzcd}[ampersand replacement=\&]
    \Tn{f}{\Tn{g}{}} 
    \arrow [rr, "\Tn{f}{\deno{g\subeffectn g'}
    }"]
    \arrow [d, "\bindn{f}{g}{}"]
    \&  \&
    \Tn{f}{\Tn{g'}{}}
    \arrow [rr, "\db{f \subeffectn f'}_{M, \Tn{g'}{}}"]
    \& \&
     \Tn{f'}{\Tn{g'}{}} 
     \arrow [d, "\bindn{f'}{g'}{}"]
     \\
    \Tn{f\dot g}{}
    \arrow [rrrr, "\deno{f\dot g\subeffectn f'\dot g'}"]
    \& \&
     \& \&
    \Tn{f'\dot g'}{}
\end{tikzcd}

\begin{align}
    (\dsen{f\dot g}{f'\dot g'}_A\after\bindn{f}{g}{A})\ev & = \dsez{(f\ev)\dot(g\ev)}{(f'\ev)\dot (g\ev)}_{A\ev}\after\bindz{(f\ev)}{(g\ev)}{(A\ev)}\\
    & = \bindz{(f\ev)}{(g\ev)}{(A\ev)}\after\dsez{f\ev}{f'\ev}_{\Tz{g'\ev}{(A\ev)}}\after \Tz{f\ev}{\dsez{g\ev}{g'\ev}}_{(A\ev)}\\
    &= \bindn{f}{g}{A}\after\dsen{f}{f'}_{\Tn{g'}{A}}\after\Tn{f}{\dsen{g}{g'}}_A
\end{align}

\subsection{Sub-Typing}
Sub-typing in $[E^n, \C]$ holds via sub-typing in $\C$

\begin{align}
    \deno{A\subtype_n B}:&\quad A\rightarrow B\\
    \deno{A\subtype_n B}\ev=&\quad \deno{A\ev\subtype_0 B\ev}
\end{align}

So the subtyping relation $A \subtype B$ forms a morphism in $[E^n, \C]$

\section{Functors Between S-Categories}
For a function $\theta: E^m \rightarrow E^n$ , the re-indexing functor $\theta\star$ is defined as follows:

\begin{align}
    \theta\star:\quad& [E^n, \C] \rightarrow [E^m, \C]\\
    \theta\star(A)\emv=&\quad A(\theta (\emv))\\
    f:\quad& A \rightarrow B \in[E^n, \C]\\
    \theta\star(f)\emv =&\quad f(\theta(\emv)): A(\theta(\emv)\rightarrow B(\theta(\emv)))
\end{align}

\subsection{Quantification}
We need to define $\allEn: [E^{n+1}, \C]\rightarrow[E^n, \C]$

So
\begin{align}
    (\allEn A)\env =&\quad \Pi_{\e\in E} A(\env, \e)\\
    m:&\quad A\rightarrow B\\
    (\allEn m):&\quad \allEn A\rightarrow \allEn B\\
    (\allEn m)\env =&\Pi_{\e\in E} m(\env, \e)\\
\end{align}

\subsection{Adjunction}
It is the case that:
$$\pstar \dashv \allEn$$

With unit:

\begin{align}
    \unit{A}:&\quad A \rightarrow \allEn\pstar A\\
    \unit{A}(\env)=&\quad\finpr{\Id{A(\env,e)}}{\e\in E}
\end{align}


And co-unit


\begin{align}
    \counit{B} : &\quad \pstar\allEn B \rightarrow B\\
    \counit{B}(\env, \e) =&\quad \pi_\e : \Pi_{e\in E}B(\env, \e)\rightarrow \Pi_{e\in E}B(\env, \e)
\end{align}


We then define the natural bi-jection as so:

\begin{equation}
    \bar{(-)}: [E^{n_1}, \C](\pstar A, B) \leftrightarrow [E^n,\C] (A, \allEn B): \widehat{(-)}
\end{equation}

\begin{align}
    m:&\quad \pstar A \rightarrow B\\
    \bar{m}:&\quad A\rightarrow \allEn B\\
    \bar{m}(\env) =&\quad \finpr{m(\env, \e)}{e\in E}
\end{align}


\begin{align}
    n:&\quad A \rightarrow \allEn B\\
    \widehat{n}:&\quad \pstar A\rightarrow B\\
    \widehat{n}(\env, \e_{n+1}) =&\quad\pi_\e\after g(\env)
\end{align}

\subsubsection{This is an Adjunction}
For any $g: \pstar A \rightarrow B$,

\begin{align}
    (\counit{B}\after\pstar(\bar{g}))(\env, \e_{n+1}) & = \pi_{\e_{n+1}}\after\finpr{g(\env, \e')}{\e'\in E}\\
    &= g(\env, \e_{n+1})
\end{align}

\subsection{Beck-Chevalley Condition}

For $\theta: E^m \rightarrow E^n$:
\begin{align}
    ((\theta\star\after\allEn)A)\env &= \theta\star(\allEn A)\env\\
    &= (\allEn A)(\theta(\env))\\
    &= \Pi_{\e\in E}(A(\theta(\env), \e))\\
    &= \Pi_{\e\in E}(((\theta\times \Id{U})\star A)(\env, \e))\\
    &= \allEm((\theta\times\Id{E})\star A)\env\\
    &= ((\allEm\after(\theta\times\Id{E})\star)A)\env
\end{align}

And $\bar{(\theta\times\Id{U})\star \counit{}} = \Id{\theta\star\after\allI}$.

\begin{align}
    \bar{(\theta\times\Id{U})\star \counit{A}} \ev  & = \finpr{(\theta\times\Id{U})\star\counit{A}(\ev, \e)}{\e\in E}\\
    & = \finpr{\counit{A}(\theta\ev, \e)}{e\in E}\\
    & = \finpr{\pi_\e}{\e\in E}: \Pi_{\e\in E}A(\theta\ev, \e) \rightarrow \Pi_{\e\in E}A(\theta\ev, \e)\\
    & = \Id{\Pi_{\e\in E}A(\theta\ev, \e)}\\
    & = \Id{\allII\after(\theta\times\Id{U})\star A}\ev\\
    & = \Id{\theta\star\after\allI}
\end{align}

\end{document}
