
%% Don't import the header multiple times

\ifdefined\HEADERIMPORTED
\else
\newcommand\HEADERIMPORTED[0]{This file is HEADERIMPORTED}
\usepackage{amssymb}

\usepackage{amsmath}
\usepackage[a4paper,includeheadfoot,margin=2.54cm]{geometry}

% For typesetting tree rules
\usepackage{mathpartir}


\usepackage{array}   % for \newcolumntype macro
\usepackage{tikz-cd}
\usepackage{tabstackengine}
\usepackage{breqn}
\usepackage{stmaryrd}

\usepackage{framed}
\usepackage{listings}

\usepackage{amsthm}
%Theorems
\usepackage[utf8]{inputenc}
\usepackage[english]{babel}
\newtheorem{theorem}{Theorem}[section]
\newtheorem{corollary}{Corollary}[theorem]
\newtheorem{lemma}[theorem]{Lemma}
\theoremstyle{definition}
\newtheorem{definition}{Definition}[section]

\usepackage{tikz}

\usetikzlibrary{shapes.geometric,fit}
\usetikzlibrary{arrows,automata,positioning}
\usetikzlibrary{decorations.pathreplacing,calc}



\setstackEOL{\cr}
\setstackgap{L}{\normalbaselineskip}

\newcommand\todo[1]{\textbf{TODO: #1}}
\newcommand\needsRef[1]{\textbf{Reference Needed: (#1)}}
\newcommand\fixLayout[1]{\textbf{Fix Layout: #1}}

\newcommand{\s}{\;}
\newcommand{\doin}[3]{\texttt{do}\s #1 \leftarrow #2 \s\texttt{in}\s #3\s}
\newcommand\apply[2]{#1\s#2}
\newcommand{\pifthenelse}[4]{\texttt{if}_{#1}\s#2\s \texttt{then}\s #3 \s\texttt{else} \s#4\s}
\newcommand\ifthenelse[5]{\pifthenelse{#1, #2}{#3}{#4}{#5}}
\newcommand\const[1]{\texttt{C}^{#1}}
\newcommand\return[1]{\texttt{return} \s#1\s}
\newcommand\lam[3]{\lambda #1 : #2. #3\s}
\renewcommand\u[0]{\texttt{()}}
\newcommand{\U}[0]{\texttt{Unit}}
\renewcommand\t[0]{\texttt{true}}
\newcommand\f[0]{\texttt{false}}
\newcommand{\B}[0]{\texttt{Bool}}
\newcommand{\G}[0]{\Gamma}
\newcommand\D{\Delta}


% draw type relations
\newcommand{\typerelation}[3]{#1 \vdash #2 \colon #3}
\newcommand\wellformed[2]{#1\vdash #2}
\newcommand\wellformedok[2]{\ok{#1\vdash #2}}



\newcommand{\gtyperelation}[2]{\typerelation{\G}{#1}{#2}}
 

\newcommand\treerulez[1]{\inferrule{ }{#1}}
\newcommand\treeruleI[2]{\inferrule{#1}{#2}}
\newcommand\treeruleII[3]{\inferrule{#1 \\ #2}{#3}}
\newcommand\treeruleIII[4]{\inferrule{#1 \\ #2 \\ #3}{#4}}
\newcommand\treeruleIV[5]{\inferrule{#1 \\ #2 \\ #3 \\ #4}{#5}}
\newcommand\treeruleV[6]{\inferrule{#1 \\ #2 \\ #3 \\ #4 \\ #5}{#6}}

\newcommand\ntreerulez[2]{(\text{#1})\inferrule{ }{#2}}
\newcommand\ntreeruleI[3]{(\text{#1})\inferrule{#2}{#3}}
\newcommand\ntreeruleII[4]{(\text{#1})\inferrule{#2 \\ #3}{#4}}
\newcommand\ntreeruleIII[5]{(\text{#1})\inferrule{#2 \\ #3 \\ #4}{#5}}
\newcommand\ntreeruleIV[6]{(\text{#1})\inferrule{#2 \\ #3 \\ #4 \\ #5}{#6}}
\newcommand\ntreeruleV[7]{(\text{#1})\inferrule{#2 \\ #3 \\ #4 \\ #5 \\ #6}{#7}}

\newcommand\condtreerulez[3]{(\text{#1})\inferrule{ }{#2}(\text{if } #3)}
\newcommand\condtreeruleI[4]{(\text{#1})\inferrule{#2}{#3}(\text{if } #4)}
\newcommand\condtreeruleII[5]{(\text{#1})\inferrule{#2 \\ #3}{#4}(\text{if } #5)}
\newcommand\condtreeruleIII[6]{(\text{#1})\inferrule{#2 \\ #3 \\ #4}{#5}(\text{if } #6)}
\newcommand\condtreeruleIV[7]{(\text{#1})\inferrule{#2 \\ #3 \\ #4 \\ #5}{#6}(\text{if } #7)}
\newcommand\condtreeruleV[8]{(\text{#1})\inferrule{ #2 \\ #3 \\ #4 \\ #5 \\ #6 }{#7}(\text{if } #8)}



\newcommand{\subtype}[0]{\leq:}
\newcommand\subeffect[0]{\leq}

\newcommand{\M}[2]{\texttt{M}_{#1}{#2}}

\newcommand\lamtype[3]{#1 \rightarrow \M{#2}{#3}}
\newcommand{\1}[0]{\texttt{1}}

\newcommand\e[0]{\epsilon}

\newcommand{\db}[1]{{\bf [\![}#1{\bf ]\!]}}
\newcommand{\deno}[1]{\db{#1}}
\newcommand\after\circ
\newcommand\term[1]{\left\langle\right\rangle_{#1}}

\newcommand\bindmu[0]{\mu}
\newcommand\point[1]{\eta_{#1}}
\newcommand\bind[3]{\bindmu_{#1, #2, #3}}

\newcommand\T[2]{T_{#1}{#2}}

\newcommand\pr[2]{\left\langle#1, #2\right\rangle}
\newcommand\finpr[2]{\left\langle#1\right\rangle_{#2}}

\newcommand\strengtht[0]{\texttt{t}}
% tensor strength Nat-tran
\newcommand\tstrength[3]{\strengtht_{#1, #2, #3}}

% Id morphism
\newcommand\Id[1]{\texttt{Id}_{#1}}

\newcommand\idg[0]{\Id{\G}}
% beta-eta equivalence
\newcommand\beequiv[0]{\approx}
% Substitutions
\newcommand\si{\sigma}

\newcommand{\sub}[1]{\left[#1\right]}
\newcommand{\ssub}[2]{\left[#2 / #1\right]}
\newcommand{\ssi}[0]{\sub{\si}}

% beta-eta equivalence relation
\newcommand{\berelation}[4]{\typerelation{#1}{#2 \beequiv #3}{#4}}
\newcommand{\gberelation}[3]{\gtyperelation{#1 \beequiv #2}{#3}}


% Shortcuts for denotational equality
\newcommand{\denoequality}[4]{\deno{\typerelation{#1}{#2}{#4}} = \deno{\typerelation{#1}{#3}{#4}}}
\newcommand{\gdenoequality}[3]{\denoequality{\G}{#1}{#2}{#3}}

% Shorthand for monad types
\newcommand\mea[0]{\M{\e}{A}}
\newcommand\meb[0]{\M{\e}{B}}
\newcommand\mec[0]{\M{\e}{C}}

\newcommand\tea[0]{\T{\e}{A}}
\newcommand\teb[0]{\T{\e}{B}}
\newcommand\tec[0]{\T{\e}{C}}


\newcommand\moa[0]{\M{\1}{A}}
\newcommand\mob[0]{\M{\1}{B}}
\newcommand\moc[0]{\M{\1}{C}}

\newcommand\toa[0]{\T{\1}{A}}
\newcommand\tob[0]{\T{\1}{B}}
\newcommand\toc[0]{\T{\1}{C}}

\newcommand\aeb[0]{\lamtype{A}{\e}{B}}

% Shorthand for Gammas
\newcommand{\gax}[0]{\G, x: A}
\newcommand{\gby}[0]{\G, y: B}

% reduction function
\newcommand{\reduce}[0]{reduce}



% Combinators for building delta-based tree proof terms
\newcommand{\deltavrule}[4]{
    \ntreeruleII{Subtype}{\treeruleI{\D}{\typerelation{#1}{#2}{#3}}}{#3 \subtype #4}{\typerelation{#1}{#2}{#4}}}

\newcommand{\deltavruleprime}[4]{
    \ntreeruleII{Subtype}{\treeruleI{\D'}{\typerelation{#1}{#2}{#3}}}{#3 \subtype #4}{\typerelation{#1}{#2}{#4}}}

\newcommand{\deltavruleprimeprime}[4]{
        \ntreeruleII{Subtype}{\treeruleI{\D'}{\typerelation{#1}{#2}{#3}}}{#3 \subtype #4}{\typerelation{#1}{#2}{#4}}}
    
\newcommand{\deltacrule}[6]{
            \ntreeruleIII{Subeffect}{\treeruleI{\D}{\typerelation{#1}{#2}{\M{#3}{#4}}}}{#4 \subtype #6}{#3 \subeffect #5}{\typerelation{#1}{#2}{\M{#5}{#6}}}}
\newcommand{\deltacruleprime}[6]{
    \ntreeruleIII{Subeffect}{\treeruleI{\D'}{\typerelation{#1}{#2}{\M{#3}{#4}}}}{#4 \subtype #6}{#3 \subeffect #5}{\typerelation{#1}{#2}{\M{#5}{#6}}}}
\newcommand{\deltacruleprimeprime}[6]{
    \ntreeruleIII{Subeffect}{\treeruleI{\D''}{\typerelation{#1}{#2}{\M{#3}{#4}}}}{#4 \subtype #6}{#3 \subeffect #5}{\typerelation{#1}{#2}{\M{#5}{#6}}}}
                            

\newcommand{\p}[0]{\pi_1}
\newcommand{\pp}[0]{\pi_2}

% short-hands for weakening
\newcommand{\wrel}[3]{#1 : #2 \triangleright #3}
\newcommand{\ok}[1]{#1 \texttt{Ok}}
\renewcommand\i[0]{\iota}
\newcommand\w{\omega}
\newcommand\dom[1]{\texttt{dom}(#1)}
\newcommand\x{\times}


% Combinators to build tree proofs
\newcommand{\truleconst}[0]{\ntreeruleI{Const}{\ok{\G}}{\gtyperelation{\const{A}}{A}}}
\newcommand{\truleunit}[0]{\ntreeruleI{Unit}{\ok{\G}}{\typerelation{\G}{\u}{\U}}}
\newcommand{\truletrue}[0]{\ntreeruleI{True}{\ok{\G}}{\typerelation{\G}{\t}{\B}}}
\newcommand{\trulefalse}[0]{\ntreeruleI{False}{\ok{\G}}{\typerelation{\G}{\f}{\B}}}


\newcommand{\E}[0]{\mathbb{E}}
\renewcommand{\dot}{\cdot}
\newcommand{\gens}[0]{::=}
\newcommand{\nil}[0]{\diamond}
\newcommand{\ground}[0]{\gamma}

% Terminal object of C
\newcommand{\terminal}[0]{\texttt{\1}}

% The category C
\newcommand{\C}[0]{\mathbb{C}}
\newcommand{\Cz}[0]{\C_0}
\newcommand\DC[0]{\mathbb{D}}

% The category of locally-small categories
\newcommand{\Cat}[0]{\texttt{Cat}}
% Sub-effect Nat-trans
\newcommand{\dse}[2]{\db{#1 \subeffect #2}}

\newcommand\app[0]{\texttt{app}}
\newcommand\cur[1]{\texttt{cur}(#1)}
\newcommand{\ifnt}[1]{\texttt{If}_{#1}}


\newcommand{\setto}{:=}
\newcommand{\fv}[1]{\texttt{fv}(#1)}

% shorthand for inserting text to equations
\newcommand\qt[1]{\quad\text{#1}}

% Co-product short-hands
\newcommand\inr[0]{\texttt{inr}}
\newcommand\inl[0]{\texttt{inl}}
    
\newcommand\fld[2]{\left[#1,#2\right]}
\newcommand{\diag}[1]{\delta_{#1}}
\newcommand{\twist}[2]{\tau_{#1, #2}}

\newcommand\ifMorph[3]{\app\after((\fld{\cur{#2\after\pp}}{\cur{#3\after\pp}}\after #1)\times \idg)\after \diag{\G}}


% Polymorphic short-hands
\newcommand\elam[2]{\Lambda #1. #2}
\newcommand{\eapp}[2]{#1\s#2}
\renewcommand{\a}[0]{\alpha}
\newcommand{\all}[2]{\forall #1. #2}
\renewcommand{\P}[0]{\Phi}

\renewcommand{\b}[0]{\beta}
\newcommand{\g}[0]{\gamma}
\renewcommand\d[0]{\delta}
\newcommand\oke[2]{\wellformed{#1}{\ok{#2}}}
\newcommand\etyperelation[4]{\typerelation{#1\mid#2}{#3}{#4}}
\newcommand{\gpetyperelation}[2]{\etyperelation{\P}{\G}{#1}{#2}}
\newcommand{\gppetyperelation}[2]{\etyperelation{\P'}{\G}{#1}{#2}}


\newcommand{\eberelation}[5]{\berelation{#1\mid#2}{#3}{#4}{#5}}
\newcommand{\gpeberelation}[3]{\berelation{\P\mid\G}{#1}{#2}{#3}}
\newcommand{\gppeberelation}[3]{\berelation{\P'\mid\G}{#1}{#2}{#3}}

\newcommand{\dotp}[0]{\dot_\P}
\newcommand{\fntype}[2]{#1\rightarrow #2}
\newcommand{\ab}[0]{\fntype{A}{B}}

\newcommand\wrelw[2]{\wrel{\w}{#1}{#2}}
\renewcommand\proof[0]{\paragraph{Proof:}}
\newcommand{\case}[1]{\paragraph{Case #1:}}
\newcommand{\subcase}[1]{\subparagraph{Case: #1}}
\newcommand\bi[0]{By inversion}

%pre-filled effect-weakening relations
\newcommand\ewrel[4]{\wellformed{#1}{\wrel{#2}{#3}{#4}}}
\newcommand\pewrel[3]{\ewrel{\P}{#1}{#2}#3}
\newcommand\ppewrel[3]{\ewrel{\P'}{#1}{#2}#3}

\newcommand\subtypep[0]{\subtype_\P}
\newcommand\subtypepp[0]{\subtype_{\P'}}
\newcommand\subeffectp[0]{\subeffect_{\P}}
\newcommand\subeffectpp[0]{\subeffect_{\P'}}


\newcommand{\allI}[0]{\forall_I}
\newcommand{\allII}[0]{\forall_{I'}}
\newcommand\allIU[0]{\forall_{I\times U}}
\newcommand\type[0]{\texttt{Type}}
\newcommand\effect[0]{\texttt{Effect}}
\newcommand\ciw[0]{\C(I, W)}
\newcommand\ciu[0]{\C(I, U)}
\newcommand\ciuw[0]{\C(I\times U, W)}
\newcommand\cipw[0]{\C(I', W)}
\newcommand\cipu[0]{\C(I', U)}
\newcommand\ciuu[0]{\C(I\times U, U)}
\newcommand\cii[0]{\C(I', I)}
\newcommand\Eff[0]{\texttt{Eff}}
\newcommand\Mul[0]{\texttt{Mul}}
\newcommand\singleton[0]{\ast}
\renewcommand\star[0]{^*}
\renewcommand\bar[1]{\overline{#1}}

\newcommand\subtypeg[0]{\subtype_\g}
\newcommand\subtypepa[0]{\subtype_{\P, \a}}
\newcommand\subtypeppa[0]{\subtype_{\P', \a}}

\usepackage{scalerel,stackengine}
\stackMath
\renewcommand\widehat[1]{%
\savestack{\tmpbox}{\stretchto{%
  \scaleto{%
    \scalerel*[\widthof{\ensuremath{#1}}]{\kern.1pt\mathchar"0362\kern.1pt}%
    {\rule{0ex}{\textheight}}%WIDTH-LIMITED CIRCUMFLEX
  }{\textheight}% 
}{2.4ex}}%
\stackon[-6.9pt]{#1}{\tmpbox}%
}
\parskip 1ex

\newcommand\pstar[0]{\p\star}

\newcommand\edeltavrule[5]{\deltavrule{#1 \mid #2}{#3}{#4}{#5}}

\newcommand\subeffecttree[4]{\ntreeruleII{Computation}{
    #1\subeffectp #3}{#2 \subtypep #4
}{\M{#1}{
    #2
}\subtypep\M{#3}{#4}}}

\newcommand{\edeltavruleprime}[5]{
        \deltavruleprime{#1\mid #2}{#3}{#4}{#5}}
    
\newcommand{\edeltavruleprimeprime}[5]{
        \deltavruleprimeprime{#1\mid #2}{#3}{#4}{#5}}
    
\newcommand{\edeltacrule}[6]{
            \ntreeruleII{Subtype}{
                \treeruleI{
                    \D
                }{
                    \typerelation{\P\mid#1}{#2}{\M{#3}{#4}}
                }
            }{
                \ntreeruleII{Computation}{
                    #4 \subtypep #6
                    }{
                         #3 \subeffectp #5
                }{
                    \M{#3}{#4}\subtypep{\M{#5}{#6}}
                }
            }{
                \typerelation{\P\mid #1}{#2}{\M{#5}{#6}}
            }
        }
        

        \newcommand{\edeltacruleprime}[6]{
            \ntreeruleII{Subtype}{
                \treeruleI{
                    \D'
                }{
                    \typerelation{\P\mid #1}{#2}{\M{#3}{#4}}
                }
            }{
                \ntreeruleII{Computation}{
                    #4 \subtypep #6
                }{#3 \subeffectp #5
                }{
                    \M{#3}{#4}\subtypep{\M{#5}{#6}}
                }
            }{
                \typerelation{\P\mid #1}{#2}{\M{#5}{#6}}
            }
        }
                   

        \newcommand{\edeltacruleprimeprime}[6]{
            \ntreeruleII{Subtype}{
                \treeruleI{
                    \D''
                }{
                    \typerelation{\P\mid #1}{#2}{\M{#3}{#4}}
                }
            }{
                \ntreeruleII{Computation}{
                    #4 \subtypep #6
                    }{ #3 \subeffectp #5
                }{
                    \M{#3}{#4}\subtypep{\M{#5}{#6}}
                }
            }{
                \typerelation{\P\mid #1}{#2}{\M{#5}{#6}}
            }
        }

        \newcommand\obj[0]{\texttt{obj }}


        \newcommand{\Tz}[2]{\texttt{T}^0_{#1}#2}
        \newcommand{\Tn}[2]{\texttt{T}^n_{#1}#2}
        \newcommand{\Tm}[2]{\texttt{T}^m_{#1}#2}
        
        \newcommand{\pointz}[1]{\point{#1}^0}
        \newcommand{\pointn}[1]{\point{#1}^n}
        \newcommand{\pointm}[1]{\point{#1}^m}
        
        \newcommand{\bindz}[3]{\bind{#1}{#2}{#3}^0}
        \newcommand{\bindn}[3]{\bind{#1}{#2}{#3}^n}
        \newcommand{\bindm}[3]{\bind{#1}{#2}{#3}^m}
        
        \newcommand\tstrengthz[3]{\tstrength{#1}{#2}{#3}^0}
        \newcommand\tstrengthn[3]{\tstrength{#1}{#2}{#3}^n}
        \newcommand\tstrengthm[3]{\tstrength{#1}{#2}{#3}^m}
        
        \newcommand\set[0]{\texttt{Set}}
        \newcommand\cccat[0]{\textit{CCCat}}

        \newcommand\ev[0]{\vec{\e}}
        \newcommand\emv[0]{\vec{\e_m}}
        \newcommand\env[0]{\vec{\e_n}}
        
        \newcommand\subeffectz[0]{\subeffect_0}
        \newcommand\subeffectn[0]{\subeffect_n}
        \newcommand\subeffectm[0]{\subeffect_m}
        
        \newcommand\dsem[2]{\db{#1 \subeffectm #2}}
        \newcommand\dsen[2]{\db{#1 \subeffectn #2}}
        \newcommand\dsez[2]{\db{#1 \subeffectz #2}}
        \newcommand\dsep[2]{\db{#1 \subeffectp #2}}
        \newcommand\dsepp[2]{\db{#1 \subeffectpp #2}}
        
        \newcommand\allEn[0]{\forall_{E^n}}
        \newcommand\allEm[0]{\forall_{E^m}}
        
        \newcommand\counit[1]{\boldsymbol{\epsilon}_{#1}}
        \newcommand\unit[1]{\boldsymbol{\eta}_{#1}}
\fi